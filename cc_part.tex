\subsection{Problem setting and applications}
\label{sub:problem_setting}


Like other clustering frameworks, in \pcc{}, we are given a set of objects and we want to gather
them into groups (called clusters) so that objects belonging to one cluster are similar to each
other while being dissimilar to objects from all the other clusters.

In \pcc{}, we formalize this problem by considering objects as the nodes of a graph $G$, whose edges
weight encode similarity. Namely, in the most general case, for nodes $u$ and $v$, the edge between
$u$ and $v$ is associated with two positive numbers:
$w_{u,v}^+$ denotes the strength of the similarity between $u$ and $w$;
$w_{u,v}^-$ denotes the strength of the dissimilarity between $u$ and $w$.
Note however that in many applications, only one of these two numbers is non zero in which case we
more conveniently set $w_{u,v} = \begin{cases}
	 w_{u,v}^+ & \quad \text{if } w_{u,v}^+ > 0 \text{ and } w_{u,v}^-=0 \\
	-w_{u,v}^- & \quad \text{if } w_{u,v}^+ = 0 \text{ and } w_{u,v}^->0 \\
\end{cases}$

Now consider a clustering \cluster{} of $V$, that is a function from $V$ to $\Nbb{}^{|V|}_{>0}$
that assigns to each node a cluster index. For instance, $\cluster(u) = 3$ means that $u$ belongs
to the third cluster. We can evaluate how \cluster{} fits our clustering paradigm in two ways,
either by the number of \emph{agreements}, that is the weighted number of positive edges inside
clusters plus the weighted number of negative edges across clusters; or by the number of
\emph{disagreements}, that is the weighted number of negative edges inside clusters plus the
weighted number of positive edges across clusters. Given a cost function $c$, which is usually the
identity, \pcc{} can then be seen as graph optimization problem, either of maximizing agreements
(\maxa{}):
\begin{equation}
	\max_{\cluster{}} \sum_{\cluster(u) = \cluster(v)} c(w_{uv}^+) +
	\sum_{\cluster(u) \neq \cluster(v)} c(w_{uv}^-)
	\label{eq:maxa}
\end{equation}
or minimizing disagreements (\mind{}):
\begin{equation}
	\min_{\cluster{}} \sum_{\cluster(u) = \cluster(v)} c(w_{uv}^-) +
	\sum_{\cluster(u) \neq \cluster(v)} c(w_{uv}^+)
	\label{eq:mind}
\end{equation}

Although an optimal clustering $\cluster^\star$ achieves the same value on both \eqref{eq:maxa}
and \eqref{eq:mind}, we will see in \autoref{sub:state_of_the_art} that the latter objective is in
some sense \enquote{easier}.

\begin{figure}[hbt]
	\centering
	\includegraphics[width=0.9\linewidth]{assets/tikz/cc_objectives_tikz.pdf}
	\caption[]{A clustering of a small with two disagreements due to two cycles with one negative edge.}
	\label{fig:cc_objectives}
\end{figure}

\subsection{Relation with edge sign prediction}
\label{sub:relation_with_edge_sign_prediction}

\iffalse
For the 2 clusters case ($k=2$), characterization proven already in 1936 by \textcite{Konig36}, as noted in
\cite{Zaslavsky2012} (commenting on \cite{harary1953}: \enquote{Although Thm. 3 was anticipated by
\textcite[Theorem X.11]{Konig36}  without the terminology of signs, here is the
first recognition of the crucial fact that labelling edges by elements of a
group—specifically, the sign group—can lead to a general theory.})
and \cite{Huffner2010} (\enquote{\textcite{Konig36} proved the following characterization of
	balanced graphs. For a graph $G = (V , E)$, the following are equivalent:\marginpars{There is a
	proof in \autocite[p. 111]{BookKleinberg2010}, maybe I can rewrite it as well}
	\begin{enumerate}
		\item $V$ can be partitioned into two sets $V_1$ and $V_2$ called sides such that there is no
			negative edge $\{v, w\} \in E$ with both $v, w \in V_1$ or both $v, w \in V_2$ and no negative
			edge $\{v, w\}$ with $v \in V_1$ and $w \in V_2$ .
		\item $V$ can be colored with two colors such that for all $\{v, w\} \in E^-$, the vertices $v$
			and $w$ have different colors, and for all $\{v, w\} \in E^+$, the vertices $v$ and $w$ have
			the same color. The color classes correspond to the sides.
		\item $G$ does not contain cycles with an odd number of negative edges.
	\end{enumerate}
	Using the characterization by a coloring, it is easy to see that balance of a signed
graph can be checked in linear time by depth-first search.})
\fi

\subsection{State of the art}
\label{sub:state_of_the_art}

% ICML paper this year that touch something very related multicut and give recent applications in
% vision https://arxiv.org/abs/1503.03791

\subsection{Variants and extensions}
\label{sub:variants_and_extensions}

\subsubsection{\pcc{} under stability assumption}
\label{ssub:cc_under_stability_assumption}

\iffalse
Haris Angelidakis, Konstantin Makarychev, and Yury Makarychev. 2017.
Algorithms for Stable and Perturbation-Resilient Problems. STOC’17
\href{http://ttic.uchicago.edu/~yury/papers/two-stable.pdf}{10.1145/3055399.3055487}
improves over the one cited in the internship description
\fi

\subsubsection{Parallel \pcc{}}
\label{ssub:parallel_cc}

\subsection{Empirical evaluation?}
\label{sub:cc_empiracal_evaluation}
