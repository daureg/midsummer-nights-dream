\subsection{Problem setting and applications}
\label{sub:problem_setting}

Like other clustering frameworks, in \pcc{}, we are given a set of objects and we want to gather
them into groups (called clusters) so that objects belonging to one cluster are similar to each
other while being dissimilar to objects from all the other clusters.

In \pcc{}, we formalize this problem by considering objects as the nodes of a graph $G$, whose edges
weight encode similarity. Namely, in the most general case, for nodes $u$ and $v$, the edge between
$u$ and $v$ is associated with two positive numbers:
$w_{u,v}^+$ denotes the strength of the similarity between $u$ and $w$;
$w_{u,v}^-$ denotes the strength of the dissimilarity between $u$ and $w$.
Note however that in many applications, only one of these two numbers is non zero in which case we
more conveniently set $w_{u,v} = \begin{cases}
	 w_{u,v}^+ & \quad \text{if } w_{u,v}^+ > 0 \text{ and } w_{u,v}^-=0 \\
	-w_{u,v}^- & \quad \text{if } w_{u,v}^+ = 0 \text{ and } w_{u,v}^->0 \\
\end{cases}$

Now consider a clustering \cluster{} of $V$, that is a function from $V$ to $\Nbb{}^{|V|}_{>0}$
that assigns to each node a cluster index. For instance, $\cluster(u) = 3$ means that $u$ belongs
to the third cluster. We can evaluate how \cluster{} fits our clustering paradigm in two ways,
either by the number of \emph{agreements}, that is the weighted number of positive edges inside
clusters plus the weighted number of negative edges across clusters; or by the number of
\emph{disagreements}, that is the weighted number of negative edges inside clusters plus the
weighted number of positive edges across clusters. Given a cost function $c$, which is usually the
identity, \pcc{} can then be seen as graph optimization problem, either of maximizing agreements
(\maxa{}):
\begin{equation}
	\max_{\cluster{}} \sum_{\cluster(u) = \cluster(v)} c(w_{uv}^+) +
	\sum_{\cluster(u) \neq \cluster(v)} c(w_{uv}^-)
	\label{eq:maxa}
\end{equation}
or minimizing disagreements (\mind{}):
\begin{equation}
	\min_{\cluster{}} \sum_{\cluster(u) = \cluster(v)} c(w_{uv}^-) +
	\sum_{\cluster(u) \neq \cluster(v)} c(w_{uv}^+)
	\label{eq:mind}
\end{equation}

Although an optimal clustering $\cluster^\star$ achieves the same value on both \eqref{eq:maxa}
and \eqref{eq:mind}, we will see in \autoref{sub:state_of_the_art} that the latter objective is in
some sense \enquote{easier}. Another interesting feature of the \pcc{} problem is that contrary to
other clustering formulations, it does not require us to set the number of clusters $k$ beforehand.
Instead, $k$ emerges as a natural property of the solution. Since clustering is an unsupervised
problem, this is generally handy. However, in some situations, we may have prior knowledge on how
many clusters are the data, or external constraints. This can be handled with parametrized version
of \pcc{} \autocites{Giotis2006}{Fomin2014}\marginpars{According to many articles, \pcc{} and
cluster editing are equivalent, but even on complete graphs or on $(V,E^+)$ I don't see why.}.
% The cluster edit distance of a graph is the smallest number of edges to change for it to admit a
% perfect clustering (i.e., a union of cliques).  Equivalently, it is the cost of the optimal
% correlation clustering. [From KDDTuto14, slides 204]

In \autoref{fig:cc_objectives}, we show a simple instance of \pcc{} and one of its optimal solution.
\begin{figure}[hbt]
	\centering
	\includegraphics[width=0.8\linewidth]{assets/tikz/cc_objectives_tikz.pdf}
	\caption[Small example of \pcc{}]{A small graph with eight nodes and ten edges. Solid edges
	represent positive edges and dashed edges represent negative edges. A clustering \cluster{}
	is showed with 3 clusters: $\{1, 2, 4\}$, $\{3, 5, 6\}$ and $\{7, 8\}$. \cluster{} incurs two
	disagreements: the negative edge between nodes $1$ and $2$ within the blue cluster, and the
	positive edge $6,7$ between the orange and green clusters. Those disagreements are created
	by two cycles with one negative edge and thus cannot be avoided, meaning that \cluster{} is
	optimal. However it is not the unique solution, for instance merging the orange and green
	clusters would also yields two disagreements.}
	\label{fig:cc_objectives}
\end{figure}

\subsubsection{Applications}
\label{ssub:cc_applications}

According to \textcite[Section 5]{Demaine2006}, \pcc{} is well suited to several situations:
\begin{itemize}
   \item when the items to be clustered do not belong to a natural metric space (preventing
      approaches such as $k$-means) but we still know for some pairs whether they are similar or
      not.
   \item when we do not know the number of clusters beforehand but we have a similarity measure. In
      that case, we can select a problem-specific similarity threshold and set all edges with a
      similarity larger than the threshold to be positive while the others are set to negative.
   \item when we have a classic clustering problem (that is a set of objects, a distance between
      them and an objective function to minimize) with additional pairwise constraints of the form
      \emph{must-link/cannot-link}. Instead of restraining a clustering algorithm to the space of
      feasible solution, we convert the distances between objects and the constraints into signed
      edges and solve the resulting \pcc{} problem. 
\end{itemize}

In addition to these general considerations, \pcc{} has also been used in several domains:
\begin{description}
   \item[Computer Vision] For instance image segmentation
      \autocites{Bagon2011}{Kim2011}{VolumeSegmentation12}.

\Textcite{Shape3D17} develop a method to extract a network of descriptive curves from 3D shapes.
After an initial stage of generating many such \emph{flowlines}, they describe in Section 6 a \pcc{}
formulation to extract \emph{reliable} representative flowlines, using geometric constraints to give 

\textsc{Learning to Divide and Conquer for Online Multi-Target Tracking}
In order to track several targets across sequential video frames, \textcite{multiTracking15} propose
a multistage framework. One step revolves around a cost matrix $A$ and zones

\item[Natural Language Processing] 
   coreference resolution is solved by \textcite[Section 2.3]{graphicalCoreference04} using a
undirected graphical model on which performing inference is equivalent to \pcc{}.
	     \Textcite{Elsner2009} compared various heuristics with a bound of the optimal solution
	     obtained through SDP relaxation and show that best performing ones are within few
	     percents of it, provided they are followed by a local search step.
Another application in NLP is clustering words based on distributional embedding while adding
antonym constraints~\autocite{SignedWordRatings}.
\item[Biology]

		\Textcite{Mason2009} analyze a signed co-expression networks of genes
		involved in embryonic stem cells to find which genes are related to
		pluripotency or self-renewal.  Also something about haplotype assembly
		\autocite{Das2015}.

\Textcite{Ben-Dor99} assume there exists a perfect clustering, and receive as
input a complete similarity matrix, which is corrupted by measurement error.
They give an $O(n^2(\log n)^c)$ algorithm that recover the planted disjoint
cliques with high probability.

	   \enquote{They examined dynamical systems, where a gene is modeled as a vertex and an
	      activating connection is modeled as a positive edge and an inhibiting connection is
	      modeled as a negative edge. The claim is that biological dynamical systems are close
	      to being balanced, and that finding a minimum set of edges to delete to make the graph
	      balanced can be used to decompose the graph into “monotone subsystems”, which exhibit
	   stable behavior and thus allow a better understanding of the dynamics of a
	system.}\autocite{monotoneBiology07}

     \item[Network science]
study political vote \autocites{BrazilCC17}{Jiang2015}{Mendonca2015}. Also in politics
community structure \autocites{Traag2009}{CommunityUN12} (more in the variants part)

Visualization of signed social graph \autocite{Luca10}


\item[Others]
   \begin{itemize}
	\item duplicate detection, also called entity resolution \autocite{DeDup09}

statistical physics (Barahona 1982), Barahona F (1982) On the computational complexity of Ising
	   spin glass models. J Phys A: Math Gen 15(10):3241–3253 \autocite{Ising82}

portfolio risk analysis
	   \enquote{In this context, each security is represented by a vertex in the signed graph
	      while the correlation between securities is represented by the set of signed edges. A
	      balanced signed graph with only positive edges represents a speculative portfolio
	      since all its securities tend to move in the same direction, either on the upside or
	      on the downside. On the other hand, a balanced signed graph with at least one negative
	      edge is associated with a limited risk portfolio. Such a portfolio is defined by two
	      sets of securities, each set with a tendency to move in tandem, while some pair of
	      assets (connected by negative edges) tend to move in opposite directions providing the
	      investors with a hedging guarantee. According to Harary et al. (2003), an unbalanced
	   signed graph represents an unpredictable portfolio.}\autocite{portfolio02}

Given an electrical circuit layout, \textcite{circuitDesign07} extract a graph of its components
(called shifter) that must be assigned one of two possible phases. Because two shifters next to some
specific shape must be in opposite phase and two shifters separated by less than a specified
distance must be of the same phase, the authors look for a two-clustering of the nodes that will
minimize the number of disagreements.
% [21] N. Gülpinar, G. Gutin, G. Mitra, and A. Zverovitch. Extracting pure network submatrices in linear programs using signed graphs. Discrete Applied Mathematics, 137:359–372, 2004
% + Figueiredo2011
   \end{itemize}
\end{description}

\subsection{Relation with edge sign prediction}
\label{sub:relation_with_edge_sign_prediction}

\iffalse
For the 2 clusters case ($k=2$), characterization proven already in 1936 by \textcite{Konig36}, as noted in
\cite{Zaslavsky2012} (commenting on \cite{harary1953}: \enquote{Although Thm. 3 was anticipated by
\textcite[Theorem X.11]{Konig36}  without the terminology of signs, here is the
first recognition of the crucial fact that labelling edges by elements of a
group—specifically, the sign group—can lead to a general theory.})
and \cite{Huffner2010} (\enquote{\textcite{Konig36} proved the following characterization of
	balanced graphs. For a graph $G = (V , E)$, the following are equivalent:\marginpars{There is a
	proof in \autocite[p. 111]{BookKleinberg2010}, maybe I can rewrite it as well}
	\begin{enumerate}
		\item $V$ can be partitioned into two sets $V_1$ and $V_2$ called sides such that there is no
			negative edge $\{v, w\} \in E$ with both $v, w \in V_1$ or both $v, w \in V_2$ and no negative
			edge $\{v, w\}$ with $v \in V_1$ and $w \in V_2$ .
		\item $V$ can be colored with two colors such that for all $\{v, w\} \in E^-$, the vertices $v$
			and $w$ have different colors, and for all $\{v, w\} \in E^+$, the vertices $v$ and $w$ have
			the same color. The color classes correspond to the sides.
		\item $G$ does not contain cycles with an odd number of negative edges.
	\end{enumerate}
	Using the characterization by a coloring, it is easy to see that balance of a signed
graph can be checked in linear time by depth-first search.})
\fi

\subsection{State of the art}
\label{sub:state_of_the_art}

% ICML paper this year that touch something very related multicut and give recent applications in
% vision https://arxiv.org/abs/1503.03791

\subsection{Variants and extensions}
\label{sub:variants_and_extensions}

\subsubsection{\pcc{} under stability assumption}
\label{ssub:cc_under_stability_assumption}

\iffalse
Haris Angelidakis, Konstantin Makarychev, and Yury Makarychev. 2017.
Algorithms for Stable and Perturbation-Resilient Problems. STOC’17
\href{http://ttic.uchicago.edu/~yury/papers/two-stable.pdf}{10.1145/3055399.3055487}
improves over the one cited in the internship description
\fi

\subsubsection{Parallel \pcc{}}
\label{ssub:parallel_cc}

\subsection{Empirical evaluation?}
\label{sub:cc_empiracal_evaluation}
