\subsection{Problem setting and applications}
\label{sub:problem_setting}

\subsection{Relation with edge sign prediction}
\label{sub:relation_with_edge_sign_prediction}

\iffalse
For the 2 clusters case ($k=2$), characterization proven already in 1936 by \textcite{Konig36}, as noted in
\cite{Zaslavsky2012} (commenting on \cite{harary1953}: \enquote{Although Thm. 3 was anticipated by
\textcite[Theorem X.11]{Konig36}  without the terminology of signs, here is the
first recognition of the crucial fact that labelling edges by elements of a
group—specifically, the sign group—can lead to a general theory.})
and \cite{Huffner2010} (\enquote{\textcite{Konig36} proved the following characterization of
	balanced graphs. For a graph $G = (V , E)$, the following are equivalent:\marginpars{There is a
	proof in \autocite[p. 111]{BookKleinberg2010}, maybe I can rewrite it as well}
	\begin{enumerate}
		\item $V$ can be partitioned into two sets $V_1$ and $V_2$ called sides such that there is no
			negative edge $\{v, w\} \in E$ with both $v, w \in V_1$ or both $v, w \in V_2$ and no negative
			edge $\{v, w\}$ with $v \in V_1$ and $w \in V_2$ .
		\item $V$ can be colored with two colors such that for all $\{v, w\} \in E^-$, the vertices $v$
			and $w$ have different colors, and for all $\{v, w\} \in E^+$, the vertices $v$ and $w$ have
			the same color. The color classes correspond to the sides.
		\item $G$ does not contain cycles with an odd number of negative edges.
	\end{enumerate}
	Using the characterization by a coloring, it is easy to see that balance of a signed
graph can be checked in linear time by depth-first search.})
\fi

\subsection{State of the art}
\label{sub:state_of_the_art}

% ICML paper this year that touch something very related multicut and give recent applications in
% vision https://arxiv.org/abs/1503.03791

\subsection{Variants and extensions}
\label{sub:variants_and_extensions}

\subsubsection{\pcc{} under stability assumption}
\label{ssub:cc_under_stability_assumption}

\iffalse
Haris Angelidakis, Konstantin Makarychev, and Yury Makarychev. 2017.
Algorithms for Stable and Perturbation-Resilient Problems. STOC’17
\href{http://ttic.uchicago.edu/~yury/papers/two-stable.pdf}{10.1145/3055399.3055487}
improves over the one cited in the internship description
\fi

\subsubsection{Parallel \pcc{}}
\label{ssub:parallel_cc}

\subsection{Empirical evaluation?}
\label{sub:cc_empiracal_evaluation}
