\section{Conclusion}

In this chapter, we started our characterization of edges in complex networks by studying the \esp{}
problem in \dssn{}. As mentioned in the introduction, given a network topology and a few labeled
interactions, being able to infer the polarity of the remaining interactions is valuable to improve
user experience. Our main insight is that two features of users behavior, trollness and
(un)trustworthiness, are key to understand their interactions. This leads us to design a simple sign
generative model. Such a model serves both as a theoretical justification for many successful
heuristics and as the underpinning of our \usrule{} algorithm. This algorithm is fast, trivially
parallelizable, provably close to the Bayes optimal predictor on dense graphs and experimentally
accurate even on sparse graphs. We further exploit this model in the batch setting through an
approximation of the maximum likelihood estimation procedure. While making this approximation, we
moreover cast the problem as a node prediction problem. This can be efficiently tackled by standard
Label Propagation algorithms.

We validate our theoretical results by experimentally assessing these two methods in the small
training set regime, on synthetic and real-world datasets. We draw two main conclusions from our
experiments. First, our generative model is robust across domains. Indeed, it produces predictors
that are empirically both close to the Bayes optimal and to general linear models trained on
trollness and trustworthiness features. This is notable as this result is achieved by simply
counting negative edges, without resorting to Stochastic Gradient Descent methods typically involved
in training large scale linear models. Second, our methods are in practice either strictly better
than their competitors in terms of prediction quality or, when they are not, they are faster.

Finally, we study the online adversarial setting, where trollness and (un)trustworthiness naturally
lend themselves to define a notion of edge sign regularity. Based on this regularity, we provide an
upper and an (almost matching) lower bounds on the expected number of prediction mistakes. Before
exploring in the next chapter another learning bias for \esp{}, which is not based on trollness and
(un)trustworthiness, we present directions in which this work could be extended.

\bigskip

Given the fundamental role of our generative model, we would like to adapt it in order to handle
the three problem extensions presented in the related work of \autoref{sec:troll_related}.

\begin{enumerate}[1.,]

  \item When available, using \textbf{side information} about users is an alluring option to
    alleviate the cold start problem. Recall that in our experiments, we saw that if we learn our
    model on historical data, we cannot successfully predict the interactions involving new users,
    for their trollness and trustworthiness are not yet known to us. However, we expect that for
    existing users, side information is correlated with their value of $p$ and $q$. This knowledge
    thus provides a parameterised prior for the $p$ and $q$ values of news users. As we observe
    their behavior in the network, our estimation of $p$ and $q$ would rely less and less on prior
    information and more and more on observed data, as it is common in a Bayesian approach. Note
    that this introduces an additional learning stage.

  \item Our generative model, and in particular the prior distribution $\mu(p,q)$, could be
    exploited to guide label queries in an adaptive \textbf{active learning} setting. Given a budget
    of queries $B$, and after an initial phase of querying edge labels \uar{}, one might use
    concentration results on $\mu$\footnote{As done in the proof of \autoref{t:active},
    \vpageref{ssec:troll_proof_bayes}.} to devise an informativeness criterion, indicating for which
    node the value of $p$ and $q$ are the most uncertain. Combined with the graph topology, this
    would suggest which edge is the most important to query next, until the budget is exhausted.

  \item So far we discussed binary classification of edges. In \textbf{weighted graphs} though, a
    refined problem would be to predict the weight of unlabeled edges. Let us assume for simplicity
    that the weights are bounded and within $[-1, 1]$. The most immediate solution would be, for an
    edge \euv{}, to linearly shift our estimation of $\frac{p_u+q_v}{2}=\etauv{}$ from $[0,1]$ to
    $[-1,1]$ and use this quantity as our prediction. We could also change the semantic of \etauv{}
    from being the probability of \euv{} to be positive to the mean of a narrow Gaussian from which
    the weight of \euv{} is drawn. Finally, by letting $p_u$ range from $-1$ to $+1$, we note a
    similarity with fairness and goodness, suggesting $p_u\times q_v$ could approximate the weights.
\end{enumerate}
