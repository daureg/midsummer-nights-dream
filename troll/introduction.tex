As we saw in the introduction, many situations in various fields can be modeled by signed graphs.
One of the most natural usage of signed graphs, though, is the study of social networks. That is
indeed what first motivated their inception. Quite naturally, we thus begin our exploration of the
role of edges in complex networks by a binary classification problem in signed graphs. Namely, we
want to predict whether an observed interaction between two nodes is positive or negative and we
call this problem \esp{}.\footnote{This chapter is closely based on a paper written with several
co-authors~\autocite{trollSign17}.} To motivate this problem from a practical point of view, let us
first define precisely the type of social networks we consider. We call \emph{Directed Signed Social
Network} (DSSN) a directed graph whose nodes are either human beings (often called users if they are
the members of an online community) or artifacts directly created by individuals\footnote{This
semantic detail allows us to treat citation networks as social networks.}. Furthermore, a directed
edge from node $u$ to node $v$ denote an interaction initiated by the user $u$ and whose object is
user $v$. Such an interaction can be positive (to praise, to support, to trust, to befriend) or
negative (to criticize, to oppose, to distrust, to make enemy with).

Known examples are \textsc{Ebay}, where users trust or distrust agents in the network based on their
personal interactions, \textsc{Slashdot}, where each user can tag another user as friend or foe,
\textsc{BlaBlaCar}, a carpooling website where users can evaluate drivers and passengers as pleasant
travelers or not, and \textsc{Epinion}, where users can rate positively or negatively not only
products, but also other users. In such examples, the sign of the interaction is explicit. However,
there are other cases where such interactions are only displayed in their raw form and need further
processing to be given a sign. The typical situation where this happens is when interactions are
mediated through text, for instance in Twitter or in the comment section of some online content. It
is then possible to extract signs using sentiment analysis methods~\autocite{signedText12}.

Another distinction is that in most cases, only the owner of the network can observe its graph.
Think for instance of \textsc{YouTube}, where users can like or dislike the video of another user
but where only aggregate anonymous statistics are available publicly. The situation is similar on
the \textsc{Stack Overflow} community, where users can upvote or downvote answers according to their
perceived quality, but where only the total score are available.

\bigskip

With this definition of social networks, we now look at how the learning problem of \esp{} can be
used in real world applications. More precisely, we present four possible applications, and order
them by how severely negative interactions impact the well-being of users of the networks.
First, it could help improve the quality of link recommendation in solely positive networks. Such
networks indeed contain implicit negative links~\autocite{Yang2012}. By asking users to label a
small proportion of existing links as truly positive or actually negative, recommended links could
be discarded if they are predicted to be negative.
%
Second, this could be used to monitor news textual comments in a scalable fashion and thus ensure
that online debates remain courteous and constructive. For instance, \textcite{betterDebate14} show
that by visually reminding comment authors of the importance of respecting the plurality of
opinions, the quality of debates is improved (other references can be found
in~\autocite{journalism15}). In a more extreme case, banning users with aggressive and hateful
behavior has proved effective in the case of Reddit~\autocite{RedditBan15}.
%
Third, like many other fields, human resource management is undergoing a transformation through the
increasing use of Machine Learning~\autocite{MLinHR16}. Among other tools, assessing personal
relationships between employees is crucial to maintain a productive work
environment~\autocite{friendshipTeam02}. Whereas this
can be done by invasive wearable sensors~\autocite{WearableBehavior09}, our methods could predict
which employees would form the tighter teams from a small amount of labeled data.
%
Fourth, and most critical, predicting the sign of interactions could be a tool to detect users with
a high proportion of negative interactions. Indeed, because people tend to be less inhibited in
their online interactions~\cite{Suler04}, some users may join an online community with the main goal
to disrupt it, by engaging into anti-social behavior and creating conflictual relationships with
other members. This kind of attitude expressed publicly on social media leads to the following
definition of \emph{trolls}: \enquote{users whose real intentions are to cause disruption and/or to
trigger or exacerbate conflict for the purposes of their own amusement}~\autocite{Hardaker10}.
\Textcite{Shachaf10} elaborate on their motives, adding that boredom, attention seeking, and
revenge motivate trolls; they find pleasure from causing damage to other people or to the community
as a whole. An extreme form of trolling is bullying (\ie{} the behavior of someone intentionally and
repeatedly harming a victim that is unable to defend himself or herself) and just like anything
else, it has an online version called cyber bullying~\autocite{cyberbullying13}. Such behavior is
rather widespread, one study revealing that 72\% of \np{1454} surveyed teenagers reported at least
one occurrence of cyberbullying, most of them through instant messaging~\autocite{Juvonen08}. While
we have not conducted experimental study on this topic, we believe that being able to predict such
harmful interactions would be beneficial for the majority of users.

%Consider, in particular, a user joining an online community within a social network. His/her behavior will often fit one of these two stereotypes: the new member could play well with other users, establishing positive relationships with those who have been helpful. Or, the new user could try and disrupt the community by engaging into anti-social behavior, and creating conflictual relationships with other members. This behavioral dichotomy is supported by decades of research in psychology, starting with the seminal work~\autocite{Dissonance57} about cognitive dissonance (when someone is acting in contradiction with his/her personal beliefs, ideas, or values). This kind of attitude expressed publicly on social media leads to the definition of \enquote{trolls}~\autocite{Hardaker10}: users whose real intentions are to cause disruption and/or to trigger or exacerbate conflict for the purposes of their own amusement. The paper~\autocite{Shachaf10} elaborates on their motives, adding that boredom, attention seeking, and revenge motivate trolls; they find pleasure from causing damage to other people or to the community as a whole.

% When the social network is signed, specific challenges arise in both network analysis and
% learning. On the one hand, novel methods are required to tackle standard tasks (e.g., user
% clustering, link prediction, targeted advertising/recommendation, prediction of user interests,
% and analysis of the spreading of diseases in epidemiological models). On the other hand, new
% problems such as edge sign prediction, which we consider here, naturally emerge.

\bigskip

Such applications motivate studying \esp{}, which is the problem of classifying the positive or
negative nature of the links based on the
network topology. Prior knowledge of the network topology is often a realistic assumption, for in
several situations the discovery of the link sign can be more costly than acquiring the topological
information of the network. For instance, when two users of an online social network communicate on
a public web page, we immediately detect a link. Yet, the classification of the link sign as
positive or negative may require more involved techniques.

From the modeling and algorithmic viewpoints, because of the huge amount of available networked
data, a major concern in developing learning methods for \esp{} is algorithmic
scalability. Many successful, yet simple heuristics for \esp{} are based on the
troll-trust features, \ie{} on the fraction of outgoing negative links (trollness) and incoming
positive links (trustworthiness) at each node. We first define these notions and more notations in
\autoref{s:prel}, along with others tools and precise problem statements. Then we study such
troll-trust heuristics by defining in \autoref{s:gen} a probabilistic
generative model for the signs on the directed links of a given network. We also show that these
heuristics can be understood and analyzed as approximators to the Bayes optimal classifier for our
generative model. In this context, we design our first batch algorithm and show in
\autoref{ss:bayes_approx} that it provably approximates the Bayes classifier on dense graphs.
We furthermore gather empirical evidence supporting our probabilistic model by observing
that a logistic model trained on trollness and trustworthiness features alone is able to learn
weights that, on all datasets considered in our experiments, consistently satisfy the properties
predicted by our model.

We then introduce suitable graph transformations defining reductions from \esp{} to
node sign prediction problems. This opens up the possibility of using the arsenal of known
algorithmic techniques developed for node classification. In particular, we introduce in
\autoref{ss:passive} our second batch algorithm. It takes the form of a Label
Propagation algorithm that, combined with our problem reduction, approximates the maximum likelihood
estimator of our probabilistic generative model.
In order to compare our two algorithms with existing work, we then describe with more details
previous heuristics and related methods in \autoref{sec:troll_related}.
We then experimentally evaluate our proposed approach in \autoref{s:exp}. On synthetic data, we
confirm the quality of our approximation of the Bayes predictor. More importantly, on real-world
data, we show the competitiveness of our approach in terms of both prediction performance
(especially in the regime where training data are scarce) and scalability.

Finally, in \autoref{s:algonline}, we point out that the notions of trollness and trustworthiness
naturally define a measure
of complexity, or learning bias, for the signed network that can also be used to design
\emph{online} (\ie{} sequential) learning algorithms for the \esp{} problem. The
learning bias encourages settings where the nodes in the network have polarized features (e.g.,
trollness/trustworthiness are either very high or very low). Our online analysis holds under
adversarial conditions, namely, without any stochastic assumption on the assignment of signs to the
network links.
