\section{Generative Model for Edge Labels}\label{s:gen}

We now define the stochastic generative model for edge labels we use in the batch learning setting.
Given the graph $G = (V,E)$, let the label $\yuv \in \spin$ of directed edge $(u,v) \in E$ be
generated as follows. Each node $u \in V$ is endowed with two latent parameters $p_u, q_u \in
[0,1]$, which we assume to be generated, for each node $u$, by an independent draw from a fixed but
unknown joint prior distribution $\mu(p,q)$ over $[0,1]^2$. Each label $\yuv \in \spin$ is then
generated by an independent draw from the mixture of $p_u$ and $q_v$
$$\Pr\big( \yuv = 1 \big) = \tfrac{p_u + q_v}{2}$$
This can be seen pictorially in \autoref{fig:troll_genmodel}.
\begin{marginfigure}
	\centering
	\includegraphics[width=0.98\linewidth]{tikz/troll_genmodel_tikz}
	\caption{The sign \yuv{} of the edge \euv{} is positive with probability $\frac{1}{2}(p_u+q_v)$.
	\label{fig:troll_genmodel}}
\end{marginfigure}

The basic intuition is that the nature $\yuv$ of a relationship \euv{} is stochastically
determined by a mixture between how much node $u$ tends to like other people ($p_u$) and how much
node $v$ tends to be liked by other people ($q_v$). In a certain sense, $1-tr(u)$ is the empirical
counterpart to $p_u$, and $1-un(v)$ is the empirical counterpart to $q_v$.\footnote{One might view
our model as reminiscent of standard models for link generation in social network analysis, like the
classical $p_1$ model from \cite{hl81}. Yet, the similarity falls short, for all these models aim at
representing the likelihood of the network topology, rather than the probability of edge signs, once
the topology is \emph{given}.} 

Notice that the Bayes optimal prediction for $\yuv$ is $y^*(u,v) = \sgn\big(\eta(u,v) -
\tfrac{1}{2}\big)$, where $\eta(u,v) = \Pr\big( \yuv = 1 \big)$. Moreover, the probability of
drawing at random a $+1$-labeled edge from $\Nout(u)$ equals
\begin{equation}\label{e:pout}
	\frac{1}{2}\,\Biggl(p_u + \frac{1}{\dout(u)} \sum_{v\in \NNout(u)} q_v \Biggl)
\end{equation}
Similarly, the probability of drawing at random a $+1$-labeled edge from $\Nin(v)$ equals
\begin{equation}\label{e:pin}
	\frac{1}{2}\,\Biggl(q_v + \frac{1}{\din(v)} \sum_{u\in \NNin(v)} p_u \Biggl)
\end{equation}
