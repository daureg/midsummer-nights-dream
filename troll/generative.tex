
\section{Generative Model for Edge Labels}\label{s:gen}
We now define the stochastic generative model for edge labels we use in the batch learning setting. Given the graph $G = (V,E)$, let the label $y_{i,j} \in \spin$ of directed edge $(i,j) \in E$ be generated as follows. Each node $i \in V$ is endowed with two latent parameters $p_i, q_i \in [0,1]$, which we assume to be generated, for each node $i$, by an independent draw from a fixed but unknown joint prior distribution $\mu(p,q)$ over $[0,1]^2$. Each label $y_{i,j} \in \spin$ is then generated by an independent draw from the mixture of $p_i$ and $q_j$,
$
	\Pr\big( y_{i,j} = 1 \big) = \tfrac{p_i + q_j}{2}~.
$
The basic intuition is that the nature $y_{i,j}$ of a relationship $i\rightarrow j$ is stochastically determined by a mixture between how much node $i$ tends to like other people ($p_i$) and how much node $j$ tends to be liked by other people ($q_j$). In a certain sense, $1-tr(i)$ is the empirical counterpart to $p_i$, and $1-un(j)$ is the empirical counterpart to $q_j$.\footnote
{
One might view our model as reminiscent of standard models for link generation in social network analysis, like the classical $p_1$ model from \cite{hl81}. Yet, the similarity falls short, for all these models aim at representing the likelihood of the network topology, rather than the probability of edge signs, once the topology is {\em given}.
} 
Notice that the Bayes optimal prediction for $y_{i,j}$ is
$
	y^*(i,j) = \sgn\big(\eta(i,j) - \tfrac{1}{2}\big)~,
$
where $\eta(i,j) = \Pr\big( y_{i,j} = 1 \big)$. Moreover, the probability of drawing at random a $+1$-labeled edge from $\Nout(i)$ and the probability of drawing at random a $+1$-labeled edge from $\Nin(j)$ are respectively equal to
%

\begin{small}
\begin{equation}\label{e:pout}
\frac{1}{2}\,\Biggl(p_i + \frac{1}{\dout(i)}\!\!\sum_{j\in \NNout(i)} \!\!\! q_j \Biggl) \,\,\,\,\text{and}\,\,\,\,
%\end{equation}
%and similarly, the probability of drawing at random a $+1$-labeled edge from $\Nin(j)$ %equals
%
%\begin{equation}\label{e:pin}
\frac{1}{2}\,\Biggl(q_j + \frac{1}{\din(j)}\!\!\sum_{i\in \NNin(j)} \!\!\! p_i \Biggl)~.
\end{equation}
\end{small}
%
