In this thesis we will sometimes write remarks in a smaller font and with a light blue edging.
\begin{aside}
  Such remarks provide additional information to the topic discussed above, but can be skipped
  without harming the comprehension of the main material.
\end{aside}

\autoref{tab:notations} presents a tentative list of notations, but is still subject to change.

\begin{table*}[thpb]
\begin{adjustwidth}{-2cm}{}
  \centering
  \caption{List of notations used in this thesis}\label{tab:notations}%
  \vspace{-.5\baselineskip}
  \begin{tabulary}{179mm}{LL}
    \toprule
    Symbol & Meaning \\
    \midrule
    \rangek{} & The natural integers from $1$ to $K$, \ie{} $\{1, 2, \ldots, K \}$ \\
    $G$ & An arbitrary graph. It should be clear from the context whether it is weighted or not, and directed or not \\
    $V$ & The set of all the nodes of a graph, with $|V|=n$. Unless noted otherwise, nodes are indexed from $1$ to $n$ \\
    $u$ & A generic node of $G$. When referring to several nodes, we naturally use $u$, $v$, $w$ and so on. \\
    $E$ & The set of all the edges of a graph, with $|E|=m$ \\
    $(u,v)$ & An undirected edge between nodes $u$ and $v$ \\
    \euv{} & A directed edge from node $u$ to node $v$ \\
    $\yuv{}$ & The sign of the edge $(u,v)$, which can be either $+1$ or $-1$ \\
    $Y(E)$ & The labeling of $E$, that is the set of all signs of $E$: $Y(E) = \{\yuv : (u,v)\in E\}$ \\
    $\etrain$ & A subset of $E$, given or chosen, of which we observe the signs \\
    $\degr(u)$ & The total degree of node $u$ (that is, the number of edges incident to $u$,
    regardless of their direction) \\
    $\nei(u)$ & The set of all neighbors of $u$, regardless of edge direction. It thus holds
    that $|\nei(u)| = \degr(u)$ \\
    $T$ & An unweighted and undirected tree \\
    $\pathtuv$ & The unique path between $u$ and $v$ in the tree $T$, represented by an ordered list of edges \\
    $|\pathtuv|$ & The length of the path between $u$ and $v$ in the tree $T$, that is its number of edges \\
    \bottomrule
  \end{tabulary}
\end{adjustwidth}
\end{table*}
