Dans cette thèse, nous proposons des méthodes pour caractériser efficacement et précisément les
arêtes au sein de réseaux complexes. Dans les graphes simples, les nœuds sont liés au travers d'une
sémantique unique. Par exemple, deux utilisateurs sont amis dans un réseau social, ou une page web
contient un lien hypertexte pointant vers un autre page. De plus, ces connexions sont généralement
guidées par la similarité entre les nœuds, au travers d'un mécanisme appelé homophilie. Dans les
exemples précédents, les utilisateurs deviennent amis à cause de caractéristiques communes, et les
pages web sont reliées les unes aux autres sur la base de sujets communs. En revanche, les réseaux
complexes sont des graphes où chaque connexion possède une sémantique parmi $k$ possibles. Ces
connexions sont en outre basées à la fois sur une homophilie et une hétérophilie partielle des nœuds
à leurs extrémité. Cette information supplémentaire permet une analyse plus fine des graphes issus
d'applications réelles. Cependant, elle peut être coûteuse à acquérir, ou n'est pas toujours
disponible a priori. Nous abordons donc le problème d'inférer la sémantique des arêtes dans
plusieurs contextes. Tout d'abord, nous considérons les graphes où les arêtes ont deux sémantiques
opposées, et où nous observons l'étiquette de certaines arêtes. Ces \enquote{graphes signés} sont
une façon élégante de représenter des interactions polarisées. Nous proposons deux biais
d'apprentissage, adaptés respectivement aux graphes signés dirigés et non dirigés. Ceci nous amène à
concevoir plusieurs algorithmes utilisant la topologie du graphe pour résoudre un problème de
classification binaire que nous appelons \esp{}. Deuxièmement, nous considérons les graphes avec $k
\geq 2$ sémantiques possibles pour les arêtes. Dans ce cas, nous ne recevons pas d'étiquette
d'arêtes, mais plutôt un vecteur de caractéristiques pour chaque nœud. Face à ce problème non
supervisé d'\ecp{}, nous concevons un critère de qualité exprimant dans quelle mesure une
$k$-partition des arêtes et $k$ vecteurs sémantiques expliquent les connexions observées. Nous
optimisons ce critère \enquote{qualité explicative} sous une forme vectorielle et matricielle et
illustrons le comportement de ces deux méthodes sur des données synthétiques.
