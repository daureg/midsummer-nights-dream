The birth of graph theory is credited to \textcite{euler1741} in 1736 for his elegant solution of
the Seven Bridges of Königsberg
problem\footnote{\url{https://en.wikipedia.org/wiki/Seven_Bridges_of_Königsberg}}. Since then,
it has been a very active branch of mathematics ~\autocite{biggs1976graph}. Indeed, it provides a
conceptually simple yet immensely rich framework to model phenomena where entities are connected
with each other~\autocite{ComplexNetworksApp11}. Coupled with the increasing availability of large
amount of relational data, learning on graphs has recently spurred a lot of interest across various
lines of research, with tangible benefits.

\begin{description}[leftmargin=*]

  \item[community detection~\autocite{FortunatoSurvey10}] The goal is to cluster nodes in tightly
    connected groups that are loosely connected with the rest of the graph. This allows for a better
    understanding of the graph organization, and present a higher level view than looking at the
    individual node relationships. For instance, it has been used to identify proteins functional
    groups~\autocite{clusterBio03}, or to see how different scientific fields relate based on
    publication data~\autocite{clusterScience08}.

  \item[semi-supervised learning~\autocites{SSL06}{graphSSL14}] In addition to labeled data,
    the learner is also provided with unlabeled data at training time, and its goal is to classify
    nodes. Connecting similar instances allow propagating information along the graph. This has
    found applications in classifying text documents~\autocite{sslText09} or aligning categories and
    relations across multiple knowledge bases~\autocite{sslKB13}.

    % \item node classification~\autocite{nodeClassif11} Idem
    % could add visualisation (Roberto Tamassia. 2013. Handbook of Graph Drawing and Visualization.
    % CRC Press.) but it's not exactly learning

  \item[node embedding~\autocite{representationLearning17}] The goal is to find a low dimensional
    representation of the nodes, based on their structural patterns. This usually performed in an
    unsupervised way, although it is also possible to include problem supervision when available.
    Such representation can then be used in downstream tasks, for instance
    visualization~\autocite{LINE15} or the aforementioned node classification, even in the inductive
    setting where new nodes can join the graph after training~\autocite{inductiveRepresentation17}.

  \item[link prediction~\autocite{linkPredSurvey16}] Given a snapshot of the graph at time $t$, the
    goal of link prediction is to return a set of links that do not exist at time $t$ but will be
    created by time $t+\Delta_t$. Most methods are based on the assumption that link creation is
    driven by node similarity. It has been successfully applied to inferring potential interactions
    between proteins without expensive experiments~\autocite{linkPredBio06} and uncovering hidden
    associations in criminal networks~\autocite{linkPredCrime08}.

  \item[information and influence propagation~\autocite{infmax13}] The study of processes by which
    content is spread across networks, and how such processes can be influenced to speed them up or
    slow them down. Two representative applications are selecting the best seeds in a social network
    to promote a viral marketing campaign~\autocite{infmaxKempe15} and containing more effectively
    the diffusion of actual biological viruses~\autocite{influenceBio13}.

  \item[network evolution~\autocite{networkEvolution14}] These methods focus on the mechanisms and
    consequences of the growth of networks. They also seek ways to keep the results of some data mining
    algorithms up to date and relevant. Monitoring the changes in the interactions of proteins
    can indeed be used as an early indicator for some kind of diabetes~\autocite{evolBio10}.
    Furthermore, sudden changes in a network of computers are usually worth investigating, for they
    might signal external attacks~\autocite{evolSecurity04}.

\end{description}

This list of graph learning problems and their applications to real world scenario is incomplete.
Yet it already demonstrates the impact of inferring patterns in relational data over many aspects of
our lives. However, we argue that more can be done. Indeed, all the methods presented above only
consider graphs with a single type of edge and where nodes are connected based on some
domain-dependant notion of similarity.
