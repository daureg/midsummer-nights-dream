The three graphs we described in the introduction are examples of what we call complex networks.
Such graphs have more than one type of edge, and edges are not explained by mere global node
similarity. The point of this section is to illustrate that this notion is not simply a mathematical
variation of a \enquote{simple} graph, where the adjacency matrix would take values in $\{-1, 0, 1\}$ or
$\{0, 1, \ldots, k\}$ instead of $\{0, 1\}$. Rather, we reviews many applications in various domains
where complex networks are the right model to represent rich networked data. First, we showcase many
uses of signed networks. In that case, we do not insist on the fact that similarity is not the only
driver of node connections, for it is implicit that negative edges actually denote dissimilarity.
Second, we present multilayer graphs, that is graphs with more than two types of edges. When
applicable, we highlight how the connections in such graphs indeed rely on a nuanced similarity or
dissimilarity across some of the nodes attributes.

To elaborate on this last point, our hypothesis is that connections in complex networks are the
results of both partial homophily (that is, nodes are connected when they are similar on a subset of
the attributes) and partial heterophily (that is, nodes are connected when they are dissimilar on a
subset of the attributes). As examples of the latter, think of dating websites ---where most users
are linked with users of the opposite gender~\autocites{homophilyMyspace09}{Tinder16}; diffusing
innovations ---where meeting people with different backgrounds and point of views is crucial to
favor diversity and creativity~\autocite{rogers2003diffusion}; and online news consumption ---where
connecting people from different sides of the political spectrum helps to avoid echo chambers and
fuel a democratic debate~\autocite{balancedNews17}.

\subsection{Signed graphs}
\label{sub:intro_signed_graphs}

In this section, we present a list of signed graph usages, sorted by domains~\autocite{Tang2015a}. Many of those signed
graphs are the input of some clustering algorithms. In the context of signed graph, the clustering
task can be captured by the \pcc{} problem. We provide a thorough overview of this problem in
\autoref{chap:cc}. Here we simply give a broad, informal definition. The objective in \pcc{} is to
cluster the nodes of a signed graph in a way that minimizes the number of positive edges across
clusters and the number of negative edges within clusters. Such edges are called disagreements.

%TODO turn those items into paragraphs?
% \begin{description}[leftmargin=*]
   \paragraph{Computer Vision}
      The ubiquitous task of segmenting an image into homogeneous regions is a prerequisite for many
      further processing. As we mentioned earlier in one example, building a signed graph can help,
      although it might be costly to do it at the pixel level. For instance, to segment cell in
      microscopy imagery, \textcite{CellSeg14} first use generic image features to classify pixel in
      belonging to region boundaries or not. Then, they extract small scale regions called
      superpixels. After building the adjacency graph of these superpixels, they assigned edge
      weights by averaging the boundary probabilities of the pixels separating adjacent
      superpixels. They also add strong negative constraints between distant superpixels, and lastly
      cluster these superpixels according to the \pcc{} objective to obtain the final segmentation.
      A similar approach was used earlier in~\textcite{Kim2011}, who stress the importance of
      considering such higher order constraints between distant superpixels in order to achieve good
      performance. This was also extended to 3D segmentation~\autocite{VolumeSegmentation12}, where
      additional tuning allows to segment a volume image of a mouse cortex with up to billions
      voxels. \Textcite{Beier2015} segment 2D and 3D images with an energy based formulation of
      \pcc{} and iteratively improve their solution by merging it with another clustering given by a
      proposal generator. By developing another scalable energy based optimization procedure, and
      with the help of few user-provided cues, \textcite{Bagon2011} are able to apply \pcc{}
      directly at the pixel level.
      % one more? https://link.springer.com/chapter/10.1007/978-3-642-33783-3_41
      % 3D mesh segment https://link.springer.com/article/10.1007/s41095-016-0071-3

      Beyond image segmentation, \textcite{Shape3D17} develop a method to extract a network of
      descriptive curves from 3D shapes.  After an initial stage of generating many such
      \emph{flowlines}, they describe in Section 6 a \pcc{} formulation to extract \emph{reliable}
      representative flowlines, using geometric constraints to obtain positive or negative cues that
      two flowlines are from the same reliable representative.

      Finally, in order to track several targets across sequential video frames,
      \textcite{multiTracking15} propose a multistage framework. One step revolves around a matrix
      $A$ that defines the cost of assigning an object tracked in previous frames to an object
      detected in the current frame. This matrix is turned into a symmetric affinity matrix
      $\bar{A}_{sym}$ that can be seen as a signed graph adjacency matrix. \pcc{} is then used to extract clusters
      (called zones), in which local processing is performed. This is beneficial since the
      complexity of these local methods can be adapted to the difficulty of each zone.

      \paragraph{Natural Language Processing} 
      Coreference resolution is the task of finding all expressions that refer to the same entity in
      a text. Like image segmentation, it is a preprocessing step that can later be used in document
      summarization, question answering, and information extraction. Furthermore, in that case, it
      is also natural to build a graph of words. One then add negative edges between words that
      cannot refer to the same entity (for instance because they are of different gender) and
      positive edges between words with linguistic cues indicating they might refer to the same
      entity. 
      \Textcite[Section
      2.3]{graphicalCoreference04} instead tackle coreference resolution by using an undirected graphical
      model on which performing inference is equivalent to \pcc{}. On small scale instances,
      \textcite{Elsner2009} use the signed graph procedure outlined above. They first obtain an
      upper bound of the optimal solution by solving a SDP
      relaxation of the problem. They then compare various heuristics and show that best performing
      ones are within few percents of the optimum, provided they are followed by a local search
      step, such as the Best One Element Move~\autocite{Gionis2007}. Further NLP tasks amenable to
      a signed graph representation are referenced in their paper. Another task is
      to cluster words based on distributional embedding vectors while adding antonym
      constraints~\autocite{SignedWordRatings}.

      \paragraph{Biology}
      Signed graphs are also abundant in biology.
      A typical input is a similarity matrix between genes expression level in various experimental conditions
      and the goal is to cluster those genes into groups which react similarly. \Textcite[Section
      4]{Ben-Dor99} gives three examples: 112 genes involved in the rat central nervous system, 1246
      genes of the roundworm \emph{C. elegans} and 2000 human genes obtained from 40 tumor and 22
      normal colon samples. \Textcite{Mason2009} analyze a signed co-expression network of genes
      involved in embryonic stem cells to find which genes are related to pluripotency (the ability
      to differentiate into any type of cell in the body) or self-renewal (the ability to replicate
      indefinitely). Another application is to study the variation of one individual
      DNA~\autocite{Das2015}. In the human organism, chromosomes are organized in pair, and both
      chromosomes of a pair have similar but not identical DNA sequences. This is mostly because of
      single nucleotide polymorphisms (SNPs), where a single base differs between the two DNA
      sequences, leading to different alleles of the corresponding gene. An haplotype is the list of
      all alleles at a contiguous region of a single chromosome, and this information is
      used in several medical applications. The high-throughput sequencing of one individual genome
      yields many short \emph{reads} that provides information about the order of nucleotides in a
      fragment of one chromosome and that can be used to assemble haplotypes. To do so, the authors
      build a graph of reads and define a similarity function between reads to assign weights on the
      edges. The clusters of that graph correspond to haplotypes, and are obtained by a SDP
      relaxation of the \pcc{} objective. \Textcite{monotoneBiology07} also consider graphs whose
      nodes are genes, but in a different context. In this case, positive edges represent an
      activating connection, while negative edges represent inhibiting connection. They also define
      a \emph{monotone system} as a balanced subgraph, that is a subgraph which does not contain a
      cycle with an odd number of edges. Such monotone system are stable, in the sense that
      modifying the concentration of one gene will have a predictable effect, even ignoring the
      precise kinematics of the chemical reactions involved. Their goal is to find the minimum
      number of edges to remove in order to decompose a dynamics system into a collection of
      monotone system. This allows to study
      the complete system more easily. More applications of weighted complete signed graphs in biology
      are presented in~\autocite[Section 6]{clusterEditSurvey13}.
      % Finding the 3D shapes of chromosome given pairwise contact frequencies of different regions
      % \url{https://doi.org/10.1145/3107411.3108216}
      % \enquote{They examined dynamical systems, where a gene is modeled as a vertex and an
      % activating connection is modeled as a positive edge and an inhibiting connection is modeled as
      % a negative edge. The claim is that biological dynamical systems are close to being balanced,
      % and that finding a minimum set of edges to delete to make the graph balanced can be used to
      % decompose the graph into “monotone subsystems”, which exhibit stable behavior and thus allow a
      % better understanding of the dynamics of a system.}\autocite{monotoneBiology07}

      \paragraph{Network science}
      One early use of signed graphs was to model social
      interactions~\autocites{harary1953}{HeiderBook58}. Here we present
      some recent references. For instance, one can extract all the votes of the members of a
      political parliament and form a graph whose nodes are politicians and edge weights quantify how
      much they agree or disagree on various issues they have been voting on. This can be used to
      study various social science questions such as loyalty, leadership, coalitions, political
      crisis and polarization. It has been applied to the European
      parliament~\autocite{Mendonca2015}, Slovenian parliament~\autocite{Jiang2015} and the
      Brazilian parliament~\autocites{BrazilCC17}. This can also be used at international level. For
      instance, by considering a dataset of military alliances and disputes, \textcite{Traag2009}
      cluster countries into blocks that resemble those identified by Huntington in his \emph{Clash
      of Civilizations} book. Another source of data is the vote on resolutions during the United
      Nations General Assembly~\autocite{CommunityUN12}. Finally, one can also study how to exploit
      the information contained within the negative links to enhance the visualization of social
      networks~\autocites{Luca10}{drawingSignedGraphs11}.

      \paragraph{Others}
      \begin{itemize}[leftmargin=*]
	 \item
	    Deduplication, also called duplicate detection or entity resolution, is the process
	    of identifying objects from a real-world, noisy database that refer to the same entity.
	    On a high level, a solution to this problem is to build a graph of all the available
	    objects, define a similarity between them and run a \pcc{} algorithm.
	    % This was indeed this kind of problems at Whizbang! Labs that partly motivated one of the early \pcc{} paper~\autocite{Bansal2002}.
	    The main challenge
	    thus lies in devising an appropriate similarity measure, given that object can have very
	    different features from one database to another. \Textcite{LargeScaleDeDup09} propose a
	    declarative language, expressing both hard constraints (that have to be satisfied) and
	    soft constraints (that can be seen as cues guiding the process). Because of these hard
	    constraints that admissible clusterings have to respect, the authors have to modify in
	    nontrivial ways an existing \pcc{} algorithm. This was extended to weighted and partial
	    constraints  by \textcite{WeightedER14}. Another example is given by
	    \textcite{Crosslingual07}, who cluster together news articles in different languages
	    covering the same event. \pcc{} was also evaluated among other
	    solutions to that problem by \textcite{DeDup09}, who note that their non optimized
	    implementation does not perform the best.
	 \item 
	    Given an electrical circuit layout, \textcite{circuitDesign07} extract a graph of its
	    components (called shifter) that must be assigned one of two possible phases. Because
	    two shifters next to some specific shape must be in opposite phase and two shifters
	    separated by less than a specified distance must be of the same phase, the authors look
	    for a two-clustering of the nodes that will minimize the number of disagreements.
	 \item In finance, one can represent an investment portfolio as a signed
	    graph~\autocite{portfolio02}. Each node is a security, and the edge between two
	    securities is weighted by their correlation, which can be negative. For instance, a
	    graph with only positive edges is speculative, as all the securities move in the same
	    direction, either up or down. On the other hand, if the securities can be partitioned in
	    groups without disagreement, the risk is limited, for two clusters will move in opposite
	    directions, providing the investors with a hedging guarantee.
	 \item In wireless networks, signed graphs can be used to solve optimization problems
	    involved in determining energy-saving routes~\autocite{signedRouting12} or to exchange
	    cryptographic keys in a secure and efficient manner~\autocite{signedKey17}.
      \end{itemize}
% \end{description}


\subsection{Multilayer graphs}
\label{sub:intro_multilayer_graphs}

Besides signed graphs, in this thesis we also consider at multilayer
graphs~\autocites{Kivela2014}{multiSurvey14}. Those are graphs with $k$ edge types, and the name
refer to the fact we can see them as the superposition of the $k$ subgraphs induced by each edge
type. Even when $k=2$, we make a distinction between signed graphs and multilayer graphs. In signed
graphs, the two types of edge have reverse semantic, whereas in a 2-layers graph, they simply
denotes two possible interactions, for instance advisor-advisee or regular coauthors relation in a
citation network~\autocite{Advisor10}. In general though, we focus on cases where $k$ is larger than
$2$, and not larger than a few dozen in order to preserve interpretability.
Like signed graphs, these multilayer graphs are versatile enough to be used in many fields.

\paragraph{Social networks} \Textcite{Szell2010} model the interactions of the players of
\emph{Pardus}, a massively multilayer online game. These players can be friend or enemy, send
private messages, trade resources, attack each other and set a bounty on the head of another
players. These six types of interactions are either positive or negative, but their nuances cannot
simply by explained by global similarity and dissimilarity. Another examples are the users of the
photo sharing website Flickr.  They can interact in eleven ways, either directly or through
comments, shared tags, groups membership and so on~\autocite{RecoFlickrMulti11}. Again, while being
part of a common group denote shared interests between two users, overall they must also differ in
some other attributes (for instance location) in order to bring diversity to this group.

\paragraph{Citation networks} In these networks, nodes are research papers or authors, and edges
typically connect two nodes whenever one cites the other. For instance, using the DBLP dataset,
\textcite{communityDBLPbyConf05} connect two authors if they have co-authored a paper in one the
\np{1000} conferences appearing in the data. One can then consider that the resulting graph was
obtained as the superposition of these \np{1000} subgraphs. Besides direct citations,
\textcite{articlesMultiSim11} consider four others reasons to connect \np{5000} SIAM papers, based
on their similarity in terms of abstract words, title words, keywords and authors.

\paragraph{Economic networks} In our ever increasingly globalized economy, entities around the world
are getting more and more tightly connected. However, it would be overly simplistic to consider
there is only kind of link between entities. For something seemingly as simple as the connections
between the largest \np{951} ports, one must already noticed that these connections can be
implemented by any combinations of three kinds of ships: bulk dry carriers, container ships and oil
tankers~\autocite{ports3kindofships10}. Likewise, the 162 countries of the International Trade
Network are connected by 96 kinds of commodities they can exchange~\autocite{worldTradeNetwork10}.
Finally, \textcite{KantPeace15} studies international relations through the lens of Kant three
folded program for peace,  based on democracy, economic dependence and supra national governance.
They build the graph of all countries and connect them in three layers. All democracies form a
clique in the first layer, countries are connected with weight proportional to the amount of yearly
trade in the second layer and with weight proportional to the number of international organisations
they belong to together in the third layer. While the essence of commerce is to exploit differential
between the partners involved (\ie{} heterophily), trade (and in the last example participating in
common institution) intensity also involve geographical, historical and cultural ties (\ie{}
homophily).

\paragraph{Biological networks} The interactions in biology also take several forms and studying
them as a whole has proved fruitful. One example is a genes co-expression network, where each
connection was tested under 130 different experimental conditions, providing as many
layers~\autocite{bioLayerExp11}.
Multilayer graphs are also a relevant way to represent ecological
networks~\autocite{EcologyMultiReview17}. For instance, \textcite{EcoChile15} build the graph of
more than 100 species living on the Chilean
coast\footnote{\url{http://app.mappr.io/play/chile-marine-intertidal-network}} and divide their
interactions in three categories: trophic (that is one specie eats another one), positive
non-trophic (\eg{} refuge providing) and negative non-trophic (\eg{} competition for shared
resources).
Finally, in neuroscience, multilayer graphs have recently emerged as a useful tool to better
understand the human brain~\autocite{Neuroscience16}. The nodes of such graphs are neurons, and the
edges can be labeled in various ways: some correspond to actual physical links while others are
functional ---\ie{} neurons responding in the same way to external stimuli, some are present in
healthy subject and others in patient cohort, some are acquired through MRI and others by EEG.
