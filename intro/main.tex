Graphs are a natural way to represent the relationships over a set of entities. Because of the
simplicity and flexibility of this formalism, graphs are ubiquitous and have been used in countless
fields. To make the rest of our explanations more concrete, we now give three examples of graphs
that we consider later on.
\begin{enumerate}[nosep]%,leftmargin=*]
\item In a social network\footnote{In the rest of this thesis, we use the terms \emph{graph} and
\emph{network} interchangeably.}, the nodes of the graph are human users, and relationships
between nodes denote interactions between the corresponding users. One social network we study
in \autoref{chap:troll} summarizes the connections between the editors of \wik{}. These editors
can be promoted to administrators after a vote from their peers. An edge \euv{} in this network
means that user $u$ has voted on the possible promotion of $v$.
\item In computer vision, we can represent an image as a graph. Each pixel is a node, and those
nodes are connected to four neighbors, namely the adjacent pixels from the top, bottom, left and
right sides. The relationship in this graph is therefore adjacency in an image.
\item In e-commerce, we can consider a co-purchase network. Nodes are products being sold on a
website, and two nodes are connected if the corresponding products are frequently bought
together by customers, for instance a phone and a memory card.
\end{enumerate}

Not only are there graphs in many domains, but the progress of technology in the last few decades
has made it easier to collect many graphs in every single domain, sometimes with up to billions of
nodes and hundreds of billions of edges. The availability of such large amounts of structured data has
prompted the development of automated methods to extract insights from them. For instance, it is
possible to cluster nodes into coherent groups, predict the category of the nodes or study how to
best propagate information within a graph. We describe in more details these possibilities and
others in \autoref{sec:learning_in_graphs}. By doing so, our goal is to illustrate the wide
potential of \emph{learning in graphs}.

\medskip

At this point of the discussion though, we have only considered the most common and simple kind of
graph, one representing a single type of relation between nodes.\footnote{Note that by \enquote{type
of relation}, we do not refer to some graph-theoretical characteristic of an edge, such as being directed or weighted.
Indeed, all the graphs we consider have homogeneous edges with that respect. Instead, we mean \emph{domain
specific semantic}, as we will make clear in examples.} However, in many situations, there
are two dimensions along which graphs exhibit more complexity. First, nodes have more than one type
of relation among each other. Second, two nodes are not only connected because of their global
similarity but also for more nuanced reasons. As a case in point, let us look again at our
three examples. In the \wik{} network, a vote can support the promotion or oppose it. This
additional information enrich our understanding of the relationships among the nodes in the graph.
It also points out that two editors can be connected because they share a common topic of interest,
but come from different socio-economic backgrounds and therefore cannot agree on this topic.
Likewise in an image, an edge between two pixels is positive if the two pixels belong to the same
object (say a car or a building) and negative otherwise. This information could be used to segment
the image. In the co-purchase network, there are even more than two types of relation. Assuming the
products are movies, two movies can be frequently bought together because they are part of a series
(like Star Wars), because they have the same director but different genre, because they receive the
same prestigious award in different years, and so on. The second type of relation (\enquote{same
director} and \enquote{different genre}) is actually an example of a mixed relation, for it combines
similarity and dissimilarity over several features of the nodes.

We call \emph{complex networks} graphs where edges have different \emph{semantics}, or types, and
where connections are explained by more than simple global similarity. In
\autoref{sec:edge_semantics}, we give additional examples of such complex networks. As
showed in our three previous examples, we distinguish between two cases. The first is when there are two types of
edge having opposite semantics. Such graphs are called \emph{signed graphs} and have been
extensively studied since the fifties, for we shall see they have many applications. Their name
comes from the fact that edges are typically labeled $+1$ and $-1$. The second case
is when there are more than two types of edge. Such graphs have also been studied for a long time
under different names and we refer to them as \emph{multilayer graphs}.

\medskip

It is natural to expect that we can extract finer insights from graphs with edge semantics. However,
in many cases, the information about edge types is not available, at least not in a convenient form.
We therefore present in
\autoref{sec:predicting_edge_type} the problem of characterizing edges in complex networks.
Informally, given an input graph and possibly some extra information, we want to predict the type of
every edge. This problem can take several forms depending on what information is available as input.
First, the graph may be directed or undirected. Second, there might be two edge types (in the case of
signed graphs) or more (for general multilayer graphs). Third, the problem can be supervised or not.
In the supervised setting, we are provided with labels for some of the edges. In the unsupervised
setting, there is no label at all. We thus make the additional assumption that we observe some
\emph{attributes} of the nodes. Once again, we return to our three examples to illustrate what
those attributes can be. In the \wik{} graph, we could have for each user data about her age, experience and
area of expertise. In an image, each pixel is associated with a color, along with higher order
visual features. Each product of a co-purchase network comes with information about, \eg{}, its price,
popularity, category and availability. In the absence of label, our intuition is that these
attributes can inform us about the types of edge among nodes. 

\medskip

We list in \autoref{sec:outline} three concrete instantiations of this general problem that we
consider in this thesis. Our solutions to these three problems offer evidence in the defense of our
thesis statement:
\begin{quote}
\bf
There exist efficient and accurate methods to predict edge type in complex networks, relying
only on the graph topology or also on node attributes.
\end{quote}

\section{Learning in graphs}
\label{sec:learning_in_graphs}
The birth of graph theory is credited to \textcite{euler1741} in 1736 for his elegant solution of
the Seven Bridges of Königsberg
problem\footnote{\url{https://en.wikipedia.org/wiki/Seven_Bridges_of_Königsberg}}. Since then,
it has been a very active branch of mathematics ~\autocite{biggs1976graph}. Indeed, it provides a
conceptually simple yet immensely rich framework to model phenomena where entities are connected
with each other~\autocite{ComplexNetworksApp11}. Coupled with the increasing availability of large
amount of relational data, learning on graphs has recently spurred a lot of interest across various
lines of research, with tangible benefits.

\begin{description}[leftmargin=*]

  \item[community detection~\autocite{FortunatoSurvey10}] The goal is to cluster nodes in tightly
    connected groups that are loosely connected with the rest of the graph. This allows for a better
    understanding of the graph organization, and present a higher level view than looking at
    individual nodes relationships. For instance, it has been used to identify proteins functional
    groups~\autocite{clusterBio03} or see how different scientific fields relate based on
    publication data~\autocite{clusterScience08}.

  \item[semi-supervised learning~\autocites{SSL06}{graphSSL14}] In addition to labeled data,
    the learner is also provided with unlabeled data at training time, and its goal is to classify
    nodes. Connecting similar instances allow propagating information along the graph. This has
    found applications in classifying text documents~\autocite{sslText09} or aligning categories and
    relations across multiple knowledge bases~\autocite{sslKB13}.

    % \item node classification~\autocite{nodeClassif11} Idem
    % could add visualisation (Roberto Tamassia. 2013. Handbook of Graph Drawing and Visualization.
    % CRC Press.) but it's not exactly learning

  \item[node embedding~\autocite{representationLearning17}] The goal is to find a low dimensional
    representation of the nodes, based on their structural patterns. This usually performed in an
    unsupervised way, although it is also possible to include problem supervision when available.
    Such representation can then be used in downstream tasks, for instance
    visualization~\autocite{LINE15} or the aforementioned node classification, even in the inductive
    setting where new nodes can join the graph after training~\autocite{inductiveRepresentation17}.

  \item[link prediction~\autocite{linkPredSurvey16}] Given a snapshot of the graph at time $t$, the
    goal of link prediction is to return a set of links that do not exist at time $t$ but will be
    created by time $t+\Delta_t$. Most methods are based on the assumption that link creation is
    driven by node similarity. It has been successfully applied to inferring potential interactions
    between proteins without expensive experiments~\autocite{linkPredBio06} and uncovering hidden
    associations in criminal networks~\autocite{linkPredCrime08}.

  \item[information and influence propagation~\autocite{infmax13}] The study of processes by which
    content is spread across networks, and how such processes can be influenced to speed them up or
    slow them down. Two representative applications are selecting the best seeds in a social network
    to promote a viral marketing campaign~\autocite{infmaxKempe15} and containing more effectively
    the diffusion of actual biological viruses~\autocite{influenceBio13}.

  \item[network evolution~\autocite{networkEvolution14}] These methods focus on the mechanisms and
    consequences of the growth of networks, as well as how to keep the results of some data mining
    algorithms up to date and relevant. Monitoring the changes in the interactions of proteins
    can indeed be used as an early indicator for some kind of diabetes~\autocite{evolBio10}.
    Furthermore, sudden changes in a network of computers are usually worth investigating, for they
    might signal external attacks~\autocite{evolSecurity04}.

\end{description}

This list of graph learning problems and their applications to real world scenario is incomplete.
Yet it already demonstrates the impact of inferring patterns in relational data over many aspects of
our lives. However, we argue that more can be done. Indeed, all the methods presented above only
consider graphs with a single type of edge and where nodes are connected based on some
domain-dependant notion of similarity.


\section{Graph with several edge semantics}
\label{sec:edge_semantics}
The three graphs we described in the introduction are examples of what we call complex networks.
Such graphs have more than one type of edge, and edges are not explained by mere global node
similarity. The point of this section is to illustrate that this notion is not simply a mathematical
variation of a \enquote{simple} graph, where the adjacency matrix would take values in $\{-1, 0, 1\}$ or
$\{0, 1, \ldots, k\}$ instead of $\{0, 1\}$. Rather, we review many applications in various domains
where complex networks are the right model to represent rich networked data. First, we showcase many
uses of signed networks. In that case, we do not insist on the fact that similarity is not the only
driver of node connections, for it is implicit that negative edges actually denote dissimilarity.
Second, we present multilayer graphs, that is graphs with more than two types of edges. When
applicable, we highlight how the connections in such graphs indeed rely on a nuanced similarity or
dissimilarity across some of the nodes attributes.

To elaborate on this last point, our hypothesis is that connections in complex networks are the
results of both partial homophily (that is, nodes are connected when they are similar on a subset of
the attributes) and partial heterophily (that is, nodes are connected when they are dissimilar on a
subset of the attributes). As examples of the latter, think of dating websites ---where most users
are linked with users of the opposite gender~\autocites{homophilyMyspace09}{TinderDesc16}; diffusing
innovations ---where meeting people with different backgrounds and point of views is crucial to
favor diversity and creativity~\autocite{rogers2003diffusion}; and online news consumption ---where
connecting people from different sides of the political spectrum helps to avoid echo chambers and
instead fuel a democratic debate~\autocite{balancedNews17}.

\subsection{Signed graphs}
\label{sub:intro_signed_graphs}

In this section, we present a list of signed graph usages, sorted by domains~\autocite{Tang2015a}. Many of those signed
graphs are the input of some clustering algorithms. In the context of signed graph, the clustering
task can be captured by the \pcc{} problem. We provide a thorough overview of this problem in
\autoref{chap:cc}. Here we simply give a broad, informal definition. The objective in \pcc{} is to
cluster the nodes of a signed graph in a way that minimizes the number of positive edges across
clusters and the number of negative edges within clusters. Such edges are called disagreements.

%TODO turn those items into paragraphs?
% \begin{description}[leftmargin=*]
   \paragraph{Computer Vision}
      The ubiquitous task of segmenting an image into homogeneous regions is a prerequisite for many
      further processing. As we mentioned earlier in one example, building a signed graph can help,
      although it might be costly to do it at the pixel level. For instance, to segment cell in
      microscopy imagery, \textcite{CellSeg14} first use generic image features to classify pixel in
      belonging to region boundaries or not. Then, they extract small scale regions called
      superpixels. After building the adjacency graph of these superpixels, they assigned edge
      weights by averaging the boundary probabilities of the pixels separating adjacent
      superpixels. They also add strong negative constraints between distant superpixels, and lastly
      cluster these superpixels according to the \pcc{} objective to obtain the final segmentation.
      A similar approach was used earlier in~\textcite{Kim2011}, who stress the importance of
      considering such higher order constraints between distant superpixels in order to achieve good
      performance. This was also extended to 3D segmentation~\autocite{VolumeSegmentation12}, where
      additional tuning allows to segment a volume image of a mouse cortex with up to billions
      voxels. \Textcite{Beier2015} segment 2D and 3D images with an energy based formulation of
      \pcc{} and iteratively improve their solution by merging it with another clustering given by a
      proposal generator. By developing another scalable energy based optimization procedure, and
      with the help of few user-provided cues, \textcite{Bagon2011} are able to apply \pcc{}
      directly at the pixel level.
      % one more? https://link.springer.com/chapter/10.1007/978-3-642-33783-3_41
      % 3D mesh segment https://link.springer.com/article/10.1007/s41095-016-0071-3

      Beyond image segmentation, \textcite{Shape3D17} develop a method to extract a network of
      descriptive curves from 3D shapes.  After an initial stage of generating many such
      \emph{flowlines}, they describe in Section 6 a \pcc{} formulation to extract \emph{reliable}
      representative flowlines, using geometric constraints to obtain positive or negative cues that
      two flowlines are from the same reliable representative.

      Finally, in order to track several targets across sequential video frames,
      \textcite{multiTracking15} propose a multistage framework. One step revolves around a matrix
      $A$ that defines the cost of assigning an object tracked in previous frames to an object
      detected in the current frame. This matrix is turned into a symmetric affinity matrix
      $\bar{A}_{sym}$ that can be seen as a signed graph adjacency matrix. \pcc{} is then used to extract clusters
      (called zones), in which local processing is performed. This is beneficial since the
      complexity of these local methods can be adapted to the difficulty of each zone.

      \paragraph{Natural Language Processing} 
      Coreference resolution is the task of finding all expressions that refer to the same entity in
      a text. Like image segmentation, it is a preprocessing step that can later be used in document
      summarization, question answering, and information extraction. Furthermore, in that case, it
      is also natural to build a graph of words. One then add negative edges between words that
      cannot refer to the same entity (for instance because they are of different gender) and
      positive edges between words with linguistic cues indicating they might refer to the same
      entity. 
      \Textcite[Section
      2.3]{graphicalCoreference04} instead tackle coreference resolution by using an undirected graphical
      model on which performing inference is equivalent to \pcc{}. On small scale instances,
      \textcite{Elsner2009} use the signed graph procedure outlined above. They first obtain an
      upper bound of the optimal solution by solving a SDP
      relaxation of the problem. They then compare various heuristics and show that best performing
      ones are within few percents of the optimum, provided they are followed by a local search
      step, such as the Best One Element Move~\autocite{Gionis2007}. Further NLP tasks amenable to
      a signed graph representation are referenced in their paper. Another task is
      to cluster words based on distributional embedding vectors while adding antonym
      constraints~\autocite{SignedWordRatings}.

      \paragraph{Biology}
      Signed graphs are also abundant in biology.
      A typical input is a similarity matrix between genes expression level in various experimental conditions
      and the goal is to cluster those genes into groups which react similarly. \Textcite[Section
      4]{Ben-Dor99} gives three examples: 112 genes involved in the rat central nervous system, 1246
      genes of the roundworm \emph{C. elegans} and 2000 human genes obtained from 40 tumor and 22
      normal colon samples. \Textcite{Mason2009} analyze a signed co-expression network of genes
      involved in embryonic stem cells to find which genes are related to pluripotency (the ability
      to differentiate into any type of cell in the body) or self-renewal (the ability to replicate
      indefinitely). Another application is to study the variation of one individual
      DNA~\autocite{Das2015}. In the human organism, chromosomes are organized in pair, and both
      chromosomes of a pair have similar but not identical DNA sequences. This is mostly because of
      single nucleotide polymorphisms (SNPs), where a single base differs between the two DNA
      sequences, leading to different alleles of the corresponding gene. An haplotype is the list of
      all alleles at a contiguous region of a single chromosome, and this information is
      used in several medical applications. The high-throughput sequencing of one individual genome
      yields many short \emph{reads} that provides information about the order of nucleotides in a
      fragment of one chromosome and that can be used to assemble haplotypes. To do so, the authors
      build a graph of reads and define a similarity function between reads to assign weights on the
      edges. The clusters of that graph correspond to haplotypes, and are obtained by a SDP
      relaxation of the \pcc{} objective. \Textcite{monotoneBiology07} also consider graphs whose
      nodes are genes, but in a different context. In this case, positive edges represent an
      activating connection, while negative edges represent inhibiting connection. They also define
      a \emph{monotone system} as a balanced subgraph, that is a subgraph which does not contain a
      cycle with an odd number of edges. Such monotone system are stable, in the sense that
      modifying the concentration of one gene will have a predictable effect, even ignoring the
      precise kinematics of the chemical reactions involved. Their goal is to find the minimum
      number of edges to remove in order to decompose a dynamics system into a collection of
      monotone system. This allows to study
      the complete system more easily. More applications of weighted complete signed graphs in biology
      are presented in~\autocite[Section 6]{clusterEditSurvey13}.
      % Finding the 3D shapes of chromosome given pairwise contact frequencies of different regions
      % \url{https://doi.org/10.1145/3107411.3108216}
      % \enquote{They examined dynamical systems, where a gene is modeled as a vertex and an
      % activating connection is modeled as a positive edge and an inhibiting connection is modeled as
      % a negative edge. The claim is that biological dynamical systems are close to being balanced,
      % and that finding a minimum set of edges to delete to make the graph balanced can be used to
      % decompose the graph into “monotone subsystems”, which exhibit stable behavior and thus allow a
      % better understanding of the dynamics of a system.}\autocite{monotoneBiology07}

      \paragraph{Network science}
      One early use of signed graphs was to model social
      interactions~\autocites{harary1953}{HeiderBook58}. Here we present
      some recent references. For instance, one can extract all the votes of the members of a
      political parliament and form a graph whose nodes are politicians and edge weights quantify how
      much they agree or disagree on various issues they have been voting on. This can be used to
      study various social science questions such as loyalty, leadership, coalitions, political
      crisis and polarization. It has been applied to the European
      parliament~\autocite{Mendonca2015}, Slovenian parliament~\autocite{Jiang2015} and the
      Brazilian parliament~\autocites{BrazilCC17}. This can also be used at international level. For
      instance, by considering a dataset of military alliances and disputes, \textcite{Traag2009}
      cluster countries into blocks that resemble those identified by Huntington in his \emph{Clash
      of Civilizations} book. Another source of data is the vote on resolutions during the United
      Nations General Assembly~\autocite{CommunityUN12}. Finally, one can also study how to exploit
      the information contained within the negative links to enhance the visualization of social
      networks~\autocites{Luca10}{drawingSignedGraphs11}.

      \paragraph{Others}
      \begin{itemize}[leftmargin=*]
	 \item
	    Deduplication, also called duplicate detection or entity resolution, is the process
	    of identifying objects from a real-world, noisy database that refer to the same entity.
	    On a high level, a solution to this problem is to build a graph of all the available
	    objects, define a similarity between them and run a \pcc{} algorithm.
	    % This was indeed this kind of problems at Whizbang! Labs that partly motivated one of the early \pcc{} paper~\autocite{Bansal2002}.
	    The main challenge
	    thus lies in devising an appropriate similarity measure, given that object can have very
	    different features from one database to another. \Textcite{LargeScaleDeDup09} propose a
	    declarative language, expressing both hard constraints (that have to be satisfied) and
	    soft constraints (that can be seen as cues guiding the process). Because of these hard
	    constraints that admissible clusterings have to respect, the authors have to modify in
	    nontrivial ways an existing \pcc{} algorithm. This was extended to weighted and partial
	    constraints  by \textcite{WeightedER14}. Another example is given by
	    \textcite{Crosslingual07}, who cluster together news articles in different languages
	    covering the same event. \pcc{} was also evaluated among other
	    solutions to that problem by \textcite{DeDup09}, who note that their non optimized
	    implementation does not perform the best.
	 \item 
	    Given an electrical circuit layout, \textcite{circuitDesign07} extract a graph of its
	    components (called shifter) that must be assigned one of two possible phases. Because
	    two shifters next to some specific shape must be in opposite phase and two shifters
	    separated by less than a specified distance must be of the same phase, the authors look
	    for a two-clustering of the nodes that will minimize the number of disagreements.
	 \item In finance, one can represent an investment portfolio as a signed
	    graph~\autocite{portfolio02}. Each node is a security, and the edge between two
	    securities is weighted by their correlation, which can be negative. For instance, a
	    graph with only positive edges is speculative, as all the securities move in the same
	    direction, either up or down. On the other hand, if the securities can be partitioned in
	    groups without disagreement, the risk is limited, for two clusters will move in opposite
	    directions, providing the investors with a hedging guarantee.
	 \item In wireless networks, signed graphs can be used to solve optimization problems
	    involved in determining energy-saving routes~\autocite{signedRouting12} or to exchange
	    cryptographic keys in a secure and efficient manner~\autocite{signedKey17}.
      \end{itemize}
% \end{description}


\subsection{Multilayer graphs}
\label{sub:intro_multilayer_graphs}

Besides signed graphs, in this thesis we also consider multilayer
graphs~\autocites{Kivela2014}{multiSurvey14}. Those are graphs with $k$ edge types, and the name
refer to the fact we can see them as the superposition of the $k$ subgraphs induced by each edge
type. Even when $k=2$, we make a distinction between signed graphs and multilayer graphs. In signed
graphs, the two types of edge have reverse semantic, whereas in a 2-layers graph, they simply
denote two possible interactions, for instance advisor-advisee or regular coauthors relation in a
citation network~\autocite{Advisor10}. In general though, we focus on cases where $k$ is larger than
$2$, and not larger than a few dozen in order to preserve interpretability.
Like signed graphs, these multilayer graphs are versatile enough to be used in many fields.

\paragraph{Social networks} \Textcite{Szell2010} model the interactions of the players of
\emph{Pardus}, a massively multilayer online game. These players can be friend or enemy, send
private messages, trade resources, attack each other and set a bounty on the head of another
player. These six types of interactions are either positive or negative, but their nuances cannot
simply by explained by global similarity and dissimilarity. Another example is photo sharing website
Flickr and its users. They can interact in eleven ways, either directly or through
comments, shared tags, groups membership and so on~\autocite{RecoFlickrMulti11}. Again, while being
part of a common group denotes shared interests between two users, overall they must also differ in
some other attributes (for instance location) in order to bring diversity to this group.

\paragraph{Citation networks} In these networks, nodes are research papers or authors, and edges
typically connect two nodes whenever one cites the other. For instance, using the DBLP dataset,
\textcite{communityDBLPbyConf05} connect two authors if they have co-authored a paper in one the
\np{1000} conferences appearing in the data. One can then consider that the resulting graph was
obtained as the superposition of these \np{1000} subgraphs. Besides direct citations,
\textcite{articlesMultiSim11} consider four others reasons to connect \np{5000} SIAM papers, based
on their similarity in terms of abstract words, title words, keywords and authors.

\paragraph{Economic networks} In our ever increasingly globalized economy, entities around the world
are getting more and more tightly connected. However, those connections take on many different
forms simultaneously. For something seemingly as simple as the connections
between the largest \np{951} ports in the world, one must already notice that these connections can be
implemented by any combinations of three kinds of ships: bulk dry carriers, container ships and oil
tankers~\autocite{ports3kindofships10}. Likewise, the 162 countries of the International Trade
Network are connected by 96 kinds of commodities they can exchange~\autocite{worldTradeNetwork10}.
Finally, \textcite{KantPeace15} studies international relations through the lens of Kant's three
folded program for peace,  based on democracy, economic dependence and supra national governance.
They build the graph of all countries and connect them in three layers. All democracies form a
clique in the first layer, countries are connected with weight proportional to the amount of yearly
trade in the second layer and with weight proportional to the number of international organizations
they belong to together in the third layer. While the essence of commerce is to exploit differential
between the partners involved (\ie{} heterophily), trade intensity and participating in
common institutions also involve geographical, historical and cultural ties (\ie{}
homophily).

\paragraph{Biological networks} The interactions in biology also take several forms and studying
them as a whole has proved fruitful. One example is a genes co-expression network, where each
connection was tested under 130 different experimental conditions, providing as many
layers~\autocite{bioLayerExp11}.
Multilayer graphs are also a relevant way to represent ecological
networks~\autocite{EcologyMultiReview17}. For instance, \textcite{EcoChile15} build the graph of
more than 100 species living on the Chilean
coast\footnote{\url{http://app.mappr.io/play/chile-marine-intertidal-network}} and divide their
interactions in three categories: trophic (\ie{} one specie eating another one), positive
non-trophic (\eg{} refuge providing) and negative non-trophic (\eg{} competition for shared
resources).
Finally, in neuroscience, multilayer graphs have recently emerged as a useful tool to better
understand the human brain~\autocite{Neuroscience16}. The nodes of such graphs are neurons, and the
edges can be labeled in various ways: some correspond to actual physical links while others are
functional (\ie{} neurons responding in the same way to external stimuli), some are present in
healthy subjects and others in treated patients, some are acquired through MRI and others by EEG.


\section{Predicting edge type}
\label{sec:predicting_edge_type}
Let us summarize in one sentence the two previous sections. Learning in graphs provides many
insights, and many graphs are complex, in the sense of having edges expressing different semantics
and created for mixed reasons. The logical conclusion is that we want to learn in complex graphs.
However, this requires the edges of such graphs to be labeled with one of $k$ types of relationship.
In the examples presented above, this was already done. Yet we argue that efficiently labeling the
edges is an interesting problem, especially in the following three situations:

\begin{enumerate}

  \item There exists a function that can perfectly label any edge because it is tailored to this
    specific graph, but calling it is expensive. For instance, say we want to know whether two
    connected users of a social network are tied to each other through family, work, school or
    hobby. The labeling function in this case simply asks users to label their relationships,
    assuming the answers do not contain any noise. At the scale of Facebook, this would require
    asking hundred of questions to each user on average, which is quite time-consuming. Furthermore,
    it would also cost marketing resource to convince users this is beneficial for them, and
    engineering time to ensure this information remain confidential. Likewise, in a biological
    network, determining whether two proteins interact positively or negatively with each other is
    achieved by a lab experiment, which requires time and material.

  \item All edges are already labeled, perfectly and without any cost, but the number $k$ of edge
    types is very large. Indeed, we have seen examples where they are hundreds or even thousands of
    edge types. Similarly to what happen in dimensionality
    reduction~\autocite{DimensionReduction10}, to make intuitive sense of such data, we want to
    reduce the number of edge types to less than ten. A natural way is to ask a domain expert to
    cluster edge types together. However, this is again time consuming, and does not necessarily
    make use of the topology of the graph.

  \item The input graph is actually unlabeled. That is, we observe interactions between the nodes,
    and we assume from domain knowledge that these interactions fall into $k$ categories. However,
    there is no reasonable way to come up with a specific labeling function. In
    \autoref{tab:edge_apps} \vpageref{tab:edge_apps}, we present several such examples, but for now
    we simply recall our earlier co-purchase network. As we mentioned, there are several reasons why
    two products could be bought together. Yet it is unlikely that customers will provide this type
    of feedback, for it does not bring them immediate advantage.
\end{enumerate}

In this thesis, we consider three versions of the problem of predicting edge types.
\begin{enumerate}

  \item The \esp{} problem, which takes as input the topology of a directed signed graph, and the
    label of some of the edges. The output is the label of the remaining edges. Therefore, it can be
    seen as a supervised binary classification problem.

  \item The same \esp{} problem, where the input graph is undirected. In this case, we consider an
    active variant where, instead of being given a random training set of labels, we have a budget of
    queries we can use to request arbitrary labels.

  \item The \ecp{} problem, which takes as input an unlabeled, undirected multilayer graph, the
    attributes of all its nodes, and a number of edge types $k$. The output is a $k$ clustering of
    the edges, and a set of $k$ vectors describing the clusters in terms of node attributes.
    Therefore, it can be seen as an unsupervised clustering problem with side information.
\end{enumerate}

We now briefly review existing approaches addressing these problems, in order to highlight our contribution
in the next section.

The modern formulation of the \esp{} on directed graphs can be attributed to
\textcite{Leskovec2010}. Their idea is to compute local features of the nodes based on the training
labels. Such features include variations on the node degree, such as the number of positive
outgoing edges or the number of negative incoming edges. These node features are combined into edge
features, and a supervised classification algorithm is trained. Several works have devised
additional features~\autocites{Bayesian15}{Yuan2017}, most notably based on scoring the nodes in a
Page Rank fashion~\autocites{traag2010exponential}{shahriari2014ranking}{wu2016troll}. These methods
are accurate, fast ---for they are local, and interpretable ---for the features are hand crafted.
The drawback is that this feature engineering is done mostly in an ad hoc way. On the other hand, it
is also possible to look at the problem from a global point of view, by completing the
signed adjacency matrix through low rank
factorization~\autocites{LowRankCompletion14}{OnlineCompletion17}. These methods are also accurate,
and bypass feature engineering, but by nature, they require careful algorithms to scale with larger
networks. Finally, during the writing of this manuscript, several papers have been published, which
use embedding of the nodes in a low dimensional space, based on the training
labels~\autocites{SIGNet17}{SNE17}. This also avoids feature engineering but is global in nature and
not so interpretable.

One way to solve the problem on undirected graphs would be to apply the previous methods, having
first replaced every edge $(u,v)$ by two directed edges \euv{} and \evu{}. This is not quite
satisfactory, both from a theoretical but also practical point of view (see
\autoref{sub:need_for_a_directed_graph}). Instead, \textcite{Cesa-Bianchi2012b} draw a connection
between the \esp{} problem and the \pcc{} problem. Recall that given a fully labeled signed graph,
the solution to \pcc{} is a partition of the nodes that minimize the number of disagreement
edges~\autocite{Bansal2002}. \Textcite{Cesa-Bianchi2012b} assume that the signs are originally
consistent with an underlying, hidden $2$-clustering, but that we only observe the signs after they have
been flipped \uar{}. In this case, they show that the optimal number of disagreements is a lower
bound of the number of mistakes made by any active \esp{} algorithm. One natural approach would then be
to solve \pcc{} based on the observed signs, and predict the remaining signs consistently with the
inferred clusters. At first, it seems hopeless, as \pcc{} is difficult to approximate on general
graphs~\autocite{Charikar2003}, even when there are only two clusters~\autocite{Giotis2006}.
However, this worst case analysis does not forbid more positive results on signed graphs that
exhibit stability under perturbation~\autocites{clusteringFeasibility15}{StableCC09}{StableLP09} or
are obtained through perturbations from an ideal case~\autocites{plantedAilon09}{Makarychev2014}.
Furthermore, in the active setting, the learner gets to choose which signs are observed. The general
idea of \textcite{Cesa-Bianchi2012b} is thus to use the query budget to build fully labeled paths
between each connected nodes. Those paths must be as short as possible, since the predicted sign of
$(u,v)$ is the product of the signs along the path from $u$ to $v$. Therefore, the shorter the path
and the less influence of the random perturbations. At the same time, those paths must span the
whole graph. This is the topic of an active research area~\autocites{Abraham2012}{Spanner17}.

As for the \ecp{} problem, to the best of our knowledge, it has not been studied under our
assumptions. A more common problem in attributed graphs is to cluster nodes into
communities~\autocites{LeskovecEgo12}{Yang2013}{Xu2014}{ZhangModelFree16}. However, it is not
immediate how such methods, generally based on generative models, could be adapted to our problem.
Direct approaches to classify edges have been proposed, based on graphical
models~\autocite{graphicalModelTies11}, nearest neighbors with a customized
distance~\autocite{Aggarwal2016a} and edge embedding in knowledge graphs~\autocite{transE13}. Yet
they all rely on having training labels as supervision. A simple unsupervised method is to cluster
the line graph of the input graph~\autocite{LineGraph09}, but this does not take advantage of
attributes. Another method uses topological features~\autocite{ahmed2017roles} but is rather complex
and thus not very interpretable. If we further assume that each edge type is associated with a
Euclidean space, and that the position of nodes in one space are not correlated with their position
in another space, then it is possible to recover an approximation of these Euclidean metrics in
polynomial time without any supervision~\autocite{Abraham2012a}.


\section{Outline}
\label{sec:outline}
The contributions of this thesis are organized as follow:

\begin{itemize}

  \item We start in \autoref{chap:troll} by addressing the \esp{} problem in \dssn{}. Our goal is to
    design a method that is scalable, principled and accurate. For that, we start by introducing a
    generative model for signs, and derive approximations of the optimal Bayes predictor and of the
    maximum likelihood predictor. We confirm the theoretical soundness of these approaches by
    performing extensive synthetic and real world experiments. Finally, to the best of our
    knowledge, we are the first to give an online algorithm for the \esp{} problem on directed
    graphs. This chapter is based on an existing publication~\autocite{trollSign17}.

  \item We then move from directed to undirected signed graphs, and explain in \autoref{chap:cc} why
    this requires another learning bias. Namely, we assume that the nodes belong to $k$ groups, and
    that the sign between two nodes is positive when both nodes belong to the same group, negative
    otherwise. We describe how this new bias is related to the \pcc{} problem, and present a
    thorough overview of existing approaches. Finally, we provide the first implementation of an
    existing spanner construction~\autocite{gtxFabio}, and give a preview of its performance on
    synthetic and real signed graphs.

 \item In \autoref{chap:vector}, we consider the \ecp{} problem on multilayer graphs with node
   attributes. Our approach is to seek a small number of vectors that, once assigned to every edge,
   best explain the graph (\ie{} maximize a score function between the vector assigned to an edge
   and the profiles of this edge's endpoints). From this initial formulation, we derive two
   optimization problems, and show how to solve them on synthetic data.

  \item Finally, reflecting on our treatment of the three previous problems, we discuss in
    \autoref{chap:conclusion} other settings and methods that might extend the problem of
    characterizing edges in complex networks beyond the frame of this thesis.

\end{itemize}

