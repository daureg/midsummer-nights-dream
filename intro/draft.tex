Graphs are well-suited to represent relationships between entities. In particular, a lot can be learned
from graphs encompassing several types of relations in one integrated structure. However, many
existing works that study interesting properties of such graphs, like community structure and
information spreading, assume that the graph if fully given. In this thesis instead, we develop
methods to predict
the type of edges given a graph topology, a few initial labels or alternatively attributes of the nodes.
\smallskip

\paragraph{Learning on graphs}

Graph theory has been around for almost three centuries~\autocite{biggs1976graph}, and it provides a
conceptually simple yet immensely rich framework to model situations where entities are connected
with each other~\autocite{ComplexNetworksApp11}. Coupled with the increasing availability of large
amount of relational data, learning on graphs has recently spurred a lot of interest across various
lines of research, with tangible benefits:
\begin{itemize}[nosep,leftmargin=*]
  \item community detection~\autocite{FortunatoSurvey10}, where the goal is to cluster nodes in
    tightly connected groups that are loosely connected with the rest of the graph, in order to
    better understand the graph organization the nodes relationships. It has been used for instance
    to identify proteins functional groups~\autocite{clusterBio03} or see how different scientific
    fields relate based on publication data~\autocite{clusterScience08}.
  \item semi-supervised learning~\autocites{SSL06}{graphSSL14}, where in addition to labeled data,
    the learner is also provided with unlabeled data at training time, and its goal is to classify
    nodes. Connecting similar instances
    allow propagating information along the graph, with applications in classifying text
    documents~\autocite{sslText09} or aligning categories and relations across multiple knowledge
    bases~\autocite{sslKB13}.
  % \item node classification~\autocite{nodeClassif11} Idem
    % could add visualisation (Roberto Tamassia. 2013. Handbook of Graph Drawing and Visualization.
    % CRC Press.) but it's not exactly learning
  \item node embedding~\autocite{representationLearning17}, where we seek a low dimensional
    representation of nodes based on their structural patterns, usually in an unsupervised way,
    although it is also possible to include problem supervision when available. Such representation can
    then be used in downstream tasks, for instance visualization~\autocite{LINE15} or the
    aforementioned node classification, even when new nodes join the graph after
    training~\autocite{inductiveRepresentation17}.
  \item link prediction~\autocite{linkPredSurvey16}, where the goal is, given a snapshot of the
    graph at time $t$, to return a set of links that do not exist at time $t$ but will be created by time
    $t+\Delta_t$. Most methods are based on the assumption that link creation is driven by node
    similarity, and it has been successfully applied to inferring potential interactions between
    proteins without expensive experiments~\autocite{linkPredBio06} and uncovering hidden
    associations in criminal networks~\autocite{linkPredCrime08}.
  \item information and influence propagation~\autocite{infmax13}, which is the study of processes
    by which content is spread across networks, and how such processes can be influenced to speed
    them up or slow them down. Two representative applications are selecting the best seeds in a social
    network to promote a viral marketing campaign~\autocite{infmaxKempe15} and containing more effectively the
    diffusion of actual biological viruses~\autocite{influenceBio13}.
  \item network evolution~\autocite{networkEvolution14}, which focuses on the mechanisms and
    consequences of the growth of networks, as well as how to keep the results on some data
    mining algorithms \enquote{fresh} and relevant. Monitoring the changes in the interactions of
    proteins can indeed be used as an early indicator for some kind of
    diabetes~\autocite{evolBio10}, while sudden changes in a network of computers are usually worth
    investigating, for they might signal external attacks~\autocite{evolSecurity04}.
\end{itemize}

\vspace{-\baselineskip}
\paragraph{Complex networks}

The prevalent setting for most of the works discussed above is that the graph connections are: (i)
homogeneous, in the sense that all edges are of a single type, having the same semantics  between
any pair of nodes; and (ii) based on the \emph{homophily} principle~\autocite{Homophily01}, roughly
meaning that nodes are linked because they are similar. However, many real world situations require
a more complex model, with several kinds of edge and where nodes can be connected if they are
similar only on some dimensions, or even if they are dissimilar. For instance, an e-commerce website
could let its users express whether they find the reviews written by other users helpful or not. As a
more involved example, consider a social network such as Facebook. Whereas the website offers only a
single \emph{friendship} relation, we may want to distinguish between family ties, work ties and
hobbies ties~\autocite{LeskovecEgo12}. Moreover, among one user's \enquote{friends}, there might be some
relations that are actually not positive~\autocite{Yang2012}, whether because of spam, noise or
friendship drift. Note finally that friendship is only one possible kind of relationship between two
users of Facebook, we could also link them whenever they attend the same event or are tagged in the
same post. Yet another possibility would be to study the same set of users across multiple social
networks~\autocite{mergingNetworks16}, in which case the edge type could denote some common
combinations of connections (\eg Twitter+LinkedIn or Facebook+Instagram). In this thesis, we thus
defend the following statement: \textbf{There exists efficient and accurate methods to predict edge
type in complex networks, relying either only on the graph topology or also on node metadata.}

A first motivation for such methods is that by labeling all edges, we can split a complex networks
in a superposition
of simple networks with a single edge type, on which we can apply existing methods. Of course,
different kind of relation are intertwined, thus a second stronger motivation is to identify the
different layers in order to study their interactions.

\vspace{-\baselineskip}
\paragraph{Predicting polarity in directed social networks}

The first kind of complex networks we consider are \emph{signed graphs}~\autocite{Tang2015a},
where edges carry a positive or a negative sign. Among the many applications of signed graphs, a
natural (and historical~\autocite{harary1953}) one is directed social networks. In such systems,
users can explicitly tag others, stating whether they trust them, support them, like them, or not.
Being able to predict edge signs in such networks improve the user experience, for instance by
promoting courteous online debates~\autocite{journalism15}, assembling efficient teams within
companies~\autocite{MLinHR16} and mitigating cyberbullying~\autocite{cyberbullying13}. Existing
approaches attack the problem along several angles: finding a low dimensional representation of the
nodes based on their connection patterns~\autocites{SIGNet17}{SNE17}, finding a low rank approximation
of the adjacency matrix~\autocites{LowRankCompletion14}{OnlineCompletion17} or extracting
topological features and training classifiers~\autocites{Leskovec2010}{Bayesian15}{wu2016troll}.
Guided by the idea that every user exhibit trollness to various extent, we propose a simple sign
generative model. This model naturally leads to two batch algorithms, one that probably approximate the
Bayes optimal predictor when the training set is dense enough, and one that approximate the maximum
likelihood estimates through label propagation. Extensive experiments on synthetic and real datasets
show that in addition to their sound theoretical foundations, our methods are competitive with the
state of the art in predictive performance, while being more scalable.
\iffalse
To illustrate this point, a
short extract of our results is presented in \autoref{tab:troll}.

\begin{table}[ptbh]
  \centering
  \caption{Matthews Correlation Coefficient measuring binary prediction performance (the closer to
    100 the better) in a \aut{}
    network as the proportion of labeled training edges grows. The best result is in red, the
  second one in blue, and our two methods are highlighted in gray.\label{tab:troll}}
  \vspace{0.5\baselineskip}
  \begin{tabular}{lccccc|r}
    \toprule
    method           & 5\%               & 10\%              & 15\%              & 20\%              & 25\%              & time (ms)            \\
    \midrule
    \uslogregp{}     & 19.8              & 27.9              & \vsecondSig{33.2} & 36.9              & 39.7              & \np{4}               \\
    \rowcolor{gray!20!white}
    \usrule{}        & \vsecondSig{19.8} & \vsecondSig{28.0} & 33.1              & \vsecondSig{37.1} & 39.7              & \textbf{\textless 1} \\
    \rowcolor{gray!20!white}
    \uslpropGsec{}   & \vfirstSig{24.2}  & \vfirstSig{31.7}  & \vfirstSig{36.1}  & \vfirstSig{38.9}  & \vfirstSig{41.1}  & \textbf{\np{19}}     \\
    \midrule
    \autocite{LowRankCompletion14}   & 12.4              & 17.9              & 22.0              & 25.7              & 29.0              & \np{3222}            \\
    \autocite{OnlineCompletion17}   & 1.2               & 12.6              & 22.2              & 30.3              & 36.5              & \np{23229}           \\
    \autocite{Leskovec2010}    & 11.4              & 17.2              & 21.0              & 24.3              & 27.0              & \textbf{\np{7}}      \\
    \autocite{wu2016troll} & 17.5              & 25.1              & 31.2              & 35.2              & 37.8              & \np{157}             \\
    \autocite{Bayesian15}  & 15.2              & 25.5              & 32.0              & 36.7              & \vsecondSig{39.8} & \np{4787}            \\
    \bottomrule
  \end{tabular}
\end{table}
\fi

\vspace{-\baselineskip}
\paragraph{Active learning in undirected signed graphs}

There are many other domains where signed graphs have proved useful. For instance in computer
vision, it is used to segment images in 2D~\autocites{Kim2011}{Bagon2011}{CellSeg14} or
3D~\autocites{VolumeSegmentation12}{Beier2015}, to simplify 3D shapes~\autocite{Shape3D17} and to
track target across sequential video frames~\autocite{multiTracking15}. When the nodes are words,
signed graphs are employed to identify coreference~\autocites{graphicalCoreference04}{Elsner2009} or
cluster synonym words~\autocite{SignedWordRatings}. They play a large role in biology, to clusters
genes~\autocite{Ben-Dor99}, identify mutation regions in chromosomes~\autocite{Das2015} or finding
stable subsystems~\autocite{monotoneBiology07}. Finally, given large databases with duplicated
records, they naturally help de-deduplicating
entities~\autocites{Crosslingual07}{DeDup09}{LargeScaleDeDup09}{WeightedER14}.

In all these cases, graphs are not directed and nodes are not human beings, meaning we can hardly
assign them trollness.  Therefore, we make a different assumption, positing that nodes are
partitioned in $K$ groups in such a way that links within groups are positive and links across
groups are negative. Recovering the optimal partition from the observation of all the edge signs is
called \pcc{}~\autocite{Bansal2002} in the Machine Learning field. Whereas it has been proved to be
\NPc{} to even approximate in the worst case~\autocites{Charikar2003}{Demaine2006}, our thorough
literature review paints a less gloomy landscape. Not only does there exist
exact~\autocite{Berg2015} and fixed-parameters tractable algorithms~\autocites{GoldenCE12}{Fomin2014},
efficient approximation~\autocites{CCPivotConf05}{ChawlaArxiv14} and large scale
heuristics~\autocites{Levorato2015}{Facchetti2011isingmodel}{Kappes2016}, but beyond the worst case,
we are interested in instances that exhibit stability under
perturbation~\autocites{clusteringFeasibility15}{StableCC09}{StableLP09} or are obtained through
random moves from an ideal case~\autocites{plantedAilon09}{Makarychev2014}. This leads us to develop
a spanning tree tailored to predicting edge signs in an active setting~\autocite{Cesa-Bianchi2012b},
where the learner is allowed to select its training labels in the most informative way. The idea is
to ensure that for every edge we will predict, there will be a path in the tree as short as
possible~\autocites{LowerBound95}{Abraham2012} from which we have queried all the signs. We call
this structure \gtx{}, as it recursively build stars of stars. 
\iffalse
We present preliminary results showing that on certain
kind of graphs, this indeed results in better predictive performance than a competitive baseline.
Namely, we create synthetic graphs of growing size with the \lpa{} model~\autocite{Barabasi1999},
partition the nodes in two clusters, set the edge signs accordingly and finally flip a fraction of
the signs \uar{}. As showed on \autoref{fig:gtx_xp_pasynthmcc}, \gtx{} always predict signs better than a
\bfs{} tree rooted at the highest degree node, especially when the noise level is low. Furthermore,
\gtx{} is less sensible to the graph size.

\begin{figure}[phtb]
  \centering
  \includegraphics[width=.8\textwidth]{gtx_exp/pasynthmcc}
  \caption{The Matthews Correlation Coefficient of \bfs{} and \gtx{} on synthetic \lpa{} graphs as a
    function of the number of nodes and of the fraction of signs who have been flipped from the
    hidden two-clustering.  \label{fig:gtx_xp_pasynthmcc}}
\end{figure}
\fi

\vspace{-\baselineskip}
\paragraph{Predicting more than two kinds of relations}

Finally, we consider the case where there are more than two possible type of relations, which is
often referred to as \emph{multilayer graphs}~\autocites{Kivela2014}{multiSurvey14}. In addition, we assume
that the graph is attributed, that is every node is described by a vector of numeric features called
the \emph{profile} of the node. Our assumption is then that links cannot be fully described by
global homophily. Thus we take a more nuanced view, combining partial homophily (that is, nodes are
connected when they are similar on a subset of the attributes) and heterophily (that is, nodes are
connected when they are dissimilar on a subset of the attributes). As examples of the latter, think
of dating websites ---where most users are linked with users of the opposite
gender~\autocites{homophilyMyspace09}{Tinder16}; diffusing innovations ---where meeting people with
different backgrounds and point of views is crucial to favor diversity and
creativity~\autocite{rogers2003diffusion}; and online news consumption ---where connecting people
from different sides of the political spectrum helps to avoid echo chambers and fuel a democratic
debate~\autocite{balancedNews17}. Our goal in this setting remains the same as before: predict the
type of a given edge. This is
widely applicable, for multilayer graphs are present  in various fields. In social networks, where
players of the Pardus massively multilayer online game can have six types of relations (friendship,
enmity, private message, trading, aggression and bounty)~\autocite{Szell2010}, or where photo sharers
of the Flickr website can interact in eleven ways (either directly or through comments, tags, groups
and so on)~\autocite{RecoFlickrMulti11}. In citation networks, where authors of the DBLP dataset are
connected if they have co-authored a paper in one the \np{1000} conferences that form the edge
types~\autocite{communityDBLPbyConf05}, or where \np{5000} SIAM papers can be connected for five
reasons (abstract similarity, title similarity, keywords similarity, author similarity or
citation)~\autocite{articlesMultiSim11}. In economic networks, where 951 ports are connected by three
kinds of ships (bulk dry carriers, container ships and oil tankers)~\autocite{ports3kindofships10} or
where 162 countries of the International Trade Network are connected by 96 kinds of commodities they
can exchange~\autocite{worldTradeNetwork10}. And in biology, where genes are connected in a
co-expression network under 130 different experimental conditions~\autocite{bioLayerExp11}.

There exists previous works that lean towards analyzing and characterizing edges in such multilayer
graphs, although they do not share exactly our goal of predicting edge type. For instance, one can
maximize the probability of a generative model in ego networks with $K$
categories~\autocites{LeskovecEgo12}{LeskovecEgo14}, look for communities that are both tightly
connected and share common keywords~\autocite{AttributedCommunity16}, find a partition maximizing
the density of intra community edges ---weighted by the endpoints profile similarities and a learned
community weight vector~\autocite{ZhangModelFree16}, recover with low distortion $K$ metric spaces
whose small-world superposition yield the observed graph~\autocite{Abraham2012a}, discover edge role
by computing high-order topological features~\autocite{ahmed2017roles}, group nodes with common
attributes and extract frequent links between such groups as edge types~\autocite{conceptualLinks12},
or cluster the profiles into a set of affine subspace of $\mathbb{R}^d$~\autocite{SCSurvey11}.

Our approach is to seek a small number of vectors that, once assigned to every edge, best explain
the graph (\ie maximize a score function between the vector assigned to an edge  and the profiles of
this edge's endpoints).
\iffalse
From this initial formulation, we derive two optimization problems, and initially solve them on
synthetic data, generated with a few natural topological constraints. We compared these solutions
with the result of a baseline approach that simply clusters the pairwise profile vectors using
$k$-means. As showed on \autoref{fig:edge}, on a graph with \np{500} nodes, \np{1300} edges and 7
types of edges (where each node is involved in at most 3 types of relation), we recover the hidden
edge assignment more accurately than the $k$-means baseline.

\begin{figure}[htpb]
  \centering
  \includegraphics[width=0.8\linewidth]{assets/raw/Gfixed_over0_step2_7dirs3_ambi.pdf}
  \caption{Distribution of the Adjusted Mutual Information (the closer to $1$ the better) over
  \np{600} generations of 7 random directions, comparing our \textsf{optim} method with the $k$-means
baseline. \label{fig:edge}}
\end{figure}
\fi

While our thesis is therefore primarily concerned with learning on graphs, instead of focusing on
nodes (to predict their class, their cluster or to find their representation), we look at edges (\ie
pairs of nodes), and instead of having one kind of relationship, we have two, or more than two when
the nodes are attributed.
