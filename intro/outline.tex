The contributions of this thesis are organized as follow:

\begin{itemize}

  \item We start in \autoref{chap:troll} by addressing the \esp{} problem in \dssn{}. Our goal is to
    design a method that is scalable, principled and accurate. For that, we start by introducing a
    generative model for signs, and derive approximations of the optimal Bayes predictor and of the
    maximum likelihood predictor. We confirm the theoretical soundness of these approaches by
    performing extensive synthetic and real world experiments. Finally, to the best of our
    knowledge, we are the first to give an online algorithm for the \esp{} problem on directed
    graphs. This chapter is based on an existing publication~\autocite{trollSign17}.

  \item We then move from directed to undirected signed graphs, and explain in \autoref{chap:cc} why
    this requires another learning bias. Namely, we assume that the nodes belong to $k$ groups, and
    that the sign between two nodes is positive when both nodes belong to the same group, negative
    otherwise. We describe how this new bias is related to the \pcc{} problem, and present a
    thorough overview of existing approaches. Finally, we provide the first implementation of an
    existing spanner construction~\autocite{gtxFabio}, and give a preview of its performance on
    synthetic and real signed graphs.

 \item In \autoref{chap:vector}, we consider the \ecp{} problem on multilayer graphs with node
   attributes. Our approach is to seek a small number of vectors that, once assigned to every edge,
   best explain the graph (\ie{} maximize a score function between the vector assigned to an edge
   and the profiles of this edge's endpoints). From this initial formulation, we derive two
   optimization problems, and show how to solve them on synthetic data.

  \item Finally, reflecting on our treatment of the three previous problems, we discuss in
    \autoref{chap:conclusion} other settings and methods that might extend the problem of
    characterizing edges in complex networks beyond the frame of this thesis.

\end{itemize}
