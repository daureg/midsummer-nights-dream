In this thesis, we develop methods to efficiently and accurately characterize edges in complex
networks. In simple graphs, nodes are connected by a single semantic. For instance, two users are
friends in a social networks, or there is a hypertext link from one webpage to another. Furthermore,
those connections are typically driven by node similarity, in what is known as the homophily
mechanism. In the previous examples, users become friends because of common features, and webpages
link to each other based on common topics. By contrast, complex networks are graphs where every
connection has one semantic among $k$ possible ones. Those connections are moreover based on both
partial homophily and heterophily of their endpoints. This additional information enable finer
analysis of real world graphs. However, it can be expensive to acquire, or is sometimes not known
beforehand. We address the problems of inferring edge semantics in various settings. First, we
consider graphs where edges have two opposite semantics, and where we observe the label of some
edges. These so-called \emph{signed graphs} are a convenient way to represent polarized
interactions. We propose two learning biases suited for directed and undirected signed graphs
respectively. This leads us to design several algorithms leveraging the graph topology to solve a
binary classification problem that we call \esp{}. Second, we consider graphs with $k \geq 2$
available semantics for edge. In that case of \emph{multilayer graphs}, we are not provided with any
edge label, but instead are given one feature vector for each node. Faced with such an unsupervised
\ecp{} problem, we devise a quality criterion expressing how well an edge $k$-partition and $k$
semantical vectors explains the observed connections. We optimize this \emph{goodness of
explanation} criterion in vectorial and matricial forms, and show how those two methods perform on
synthetic data.
