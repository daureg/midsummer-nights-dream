First and foremost, I would like to express my deepest gratitude to my
supervisor, Pr. Marc Tommasi. From the day we first got in touch in that spring
four years ago, and all the way towards my defense, he has provided me with
thorough guidance. In terms of research of course, but on many other topics as
well, from professional development to folkloric music. Moreover, this has
always been done in the kindest way possible. Therefore, I'm doubly indebted to
him, for not only I wouldn't have complete my PhD without him, but he made it a
very pleasant experience. Second, such feelings carries over to include my
advisor, Dr. Fabio Vitale. There has been time when his sense of
rigor has challenged me. But at the end, it was an important learning
experience, and his ability to crack a joke at the least expected moment of a
long research meeting has proved very useful: the fact we went to bars in at
least three different countries speaks for it. I would also like to thank Pr.
Claudio Gentile for hosting me in Varese, and giving me sharp advice along
those three years.

Then I want to thank Alessandro Provetti and Liva Ralaivola who kindly agreed
to review my manuscript, as well as Elisa Fromont and Sophie Tison for having
accepted to be part of my committee.
%
To conclude this academic paragraph, I'd like to sincerely acknowledge how
lucky I was to collaborate with all my talented co-authors: Nicolò
Cesa-Bianchi, Emre Çelikten, Claudio Gentile, Aristides Gionis, Michael Mathioudakis and Fabio
Vitale.

\medskip

Although doing a PhD is a rather personal enterprise, I was fortunate to be
part of the awesome Magnet
team\footnote{\url{https://www.inria.fr/en/teams/magnet}}, whose past and
current members I met include, in seating order: Thomas, Pauline, David,
Mathieu (Thursday French Fries forever), Thanh, Carlos, Pierre, Paul, Juhi,
Quentin, William, Mikaela, Pascal (special thanks for great accommodation and
sportive spirit), Rémi, Aurélien and Hippolyte. Thanks for stimulating
discussions on many topics, constant readiness to help and answer my questions,
life long lessons of table football, and quite importantly, thanks for
patiently bearing with me while I was finishing lunches (and/or fiddling with
my phone)! Of course, this wouldn't be proper acknowledgment without
mentioning the best office in the world, namely B224, and the best possible
office mates: Nathalie (for all the cookies) and Thibault (because there were
some genuinely good jokes among all of them). Whereas I've always been slightly
worried about logistics, I never had to think about it thanks to the fantastic
work of Julie. Finally, one last Magnet member, Antonino, left before I joined
but nonetheless deserve my gratitude for he helped me find my next job. Same
credit goes to Michal, in addition to being a great company in all occasions.
At this stage, I might as well thanks the rest of the Sequel team, and
especially Julien, Alexandre, Émilie, Frédéric, Romaric, Matteo, Daniele,
Julien, Bilal, Marta, Tomáš,
Daniele, Florian, Jean-Bastien, Olivier and Ronan (too bad my defense is not on Friday, for I know
you wouldn't be long to hit the dance floor). Finally I want to thank all the
support crews from Inria, as they made it a great place for doing research, as
well as the University Lille 3 for making my short teaching experience so
pleasant.

\medskip

Doing a PhD is also supposed to be a rather labor intensive enterprise, but
again, I was fortunate to live in the student city of Lille. There I met many
wonderful people, each of them who contributed in a way to this manuscript. It
all started when a young version of myself attended a student event organized
by Tilda and later had to pleasure to collaborate with Guillaume, Antoine,
Fabien, Alexandre, and by extension with Carmelo, Jason, Cindy, Antoine
(Dujardin), Clément, Ilkay, Benjamin and Émilie. On my way to organize an
elusive café lingua, I also enjoyed the company of many members of a meetup
group, including Samir, Nathalie, Marie-Pierre, Anne Charlotte, Léa, Géraldine,
Fabienne, George, Norida and Thomas, as well as Rashida, Mairead and Yash, who
bore with me during circumstances better not written here. Another great
event in Lille is undeniably the \enquote{apéro culture}, where I was
privileged to meet Cyrille, Élise, Audrey, Wasilla and Olmo. This picture
wouldn't be complete without Stéfana, Arthur, Esmeralda, Daniel, and my
flatmate Édouard for some nice video games nights.

I also received support from outside Lille, thanks to Tristan, Olivier (may the
for be with you guys), Aloïs, Juliette, Florent, Isabelle, Camille and Claire.
In these cold days of February, I also have a warm thought for people I met
during my visits in Aalto: Luiza, Annika, Eric, Polina and Klaudia as well as
Kiran, Vera and Sanja
for their long-standing support and comments on some part of this thesis.

You astute reader may at this point grow suspicious of whether a single man
can really be so lucky. Yet this is not over and I was again fortunate to met
just before starting my PhD a wonderful friend. Over those years, Nataša has
given me support that words probably can't describe. But it's safe to say that
without her, this thesis and even myself would both be lacking something
essential.

\medskip

\begin{otherlanguage}{french}
Enfin, j'aimerais remercier ma mère et mon frère, pour leur soutien
inconditionel sans lequel toute cette aventure n'aurais pas été possible.
\end{otherlanguage}

\vspace{5\baselineskip}
\begin{flushright}
  {\itshape
    Berlin, Germany, February 27, 2018 \\
    Géraud Le Falher
  }
\end{flushright}
