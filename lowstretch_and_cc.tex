\documentclass[nobib, a4paper, 10pt, oneside, onecolumn, openany, notitlepage, final,
svgnames, marginals=raggedouter, english,
%symmetric, justified,
]{article}
\usepackage{charter}
\setcounter{secnumdepth}{3}
\setcounter{tocdepth}{3}
\usepackage[utf8]{inputenc}
\usepackage[T1]{fontenc}
\usepackage{csquotes}
\usepackage{babel}
\usepackage[heightrounded]{geometry}
\usepackage{setspace}
\usepackage{microtype}
\usepackage[citestyle=alphabetic, bibstyle=ieee-alphabetic, isbn=false, maxnames=1, minnames=1,
sorting=nyvt, backref=true, backend=biber, defernumbers=true]{biblatex}
\AtEveryBibitem{
   \clearfield{arxivId}
   % \clearfield{booktitle}
   % \clearfield{doi}
   \clearfield{eprint}
   \clearfield{eventdate}
   \clearfield{isbn}
   \clearfield{issn}
   % \clearfield{journaltitle}
   \clearfield{month}
   % \clearfield{number}
   % \clearfield{pages}
   \clearfield{series}
   \clearfield{url}
   \clearfield{urldate}
   \clearfield{venue}
   % \clearfield{volume}
   \clearlist{location} % alias to field 'address'
   \clearlist{publisher}
   \clearname{editor}
}
\addbibresource{/home/orphee/data/projects/biblio/library.bib}
\addbibresource{more.bib}
\addbibresource{gtx.bib}
\usepackage{xcolor}
\usepackage{amsmath}
\usepackage{amsthm}
\usepackage{mathtools}
\usepackage{sidenotes}
\renewcommand{\footnote}[1]{\sidenote{\footnotesize #1}}
\usepackage{enumitem}
\usepackage{tabulary}
\usepackage{multirow}
\usepackage[marginpar]{todo}
\usepackage[np,autolanguage]{numprint}
\usepackage{listings}
\lstset{frame=none, language=Python, tabsize=4,}
% \usepackage{inconsolata}
\usepackage{nicefrac}
\usepackage{tikz}
\usetikzlibrary{arrows,automata,calc}
\usetikzlibrary{arrows.meta}
\usetikzlibrary{decorations.pathreplacing}
\usetikzlibrary{backgrounds}
\tikzset{%
  show curve controls/.style={
    postaction={
      decoration={
        show path construction,
        curveto code={
          \draw [blue] 
            (\tikzinputsegmentfirst) -- (\tikzinputsegmentsupporta)
            (\tikzinputsegmentlast) -- (\tikzinputsegmentsupportb);
          \fill [red, opacity=0.5] 
            (\tikzinputsegmentsupporta) circle [radius=.25ex]
            (\tikzinputsegmentsupportb) circle [radius=.25ex];
        }
      },
      decorate
}}}
\tikzstyle{vertex}=[draw,circle,black,inner sep=2pt]
\tikzstyle{edge}=[line width=1.3pt,color=Black]
\tikzstyle{rare}=[fill=black,text=white]
\tikzstyle{medium}=[fill=black!15!white]


\usepackage{algorithm}
\usepackage[noend]{algpseudocode}
\newcommand*\Let[2]{\State #1 $\gets$ #2}
\def\algorithmautorefname{Algorithm}

\usepackage{varioref}
\usepackage{hyperref}
\hypersetup{%
    % draft,    % = no hyperlinking at all (useful in b/w printouts)
    colorlinks=true, linktocpage=true, pdfstartpage=3, pdfstartview=FitV,%
    % uncomment the following line if you want to have black links (e.g., for printing)
    %colorlinks=false, linktocpage=false, pdfborder={0 0 0}, pdfstartpage=3, pdfstartview=FitV,%
    breaklinks=true, pdfpagemode=UseNone, pageanchor=true, pdfpagemode=UseOutlines,%
    plainpages=false, bookmarksnumbered, bookmarksopen=true, bookmarksopenlevel=1,%
    hypertexnames=true, pdfhighlight=/O,%nesting=true,%frenchlinks,%
    urlcolor=Chocolate, linkcolor=RoyalBlue, citecolor=LimeGreen, %pagecolor=RoyalBlue,%
}
\newcommand{\marginpars}[1]{\marginpar{\small#1}}
\usepackage{caption}
\usepackage[margin=0pt,font+=small,labelformat=parens,labelsep=space,
skip=6pt,list=false,hypcap=false]{subcaption}
\captionsetup{compatibility=false}
\usepackage{graphicx}
\usepackage{booktabs}
\graphicspath{{./assets/}}
% \usepackage[capitalize,noabbrev]{cleveref}
\usepackage[]{extdash}

\newcommand{\asym}{\emph{A sym exp}}
\newcommand{\bfs}{\textsc{Breadth First Tree}}
\newcommand{\ccPivot}{\textsc{CC-Pivot}}
\newcommand{\epi}{\textsc{Epinion}}
\newcommand{\etest}{\ensuremath{E_{\mathrm{test}}}}
\newcommand{\etrain}{\ensuremath{E_{\mathrm{train}}}}
\newcommand{\gplus}{\textsc{Google+}}
\newcommand{\grid}{\textsc{Grid}}
\newcommand{\gtx}{\textsc{Galaxy Tree}}
\newcommand{\extractStar}{\textsc{Extract-Stars}}
\newcommand{\collapseStar}{\textsc{Collapse-Stars}}
\newcommand{\lpa}{\textsc{Preferential Attachment}}
\newcommand{\pcc}{\textsc{Correlation Clustering}}
\newcommand{\rst}{\textsc{Random Spanning Tree}}
% \newcommand{\sgt}{\textsc{Galaxy Tree}}
\newcommand{\shz}{\textsc{Shazoo}}
\newcommand{\sla}{\textsc{Slashdot}}
\newcommand{\wik}{\textsc{Wikipedia}}
\renewcommand{\triangle}{\textsc{Triangle}}

\newcommand{\ith}{\ensuremath{i^{\mathrm{th}}}}
\newcommand{\jth}{\ensuremath{j^{\mathrm{th}}}}
\newcommand{\tth}{\ensuremath{t^{\mathrm{th}}}}
\newcommand{\uar}{uniformly at random}
\newcommand{\ie}{i.e.\@}
\DeclareMathOperator{\degr}{deg}
\DeclareMathOperator*{\argmin}{arg\,min}

\newcommand{\starone}[1]{\ensuremath{\textcolor{DodgerBlue}{S_{#1}^1}}}
\newcommand{\startwo}[1]{\ensuremath{\textcolor{Orange}{S_{#1}^2}}}
\newcommand{\starthree}[1]{\ensuremath{\textcolor{Green}{S_{#1}^3}}}

\newcommand{\euv}{\ensuremath{u\rightarrow v}}
\newcommand{\evu}{\ensuremath{v\rightarrow u}}
\newcommand{\yuv}{\ensuremath{y_{u, v}}}
% \newcommand{\pathtuv}{\ensuremath{\text{\textsc{Path}}^T_{u,v}}}
\DeclareMathOperator{\pathm}{path}
\newcommand{\pathtuv}{\ensuremath{\pathm^T(u, v)}}


\title{Edge sign prediction in general graphs and connection with Correlation Clustering}
% \usetikzlibrary{external}
% \tikzexternalize[mode=list and make, failed ref warnings for={\ref,\cite,\pageref},]
\begin{document}
\newgeometry{vmargin=1.3cm,left=1.5cm,textwidth=130mm,marginparsep=5mm,marginparwidth=52mm}
\maketitle
% \tableofcontents

% In this thesis we will sometimes write remarks in a smaller font and with a light blue edging.
\begin{aside}
  Such remarks provide additional information to the topic discussed above, but can be skipped
  without harming the comprehension of the main material.
\end{aside}

\autoref{tab:notations} presents a tentative list of notations, but is still subject to change.

\begin{table*}[thpb]
  \centering
  \caption{List of notations used in this thesis}\label{tab:notations}%
  \vspace{-.5\baselineskip}
  \begin{tabulary}{187mm}{LL}
    \toprule
    Symbol & Meaning \\
    \midrule
    \rangek{} & The natural integers from $1$ to $K$, \ie{} $\{1, 2, \ldots, K \}$ \\
    $G$ & An arbitrary graph. It should be clear from the context whether it is weighted or not, and directed or not \\
    $V$ & The set of all the nodes of a graph, with $|V|=n$. Unless noted otherwise, nodes are indexed from $1$ to $n$ \\
    $u$ & A generic node of $G$. When referring to several nodes, we naturally use $u$, $v$, $w$ and so on. \\
    $E$ & The set of all the edges of a graph, with $|E|=m$ \\
    $(u,v)$ & An undirected edge between nodes $u$ and $v$ \\
    \euv{} & A directed edge from node $u$ to node $v$ \\
    $\yuv{}$ & The sign of the edge $(u,v)$, which can be either $+1$ or $-1$ \\
    $Y(E)$ & The labeling of $E$, that is the set of all signs of $E$: $Y(E) = \{\yuv : (u,v)\in E\}$ \\
    $\etrain$ & A subset of $E$, given or chosen, of which we observe the signs \\
    $\degr(u)$ & The total degree of node $u$ (that is, the number of edges incident to $u$,
    regardless of their direction) \\
    $\nei(u)$ & The set of all neighbors of $u$, regardless of edge direction. It thus holds
    that $|\nei(u)| = \degr(u)$ \\
    $T$ & An unweighted and undirected tree \\
    $\pathtuv$ & The unique path between $u$ and $v$ in the tree $T$, represented by an ordered list of edges \\
    $|\pathtuv|$ & The length of the path between $u$ and $v$ in the tree $T$, that is its number of edges \\
    \bottomrule
  \end{tabulary}
\end{table*}


\section{Limitations of the troll method}
\label{sec:limitations_of_the_troll_method}

\subsection{Reliance on the sign generative model}
\label{sub:reliance_on_the_sign_generative_model}

The method presented in the previous chapter hinges crucially upon our sign generative model. Yet
one can imagine contexts where this model is not applicable, especially when nodes do not represent
human beings. One way to alleviate this issue was introduced with our online algorithm, where this
time, signs are generated by an arbitrary adversary. However, in that case, we are still facing two
limitations\marginpars{limitations is not the right word, maybe \emph{strong constraints}…}
\begin{enumerate}
	\item Our bias remains that the labeling is regular (recall this means informally than the all
		the outgoing signs from a given node tend to be the same, and likewise for the incoming signs),
		since irregularities are the cost payed by the adversary to make our algorithm mispredict. While
		this bias is well suited to social networks, other applications may require other
		bias\marginpars{namely the \pcc{} bias we'll introduce next}.
	\item In the online setting, we evaluate our performance by the regret\marginpars{ref to regret equation},
		whereas in general we are interested in more classical measures of predictive accuracy.
\end{enumerate}

\subsubsection{Experience on directed biological network}
\label{ssub:experience_on_directed_biological_network}

Take for instance a biological network.

According to \href{https://web.stanford.edu/class/cs224w/slides/handout-bionets.pdf}%
{this description of various biological networks}, \emph{gene regulatory network} are directed and
contains activation and inhibition links, as \href{https://en.wikipedia.org/wiki/Gene_regulatory_network#Overview}%
{showed on Wikipedia}. There are some online databases such as
\href{http://regulondb.ccg.unam.mx/menu/download/datasets/index.jsp}%
{RegulonDB} or \href{http://www.pathwaycommons.org/pcviz/}{Pathway Commons} (this last one provide
network in the BioPAX format, which can be visualized by
\href{http://www.cytoscape.org/}{CytoScape}).
Another source is the \href{https://www.ncbi.nlm.nih.gov/pmc/articles/PMC2708159/table/T1}{Table 1}
of \cite{BioSigned09}.
%DasGupta also describe (in
%http://www.sciencedirect.com/science/article/pii/S0303264706001419#sec14) how
%they built a directed signed network from this SBML file
%https://www.ncbi.nlm.nih.gov/pmc/articles/PMC1681468/bin/msb4100014-sd1.xml

other domain? coref, images (is it directed?), entity resolution…\marginpars{Actually it's difficult
to find other kind of graph because all constructed ones (like image, coreference,
deduplication) are inherently symmetric}

\subsection{Need for a directed graph}
\label{sub:need_for_a_directed_graph}


A more serious limitation is that our sign model only applies to directed graphs. Consider
another model, where each node $i$ is endowed with an integer $c_i \in \{1, \ldots, k\}$ that
specifies its cluster and let the sign of the undirected edge $i,j$ be $+$ if $c_i = c_j$ and $-$
otherwise.  This corresponds naturally to the situation modeled by the \pcc{} problem.

\section{\pcc{}}
\label{sec:correlation_clustering}

\subsection{Problem setting}
\label{sub:problem_setting}

\subsection{Relation with edge sign prediction}
\label{sub:relation_with_edge_sign_prediction}

\iffalse
For the 2 clusters case ($k=2$), characterization proven already in 1936 by \textcite{Konig36}, as noted in
\cite{Zaslavsky2012} (commenting on \cite{harary1953}: \enquote{Although Thm. 3 was anticipated by
\textcite[Theorem X.11]{Konig36}  without the terminology of signs, here is the
first recognition of the crucial fact that labelling edges by elements of a
group—specifically, the sign group—can lead to a general theory.})
and \cite{Huffner2010} (\enquote{\textcite{Konig36} proved the following characterization of
	balanced graphs. For a graph $G = (V , E)$, the following are equivalent:\marginpars{There is a
	proof in \autocite[p. 111]{BookKleinberg2010}, maybe I can rewrite it as well}
	\begin{enumerate}
		\item $V$ can be partitioned into two sets $V_1$ and $V_2$ called sides such that there is no
			negative edge $\{v, w\} \in E$ with both $v, w \in V_1$ or both $v, w \in V_2$ and no negative
			edge $\{v, w\}$ with $v \in V_1$ and $w \in V_2$ .
		\item $V$ can be colored with two colors such that for all $\{v, w\} \in E^-$, the vertices $v$
			and $w$ have different colors, and for all $\{v, w\} \in E^+$, the vertices $v$ and $w$ have
			the same color. The color classes correspond to the sides.
		\item $G$ does not contain cycles with an odd number of negative edges.
	\end{enumerate}
	Using the characterization by a coloring, it is easy to see that balance of a signed
graph can be checked in linear time by depth-first search.})
\fi

\subsection{State of the art}
\label{sub:state_of_the_art}

% ICML paper this year that touch something very related multicut and give recent applications in
% vision https://arxiv.org/abs/1503.03791

\subsection{Variants and extensions}
\label{sub:variants_and_extensions}

\subsubsection{\pcc{} under stability assumption}
\label{ssub:cc_under_stability_assumption}

\iffalse
Haris Angelidakis, Konstantin Makarychev, and Yury Makarychev. 2017.
Algorithms for Stable and Perturbation-Resilient Problems. STOC’17
\href{http://ttic.uchicago.edu/~yury/papers/two-stable.pdf}{10.1145/3055399.3055487}
improves over the one cited in the internship description
\fi

\subsubsection{Parallel \pcc{}}
\label{ssub:parallel_cc}

\subsection{Empirical evaluation?}
\label{sub:cc_empiracal_evaluation}

\newpage

\section{Low stretch trees and spanners}
\label{sec:low_stretch_trees_and_spanners}

In this thesis we will sometimes write remarks in a smaller font and with a light blue edging.
\begin{aside}
  Such remarks provide additional information to the topic discussed above, but can be skipped
  without harming the comprehension of the main material.
\end{aside}

\autoref{tab:notations} presents a tentative list of notations, but is still subject to change.

\begin{table*}[thpb]
  \centering
  \caption{List of notations used in this thesis}\label{tab:notations}%
  \vspace{-.5\baselineskip}
  \begin{tabulary}{187mm}{LL}
    \toprule
    Symbol & Meaning \\
    \midrule
    \rangek{} & The natural integers from $1$ to $K$, \ie{} $\{1, 2, \ldots, K \}$ \\
    $G$ & An arbitrary graph. It should be clear from the context whether it is weighted or not, and directed or not \\
    $V$ & The set of all the nodes of a graph, with $|V|=n$. Unless noted otherwise, nodes are indexed from $1$ to $n$ \\
    $u$ & A generic node of $G$. When referring to several nodes, we naturally use $u$, $v$, $w$ and so on. \\
    $E$ & The set of all the edges of a graph, with $|E|=m$ \\
    $(u,v)$ & An undirected edge between nodes $u$ and $v$ \\
    \euv{} & A directed edge from node $u$ to node $v$ \\
    $\yuv{}$ & The sign of the edge $(u,v)$, which can be either $+1$ or $-1$ \\
    $Y(E)$ & The labeling of $E$, that is the set of all signs of $E$: $Y(E) = \{\yuv : (u,v)\in E\}$ \\
    $\etrain$ & A subset of $E$, given or chosen, of which we observe the signs \\
    $\degr(u)$ & The total degree of node $u$ (that is, the number of edges incident to $u$,
    regardless of their direction) \\
    $\nei(u)$ & The set of all neighbors of $u$, regardless of edge direction. It thus holds
    that $|\nei(u)| = \degr(u)$ \\
    $T$ & An unweighted and undirected tree \\
    $\pathtuv$ & The unique path between $u$ and $v$ in the tree $T$, represented by an ordered list of edges \\
    $|\pathtuv|$ & The length of the path between $u$ and $v$ in the tree $T$, that is its number of edges \\
    \bottomrule
  \end{tabulary}
\end{table*}


Given a graph $G$, and following the social balance theory roughly summarized as \enquote{the enemy
of my enemy is my friend}, assume that the labelling of $E$ is consistent with a two-clustering of
$V$. Namely, assume that $V$ can be partitioned in two clusters such that edges within each cluster
are positive and edges across clusters are negative. In that case, the following
\emph{multiplicative rule} holds: for any nodes $u$, $v$ in $V$, and any path $p$ between $u$ and
$v$ in $G$, the sign $\yuv$ is equal to the product of the signs along $p$ (we call this product the
parity of $p$ and denote it $\pi(p)$). While it is a simple and convenient model, this is too strong
of a requirement to be satisfied in practice. Therefore, we relax it by assuming that we start with
a consistent labeling $Y$ but can only observe a randomly perturbed version $Y'$ of $Y$.
Specifically, given a constant $q\in [0, \nicefrac{1}{2})$, each sign of $Y$ is flipped with a
probability smaller than $q$.

In this section, we are interested in active learning algorithms, that first query a subset
$\etrain$ of the edges, observe the signs in $\etrain$ and use them to predict the remaining signs.
More precisely, we focus on an algorithm that queries spanning tree $T$ of $G$ and predicts the sign
of an edge $(u,v) \in \etest = E \setminus E_T$ as the parity of $\pathtuv$. While this
can\Todo{state it formally then ;)} be stated more formally~\autocite[Section
4.1]{Cesa-Bianchi2012b}, it makes sense intuitively that since each sign has been potentially
flipped, the longer the path in $T$, the more likely its parity will be not be equal to the true
sign $\yuv{}$. Therefore we would like each such path to be as short as possible. In the following,
we describe a way to build spanning trees tailored for this situation.

\subsection{\gtx{}: a spanning tree designed for sign prediction}
\label{sub:gtx_a_simple_low_stretch_tree_construction}

The \gtx{} algorithm takes as input a graph topology $G_0=(V_0, E_0)$ and produces a sequence of
graphs $\{G_t\}_{t=1}^T$ of decreasing size until each connected component of $G_0$ is reduced to a
single node. As we will prove later, after reaching this point, the algorithm has selected $|V_0| -
1$ edges that form a spanning tree of $G_0$. $G_{t+1}$ is obtained from $G_t$ by composing two
primitives so that we can informally write $G_{t+1} = \left(\collapseStar{} \circ
\extractStar{}\right)(G_t)$.

\extractStar{} partitions the graph $G_t$ into a set of stars and \collapseStar{} build the graph
made of those stars using the edges in $E_t$. We provide more details on those two operations in the
following, as well as their complexity analysis. Then we state formally the \gtx{} algorithm and prove
its termination and correctness.  Finally, we study its properties, such as the number of iterations
needed to finish and the stretch of the resulting tree. For simplicity and without loss of
generality, we assume that $G_0$ consist of a single connected component.

\medskip

\extractStar{} takes as input a graph $G_t=(V_t, E_t)$, and optionally a \emph{threshold function}
$t_f$ or a \emph{degree function} $d_f$. While the nodeset $V_t$ is not exhausted, it repeatedly samples a
star center $c_i$, creates a star $S_i^t$ with the neighbors of $c_i$, removes all the nodes of $S_i^t$ from
$V_t$ and all the edges incident to $S_i^t$ from $E_t$, and finally decrements accordingly the
degree of the 2-hop neighbors of $c_i$ (see \autoref{fig:gtx_star_simple} for a visual
representation of this notation).
\begin{marginfigure}
  \centering
  \includegraphics[height=0.15\textheight]{assets/tikz/gtx_star_tikz.pdf}
  \caption[A sample star]{A sample star created during the \tth{} extraction level. The black node
    % \tikz{\node[vertex,rare] {$c_i$};}
    is the center $c_i$ of the star $S_i^t$, which is made of the four light gray peripheral nodes
  % \tikz{\node[vertex,medium] {$p_1$};} to \tikz{\node[vertex,medium] {$p_4$};}
  as well as the solid edges. The 2-hops neighbors of $c_i$ are the white nodes
  % \tikz{\node[vertex] {$h_1$};} to \tikz{\node[vertex] {$h_3$};}
  $h_1$ to $h_3$, whose degree will decrease once we $S_i^t$ is removed from $G_t$.}
  \label{fig:gtx_star_simple}
\end{marginfigure}
Upon completion, it returns a list of stars and a mapping of
each node of $V_t$ to the unique star it belongs to. We consider three heuristics to choose centers:

\begin{itemize}%[nosep]
  \item choose the node with the current highest degree, with ties broken arbitrarily
  \item if $n_i$ is the number of node remaining in $V_t$ before choosing the \ith{} center, choose
    a node \uar{} among those with a degree larger than $t_f(n_i)$. Setting the threshold function
    to be the identity therefore recovers the previous strategy, but the idea here is to choose
    among a small set of high degree nodes, for instance by letting $t_f(n) = \sqrt{n}$
  \item if $\degr(u)$ is the degree of node $u$, choose node proportionally to $d_f(\degr(u))$.
    Again, the degree function is designed so that it favors the selection of high degree nodes. For
    instance, one could use $d_f(\degr(u)) = \degr(u)^2$.
\end{itemize}

We now give the pseudo code of \extractStar{} for the highest degree variant.\footnote{Note that for
clarity, we removed some bookkeeping code in all listings, mainly the part related to maintaining
mapping between nodes at different level of contraction. However, the full python implementation
is available at \url{https://github.com/daureg/magnet/blob/master/veverica/new_galaxy.py\#L27}.}
We assume that $G$ is the adjacency list of the graph, so that $G[u]$ is the set of neighbors of
$u$, \ie{} $G[u] \equiv \mathcal{N}(u)$. The other piece of notation is $\textsc{Star}$, which
simply create a star given a center and a list of peripheral nodes.  \vspace{-\baselineskip}

\begin{center}
  \rule{\textwidth}{.3pt}
  \begin{algorithmic}[1]
    \Function{\extractStar{}}{$G_t=(V_t,E_t)$}
      \State Let $Q$ be a max-priority queue. The key of element $x$ is $Q[x]$
      \Let{$stars$}{[]}
      \Let{$remaining$}{$\emptyset$}
      \ForAll{node $u$ in $V_t$}
        % \State $\textsc{Insert}\\left(Q,\,u\right)$ \Comment with the key $\degr(u)$
        \State \Call{Insert}{$Q,\,u$} \Comment with the key $\degr(u)$
        \Let{$remaining$}{$remaining \bigcup \left\{u\right\}$}
      \EndFor
      \While{$Q$ is not empty}
        \Let{$c_i$}{\Call{Extract-Max}{$Q$}}
        \If{$c_i$ not in $remaining$}
          \State \textbf{continue} \Comment{$c_i$ is part of an existing star so there is
          nothing to do}
        \EndIf
        \Let{$periphery$}{$G[c_i] \bigcap remaining$}
        \Let{$stars$}{$stars \bigcap $\Call{Star}{$c_i,\, periphery$}}
        \Let{$remaining$}{$remaining \setminus\left\{c_i\right\} \setminus periphery$}
        \For{$p$ in $periphery$}
          \For{$h$ in $G[p] \bigcap remaining$}
            \State \Call{Decrease-Key}{$Q,\, h,\, Q[h]-1$}
          \EndFor
        \EndFor
      \EndWhile
      \State \textbf{return} $stars$
    \EndFunction
  \end{algorithmic}
  \rule{\textwidth}{.3pt}
\end{center}

\extractStar{} terminates because at each iteration of the while loop line 8, we remove one node
from $Q$ and never add any. Let us analyze the complexity when $|V_t|=n$ and $|E_t|=m$. We first
build a priority queue of all the nodes sorted by their degree (line 5--7), which takes $O(n)$ time.
Then, at each iteration of the inner loop, we find the center of the next star by extracting the
maximum of the queue (line 9), we build the corresponding star (line 12--14) and we decrease the
priority (\ie the degree) of all nodes adjacent to the new star (line 15--17).  Since both
operations require constant time when using a Strict Fibonacci Heap~\autocite{FibonacciHeaps12} and
there are $n$ iterations of that loop, a coarse approximation of the runtime of \extractStar{} is
$O(n^2)$. However, observe that there can be at most $m$ decrease operations (since after that, all
nodes still in the queue have an effective degree of $0$, meaning that $periphery$ will the empty
set and lines 13--17 will run in constant time), reducing the complexity to $O(m+n)$.

The other two variants are more time consuming because they require additional bookkeeping. Their
randomization make them useful in an adversarial context but it also renders their analysis more
challenging, not necessarily for the runtime of \extractStar{} but mostly for the tree construction.
Therefore, we only briefly describe the implementations here.\footnote{Although they are available
online at 
\nolinkurl{https://github.com/daureg/magnet/blob/master/veverica/}%
\{\href{https://github.com/daureg/magnet/blob/master/veverica/ThresholdSampler.py}%
{ThresholdSampler.py}, \href{https://github.com/daureg/magnet/blob/master/veverica/NodeSampler.py}%
{NodeSampler.py}\}.} For the threshold function, we
maintain two queues, $high$ and $low$, containing nodes whose degree is respectively above and below
the current threshold. We select a node \uar{} in $high$, remove the corresponding star from $G_t$,
recompute the new threshold and if necessary, move nodes which fell under the threshold from $high$
to $low$ and those who climb above the threshold from $low$ to $high$. For the degree function, we
can draw any node as center proportionally to its weight (where the weight of node $u$ is defined as
$d_f\left(\degr(u)\right)$), but we cannot use the standard method of computing the cumulative sum
of weights since each iteration change some of them. Therefore, we construct a binary tree whose
leaves are the nodes of $V_t$ and where each tree nodes maintain the sum of weights in its left and
right subtrees. To sample, we draw a random number between $0$ and the total weight of the tree.
When degrees are updated (or graph node removed), we update the weights along a path from the
corresponding leaves to the root of the tree.

\medskip

The second routine, \collapseStar{} takes as input the result of \extractStar{}, along with $E_t$
and an optional $\emph{eccentricity}$ array we will describe soon. It builds a new graph $G_{t+1}$
where each star becomes a node and there is a link between two nodes $s_1$ and $s_2$ if the nodes
making up $s_1$ and $s_2$ are connected in $E_t$. For that, we first shuffle $E_t$ and go through
it. When we find an edge whose endpoints belong to two different stars not yet connected, we use
that edge to connect these two stars. This trivially takes $O(m)$ times.

A variant instead keeps track of all edges connecting each pair of stars to choose one that will
best contribute to our low stretch objective. Namely, when connecting two stars, we would prefer to
join their centers rather than two peripheral points. For that we maintain an eccentricity count for
all of the nodes of the original $G_0$, which is incremented by $1$ each time a node is chosen to be
on the periphery of a star.\Todo{link that to the walk through example.}
For each pair of stars, we thus choose the edge across them with minimal sum of its endpoints'
eccentricity. This requires another pass over the edges, preserving the $O(m)$ runtime.

\begin{figure}[htbp]
  \centering
  \includegraphics[width=0.78\linewidth]{tikz/gtx_eccentricity_tikz.pdf}
  \caption[The hierarchical structure of stars created by \gtx{}]{%
    The execution of the \gtx{} algorithm. The original graph is made of the solid edges
    connecting the nodes labeled by  their index. Edges forming the final spanning tree are in black
    while the others are in gray. The four shades of gray, from white to dark gray
    denote increasing node eccentricity (as computed at the end of the algorithm). The \ith{} star
    created during the \jth{} iteration of the algorithm is denoted $S_i^j$. Refer to the main text
    for a complete walk through.}
  \label{fig:gtx_eccentricity}
\end{figure}

\begin{figure}[bthp]
  \centering
  \begin{subfigure}[b]{0.47\textwidth}
    \centering
    \includegraphics[height=5cm]{tikz/gtx_run_level1_tikz}
    \caption{Resulting graph after the first iteration}\label{fig:gtx_run1}
  \end{subfigure}~
  \begin{subfigure}[b]{0.47\textwidth}
    \centering
    \includegraphics[height=2.2cm]{tikz/gtx_run_level2_tikz}
    \caption{Resulting graph after the second iteration}\label{fig:gtx_run2}
    \vspace{\baselineskip}
    \includegraphics[height=2.2cm]{tikz/gtx_run_level3_tikz}
    \caption{Resulting graph after the third iteration}\label{fig:gtx_run3}
  \end{subfigure}~
  \caption{The other iterations of \gtx{}}\label{fig:gtx_run}
\end{figure}

\medskip

We illustrate the operation of the \gtx{} algorithm on small (and somewhat contrived) example.
Let us start with the initial graph $G_0$ depicted in \autoref{fig:gtx_eccentricity} and initialize
the eccentricity of all nodes to $0$. When running \extractStar{}, we see that the maximum degree is
$4$, achieved at nodes $\{1, 6, 11, 16, 21, 26, 31, 36, 41\}$. For the sake of simplicity, assume
nodes are picked according to their index. First, node $1$ is forms the star
$\textcolor{DodgerBlue}{S_1^1}$ with peripheral nodes $2$, $3$, $4$ and $5$. This increments the
eccentricity of those peripheral nodes by $1$. Then node $6$ forms its star
$\textcolor{DodgerBlue}{S_2^1}$ with $7$, $8$, $9$ and $10$. The process continues until node $41$ is
chosen to be the center of star $\textcolor{DodgerBlue}{S_9^1}$, at which point the max-priority
queue has been exhausted and \extractStar{} finishes.

We then call \collapseStar{}, with the eccentricity reducing variant. This will connect all possible
pairs of star. For instance, the edge between nodes $19$ and $29$ leads to the edge
between $\textcolor{DodgerBlue}{S_4^1}$ and $\textcolor{DodgerBlue}{S_6^1}$. This is actually the
only possible edge between $\textcolor{DodgerBlue}{S_4^1}$ and $\textcolor{DodgerBlue}{S_6^1}$.
Consider on the other hand the case of edges $(2, 6)$ and $(2, 9)$. They both connect
$\textcolor{DodgerBlue}{S_1^1}$ and $\textcolor{DodgerBlue}{S_2^1}$. Yet at this point of the algorithm,
the eccentricity of node $2$ is $1$, the eccentricity of node $6$ is $0$ and the eccentricity of node
$9$ is $1$. The edge $(2, 6)$ has therefore the smallest total eccentricity and is chosen to connect
$\textcolor{DodgerBlue}{S_1^1}$ and $\textcolor{DodgerBlue}{S_2^1}$. The full result of the
\collapseStar{} procedure can be seen on \autoref{fig:gtx_run1}.


\subsection{State of the art}
\label{sub:gtx_state_of_the_art}

\iffalse
\url{http://www.siam.org/meetings/da17/schedule.html} SODA 13B \url{http://dl.acm.org/citation.cfm?id=3039686}
for instance the Elkin paper \enquote{Our centralized randomized algorithm computes (with
probability close to 1), a $(2k - 1)$-spanner with $n \cdot (1 + O(\frac{\log k}{n}))$ edges in
$O(|E|)$ time, whenever $k = \Omega(\log n)$. Note that when $k = \omega(\log n)$, the number of
edges is $n(1+o(1))$, i.e., in this range the algorithm computes an ultra-sparse spanner in $O(|E|)$
time.} For instance, if $k=5\log n$, we get a $10\log n$-spanner with $n\left(1+O\left(\frac{\log\log
n}{n}\right)\right)$ edges in $O(|E|)$ time.

While they have many applications [see first paragraph of \url{https://arxiv.org/pdf/1401.2454.pdf},
which was later merged in a STOC'14 paper] (a major one being solving linear system), in some
practical situations their advantages are less clear [from
\url{https://link.springer.com/chapter/10.1007/978-3-319-20086-6_16}\enquote{for reasonable inputs
the constant factors make the solver much slower than methods with higher asymptotic complexity.
One other aspect predicted by theory is confirmed by our findings: Spanning trees with lower
stretch indeed reduce the solver's running time. Yet, simple spanning tree algorithms perform
better in practice than those with a guaranteed low stretch.} this is improved by
\url{https://link.springer.com/chapter/10.1007%2F978-3-319-20086-6_17} although they seem to work
	mostly with the Laplacian of the tree ]
\fi

\subsection{Empirical evaluation}
\label{sub:gtx_empirical_evaluation}

% In this \nameref{sub:gtx_empirical_evaluation}, we provide empirical evidences of the properties of
\gtx{} over several classes of graph, and compare it with a \bfs{} baseline.\marginpars{If time
allows, it would be interesting to implement some methods of \vref{sub:gtx_state_of_the_art} and add
them to the comparison}\todo*{implement more low stretch methods} Namely, we consider three kinds of
graph topology (with both synthetic and real world instances that carry signs on their edges) and
evaluate $(i)$ what average stretch is reached by various trees and $(ii)$ how accurate is the sign
prediction.

\subsubsection{Graph topology}

The three kinds of topology we consider are:
\begin{description}
	\item[\grid{}] which are 2D lattices, where each node has four neighbors except on the boundary.
		The synthetic ones are square, while the \enquote{real world} ones represents the four neighbors
		pixel connectivity of the pictures showed in \autoref{fig:gtx_xp_bwpics}.
	\item[\lpa{}] which are built synthetically according to the model of \textcite{Barabasi1999}.
		While this does not follow the more rigorous specification of \textcite{PAmodel04}, informally,
		we start with a line graph of $m$ nodes and add node one by one until the graph consists of $n$
		nodes. Each time a new node is added, it is connected to $m$ of the existing nodes with a
		probability proportional to their degree. Here we choose $m=3.13$, that is when adding a new
		node, we pick $3$ or $4$ existing neighbors such the initial expected number of neighbors for
		each new nodes is $3.13$. Such graphs are quite sparse and have short diameter, thus providing a
		crude but reasonable approximation of online social networks. Therefore, the real world
		instances of the \lpa{} model are \wik{}, \sla{} and \epi{}\marginpars{as used in the first
		chapter}\todo*{add a ref to first chapter} along with \gplus{}. The last one is constructed from
		ego networks of \gplus{}\footnote{Available at
		\url{http://snap.stanford.edu/data/egonets-Gplus.html}} by keeping the largest connected
		component of users whose gender is known. Basic statistics of those real \lpa{} graphs are
		presented in (\autoref{tab:gtx_xp_dataset}). 
	\item[\triangle{}] which consists of a Delaunay triangulation of random 2D points\footnote{As
		implemented by the \textsf{graph-tool} library (\url{https://graph-tool.skewed.de})}.
\end{description}

\begin{table}[htpb]
	\centering
	\caption{Dataset description }\label{tab:gtx_xp_dataset}
	\begin{tabular}{lrrcc}
		\toprule
             & $|V|$  & $|E|$    & fraction of $+$ edges & $\frac{2|E|}{|V|\cdot(|V|-1)}$ \\
		\midrule
		\wik{}   & \np{7065}   & \np{99936}    & 78.5\%                & $4.00\cdot 10^{3}$             \\
		\gplus{} & \np{74917}  & \np{10130461} & 67.6\%                & $3.61\cdot 10^{3}$             \\
		\sla{}   & \np{82052}  & \np{498527}   & 76.4\%                & $1.48\cdot 10^{4}$             \\
		\epi{}   & \np{119070} & \np{701569}   & 83.2\%                & $9.90\cdot 10^{5}$             \\
		\bottomrule
	\end{tabular}
\end{table}

\begin{figure}[t]
	\centering
	\begin{subfigure}[b]{0.32\textwidth}
		\includegraphics[width=\textwidth]{gtx_exp/zmonastery}
	\end{subfigure}~
	\begin{subfigure}[b]{0.32\textwidth}
		\includegraphics[width=\textwidth]{gtx_exp/zworld}
	\end{subfigure}~
	\begin{subfigure}[b]{0.32\textwidth}
		\includegraphics[width=\textwidth]{gtx_exp/nips_poster}
	\end{subfigure}

	\begin{subfigure}[b]{0.32\textwidth}
		\includegraphics[width=\textwidth]{gtx_exp/zmonastery_bin}
		\caption{monastery}
	\end{subfigure}~
	\begin{subfigure}[b]{0.32\textwidth}
		\includegraphics[width=\textwidth]{gtx_exp/zworld_bin}
		\caption{world}
	\end{subfigure}~
	\begin{subfigure}[b]{0.32\textwidth}
		\includegraphics[width=\textwidth]{gtx_exp/nips_poster_bin}
		\caption{poster}
	\end{subfigure}

	\begin{subfigure}[b]{0.32\textwidth}
		\includegraphics[width=\textwidth]{gtx_exp/nips_logo}
	\end{subfigure}~
	\begin{subfigure}[b]{0.32\textwidth}
		\includegraphics[width=\textwidth]{gtx_exp/space}
	\end{subfigure}~
	\begin{subfigure}[b]{0.32\textwidth}
		\includegraphics[width=\textwidth]{gtx_exp/waterfall}
	\end{subfigure}

	\begin{subfigure}[b]{0.32\textwidth}
		\includegraphics[width=\textwidth]{gtx_exp/nips_logo_bin}
		\caption{logo}
	\end{subfigure}~
	\begin{subfigure}[b]{0.32\textwidth}
		\includegraphics[width=\textwidth]{gtx_exp/space_bin}
		\caption{space}
	\end{subfigure}~
	\begin{subfigure}[b]{0.32\textwidth}
		\includegraphics[width=\textwidth]{gtx_exp/waterfall_bin}
		\caption{waterfall}
	\end{subfigure}
	\caption{Real world pictures and their binarized version}\label{fig:gtx_xp_bwpics}
\end{figure}

\subsubsection{Stretch}

The first property of Galaxy trees we wish to evaluate is their stretch, which depends only of graph
topology. Namely, let $G$ be a graph over vertex set $V$ with $|V|=n$ and edge set
$E$.\Todo[MOVE]{Stretch definition is likely to happen somewhere earlier} Furthermore, let $T$ be a
spanning tree of $G$ and $\etest{}$ the edges of $G$ not in $T$. Then we define the \emph{average
test edge stretch} as $\frac{1}{|\etest{}|} \sum_{(u,v) \in \etest{}} |\mathrm{path}^T_{u,v}|$,
where $|\mathrm{path}^T_{u,v}|$ is the unique path between $u$ and $v$ in $T$.

As we consider unweighted graphs, we compare \gtx{} with a natural baseline, namely a spanning tree
rooted at the highest degree node and obtained through a breadth first visit of the graph. This
involves randomness in order in which nodes are visited. Likewise in \gtx{}, the choice of the edge
linking two stars is not always unique, meaning that we have to break ties at random.  Therefore,
for each graph, we repeat the tree construction 12 times and present the average result, noting that
the variance (showed as error bar in \autoref{fig:gtx_xp_st}) is small.

On \lpa{} and \triangle{}, we see that both trees exhibits logarithmic stretch, although with a
larger constant for \gtx{}. Note that this is also the case for others low stretch tree methods
\autocite[\S 5.3.1]{papplow}. On \grid{} however, \gtx{} preserves this logarithmic stretch growth
while this is visually no longer the case for \bfs{}.
In that case, we cannot expect a better stretch than $\frac{\log n}{2048}$ according to
\autocite[Theorem 6.6]{LowerBound95}.

\begin{figure}[tbh]
	\centering
	\begin{subfigure}[b]{0.9\textwidth}
		\includegraphics[width=\textwidth]{gtx_exp/gridst}
		\caption{\grid{} }\label{fig:gtx_xp_gridst}
	\end{subfigure}

	\begin{subfigure}[b]{0.9\textwidth}
		\includegraphics[width=\textwidth]{gtx_exp/past}
		\caption{\lpa{} }\label{fig:gtx_xp_past}
	\end{subfigure}

	\begin{subfigure}[b]{0.9\textwidth}
		\includegraphics[width=\textwidth]{gtx_exp/trst}
		\caption{\triangle{} }\label{fig:gtx_xp_trst}
	\end{subfigure}
	\caption{Stretch over graphs of increasing size}\label{fig:gtx_xp_st}
\end{figure}

\subsubsection{Sign prediction}

The second design goal of Galaxy trees is to accurately predict the sign of edges in $\etest{}$.
Except for the three real datasets that already include signs\footnote{We nonetheless perform some
preprocessing in order to make them undirected to remove the small proportion of conflicting edges
(e.g. positive from $u$ to $v$ but negative from $v$ to $u$).}, all the other are constructed,
meaning we have to set sign on their edges in the first place. This is done by partitioning the
nodes into two clusters. For \gplus{} we use node gender, for pictures we use node color (black or
white), and for all others, we propagate labels $0$ and $1$ from randomly selected high degree nodes.
Once each node belongs to one of the two clusters, we set the sign of an edge between two nodes to
be $+$ if they are in the same cluster and $-$ otherwise.  Predicting using path parity will thus
gives perfect result. To test performance in real or adversarial situation, we then add noise, that
is we select a fraction of edges uniformly at random and flip their sign. 

We evaluate the performance of our prediction using the Matthews Correlation\Todo{merge this
definition of MCC with the one in first chapter} Coefficient (MCC)~\autocite{MCC00} \[
	\mathrm{MCC} = \frac{ TP \times TN - FP \times FN } {\sqrt{ (TP + FP) ( TP + FN ) (
			TN + FP ) ( TN + FN ) } } = \pm \sqrt{\frac{\chi^2}{n}}
\]
Since we do not have confidence score, we cannot use AUC. Yet we have to account for the large sign
unbalance and thus cannot rely on accuracy or $F_1$ measure.  Therefore we choose MCC, which
combines all the four numbers of the confusion matrix in a single metric. It ranges from $+1$
(perfect prediction) to $-1$ (inverse prediction) through $0$ (random prediction). As a
demonstration of MCC usefulness, predicting all edges but one to be positive on Slashdot gives
$.764$ accuracy, $.886$ $F_1$ score\marginpars{Actually the $F_1$ score is $.866$ for positive
edges, $0$ for negative ones and $.661$ if we can take an average weighted by class size.} but
$-0.0007$ MCC.

As showed in \autoref{fig:gtx_xp_mcc}, when the noise level is low, \gtx{} performs better than
\bfs{}. As the noise level gets higher, they have similar performance. Note also than in
\autoref{fig:gtx_xp_pasynthmcc}, \gtx{} is less sensible to the size of the graph.

\begin{figure}[tbh]
	\centering
	\begin{subfigure}[b]{0.47\textwidth}
		\includegraphics[width=\textwidth]{gtx_exp/grsynthmcc}
		\caption{Synthetic \grid{} }\label{fig:gtx_xp_grsynthmcc}
	\end{subfigure}~
	\begin{subfigure}[b]{0.47\textwidth}
		\includegraphics[width=\textwidth]{gtx_exp/grrwmcc}
		\caption{Pictures \grid{} }\label{fig:gtx_xp_grrwmcc}
	\end{subfigure}
	\begin{subfigure}[b]{0.47\textwidth}
		\includegraphics[width=\textwidth]{gtx_exp/pasynthmcc}
		\caption{Synthetic \lpa{} }\label{fig:gtx_xp_pasynthmcc}
	\end{subfigure}~
	\begin{subfigure}[b]{0.47\textwidth}
		\includegraphics[width=\textwidth]{gtx_exp/trmcc}
		\caption{\triangle{} }\label{fig:gtx_xp_trmcc}
	\end{subfigure}
	\begin{subfigure}[b]{0.47\textwidth}
		\includegraphics[width=\textwidth]{gtx_exp/parwmcc}
		\caption{Real world network }\label{fig:gtx_xp_parwmcc}
	\end{subfigure}
	\caption{MCC over various graphs}\label{fig:gtx_xp_mcc}
\end{figure}

To further assess the quality of our trees, we plug them in them into a successful heuristic method
to predict edge sign: \asym{}~\autocite{Kunegis2009}. \Todo{It might also be interesting to see if
that would be a good training set for our troll method, although it has to be checked it makes sense
from a running time point of view.} It computes the exponential of the adjacency matrix after it has
been reduce to $z$ dimension. This allows to count the sign of all paths between two pairs of nodes
with decreasing weight depending of their length. To simulate an active learning setting, we reveal
only a subset of edge in $A$. This subset can be: $i)$ the edges forming a \bfs{}, $ii)$ the edges
forming a \gtx{} $iii)$ $|V|-1$ edges chosen uniformly at random.

We set the parameter $z$ equal to $15$ because $i)$ it is one of the best in \cite[Fig.
11]{Kunegis2009}, $ii)$ it performs well on real dataset in \cite[Fig.3]{Cesa-Bianchi2012a}, and
$iii)$ it was good in our initial testing (\texttt{20150401\_wed\_spectral.ipynb}).

As the \asym{} has a $O(n^3)$ complexity and uses quite some memory at prediction time, the larger
graphs used previously are not all included. The conclusion of \autoref{fig:gtx_xp_asym} is that
except on social network, it is better to use spanning trees than random edges. Specifically, \gtx{}
on \grid{} and \bfs{} elsewhere.

\begin{figure}[tbh]
	\centering
	\begin{subfigure}[b]{0.47\textwidth}
		\includegraphics[width=\textwidth]{gtx_exp/grsynthasym}
		\caption{Synthetic \grid{} \label{fig:gtx_xp_grsynthasym}}
	\end{subfigure}~
	\begin{subfigure}[b]{0.47\textwidth}
		\includegraphics[width=\textwidth]{gtx_exp/grrwasym}
		\caption{\enquote{Real} \grid{} }\label{fig:gtx_xp_grrwasym}
	\end{subfigure}
	\begin{subfigure}[b]{0.47\textwidth}
		\includegraphics[width=\textwidth]{gtx_exp/pasynthasym}
		\caption{Synthetic \lpa{} }\label{fig:gtx_xp_pasynthasym}
	\end{subfigure}~
	\begin{subfigure}[b]{0.47\textwidth}
		\includegraphics[width=\textwidth]{gtx_exp/trasym}
		\caption{\triangle{} }\label{fig:gtx_xp_trasym}
	\end{subfigure}
	\begin{subfigure}[b]{0.47\textwidth}
		\includegraphics[width=\textwidth]{gtx_exp/parwasym}
		\caption{Real world network }\label{fig:gtx_xp_parwasym}
	\end{subfigure}
	\caption{\asym{} over various graphs}\label{fig:gtx_xp_asym}
\end{figure}

Finally\marginpars{Actually I never did it because \shz{} wasn't implemented at the time, so now is
a good occasion}\todo*{Run shazoo on galaxy tree} we also compare \gtx{} with \bfs{} and \rst{} on
the task of nodes prediction using \shz{} algorithm~\autocite{Vitale2012}.


\section{Conclusion}


\begingroup
\small
\setstretch{0.9}
\todos

\setlength\bibitemsep{2pt}
\printbibliography
\endgroup

% \bibliography{/home/orphee/data/projects/biblio/library.bib,more.bib}
% \bibliographystyle{plainnat}
\end{document}
