\documentclass[nobib,a4paper,11pt,oneside,onecolumn,openany,notitlepage,final,svgnames]{tufte-handout}
\usepackage[utf8]{inputenc}
\usepackage[T1]{fontenc}
\usepackage{csquotes}
\usepackage{babel}
\usepackage[heightrounded]{geometry}
\usepackage{setspace}
\usepackage{microtype}
\usepackage[citestyle=alphabetic, bibstyle=ieee-alphabetic, isbn=false, maxnames=1, minnames=1,
sorting=nyvt, backref=true, backend=biber, defernumbers=true]{biblatex}
\AtEveryBibitem{
   \clearfield{arxivId}
   % \clearfield{booktitle}
   % \clearfield{doi}
   \clearfield{eprint}
   \clearfield{eventdate}
   \clearfield{isbn}
   \clearfield{issn}
   % \clearfield{journaltitle}
   \clearfield{month}
   % \clearfield{number}
   % \clearfield{pages}
   \clearfield{series}
   \clearfield{url}
   \clearfield{urldate}
   \clearfield{venue}
   % \clearfield{volume}
   \clearlist{location} % alias to field 'address'
   \clearlist{publisher}
   \clearname{editor}
}
\addbibresource{/home/orphee/data/projects/biblio/library.bib}
\addbibresource{more.bib}
\addbibresource{gtx.bib}
\usepackage{xcolor}
\usepackage{amsmath}
\usepackage{amsthm}
\usepackage{mathtools}
\usepackage{sidenotes}
\renewcommand{\footnote}[1]{\sidenote{\footnotesize #1}}
\usepackage{enumitem}
\usepackage{tabulary}
\usepackage{multirow}
\usepackage[marginpar]{todo}
\usepackage[np,autolanguage]{numprint}
\usepackage{listings}
\lstset{frame=none, language=Python, tabsize=4,}
% \usepackage{inconsolata}
\usepackage{nicefrac}
\usepackage{tikz}
\usetikzlibrary{arrows,automata,calc}
\usetikzlibrary{arrows.meta}
\usetikzlibrary{decorations.pathreplacing}
\usetikzlibrary{backgrounds}
\tikzset{%
  show curve controls/.style={
    postaction={
      decoration={
        show path construction,
        curveto code={
          \draw [blue] 
            (\tikzinputsegmentfirst) -- (\tikzinputsegmentsupporta)
            (\tikzinputsegmentlast) -- (\tikzinputsegmentsupportb);
          \fill [red, opacity=0.5] 
            (\tikzinputsegmentsupporta) circle [radius=.25ex]
            (\tikzinputsegmentsupportb) circle [radius=.25ex];
        }
      },
      decorate
}}}
\tikzstyle{vertex}=[draw,circle,black,inner sep=2pt]
\tikzstyle{edge}=[line width=1.3pt,color=Black]
\tikzstyle{rare}=[fill=black,text=white]
\tikzstyle{medium}=[fill=black!15!white]


\usepackage{algorithm}
\usepackage[noend]{algpseudocode}
\newcommand*\Let[2]{\State #1 $\gets$ #2}
\def\algorithmautorefname{Algorithm}

\usepackage{varioref}
\usepackage{hyperref}
\hypersetup{%
    % draft,    % = no hyperlinking at all (useful in b/w printouts)
    colorlinks=true, linktocpage=true, pdfstartpage=3, pdfstartview=FitV,%
    % uncomment the following line if you want to have black links (e.g., for printing)
    %colorlinks=false, linktocpage=false, pdfborder={0 0 0}, pdfstartpage=3, pdfstartview=FitV,%
    breaklinks=true, pdfpagemode=UseNone, pageanchor=true, pdfpagemode=UseOutlines,%
    plainpages=false, bookmarksnumbered, bookmarksopen=true, bookmarksopenlevel=1,%
    hypertexnames=true, pdfhighlight=/O,%nesting=true,%frenchlinks,%
    urlcolor=Chocolate, linkcolor=RoyalBlue, citecolor=LimeGreen, %pagecolor=RoyalBlue,%
}
\newcommand{\marginpars}[1]{\marginpar{\small#1}}
\usepackage{caption}
\usepackage[margin=0pt,font+=small,labelformat=parens,labelsep=space,
skip=6pt,list=false,hypcap=false]{subcaption}
\captionsetup{compatibility=false}
\usepackage{graphicx}
\usepackage{booktabs}
\graphicspath{{./assets/}}
% \usepackage[capitalize,noabbrev]{cleveref}
\usepackage[]{extdash}

\newcommand{\asym}{\emph{A sym exp}}
\newcommand{\bfs}{\textsc{Breadth First Tree}}
\newcommand{\ccPivot}{\textsc{CC-Pivot}}
\newcommand{\epi}{\textsc{Epinion}}
\newcommand{\etest}{\ensuremath{E_{\mathrm{test}}}}
\newcommand{\etrain}{\ensuremath{E_{\mathrm{train}}}}
\newcommand{\gplus}{\textsc{Google+}}
\newcommand{\grid}{\textsc{Grid}}
\newcommand{\gtx}{\textsc{Galaxy Tree}}
\newcommand{\extractStar}{\textsc{Extract-Stars}}
\newcommand{\collapseStar}{\textsc{Collapse-Stars}}
\newcommand{\lpa}{\textsc{Preferential Attachment}}
\newcommand{\pcc}{\textsc{Correlation Clustering}}
\newcommand{\rst}{\textsc{Random Spanning Tree}}
% \newcommand{\sgt}{\textsc{Galaxy Tree}}
\newcommand{\shz}{\textsc{Shazoo}}
\newcommand{\sla}{\textsc{Slashdot}}
\newcommand{\wik}{\textsc{Wikipedia}}
\renewcommand{\triangle}{\textsc{Triangle}}

\newcommand{\ith}{\ensuremath{i^{\mathrm{th}}}}
\newcommand{\jth}{\ensuremath{j^{\mathrm{th}}}}
\newcommand{\tth}{\ensuremath{t^{\mathrm{th}}}}
\newcommand{\uar}{uniformly at random}
\newcommand{\ie}{i.e.\@}
\DeclareMathOperator{\degr}{deg}
\DeclareMathOperator*{\argmin}{arg\,min}

\newcommand{\starone}[1]{\ensuremath{\textcolor{DodgerBlue}{S_{#1}^1}}}
\newcommand{\startwo}[1]{\ensuremath{\textcolor{Orange}{S_{#1}^2}}}
\newcommand{\starthree}[1]{\ensuremath{\textcolor{Green}{S_{#1}^3}}}

\newcommand{\euv}{\ensuremath{u\rightarrow v}}
\newcommand{\evu}{\ensuremath{v\rightarrow u}}
\newcommand{\yuv}{\ensuremath{y_{u, v}}}
% \newcommand{\pathtuv}{\ensuremath{\text{\textsc{Path}}^T_{u,v}}}
\DeclareMathOperator{\pathm}{path}
\newcommand{\pathtuv}{\ensuremath{\pathm^T(u, v)}}

% \usepackage{charter}
% \usepackage{fullpage}

\title{Edge sign prediction in general graphs and connection with Correlation Clustering}

\begin{document}
\maketitle

\section{Limitations of the troll method}
\label{sec:limitations_of_the_troll_method}

The method presented in the previous chapter hinges crucially upon our sign generative model. Yet
one can imagine contexts where this model is not applicable, especially when nodes do not represent
human beings. One way to alleviate this issue was introduced with the online algorithm, where this
time, signs are generated by an arbitrary adversary. However, in that case, we are still facing two
limitations\marginpar{limitations is not the right word, maybe \emph{strong constraints}…}
\begin{enumerate}
	\item Our bias remains that the labeling is regular (recall this means informally than the all
		outgoing signs from a given node tend to be the same, and likewise for the incoming signs),
		since irregularities are the cost payed by the adversary to make our algorithm mispredict. While
		this bias is well suited to social networks, other applications may require other
		bias\marginpar{namely the \pcc{} bias we'll introduce next}.
	\item In the online setting, we evaluate our performance by the regret\marginpar{ref to equation},
		whereas in general we are interested in more classical measure of predictive accuracy.
\end{enumerate}

Take for instance a biological network.

According to \href{https://web.stanford.edu/class/cs224w/slides/handout-bionets.pdf}%
{this description of various biological networks}, \emph{gene regulatory network} are directed and
contains activation and inhibition links, as \href{https://en.wikipedia.org/wiki/Gene_regulatory_network#Overview}%
{showed on Wikipedia}. There are some online databases such as
\href{http://regulondb.ccg.unam.mx/menu/download/datasets/index.jsp}%
{RegulonDB} or \href{http://www.pathwaycommons.org/pcviz/}{Pathway Commons} (this last one provide
network in the BioPAX format, which can be visualized by
\href{http://www.cytoscape.org/}{CytoScape}).
Another source is the \href{https://www.ncbi.nlm.nih.gov/pmc/articles/PMC2708159/table/T1}{Table 1}
of \cite{BioSigned09}.

other domain? coref, images (is it directed?), entity resolution…\marginpar{Actually it's difficult
to find other kind of graph because all those are constructed (like image, coreference,
deduplication) are inherently symmetric}

A much more serious limitation is that our sign model only apply to directed graph. Consider another
model, where each node $i$ is endowed with an integer $c_i \in \{1, \ldots, k\}$ that define its
cluster and let the sign of the undirected edge $i,j$ be $+1$ if $c_i = c_j$ and $-1$ otherwise.
This corresponds naturally to situation modeled by the \pcc{} problem.


For the 2 clusters case ($k=2$), characterization proven already in 1936 by \textcite{Konig36}, as noted in
\cite{Zaslavsky2012} (commenting on \cite{harary1953}: \enquote{Although Thm. 3 was anticipated by
\textcite[Theorem X.11]{Konig36}  without the terminology of signs, here is the
first recognition of the crucial fact that labelling edges by elements of a
group—specifically, the sign group—can lead to a general theory.})
and \cite{Huffner2010} (\enquote{\textcite{Konig36} proved the following characterization of balanced graphs. For a
	graph $G = (V , E)$, the following are equivalent:
	\begin{enumerate}
		\item V can be partitioned into two sets $V_1$ and $V_2$ called sides such that there is no negative edge $\{v, w\} \in E$ with both $v, w \in V_1$ or both $v, w \in V_2$ and no negative edge \{v, w\} with $v \in V_1$ and $w \in V_2$ .
		\item $V$ can be colored with two colors such that for all $\{v, w\} \in E^-$, the vertices $v$ and $w$ have different colors, and for all $\{v, w\} \in E^+$, the vertices $v$ and $w$ have the same color. The color classes correspond to the sides.
		\item $G$ does not contain cycles with an odd number of negative edges.
	\end{enumerate}
	Using the characterization by a coloring, it is easy to see that balance of a signed
graph can be checked in linear time by depth-first search.})

\begingroup
\setstretch{0.9}
\setlength\bibitemsep{2pt}
\printbibliography
\endgroup

% \bibliography{/home/orphee/data/projects/biblio/library.bib,more.bib}
% \bibliographystyle{plainnat}
\end{document}
