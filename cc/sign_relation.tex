Now that we are familiar with the \pcc{} problem and its numerous applications, let us return to the
original problem of this chapter, predicting the edge sign of a signed graph. In order to do so, we
will first focus on the case where a signed graph has no disagreement (that is the minimum of
\eqref{eq:mind} is $0$) and is unweighted, as it highlights more clearly the connection with the
balance theory. We focus here on balance theory, for it is both compelling and well established.
Indeed, as illustrated by the account of \textcite{CSBalanceSurvey15}, it has been used extensively
in the last half century to study signed networks\marginpars{It was even used to study fairy
tales~\autocite{fairyTales80}.} and their dynamics.
However, it has shortcomings, the main one being its inability to deal with directed graphs.
Therefore alternative theories governing signs formation have been proposed, as surveyed
in~\autocite{Yap2015}.

In his seminal work on interpersonal relations,
\citeauthor*{Heider46}~\autocites{Heider46}{HeiderBook58} posits through
psychological and sociological arguments that in order to reduce their cognitive dissonance, three
people always interact in a way that preserve social balance. These social balance requirements can be
succinctly summarized by four statements~\autocite{HeiderBook58}:

\begin{enumerate}
	\item \textcolor{DodgerBlue}{my friend's} \textcolor{Orange}{friend} is
		\textcolor{Olive}{my friend}\hspace{1em}%
		\tikz[baseline=5]{
			\node[draw,circle,inner sep=2pt] (me) at (0,0) {};
			\node[draw,circle,inner sep=2pt] (mine) at (.5,.5) {};
			\node[draw,circle,inner sep=2pt] (alter) at (1,0) {};
			\draw[] (me) -- node[left,DodgerBlue,pos=.9] {$+$} (mine) ;
			\draw[] (mine) -- node[right,Orange,pos=.1] {$+$} (alter) ;
			\draw[white] (me) -- node[text=Olive] {$+$} (alter) ;
		}
	\item \textcolor{DodgerBlue}{my friend's} \textcolor{Orange}{enemy} is
		\textcolor{Olive}{my enemy}\hspace{1em}%
		\tikz[baseline=5]{
			\node[draw,circle,inner sep=2pt] (me) at (0,0) {};
			\node[draw,circle,inner sep=2pt] (mine) at (.5,.5) {};
			\node[draw,circle,inner sep=2pt] (alter) at (1,0) {};
			\draw[] (me) -- node[left,DodgerBlue,pos=.9] {$+$} (mine) ;
			\draw[] (mine) -- node[right,Orange,pos=.1] {$-$} (alter) ;
			\draw[white] (me) -- node[text=Olive] {$-$} (alter) ;
		}
	\item \textcolor{DodgerBlue}{my enemy's} \textcolor{Orange}{friend} is
		\textcolor{Olive}{my enemy}\hspace{1em}%
		\tikz[baseline=5]{
			\node[draw,circle,inner sep=2pt] (me) at (0,0) {};
			\node[draw,circle,inner sep=2pt] (mine) at (.5,.5) {};
			\node[draw,circle,inner sep=2pt] (alter) at (1,0) {};
			\draw[] (me) -- node[left,DodgerBlue,pos=.9] {$-$} (mine) ;
			\draw[] (mine) -- node[right,Orange,pos=.1] {$+$} (alter) ;
			\draw[white] (me) -- node[text=Olive] {$-$} (alter) ;
		}
	\item \textcolor{DodgerBlue}{my enemy's} \textcolor{Orange}{enemy} is
		\textcolor{Olive}{my friend}\hspace{1em}%
		\tikz[baseline=5]{
			\node[draw,circle,inner sep=2pt] (me) at (0,0) {};
			\node[draw,circle,inner sep=2pt] (mine) at (.5,.5) {};
			\node[draw,circle,inner sep=2pt] (alter) at (1,0) {};
			\draw[] (me) -- node[left,DodgerBlue,pos=.9] {$-$} (mine) ;
			\draw[] (mine) -- node[right,Orange,pos=.1] {$-$} (alter) ;
			\draw[white] (me) -- node[text=Olive] {$+$} (alter) ;
		}
\end{enumerate}

This can be readily translated into graph properties. Given a path of length two over three nodes,
note that the first part of each statement (in blue and orange) defines the four possible sign
assignments of such a path. The last part of the statement (in green) then prescribes which sign
should close the triangle to respect the social balance. Such triangles are called strongly balanced.
\begin{definition}[strongly balanced triangle]
	A triangle is strongly balanced if it has zero or two negative signs. Otherwise it is unbalanced.
\end{definition}

We will use triangles as the building block of our definition of strongly balanced graphs.
\begin{definition}[strongly balanced complete graphs]
	A complete signed graph is strongly balanced if all its triangles are strongly balanced.
\end{definition}

It is interesting to note that this local property of triangles gives raise to a global structure.
Namely a strongly balanced complete graph can be split in two (possibly empty) clusters $L$ and $R$
such that all edges within $L$ and $R$ are positive, while all edges between $L$ and $R$ are
negative. Indeed let us pick an arbitrary node $u$ and let $L=\neip(u)\setminus \{u\}$ and $R=V\setminus L$. Now
let $v,w\in L\setminus \{u\}$ and $x,y\in R$. From \autoref{fig:cc_twostrongcc} and the fact that
every triangle is balanced, we can see that $(v,w)$ and $(x,y)$ are positive edges, while $(w,y)$ is
a negative edge.
\begin{marginfigure}
  \centering
  \includegraphics[width=28mm]{assets/raw/2cc.pdf}
  \caption{A two-clustering a complete strongly balanced graph}
  \label{fig:cc_twostrongcc}
\end{marginfigure}
As we will see later, these two facts are actually equivalent.

Those two clusters represent a natural partition with no disagreements in the sense of \pcc{}. However,
the existence of two such clusters is only a sufficient but not necessary condition for having no
disagreement. As hinted in \autoref{sub:need_for_a_directed_graph}, when the nodes are partitioned
into $k\geq 2$ clusters, letting edges within clusters to be positive and edges across clusters to
be negative will by construction results in zero disagreement. Furthermore, it was noted by
sociologists that among triangles with odd number of negative edges, one is more stable and common
than that the other. Therefore, \textcite{davis1967clustering} relax the strong balance into the
weak balance by considering triangles with three negative edges to be balanced, as illustrated on
\autoref{fig:cc_balance}. Formally, we can modify our two previous definitions to arrive at weak
balance:
\begin{definition}[Weak balance]
	A triangle is weakly balanced if it has zero, two or three negative signs. Otherwise it is
	unbalanced. \\
	A complete signed graph is weakly balanced if all its triangles are weakly balanced.
\end{definition}

\begin{figure}[htpb]
	\centering
	\includegraphics[width=0.8\linewidth]{assets/raw/triangle_smaller.pdf}
	\caption{The four possible undirected triads, as classified by the two structural balance theories
	introduced in the main text} \label{fig:cc_balance}
\end{figure}

In that case as well, the local property of triangles has structural implication for the whole graph.
Namely, we can partition a complete weakly balanced graph into $k$ clusters of positively connected
nodes that are negatively tied with the rest of the graph. Consider the same construction as in
\autoref{fig:cc_twostrongcc}, putting a node $u$ and its positive neighborhood in one cluster $L$ and
the rest of the nodes in $R$. For $v,w\in L\setminus \{u\}$ and $x,y\in R$, we can still conclude
that $(v,w)$ is positive and $(v,y)$ is negative. On the other hand, this time weak balance allows
$(x,y)$ to be either positive or negative. We can split $R$ further into the positive neighborhood of
$x$ and the rest, and keep doing so until we form the $k$ clusters. We will make this argument more
formal in the following.

We gave the definitions of balance in the context of complete graphs in order to build intuition
about the partition consequences and because it easier to reason on triangles. Furthermore, if we are
given a set of objects and a similarity function, we can indeed build a complete graph. However,
they are situations were it is not realistic to make the completeness assumption, especially for
social networks that are typically very sparse. We therefore extend the concept of balance to
general graphs in the following way:
\begin{definition}[balanced general graphs]
	A general graph is strongly (respectively weakly) balanced if there is a sign assignment of all its
	missing edges such that the resulting complete signed graph is strongly (respectively weakly)
	balanced.
\end{definition}
To similarly extend the characterization of signed graphs by consistent clustering, we need to
consider longer cycles than triangles.
\begin{definition}[balanced cycles]
	A cycle is strongly (respectively weakly) unbalanced if it has an odd number of negative edges
	(respectively exactly one negative edge). Otherwise it is strongly (respectively weakly) balanced.
\end{definition}

Being strongly balanced is equivalent to having two consistent clusters, as proved in the seminal
theorem of \textcite[Theorem 3]{harary1953}:
\begin{theorem}[Structural Theorem]
  A graph $G=(V,E)$ is strongly balanced if and only if $V$ can be divided in two disjoint clusters
  such that all edges within clusters are positive and all edges between clusters are negative.
\end{theorem}

According to \textcite{Huffner2010}, a similar theorem was proved earlier by \textcite[Theorem
X.11]{Konig36}, although \textcite{Zaslavsky2012} notes that it was stated \enquote{without the
terminology of signs, while \autocite{harary1953} has the first recognition of the crucial fact that
labelling edges by elements of a group---specifically, the sign group---can lead to a general
theory.}

A useful characterization, whose proof can be found in~\autocite[page 122]{BookKleinberg2010}, is the
following:
\begin{theorem}
  A graph $G=(V,E)$ is strongly balanced if and only if all its cycles are strongly balanced.
\end{theorem}

The same results also hold for weak balance.
\begin{theorem}
  A graph $G=(V,E)$ is weakly balanced if and only if $V$ can be divided in $k$ disjoint clusters
  such that all edges within clusters are positive and all edges between clusters are negative.
\end{theorem}
\begin{proof}
  Let first assume that $G$ is weakly balanced. According to the definition, we can choose sign for
  the missing edges such that it becomes a complete graph with all its triangles weakly balanced. As
  mentioned earlier, we can then pick a node $u_1$ and let $C_1 = \neip(u) \cup \{u\}$. All
  nodes in $C_1$ are connected positively with each other and negatively with nodes in $V\setminus
  C_1$. Note also that $|V\setminus C_1| < |V|$ and therefore we can repeat this
  procedure until $V$ is exhausted, at which point we have obtained our $k$ clusters verifying the
  theorem's statement conditions.

  Conversely, assume we have such $k$ clusters. We can complete the graph $G$ by letting the edge
  $(u,v)$ be positive if $u$ and $v$ are in the same cluster and negative otherwise. Let us the pick
  three arbitrary nodes $u,v,w\in V$. There are three cases: they are either all in the same
  cluster, all in different clusters or two of them are in a first cluster and the third node is in
  a second cluster. In every case, one can check they form a weakly balanced triangle and therefore
  $G$ is weakly balanced.
\end{proof}
\begin{theorem}[Theorem 1 of \autocite{davis1967clustering}]
  A graph $G=(V,E)$ is weakly balanced if and only if all its cycles are weakly balanced.
\end{theorem}

In real data though, we do not expect neither strong nor weak balance to hold, for they are fairly
demanding model. Indeed, three of the real networks considered in \autoref{chap:troll} have been
repeatedly shown to be unbalanced, although the extent of this unbalance depends of the importance
given to longer cycles compared with triads~\autocites{Facchetti2011isingmodel}{measureUnbalance14}%
{measureUnbalance17}. For the same reason, in virtually any real instances of the \pcc{} problem, the%
% there are also \url{https://arxiv.org/abs/1509.04037} about measuring balance (or frustration
% index) which they claimed has been peer reviewed?
\marginpars{The proof makes essential use of the strong balance (which is ok since next section also
deal with 2-CC) and the completeness of the graph (which is a far less realistic assumption) yet
it would be interesting to see if it can be extended.}
optimal solution will incur some disagreements. \Textcite[Section 5.5]{BookKleinberg2010}
nonetheless prove that even when only a fraction of the triangle are strongly balanced in a complete
graph, the two clusters structure is still present, although it does not cover the whole graph
anymore.

\begin{theorem}
  Let $\epsilon < \nicefrac{1}{8}$ and $\delta = \sqrt[3]\epsilon$. If at least $1-\epsilon$ of all
  triangles in a signed complete graph are strongly balanced, then either
  \begin{enumerate}[(a),nosep]
    \item there is a set consisting of at least $1-\delta$ of the nodes in which at least $1-\delta$
      of all edges are positive, or else
    \item the nodes can be divided into two groups $L$ and $R$ such that
      \begin{enumerate}[(i),nosep]
	\item at least $1-\delta$ of the edges in $L$ are positive,
	\item at least $1-\delta$ of the edges in $R$ are positive, and
	\item at least $1-\delta$ of the edges between $L$ and $R$ are negative.
      \end{enumerate}
  \end{enumerate}
\end{theorem}

We can conclude that balance theory helps characterize zero disagreement instances of \pcc{}.
However, how is \pcc{} useful in predicting edge sign? At first sight, the two problems seem to be quite
different, for the former is an unsupervised/agnostic clustering problem, while the latter is a
supervised classification problem. Yet, getting an optimal solution of \pcc{} on the training set
would provide a principled heuristics for \esp{}. Unfortunately, as we shall see in
\autoref{sub:state_of_the_art}, exact or fixed parameter solutions are not scalable and one has to
rely on approximations. Under the Unique Games Conjecture, the approximation factor is lower bounded by
$\Theta(\log n)$ in the worst case. However, this is still a viable option, as our bias is that we
will operate on instances that are to some extent balanced. Therefore, we are not in the worst case
and we can find some comfort in results about the noisy (\vpageref{sub:variants_and_extensions}) and
stable (\autoref{ssub:cc_under_stability_assumption}) settings.

Another connection we will leverage in \autoref{sec:low_stretch_trees_and_spanners} is that the
value of the \mind{} objective is a complexity measure of the \esp{} problem~\autocite[Section
4.1]{Cesa-Bianchi2012b}. Before that, we will also present some variants and extensions of \pcc{},
in \autoref{sub:variants_and_extensions}, some of them bearing similarities with the Chapter 3's problem.


\iffalse
strong balance random graph model \url{https://www.lri.fr/~yannis/randsigned.pdf}, line index is basically
the objective value of CC and if I'm reading their bound correctly,  basically you need to flip half
the edges of a random graph to make it balanced (also most of their results hold when the
probability of an existing edge is 1/2, but it was strengthened in a follow up dealing with weak
balance \url{http://people.maths.ox.ac.uk/harutyunyan/weak-balance.pdf})

For the 2 clusters case ($k=2$), characterization proven already in 1936 by \textcite{Konig36}, as noted in
\cite{Zaslavsky2012} (commenting on \cite{harary1953}: \enquote{Although Theorem 3 was anticipated by
\textcite[Theorem X.11]{Konig36}  without the terminology of signs, here is the
first recognition of the crucial fact that labelling edges by elements of a
group---specifically, the sign group---can lead to a general theory.})
and \cite{Huffner2010} (\enquote{\textcite{Konig36} proved the following characterization of
	balanced graphs. For a graph $G = (V, E)$, the following are equivalent:\marginpars{There is a
	proof in \autocite[p. 111]{BookKleinberg2010}, maybe I can rewrite it as well}
	\begin{enumerate}
		\item $V$ can be partitioned into two sets $V_1$ and $V_2$ called sides such that there is no
			negative edge $\{v, w\} \in E$ with both $v, w \in V_1$ or both $v, w \in V_2$ and no positive
			edge $\{v, w\}$ with $v \in V_1$ and $w \in V_2$ .
		\item $V$ can be colored with two colors such that for all $\{v, w\} \in E^-$, the vertices $v$
			and $w$ have different colors, and for all $\{v, w\} \in E^+$, the vertices $v$ and $w$ have
			the same color. The color classes correspond to the sides.
		\item $G$ does not contain cycles with an odd number of negative edges.
	\end{enumerate}
	Using the characterization by a coloring, it is easy to see that balance of a signed
graph can be checked in linear time by depth-first search.})
\fi
