\paragraph{non binary edge labelling}

In the so called \textsc{Chromatic}-\pcc{} setting, \enquote{positive} edges are now associated with
one of $L$ possible color and the goal is to form clusters mostly made up of edges with the one same
color. Namely, a disagreement is now a negative edge between clusters or an within-cluster edge
whose color differs from the majority color of that cluster. This is motivated by edge-labeled graph
in social networks, biology and citation networks and will discuss such applications in
\autoref{chap:vector}. As a generalisation of \pcc{}, it is \NPc{} but \textcite{Bonchi2012a}
present a modification of the \ccpivot{} algorithm that pick edges instead of nodes as pivots, and
grow clusters by adding monochromatic triangle. This gives an approximation factor of six times the
maximum degree of the graph. They also present a method when the number $k$ of clusters is fixed
beforehand, starting with an initial partition and then alternating between finding the majority
color of clusters and finding better clusters. Unfortunately, the maximum degree of a graph can be
as large as $n$. However, \textcite{Anava2015} present constant factor approximations. Namely, they
show the problem can be reduce to classical \pcc{} by setting all edges incident to a node $u$ to
negative if they are not of the majority color of $u$. They then apply the regular \ccpivot{} and
show this gives a $11$-approximation to the original problem. Furthermore, they also write a linear
program and round it using the region growing technique of \textcite{Charikar2003} to obtain
approximation factor of $4$. \Textcite{multiChromatic15} extend their work to the case were a single
edge can carry a \emph{set} of labels and adapt their randomized algorithm so that the approximation
factor is multiplied by the size of the input label set.


\paragraph{Overlapping \pcc{}}

While in \pcc{}, each node is assigned to a single cluster, in other settings we may want to relax
this constraint. Given a complete weighted graph, \textcite{Bonchi2012} want to output a clustering
\cluster{} that minimizes the following cost: \[ \sum_{(u,v)\in E} \left| H(\cluster(u),
\cluster(v)) - w_{uv}\right|\] where $H$ is a similarity function between two sets of labels, chosen
here to be the Jaccard similarity or a $0/1$ indicator of non empty intersection. These problems are
showed \NPc{} and approximated by a local search algorithm, iteratively optimizing the assignment of
one node while all others are fixed. As one of the demonstration on their theoretical work,
\textcite{WeightedTheta15} show a faster solution based on a weighted extension of the Lovász's
theta function, the corresponding geometric embedding of graphs and a solver derived from one-class
SVM, while \textcite{GeneticOCC14} propose a genetic algorithm to solve this problem. Finally,
\Textcite{StochasticCC13} also deals with overlapping clustering by relaxing the problem to a
stochastic setting and using \enquote{the Baum-Eagon inequality, which provides an effective
iterative means for maximizing polynomial functions in probability domains}.

\paragraph{\msc{}} 

In \msc{}, the goal is to output a clustering which best summarizes (or agrees with) the several
given input clusterings of the same set of objects.  Motivations includes robustness --by using an
ensemble of clusterings from diverse methods-- and privacy --if the clusterings were computed by
different parties each considering only a subset of the objects attributes. We can build the
complete graph of these objects, with weights set to the fraction of clusterings that place two
objects in different clusters, thus representing a kind of distance in the space of clusterings.  As
first show by \textcite{Gionis2007}, finding the optimal clustering is therefore an instance of
\pcc{} where the weights obey the triangular inequality. They give a deterministic $3$-approximation
using the region growing method of \textcite{Charikar2003}. Later \textcite{Bonizzoni2008} show that
the minimization version is \APXh{}, even when the input is made of three clusterings and give a
combinatorial $\frac{4}{5}$-approximation for the maximization problem. Experimental evaluations are
conducted by \textcite{Bertolacci07} and \textcite{Filkov08}. The former describe a scalable
approach that first samples a small portion of the data, runs a (potentially computationally
expensive) approximation algorithm and finally augment the resulting partition by adding to it the
unsampled nodes one by one.  Experiments confirm that the running time is greatly improved compared
with the linear program methods while the resulting objective value is essentially the same. Note
however that LP methods can be applied in practice thanks to some tricks~\autocite{ConsensusLP10}.
% On the parameterized complexity of consensus clustering, Theoretical Computer Science, 2014,
% \url{http://dx.doi.org/10.1016/j.tcs.2014.05.002}


\paragraph{Local \pcc{}}

% that one is not really local, except for the node penalty
\Textcite{Puleo2014} adapt the linear program of~\autocite{Charikar2003} (and its rounding by the
region growing method. Namely after solving the LP, repeatedly pick a ball center $u$ \uar{} with
radius \shalf{}: if the average distance of the nodes in the ball to $u$ is less than
\nicefrac{1}{4}, the ball forms a cluster, otherwise $\{u\}$ forms a singleton cluster.) to the case
where all clusters have to contain less than $K$ nodes, by assigning to each node $u$ a penalty
$\mu_u$. If $u$ is placed in a cluster $C_i$, the original \mind{} objective is penalized by an
extra $\mu_u\left(|C_i| - (K+1)\right)$. By varying $\mu_v$ between $0$ and $1$ and because the
positive weights are assumed to smaller than $1$, this cluster size constraint can be made hard or
soft. They also handle more general weights, since they allow $w^-_{uv}$ to be as large as $\tau$
for $\tau\in [1,\infty)$ while still guaranteeing a $5-\nicefrac{1}{\tau}$-approximation on complete
graphs, and adapt \ccpivot{} to unweighted graphs with the hard cluster size constraint, obtaining a
randomized $7$-approximation.
Those soft constraints are for instance used in biological application 
\emph{A new correlation clustering method for cancer mutation analysis}
\url{https://arxiv.org/abs/1601.06476}
\autocite{Hou2016}

\Textcite{pmlr-v48-puleo16} also modify the \mind{} objective to make it more general. Based on the
classic \pcc{} linear program, they define a \enquote{\emph{fractional clustering} of $G$ as a
vector $x$ indexed by $V$ such that $x_{uv} \in [0, 1]$ for all $uv \in \binom{V}{2}$ and such that
$x_{vz} \leq x_{vw} + x_{wz}$ for all distinct $v, w, z \in V$}. They also define \enquote{The
error vector $err(x)$ of $x$, as a real vector indexed by $V$ whose coordinates are}
\begin{equation*}
  \mathrm{err}(x)_u = \sum_{v\in\nei^+(u)} x_{uv} + \sum_{v\in\nei^-(u)} (1-x_{uv})
\end{equation*}

Given a function $f: \Rbb^n_{\leq 0} \rightarrow \Rbb$ verifying two elementary conditions, the
problem is then to find a fractional clustering $x$ minimizing $f(err(x))$. The classical \pcc{}
corresponds to setting $f(x) = \lhalf{}\ell^1(x)$ whereas the authors here are interested in Minimax
\pcc{} that arises by setting $f(x) = \ell^\infty(x)$. Minimizing the maximum number of
disagreements incurred by a single node is motivated by the example of recommendation systems: if
errors corresponds to unsatisfying recommendations, we do not want a single user to suffer a large
number of them. Minimax \pcc{} is \NPc{} on both complete graphs and complete bipartite graphs but
by modifying the region growing method of \textcite{Charikar2003}, they give respectively a $48$ and
$10$ approximation, the latter for the one-sided error (that consider the nodes in only one of the
two clusters). The idea is to chose pivot not randomly but by maximizing a given criteria and
to grow ball with a ratio $\alpha$ computed numerically to optimize the approximation factor.
Interestingly, and in contrast with the classic \pcc{} situation, minimax \maxa{} is not easier than
minimax \mind{} and seems not to have a constant factor approximation, even on complete graphs.
Furthermore these algorithms are deterministic, as opposed to many \pcc{} approximation, since
bounds on expected disagreements on a edge does not translate easily on their maximum.
\Textcite{Charikar2017} improve these two factors to $7$, using a simpler version of the algorithm
of \textcite{pmlr-v48-puleo16}. Namely, find the ball of radius $\nicefrac{1}{7}$ with the largest
number node and create a cluster from its center with a radius of $\nicefrac{3}{7}$. They also show
that on general weighted graphs, the LP has a large integrality gap of $\nicefrac{n}{2}$ yet they
combine it with a combinatorial approach to reach a $O(\sqrt{n})$ approximation. Finally they
consider the complementary problem of maximizing the minimum number of agreements reach at a single
node, and provide a $\frac{1}{2+\epsilon}$ approximation.


\paragraph{recovery under noise}

As we have seen, assuming the Unique Games Conjecture, the minimum Multicut problem, and therefore
the \pcc{} problem on a general graph, cannot be approximated to a constant factor in the worst
case.  But maybe we can do better in the average case, which motivates the study of semi-random
model, where real graphs are seen as being obtained from the controlled perturbation of a perfectly
clusterable graph. In the simplest case, each edge sign is independently flipped with probability
$p \in [0, \shalf)$. This situation on complete graphs was considered in \autocite[Section
6]{Bansal2002}, showing a simple algorithm that with high probability makes
$\tilde{O}(n^\frac{3}{2})$ mistakes, and in \autocite[Theorem 2.6]{Ben-Dor99}, with an algorithm
recovering with high probability the planted partition of an unweighted graph in $O(n^2(\log n)^c)$,
where $c$ depends on the size of the smallest cluster and the noise probability.

\Textcite{Joachims2005} analyze a more refined weighted model where weights are generated by a
probability distribution whose mean on true positive edges is larger than $\mu^+>0$ and whose mean
on true negative edges is smaller than $\mu^-<0$. They give a finite-sample bounds on the number of
node misclustered w.r.t the planted partition as a function on the probability distribution
parameters. Indeed, as pointed out by \textcite{plantedAilon09}, independent and uniform noise is
not a good model of real situations, where the input of \pcc{}, \ie{} the similarity between nodes,
is often the result of a preprocessing, which may present strong correlations. Therefore, instead of
measuring the quality of a solution against in the input (\ie{} the similarity information), they
argue it is more sensible to measure it against the (unknown) true optimal clustering that gave rise
to the output, and show that \ccpivot{} allows that thanks to a new analysis. \Textcite{Mathieu2010}
also consider an adversarial model, where all edges are flipped with probability $p$ but the
adversary then decide whether to show us the true sign or the flipped sign. On complete unweighted
graphs, they find a solution of \mind{} at most $1 + O(n^{-\nicefrac{1}{6}})$ times the optimal
whenever $p \leq \shalf - n^{-\nicefrac{1}{3}}$ by rounding the usual SDP solution. If in addition
$p\leq\nicefrac{1}{3}$, there is no adversary and each planted cluster has at least
$\Theta(\sqrt{n})$ nodes, then the planted partition can be recovered exactly.
\Textcite{Makarychev2014} study the same adversarial model on general weighted graphs, giving a PTAS
for \mind{} when $p\leq \nicefrac{1}{4}$. Under additional assumptions on the density of edges, they
present another algorithm that finds the ground truth clustering with an arbitrarily small
classification error.
% http://www.jmlr.org/papers/v15/chen14a.html
