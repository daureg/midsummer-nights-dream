\paragraph{non binary edge labelling}

In the so called \textsc{Chromatic}-\pcc{} setting, \enquote{positive} edges are now associated with
one of $L$ possible color and the goal is to form clusters mostly made up of edges with the one same
color. Namely, a disagreement is now a negative edge between clusters or an within-cluster edge
whose color differs from the majority color of that cluster. This is motivated by edge-labeled graph
in social networks, biology and citation networks and will discuss such applications in
\autoref{chap:vector}. As a generalisation of \pcc{}, it is \NPc{} but \textcite{Bonchi2012a}
present a modification of the \ccpivot{} algorithm that pick edges instead of nodes as pivots, and
grow clusters by adding monochromatic triangle. This gives an approximation factor of six times the
maximum degree of the graph. They also present a method when the number $k$ of clusters is fixed
beforehand, starting with an initial partition and then alternating between finding the majority
color of clusters and finding better clusters. Unfortunately, the maximum degree of a graph can be
as large as $n$. However, \textcite{Anava2015} present constant factor approximations. Namely, they
show the problem can be reduce to classical \pcc{} by setting all edges incident to a node $u$ to
negative if they are not of the majority color of $u$. They then apply the regular \ccpivot{} and
show this gives a $11$-approximation to the original problem. Furthermore, they also write a linear
program and round it using the region growing technique of \textcite{Charikar2003} to obtain
approximation factor of $4$. \Textcite{multiChromatic15} extend their work to the case were a single
edge can carry a \emph{set} of labels and adapt their randomized algorithm so that the approximation
factor is multiplied by the size of the input label set.


\paragraph{Overlapping \pcc{}}

While in \pcc{}, each node is assigned to a single cluster, in other settings we may want to relax
this constraint. Given a complete weighted graph, \textcite{Bonchi2012} want to output a clustering
\cluster{} that minimizes the following cost: \[ \sum_{(u,v)\in E} \left| H(\cluster(u),
\cluster(v)) - w_{uv}\right|\] where $H$ is a similarity function between two sets of labels, chosen
here to be the Jaccard similarity or a $0/1$ indicator of non empty intersection. These problems are
showed \NPc{} and approximated by a local search algorithm, iteratively optimizing the assignment of
one node while all others are fixed. As one of the demonstration on their theoretical work,
\textcite{WeightedTheta15} show a faster solution based on a weighted extension of the Lovász's
theta function, the corresponding geometric embedding of graphs and a solver derived from one-class
SVM, while \textcite{GeneticOCC14} propose a genetic algorithm to solve this problem. Finally,
\Textcite{StochasticCC13} also deals with overlapping clustering by relaxing the problem to a
stochastic setting and using \enquote{the Baum-Eagon inequality, which provides an effective
iterative means for maximizing polynomial functions in probability domains}.

\paragraph{\msc{}} 

In \msc{}, the goal is to output a clustering which best summarizes (or agrees with) the several
given input clusterings of the same set of objects.  Motivations includes robustness --by using an
ensemble of clusterings from diverse methods-- and privacy --if the clusterings were computed by
different parties each considering only a subset of the objects attributes. We can build the
complete graph of these objects, with weights set to the fraction of clusterings that place two
objects in different clusters, thus representing a kind of distance in the space of clusterings.  As
first show by \textcite{Gionis2007}, finding the optimal clustering is therefore an instance of
\pcc{} where the weights obey the triangular inequality. They give a deterministic $3$-approximation
using the region growing method of \textcite{Charikar2003}. Later \textcite{Bonizzoni2008} show that
the minimization version is \APXh{}, even when the input is made of three clusterings and give a
combinatorial $\frac{4}{5}$-approximation for the maximization problem. Experimental evaluations are
conducted by \textcite{Bertolacci07} and \textcite{Filkov08}. The former describe a scalable
approach that first samples a small portion of the data, runs a (potentially computationally
expensive) approximation algorithm and finally augment the resulting partition by adding to it the
unsampled nodes one by one.  Experiments confirm that the running time is greatly improved compared
with the linear program methods while the resulting objective value is essentially the same. Note
however that LP methods can be applied in practice thanks to some tricks~\autocite{ConsensusLP10}.
% On the parameterized complexity of consensus clustering, Theoretical Computer Science, 2014,
% \url{http://dx.doi.org/10.1016/j.tcs.2014.05.002}


\paragraph{Local \pcc{}}

% that one is not really local, except for the node penalty
\Textcite{Puleo2014} adapt the linear program of~\autocite{Charikar2003} (and its rounding by the
region growing method. Namely after solving the LP, repeatedly pick a ball center $u$ \uar{} with
radius \shalf{}: if the average distance of the nodes in the ball to $u$ is less than
\nicefrac{1}{4}, the ball forms a cluster, otherwise $\{u\}$ forms a singleton cluster.) to the case
where all clusters have to contain less than $K$ nodes, by assigning to each node $u$ a penalty
$\mu_u$. If $u$ is placed in a cluster $C_i$, the original \mind{} objective is penalized by an
extra $\mu_u\left(|C_i| - (K+1)\right)$. By varying $\mu_v$ between $0$ and $1$ and because the
positive weights are assumed to smaller than $1$, this cluster size constraint can be made hard or
soft. They also handle more general weights, since they allow $w^-_{uv}$ to be as large as $\tau$
for $\tau\in [1,\infty)$ while still guaranteeing a $5-\nicefrac{1}{\tau}$-approximation on complete
graphs, and adapt \ccpivot{} to unweighted graphs with the hard cluster size constraint, obtaining a
randomized $7$-approximation.
Those soft constraints are for instance used in biological application 
\emph{A new correlation clustering method for cancer mutation analysis}
\url{https://arxiv.org/abs/1601.06476}
\autocite{Hou2016}

\Textcite{pmlr-v48-puleo16} also modify the \mind{} objective to make it more general. Based on the
classic \pcc{} linear program, they define a \enquote{\emph{fractional clustering} of $G$ as a
vector $x$ indexed by $V$ such that $x_{uv} \in [0, 1]$ for all $uv \in \binom{V}{2}$ and such that
$x_{vz} \leq x_{vw} + x_{wz}$ for all distinct $v, w, z \in V$}. They also define \enquote{The
error vector $err(x)$ of $x$, as a real vector indexed by $V$ whose coordinates are}
\begin{equation*}
  \mathrm{err}(x)_u = \sum_{v\in\nei^+(u)} x_{uv} + \sum_{v\in\nei^-(u)} (1-x_{uv})
\end{equation*}

Given a function $f: \Rbb^n_{\leq 0} \rightarrow \Rbb$ verifying two elementary conditions, the
problem is then to find a fractional clustering $x$ minimizing $f(err(x))$. The classical \pcc{}
corresponds to setting $f(x) = \lhalf{}\ell^1(x)$ whereas the authors here are interested in Minimax
\pcc{} that arises by setting $f(x) = \ell^\infty(x)$. Minimizing the maximum number of
disagreements incurred by a single node is motivated by the example of recommendation systems: if
errors corresponds to unsatisfying recommendations, we do not want a single user to suffer a large
number of them. Minimax \pcc{} is \NPc{} on both complete graphs and complete bipartite graphs but
by modifying the region growing method of \textcite{Charikar2003}, they give respectively a $48$ and
$10$ approximation, the latter for the one-sided error (that consider the nodes in only one of the
two clusters). The idea is to chose pivot not randomly but by maximizing a given criteria and
to grow ball with a ratio $\alpha$ computed numerically to optimize the approximation factor.
Interestingly, and in contrast with the classic \pcc{} situation, minimax \maxa{} is not easier than
minimax \mind{} and seems not to have a constant factor approximation, even on complete graphs.
Furthermore these algorithms are deterministic, as opposed to many \pcc{} approximation, since
bounds on expected disagreements on a edge does not translate easily on their maximum.
\Textcite{Charikar2017} improve these two factors to $7$, using a simpler version of the algorithm
of \textcite{pmlr-v48-puleo16}. Namely, find the ball of radius $\nicefrac{1}{7}$ with the largest
number node and create a cluster from its center with a radius of $\nicefrac{3}{7}$. They also show
that on general weighted graphs, the LP has a large integrality gap of $\nicefrac{n}{2}$ yet they
combine it with a combinatorial approach to reach a $O(\sqrt{n})$ approximation. Finally they
consider the complementary problem of maximizing the minimum number of agreements reach at a single
node, and provide a $\frac{1}{2+\epsilon}$ approximation.


\paragraph{recovery under noise}

As we have seen, assuming the Unique Games Conjecture, the minimum Multicut problem, and therefore
the \pcc{} problem on a general graph, cannot be approximated to a constant factor in the worst
case.  But maybe we can do better in the average case, which motivates the study of semi-random
model, where real graphs are seen as being obtained from the controlled perturbation of a perfectly
clusterable graph. In the simplest case, each edge sign is independently flipped with probability
$p \in [0, \shalf)$. This situation on complete graphs was considered in \autocite[Section
6]{Bansal2002}, showing a simple algorithm that with high probability makes
$\tilde{O}(n^\frac{3}{2})$ mistakes, and in \autocite[Theorem 2.6]{Ben-Dor99}, with an algorithm
recovering with high probability the planted partition of an unweighted graph in $O(n^2(\log n)^c)$,
where $c$ depends on the size of the smallest cluster and the noise probability.

\Textcite{Joachims2005} analyze a more refined weighted model where weights are generated by a
probability distribution whose mean on true positive edges is larger than $\mu^+>0$ and whose mean
on true negative edges is smaller than $\mu^-<0$. They give a finite-sample bounds on the number of
node misclustered w.r.t the planted partition as a function on the probability distribution
parameters. Indeed, as pointed out by \textcite{plantedAilon09}, independent and uniform noise is
not a good model of real situations, where the input of \pcc{}, \ie{} the similarity between nodes,
is often the result of a preprocessing, which may present strong correlations. Therefore, instead of
measuring the quality of a solution against in the input (\ie{} the similarity information), they
argue it is more sensible to measure it against the (unknown) true optimal clustering that gave rise
to the output, and show that \ccpivot{} allows that thanks to a new analysis. \Textcite{Mathieu2010}
also consider an adversarial model, where all edges are flipped with probability $p$ but the
adversary then decide whether to show us the true sign or the flipped sign. On complete unweighted
graphs, they find a solution of \mind{} at most $1 + O(n^{-\nicefrac{1}{6}})$ times the optimal
whenever $p \leq \shalf - n^{-\nicefrac{1}{3}}$ by rounding the usual SDP solution. If in addition
$p\leq\nicefrac{1}{3}$, there is no adversary and each planted cluster has at least
$\Theta(\sqrt{n})$ nodes, then the planted partition can be recovered exactly.
\Textcite{Makarychev2014} study the same adversarial model on general weighted graphs, giving a PTAS
for \mind{} when $p\leq \nicefrac{1}{4}$. Under additional assumptions on the density of edges, they
present another algorithm that finds the ground truth clustering with an arbitrarily small
classification error.
% http://www.jmlr.org/papers/v15/chen14a.html


\paragraph{Online \& active setting}

Except for the local algorithms of \textcite{Bonchi2013}, all works presented so far considered the
batch case of \pcc{}, where the whole graph and all the signs are available at all time and with no
cost. This might not always be the case, for instance if the graph is too large to fit in memory or
if the edge labels are given by an expensive external procedure.

% \emph{Still have to write this part. Is there a link between mistake bound and approximation
% ratio?} say I have a binary classifier that perfectly predict sign wrt to the optimal clustering
% C* those below are more concerned with sign prediction than clustering per se Beside batch
% setting, one can also consider active \autocites{Cesa-Bianchi2012b}{Cesa-Bianchi2012a} and online
% \autocite{Gentile2013} framework.
% \enquote{the work of Cesa-Bianchi et al. [CBGV+12] that take a learning-theoretic perspective on the
% problem of predicting signs. They consider three types of models: batch, online, and active
% learning, and provide theoretical bounds for prediction mistakes for each setting. They use the
% correlation clustering objective as their learning bias, and they show that the risk of the
% empirical risk minimizer is controlled by the correlation clustering objective.}

Case in point, in the context of entities resolution, \textcite{activeCoref07} consider the \pcc{}
problem where by querying an oracle, one can either reduce the uncertainty about one edge weight or
add an extra node with edges connecting it to existing nodes. However, this requires web queries are
resource bounded and thus yields an active problem, where one has to choose the most informative
queries given a budget. A formal definition is given in~\autocite{queryMoreEdgeCC07} and in
practice, after finding an initially good partition, they select in each cluster a node to be its
centroid, and query the edges connecting the centroid as ordered by an entropy-based criterion.
\textcite[Section 5]{Ailon2014} present another active algorithm that solves \mind$[k]$ with a
linear number of queries but an exponential running time in $n$.  In a noisy setting,
\textcite{Mitzenmacher2016} assume there is a planted $k$-partition and that we have access to an
oracle that for $u,v\in V^2$ returns $\cluster(u) - \cluster(v) \mod k$ with probability $1-p$ and a
noisy $\cluster(u) - \cluster(v) \pm 1 \mod k$ otherwise. For $k=2$, this is exactly $2$-\pcc{}, for
$k>2$, the oracle gives more information than simply the sign of the path from $u$ to $v$. Whenever
$p < \shalf$, they show that $O(n^{1+\frac{1}{\log\log n}}\log n)$ queries are enough are to recover
the planted partition in polynomial time and with high probability. Those queries are actually
random (\ie{} non-adaptive), and the clusters are found by looking at almost edge-disjoint path
between all pairs of nodes.

In the online setting, \textcite{greedyOnline10} give a greedy algorithm that upon node arrival creates a
singleton cluster and then merge all pairs of clusters for which it increases the total number of
agreements. For \mind{}, this is $O(n)$-competitive algorithm and they show such ratio is optimal by
exhibiting an instance\footnote{two positive cliques $A$ and $B$ joined by positive edges except
between $a\in A$ and $\{b_1,\ldots, b_k\}\in B$. Those nodes are given first and thus form a cluster
which yield at least one disagreement every time one the $n-(k+1)$ remaining nodes is added.} on
which any strategy ends up with $n - k$ disagreements whereas optimal cost is $k$. On the \maxa{}
side, this greedy strategy result in a $0.5$-competitive algorithm. If it is randomly mixed with a
\textsc{Dense} variation, it increases to $0.5+\eta$, still far from the demonstrated $0.834$ upper
bound.
% when the graph is given node by node,
% \url{https://link.springer.com/chapter/10.1007/978-3-319-18173-8_7} show to cluster it in static
% clique, so it sounds related to cluster editing, except their objective is to maximize the
% fraction of edge in cliques (ie avoid small cluster)

\paragraph{\textsc{Cluster editing}}

% \url{http://www.sciencedirect.com/science?_ob=ArticleListURL&_method=list&_ArticleListID=-1231039636&_sort=r&_st=4&md5=018b782f17918959c0cc389c33ba2cd7&searchtype=a}

In the cluster editing problem, one is given a general unsigned input graph $H=(V,E)$ and wants to
find the smallest number of edges that have to be added or deleted to turn $H$ into node-disjoint
union of cliques. This is equivalent to \pcc{} on complete graphs. To see why, let $G=(V,(E^+,E^-))$
be a complete signed graph with nodeset $V$, $E^+$ be the set $E$ of edges of $H$ and $E^-$ be all
the edges that are not in $H$. The optimal clusters of $G$ are the cliques of $H$, the negative
disagreements within clusters are edges added to $H$ and the positive disagreements between clusters
are deleted from $H$.  To the best of our knowledge, the problem was first introduced under this
name by \textcite{Shamir02}.\marginpars{However, the same problem was studied before and we refer the
reader to the comprehensive survey of \textcite{clusterEditSurvey13} for additional details,
whereas we shall only give important and more recent pointers here.} They show the problem is
\NPc{}, even if the number of clusters $k\geq2$ is set beforehand\footnote{The reduction is from
3-exact 3 cover, see \autocite[Theorems 1, 2 and Corollary 1]{Shamir02}.} and provide a
$0.878$-approximation in the weighted $k=2$ case using the standard Goemans-Williamson SDP
relaxation. 

If we parameterize the problem by the number $d$ of edges that need to be added or deleted (that is
the number of disagreements), then it can be solved in polynomial time in the size of the input
graph (but not in $d$). The best known approximation so far is
$O(1.62^d+m+n)$~\autocite{GoldenCE12}, which search for conflict triple (\ie{} unbalanced triangle
in the sign language) and branch by either deleting or merging one positive edge. If we additionally
look for exactly $k$ clusters (\ie \mind$[k]$), there is a fixed parameters algorithm running in
$O(2^{O(\sqrt{kd})}+n+m)$~\autocite{Fomin2014}. If at most $a$ edges can be added and at most $b$
edges can be deleted at each node, and if the minimum size of a cluster is at least $2(a+b)$, then
the problem can be solved in polynomial time~\autocite{Abu-Khzam2015}. Finally, On planar graph,
there is a PTAS running in $O(n2^{\epsilon^{-1}\log(\epsilon^{-1})})$ obtained by dividing the graph
into independent components of bounded treewidth~\autocite{PlanarCEPTAS17}

\paragraph{Communities detection}

The clustering problem is often named communities detection in the context of social networks, and
several methods developed by practitioners have been extended to signed graphs. While they do not
necessarily considered the \pcc{} objective, and especially not its optimum, we still give a brief
overview of them, as they tend to have been more tested experimentally. For instance, to find the
cluster of node $u$, \textcite{Yang2007} use a random walk approach on the positive subgraph $G^+$
to compute the probability of each node to reach $u$ in $k$ steps, sort the nodes accordingly and
then find a threshold  based on number of disagreements.
The one node moves local heuristic that we described in the Physics inspired paragraph
\vpageref{par:cc_physics} can also be formulated as genetic algorithms that simultaneously try to
minimize the number of disagreements and maximize a signed variant of the
modularity~\autocites{Li2013}{Amelio2013}. \Textcite{Anchuri2012} also consider these two objectives
by seeing them as eigenvalue problems and devise a iterative splitting procedure.
The overlapping community detection variant is considered by \textcite{Chen14}, who used a signed
probabilistic mixture model. Namely, an edge selects a pair of cluster $r,s$ with probability
$\omega_{rs}$ (where $r=s$ if the edge is positive and $r\neq s$ otherwise) and choose its endpoint
$u$ and $v$ with probability $\theta_{ru}$ and $\theta_{sv}$. $\theta_{ru}$ is therefore the soft
membership of node $u$ to cluster $r$, and those parameters are estimated using the
expectation-maximization algorithm. The same model is extended to directed graph by
\textcite{Jiang2015}, who strangely enough names it stochastic blockmodel, although the focus is
still on edge and not nodes.
In a similar spirit to \maxa{}, \textcite{SignedGang} focus on finding $k$ subgraphs dense in
positive edges and densely connected by negative edges to each others. They dub such graph
\emph{Oppositive Cohesive Groups}, or more vivedely \emph{Gangs in War} and after formulating the
problem as a constrained quadratic optimization, they propose a faster iterative local search
heuristic.
% relaxed structural balance \autocite{Doreian2009} but used the same approach as \textcite{Early96}
