\paragraph{non binary edge labelling}

In the so called \textsc{Chromatic}-\pcc{} setting, \enquote{positive} edges are now associated with
one of $L$ possible color and the goal is to form clusters mostly made up of edges with the one same
color. Namely, a disagreement is now a negative edge between clusters or an within-cluster edge
whose color differs from the majority color of that cluster. This is motivated by edge-labeled graph
in social networks, biology and citation networks and will discuss such applications in
\autoref{chap:vector}. As a generalisation of \pcc{}, it is \NPc{} but \textcite{Bonchi2012a}
present a modification of the \ccpivot{} algorithm that pick edges instead of nodes as pivots, and
grow clusters by adding monochromatic triangle. This gives an approximation factor of six times the
maximum degree of the graph. They also present a method when the number $k$ of clusters is fixed
beforehand, starting with an initial partition and then alternating between finding the majority
color of clusters and finding better clusters. Unfortunately, the maximum degree of a graph can be
as large as $n$. However, \textcite{Anava2015} present constant factor approximations. Namely, they
show the problem can be reduce to classical \pcc{} by setting all edges incident to a node $u$ to
negative if they are not of the majority color of $u$. They then apply the regular \ccpivot{} and
show this gives a $11$-approximation to the original problem. Furthermore, they also write a linear
program and round it using the region growing technique of \textcite{Charikar2003} to obtain
approximation factor of $4$. \Textcite{multiChromatic15} extend their work to the case were a single
edge can carry a \emph{set} of labels and adapt their randomized algorithm so that the approximation
factor is multiplied by the size of the input label set.


\paragraph{Overlapping \pcc{}}

While in \pcc{}, each node is assigned to a single cluster, in other settings we may want to relax
this constraint. Given a complete weighted graph, \textcite{Bonchi2012} want to output a clustering
\cluster{} that minimizes the following cost: \[ \sum_{(u,v)\in E} \left| H(\cluster(u),
\cluster(v)) - w_{uv}\right|\] where $H$ is a similarity function between two sets of labels, chosen
here to be the Jaccard similarity or a $0/1$ indicator of non empty intersection. These problems are
showed \NPc{} and approximated by a local search algorithm, iteratively optimizing the assignment of
one node while all others are fixed. As one of the demonstration on their theoretical work,
\textcite{WeightedTheta15} show a faster solution based on a weighted extension of the Lovász's
theta function, the corresponding geometric embedding of graphs and a solver derived from one-class
SVM, while \textcite{GeneticOCC14} propose a genetic algorithm to solve this problem. Finally,
\Textcite{StochasticCC13} also deals with overlapping clustering by relaxing the problem to a
stochastic setting and using \enquote{the Baum-Eagon inequality, which provides an effective
iterative means for maximizing polynomial functions in probability domains}.

\paragraph{\msc{}} 

In \msc{}, the goal is to output a clustering which best summarizes (or agrees with) the several
given input clusterings of the same set of objects.  Motivations includes robustness --by using an
ensemble of clusterings from diverse methods-- and privacy --if the clusterings were computed by
different parties each considering only a subset of the objects attributes. We can build the
complete graph of these objects, with weights set to the fraction of clusterings that place two
objects in different clusters, thus representing a kind of distance in the space of clusterings.  As
first show by \textcite{Gionis2007}, finding the optimal clustering is therefore an instance of
\pcc{} where the weights obey the triangular inequality. They give a deterministic $3$-approximation
using the region growing method of \textcite{Charikar2003}. Later \textcite{Bonizzoni2008} show that
the minimization version is \APXh{}, even when the input is made of three clusterings and give a
combinatorial $\frac{4}{5}$-approximation for the maximization problem. Experimental evaluations are
conducted by \textcite{Bertolacci07} and \textcite{Filkov08}. The former describe a scalable
approach that first samples a small portion of the data, runs a (potentially computationally
expensive) approximation algorithm and finally augment the resulting partition by adding to it the
unsampled nodes one by one.  Experiments confirm that the running time is greatly improved compared
with the linear program methods while the resulting objective value is essentially the same. Note
however that LP methods can be applied in practice thanks to some tricks~\autocite{ConsensusLP10}.
% On the parameterized complexity of consensus clustering, Theoretical Computer Science, 2014,
% \url{http://dx.doi.org/10.1016/j.tcs.2014.05.002}
