We already described methods based on LP and SDP, as they have proved to be the best way to obtain
tight approximations in the worst case. Here we survey other kind of approaches with different aims.

\paragraph{Exact methods} Because of the complexity of the \pcc{} we presented earlier, one has to
rely on approximations to solve large instances of the problem. However, we can imagine offline
signed social networks with only few nodes, in which case it is reasonable to expect finding the
optimal clustering. Furthermore, this can also be useful to evaluate in practice the quality of
heuristic methods, albeit in non asymptotic setting. The most straightforward idea is to use the
formulation of the LP \eqref{eq:mindLP} with the additional constraints that all variable are
binary. \Textcite{ExactMIP13} solve this binary integer program on random instances and show that
depending on the negative edge density, the FICO Xpress solver starts to be unable to finish within
a one-hour time limit when $n \geq 40$.
\Textcite{Aref2016} describe four linear and quadratic  binary integer problems
to model \pcc{} with two clusters along with some preprocessing optimizations. With the Gurobi
software, they solve instances with up to $3200$ edges in less than a second. They also sketch
extensions to weighted graphs and more than two clusters.
In a different direction, \textcite{Berg2015}, encode the linear and quadratic integer formulations
into weighted MaxSAT instances and use the state of the art solver
MaxHS\footnote{\href{http://www.maxhs.org/}{http://maxhs.org}}~\autocite{SATSolver13} to get the
exact solution on instances with at most $1000$ nodes in less than a few hours.

After defining the matrix $A$ by $A_{uv} = w_{uv}^+ - w_{uv}^-$, \textcite{LowRank16} show that the
\maxa{} problem can be written up to constant factor by defining one vector $x_u$ per node as:
\begin{align}
   \label{eq:cc_lowrank}
   \text{maximize } & \sum_{u<v} A_{uv}x_u^Tx_v \\
   \text{subject to}& \quad x_{uw} \in \{e_1,\ldots, e_n\} \nonumber
\end{align}
where $e_i$ are the canonical vectors of $\Rbb{}^n$. They show that when the matrix $A$ is positive
semidefinite of rank $k$, the \pcc{} problem can be solved exactly in $O(n^{k^2})$ time. More
practically, they also give an algorithm that closely approximate the objective of
\eqref{eq:cc_lowrank} in $O(nk)$ time.
On complete graphs, we may also turn on fixed-parameter tractable algorithms, where the parameter
can be the number of disagreements, the number of clusters or the maximum number of disagreements
per nodes, as we will now see.

\paragraph{\textsc{Cluster editing}}
\label{par:cc_editing}

% \url{http://www.sciencedirect.com/science?_ob=ArticleListURL&_method=list&_ArticleListID=-1231039636&_sort=r&_st=4&md5=018b782f17918959c0cc389c33ba2cd7&searchtype=a}

In the cluster editing problem, one is given a general unsigned input graph $H=(V,E)$ and wants to
find the smallest number of edges that have to be added or deleted to turn $H$ into node-disjoint
union of cliques. This is equivalent to \pcc{} on complete graphs. To see why, let $G=(V,(E^+,E^-))$
be a complete signed graph with nodeset $V$, $E^+$ be the set $E$ of edges of $H$ and $E^-$ be all
the edges that are not in $H$. The optimal clusters of $G$ are the cliques of $H$, the negative
disagreements within clusters are edges added to $H$ and the positive disagreements between clusters
are deleted from $H$.  To the best of our knowledge, the problem was first introduced under this
name by \textcite{Shamir02}.\marginpars{However, the same problem was studied before and we refer the
reader to the comprehensive survey of \textcite{clusterEditSurvey13} for additional details,
whereas we shall only give important and more recent pointers here.} They show the problem is
\NPc{}, even if the number of clusters $k\geq2$ is set beforehand\footnote{The reduction is from
3-exact 3 cover, see \autocite[Theorems 1, 2 and Corollary 1]{Shamir02}.} and provide a
$0.878$-approximation in the weighted $k=2$ case using the standard Goemans-Williamson SDP
relaxation. 

If we parameterize the problem by the number $d$ of edges that need to be added or deleted (that is,
the number of disagreements), then it can be solved in polynomial time in the size of the input
graph (but not in $d$). The best known approximation so far is
$O(1.62^d+m+n)$~\autocite{GoldenCE12}, which search for conflict triples (\ie unbalanced triangles
in the sign language) and branch by either deleting or merging one positive edge. If we additionally
look for exactly $k$ clusters (\ie \mind$[k]$), there is a fixed parameters algorithm running in
$O(2^{O(\sqrt{kd})}+n+m)$~\autocite{Fomin2014}. If at most $a$ edges can be added and at most $b$
edges can be deleted at each node, and if the minimum size of a cluster is at least $2(a+b)$, then
the problem can be solved in polynomial time~\autocite{Abu-Khzam2015}. Finally, for planar graph,
there is a PTAS running in $O(n2^{\epsilon^{-1}\log(\epsilon^{-1})})$ obtained by dividing the graph
into independent components of bounded treewidth~\autocite{PlanarCEPTAS17}.

\paragraph{Distributed setting}

To handle the massive size of some large real world dataset, parallelizing existing algorithms is a
natural approach. Due to its simplicity and approximation guarantee, \ccpivot{} has been adapted
three times for that purpose. First, \textcite{Chierichetti2014} describe how to uniformly sample
several pivots in the same round, remove pivots that are adjacent through positive edges and then
grow the corresponding clusters with potential conflicts solved according to the node order in a
global permutation drawn at the beginning of the algorithm. On general graphs, this requires $O(\log
n\diam(G))$ rounds\marginpars{maybe I could adapt the proof of lemma 1 to prove the number of rounds
of \gtx{}, although this is quite probabilist.}, which on complete graphs reduces to $O(\log n)$.
Furthermore, this almost preserves the approximation factor, which is
$3+\frac{14\epsilon}{1-7\epsilon}$, where $\epsilon$ is a parameter smaller than $1$. Finally, this
allows to experimentally cluster graphs with millions of nodes and billions of edges. Second,
\textcite{ParallelCCNIPS15} describe an equivalent version of \ccpivot{} where a permutation $\pi$
of the nodes is drawn at the start of the algorithm and the pivots are chosen sequentially in the
order of $\pi$. The exact same partition can be obtained when several pivots are chosen at the same
time by different threads as long as they respect two concurrency rules: (i) two nodes $u$ and $v$
can become pivot at the same time if they are not connected with a positive edge, otherwise only the
one with the smallest index in $\pi$ becomes pivot; (ii) if $w$ is a positive neighbor of two pivots
$u$ and $v$, it is affected to the cluster of the pivot with the smallest index in $\pi$. Enforcing
these rules preserves the factor $3$ approximation and the algorithms terminate after $O(\log
n\diam(G))$ rounds. The authors also present a version where each round is faster as it does not
enforce rule (i). However, this weaken the approximation guarantee to $(3 + \epsilon)OPT +
O(\epsilon n\log^2 n)$. In practice, the solution is very close the one of \ccpivot{}, although it
degrades as the number of threads increases. The number of rounds preserving the $3$-approximation
is lowered to $O(\log\log n)$ by \textcite{Ahn2015}. They also present results obtained in a single
pass, which corresponds to the streaming model: the algorithm receives the edges of $G$ one by one
and upon seeing the last one outputs its result. The additional constraint is that this algorithm
can only use $O(n\polylog n)$ space. In this setting, the authors show a polynomial time
$(1-\epsilon)$-approximation of \maxa{} if the weights are bounded (and $0.766(1-\epsilon)$ if the
weights are arbitrary); and an $O(\log |E^-|)$ approximation of \mind{} in polynomial time with arbitrary
weights. This is done by combining graph sketching and a method to solve convex programs in a space
efficient manner.

\Textcite{Bonchi2013} suggest a different paradigm to solve \pcc{}, that can be applied in a
distributed setting to obtain a scalable approach. Namely, given a node $u$, they want to output a
globally consistent cluster index $\cluster(u)$ while making at most $t$ queries to a sign oracle.
Here $t$ is a parameter that depends on the quality of clustering produced but not on the size of
the graph. And because this procedure is local to each node, it can be run in parallel. Finally, one
can get a full clustering by computing $\cluster(u)$ for all the nodes in the graph. Despite the
problem being apparently more challenging, they obtain approximation factors that are close to the
best known (which in this model would make $\Omega(n^2)$ total queries). More precisely, they use
two techniques.
The first one is inspired by \ccpivot{}. It starts by finding a good set of pivots, seeing the
problem as finding a maximal independent set on a sampled part of the positive graph. Then, for a
given node, it finds the closest such pivot or creates a singleton. Given a quality parameter
$\epsilon\in(0,1)$, it yields a $4\cdot OPT + \epsilon n^2$ approximation of \mind{} requiring
$O(\frac{n}{\epsilon})$ time and queries~\autocite[Theorem 3.3]{Bonchi2013}.
Roughly stated, the second one relies on an existing low-rank approximation of the adjacency matrix,
that partition the graph into similar sized classes such that edges between those classes behave as
in a random graph. This initial partition is \enquote{coarsened} into a good clustering by
considering all possible ways of assigning classes to clusters. It gives an $OPT + \epsilon n^2$
additive approximation  for \mind{} that runs in time $n \cdot poly(\nicefrac{1}{\epsilon}) +
2^{poly(\nicefrac{1}{\epsilon})}$ \autocite[Corollary 3.7]{Bonchi2013}.

\paragraph{Heuristic methods}

While all the methods we discuss so far
are either exact or come with some approximation guarantees, practitioners have also develop
approaches that are designed to efficiently reach a solution that is satisfactory enough for the
application at hand. For instance, while studying small scale signed social networks,
\textcite{Early96} describe the following procedure. Start with a random initial $k$ clustering of
the graphs and for $T$ steps, sample randomly $P$ neighboring partitions, compute their number of
disagreements and move the one with the least disagreements. Neighboring partitions are obtained
either by moving one node from cluster to another or by exchanging a pair of nodes between two
clusters. The overall complexity is $O(nTP)$. This is quite similar to the \emph{Best One Element
Move} described in~\autocite{Gionis2007}, except that there the exchange operation is replaced by moving
one node to its own singleton cluster. The \emph{Cluster Affinity Search Technique}
algorithm~\autocite{Ben-Dor99} instead grows clusters one by one by maintaining for the current
cluster the affinity of all nodes, which is the sum of the weights between that node and the nodes in
the cluster. Nodes above a certain threshold are added to the current cluster and nodes below the
threshold are removed, with the affinity to the current cluster being recomputed after each addition
or deletion.

After defining the net weight of an edge to be $w^\pm_{uv} = w^+_{uv} - w^-_{uv}$,
\textcite{Elsner2009} describe three folklore heuristics that start with empty clusters and add node
one by one: \enquote{The \textsc{Best} algorithm adds each node $u$ to the cluster with the
strongest $w^\pm$ connecting to $u$, or to a new singleton if none of the $w^\pm$ are positive.
The \textsc{First} algorithm adds each node $u$ to the cluster containing the most recently
considered node $v$ with $w^\pm_{uv} > 0$. The \textsc{Vote} algorithm adds each node to the cluster
that minimizes the \pcc{} objective, \ie{} to the cluster maximizing the total net weight
or to a singleton if no total is positive.} Empirically, \textsc{Vote} turns out to be the best.
Among other related heuristics, \textcite{mergingHeuristics14} describe the random maximum merging
algorithm, that starts with singleton clusters and keep merging two clusters chosen at random among
those whose merge would result in the maximum improvement of the score function.  This runs in
$O(n^2\log n)$, provides a $2$-approximation of \maxa{}, a $O(n\cdot \text{size of the largest
cluster})$-approximation of \mind{}, and empirically results in fewer disagreements than \ccpivot{}.
Building upon their previous GRASP work~\autocite{GRASP13}, an iterated local search (ILS) heuristic is
presented in~\autocites{Levorato2015}{Levorato2017}. Each iteration of this algorithm starts by
greedily build a clustering in fashion similar to \textsc{Vote}, albeit with more randomness in the
node ordering. This clustering is locally improved by moving blocks of $r\in\{1,2\}$ nodes from one
cluster to another as long as the number of disagreements decreases, a phase called neighborhood
descent. The algorithm then enters an inner loop where the current clustering is perturbed by $t$
random one-node-moves and updated if a subsequent neighborhood descent can improve it compared with
before the perturbation. The authors note that both the outer loop and the neighborhood descent can
be run in parallel, which allow them to process a \np{10000} nodes graph on 10 cores in around 700
seconds. \Textcite{heuristicCE16} show how to use ILS to generate initial solutions of the Cluster
Editing problem that are then fed to an integer program.
This local search followed by random perturbation is reminiscent of genetic algorithms, and such an
approach is sketched in~\autocite{GeneticCC08}.\footnote{but that's a terrible paper!} Instead of
moving nodes, \textcite{restoreNeighborhood13} starts from the observation that in a balanced
graph, $G^+$ is a disjoint union of cliques in which all the nodes of a given cluster share the shame
neighbors. Their algorithm is initialized with the set $E_s^+$ of edges where $w^+_{uv} > w^-_{uv}$,
and repeatedly samples an edge $(u,v)$ from $E_s^+$ before trying to make the neighborhoods of $u$
and $v$ coincide by adding or removing edges in $E_s^+$.  Yet another idea is to modify the LP
\eqref{eq:mindLP} to replace the binary variable $x_{uv}$ by a $k$ clusters indicator matrix
$L\in\Rbb^{k\times n}$ where $L_{iu}$ is $1$ is $u\in C_i$ and $-1$
otherwise~\autocite{AltOptimLP13}. Indeed, $x_{u\cdot} \equiv L_{\cluster(u)\cdot}$, where $\cluster$
is the cluster assignment. By relaxing the integer constraints on $L$, it is possible to do
alternate optimization on $L$ and $\cluster$.

\paragraph{Physics inspired energy methods}
\label{par:cc_physics}

Here we describe a physics particle model that can readily be adapted to model \pcc{} and provide
additional heuristic methods.
The Potts model~\autocite{PottsSurvey82} describes a general model of \emph{spins} organized in a
lattice and being in one of $k$ possible states. A spin $u$ interacts with each of its neighbors $v$
through a coupling $J_{uv}$. The energy of this system, called the Hamiltonian, is defined by
$\mathcal{H}(\bm{\sigma}) = -\sum_{u,v} J_{uv} \delta(\sigma_u, \sigma_v)$ where $\bm{\sigma}$ is
the spin configuration (that is, $\sigma_u \in \{1,\ldots,k\}\,\forall u$) and $\delta(\sigma_u,
\sigma_v)$ is equal to one if $u$ and $v$ are in the same state and zero otherwise. It is a general
principle of physics that isolated systems tend to minimize their energy, which in this case amounts
to finding a spin configuration minimizing the Hamiltonian. Viewing the spins as nodes of a graph,
the couplings as the graph weighted edges and the $k$ possible states as clusters, it is quite
natural to formulate the clustering problem as a Hamiltonian minimization
problem~\autocite{CommunityPhysics06}.

Letting $A$ be the matrix such that $A_{uv} = w_{uv}^+ -
w_{uv}^-$ on a general directed graph, \textcite{Traag2009} reward positive and absent negative
links within cluster and penalize negative and absent positive links across clusters to come up with
the following Hamiltonian: $\mathcal{H}(\bm{\sigma}) = -\sum_{u,v} \left(A_{uv}-(\gamma^+p^+_{uv} -
\gamma^-p^-_{uv})\right) \delta(\sigma_u, \sigma_v)$ where $\gamma^+$ and $\gamma^-$ are user
parameters and $p^\pm_{uv}$ are null model probabilities, which are equal to $p^\pm_{uv} =
\frac{|E^\pm|}{|V|(|V|-1)}$, or $p^\pm_{uv}=\frac{\dout^\pm(u) \din^\pm(v)}{|E^\pm|}$ in the degree
corrected model. They note that setting $\gamma^+ = 0 = \gamma^-$ make minimizing the Hamiltonian
equivalent to the \mind{} problem, which they do by simulated
annealing~\autocite{SimulatedAnnealing83}. The idea is to start from a random partition, and jump to
another partition by moving a single node from one cluster to another. Such moves are made with a
probability proportional to how each move reduces the Hamiltonian and as the procedure goes on, large
jumps are made less and less probable by reducing a parameter called the temperature.
One advantage of this energy formulation is that it requires only one variable per node instead of
one variable per edge as in the case of the linear program.

In the case of $2$-\pcc{}, \textcite{Facchetti2011isingmodel} rewrite the Hamiltonian as
$\mathcal{H}(\bm{\sigma}) = -\frac{1}{2}\bm{\sigma}^T A \bm{\sigma} = -\frac{1}{2}\bm{1}^TT_\sigma A
T_\sigma\bm{1}$. There $T_\sigma = diag(\bm{\sigma})$ (where $\bm{\sigma} \in \{0,1\}^n$) is the
outcome of a local search algorithm such that $A_\sigma = T_\sigma A T_\sigma$ has the same number
of disagreements as $A$ but the smallest number of negative edges. This is called a gauge
transformation in the spin glass literature and the benefit of that heuristic is that it scales
gracefully to large graphs.
\Textcite{Bagon2011} also write the \maxa{} objective as a Potts model, and show that it can be
interpreted as the log posterior of a partition matrix under a simple generative model and as a
pair-wise conditional random field energy without unary terms. This allows them to adapt existing
discrete energy optimization algorithms in order to cope with the following three challenges of the
\pcc{} energy: \enquote{(i) the energy is non sub-modular, (ii) the number of clusters is not known
in advance, and (iii) there is no unary term}. Doing so, they are able to handle large problems with
more than 100K nodes.  Also adopting an energy minimization approach, \textcite{Kappes2016} assign a
probability to each cut of a signed graph proportional to the exponential of the number of
disagreements of that cut. They also develop efficient cut sampling methods.

% something about local minima when defining another energy function, and how it is achieved by a
% dynamic model but to be honest it has little to do with clustering {Marvel2009landscape}
% Kappes2016 \enquote{Furthermore, due to the lack of an external field (unary terms), any permutation of an
% optimal assignment results in another optimal labeling.}
% like an ICML'17 paper that touches something very related (multicut) and gives recent applications in
% vision https://arxiv.org/abs/1503.03791 represents a cut by the set of interclusters edges
