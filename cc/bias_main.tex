Let us recall first the sign model we used for \dssn{}. Each node $u$ is endowed with two parameters
$p_u$ and $q_u$ drawn from an arbitrary joint distribution $\mu$ over $[0,1]\times [0,1]$. $p_u$ can be
interpreted as the tendency of $u$ to send positive edges to other nodes (\ie{} the
\enquote{niceness} of $u$), while $q_u$ can be interpreted as the tendency of $u$ to receive positive
edges from other nodes (\ie{} the \enquote{popularity} of $u$). The sign of an edge \euv{} is then
chosen to be positive with probability $\frac{p_u+q_v}{2}$. This suggests that nodes in the
graph have a form of agency, and further imposes that edges are directed. We show experimentally that
failing to meet these two assumptions harms the performance of our previous methods. This is not
surprising, for our bias is no more justified in that case. Therefore, we suggest a different bias,
drawing heavily on social balance theories, although we shall see later this holds for many
nonsocial graphs as well.

\subsection{Sign generative model and behavior}
\label{sub:bias_sign_behavior}

The method presented in the previous chapter hinges crucially upon our sign generative model. Yet
one can imagine contexts where this model is not applicable, especially when nodes do not represent
human beings. One way to alleviate this issue was introduced with our online algorithm, where this
time, signs are generated by an arbitrary adversary. However, in that case, we are still facing two
limitations\marginpars{limitations is not the right word, maybe \emph{strong constraints}…}
\begin{enumerate}
	\item Our bias remains that the labeling is regular (recall this means informally that the all
		the outgoing signs from a given node tend to be the same, and likewise for the incoming signs),
		since irregularities are the cost payed by the adversary to make our algorithm mispredict. While
		this bias is well suited to social networks, other applications may require other
		bias\marginpars{namely the \pcc{} bias we'll introduce next}.
	\item In the online setting, we evaluate our performance by the regret\marginpars{ref to regret equation},
		whereas in general we are interested in more classical measures of predictive accuracy.
\end{enumerate}

\subsubsection{Experience on directed biological networks}
\label{ssub:experience_on_directed_biological_networks}

Take for instance a biological network.

According to \href{https://web.stanford.edu/class/cs224w/slides/handout-bionets.pdf}%
{this description of various biological networks}, \emph{gene regulatory network} are directed and
contains activation and inhibition links, as \href{https://en.wikipedia.org/wiki/Gene_regulatory_network#Overview}%
{showed on Wikipedia}. There are some online databases such as
\href{http://regulondb.ccg.unam.mx/menu/download/datasets/index.jsp}%
{RegulonDB} or \href{http://www.pathwaycommons.org/pcviz/}{Pathway Commons} (this last one provide
network in the BioPAX format, which can be visualized by
\href{http://www.cytoscape.org/}{CytoScape}).
Another source is the \href{https://www.ncbi.nlm.nih.gov/pmc/articles/PMC2708159/table/T1}{Table 1}
of \cite{BioSigned09}.
%DasGupta also describe (in
%http://www.sciencedirect.com/science/article/pii/S0303264706001419#sec14) how
%they built a directed signed network from this SBML file
%https://www.ncbi.nlm.nih.gov/pmc/articles/PMC1681468/bin/msb4100014-sd1.xml

other domain? coref, images (is it directed?), entity resolution…\marginpars{Actually it's difficult
to find other kind of graph because all constructed ones (like image, coreference,
deduplication) are inherently symmetric}


\subsection{Directed edges requirement}
\label{sub:need_for_a_directed_graph}

Having two parameters per node, one for outgoing edges and another one for incoming edges, clearly
targets directed graphs. Many online interactions are inherently directed, for instance friendship,
trust or communication. On the other hand, predicting edge signs in undirected graphs is an equally
relevant objective. A prime example of such a situation is when we are given $n$ objects, some
pairwise similarities them, and the similarity function itself, which unfortunately takes an
exorbitant time to be evaluated. There is an underlying graph and being able to predict the sign of
its undirected edges would save us expensive evaluations of the symmetric similarity function.

A trivial way to turn an undirected graph $G$ into a directed graph $G'$ is to let $V'=V$ and, for
every edge $(u,v)$ in $E$, to add both \euv{} and \evu{} to $E'$ with the same sign as $(u,v)$. In
terms of our generative model, this corresponds to putting all the probability mass of $\mu$ on the
$p=q$ diagonal. This is clearly not a very satisfactory solution, for it removes one degree of
freedom from the model. To illustrate this point, we conduct the following experiment. We use our
previous \dssn{} datasets and remove the edge direction. As mentioned earlier, for a few pair of
nodes, there are reciprocal edges of different signs, in which case we pick a sign arbitrarily. Given
those undirected graphs, we orient them using the approach described above and compare our method
with the \complowrank{} approach ran directly on the undirected graphs. This is a fair comparison,
for \complowrank{} is designed to works solely on undirected graphs.

As we can see in \autoref{tab:bias_exp_undir}, our methods perform worse than when running on the
original directed graphs. Looking at \uslpropGsec{} on a 15\% training set as an example, we observe
that the MCC decreases by one (\kiw{}) to almost ten (\epi{}) points. On the other hand, the
\complowrank{} seems to perform better. Recall however that because it is an undirected method in
the first place and because we roughly double the number of signed edges, a 15\% training size means
the algorithm gets to observe more signs and therefore perform better. This small experiment thereby
shows that although the performance remains decent, not having direction information hurts our bias.

\begin{table}[t]
\begin{adjustwidth}{-2cm}{}
\centering
% \setlength{\tabcolsep}{3pt}
% \scriptsize
\small
\caption[MCC on the six previous datasets with direction removed]{MCC results on the six datasets from \autoref{chap:troll}, after removing the directions as
    described in the main text.  These results are presented like in \autoref{tab:all_mcc}, except
    we have transposes the rows and the columns, and we show only three relevant methods.
\label{tab:bias_exp_undir}}
% \hspace*{-0.2in}%
{\renewcommand{\arraystretch}{0.9}%
\begin{tabular}{lrrrrrrr}
\toprule
$\frac{|\trainset{}|}{|E|}$ & $5\%$                       & $10\%$                    & $15\%$                    & $20\%$                    & $25\%$                    & $50\%$                    & $90\%$                    \\
\midrule
                            & \multicolumn{7}{c}{\aut{}} \\ \cmidrule(lr){2-8}
\uslpropGsec{}              & $\vfirstSig{22.2}\pm0.7$    & $\vfirstSig{28.6}\pm0.5$  & $\vfirst{32.0}\pm0.5$     & $\vsecondSig{34.4}\pm0.3$ & $\vsecondSig{36.9}\pm0.5$ & $41.7\pm0.4$              & $44.9\pm0.9$              \\
\usrule{}                   & $\vsecondSig{18.1}\pm0.7$   & $25.3\pm0.8$              & $29.7\pm0.4$              & $32.6\pm0.4$              & $35.8\pm0.4$              & $\vsecondSig{42.2}\pm0.4$ & $\vsecondSig{45.0}\pm1.0$ \\
\complowrank{}              & $17.3\pm0.5$                & $\vsecondSig{25.3}\pm0.3$ & $\vsecond{31.4}\pm0.5$    & $\vfirstSig{36.3}\pm0.8$  & $\vfirstSig{41.2}\pm1.2$  & $\vfirstSig{55.4}\pm0.8$  & $\vfirstSig{60.6}\pm3.9$  \\[2pt]
    & \multicolumn{7}{c}{\adv{}} \\ \cmidrule(lr){2-8}
\uslpropGsec{}              & $\vfirstSig{36.5}\pm0.4$    & $\vfirstSig{41.4}\pm0.5$  & $\vfirstSig{43.9}\pm0.4$  & $\vfirstSig{45.7}\pm0.5$  & $\vfirst{47.0}\pm0.5$     & $\vfirst{50.4}\pm0.4$     & $\vfirst{52.7}\pm0.7$     \\
\usrule{}                   & $\vsecondSig{32.8}\pm0.5$   & $\vsecondSig{38.9}\pm0.7$ & $\vsecondSig{41.8}\pm0.8$ & $\vsecondSig{44.3}\pm0.8$ & $\vsecond{46.3}\pm0.7$    & $\vsecond{49.8}\pm0.5$    & $\vsecond{52.6}\pm0.8$    \\
\complowrank{}              & $28.8\pm0.6$                & $32.2\pm0.4$              & $33.9\pm0.6$              & $35.0\pm0.5$              & $36.3\pm0.7$              & $41.1\pm1.0$              & $46.2\pm3.4$              \\[2pt]
    & \multicolumn{7}{c}{\wik{}} \\ \cmidrule(lr){2-8}
\uslpropGsec{}              & $\vfirstSig{36.3}\pm0.3$    & $\vfirstSig{42.6}\pm0.4$  & $\vfirstSig{45.8}\pm0.2$  & $\vfirst{47.8}\pm0.4$     & $\vfirstSig{49.2}\pm0.2$  & $\vfirst{52.5}\pm0.3$     & $\vsecond{54.1}\pm0.5$    \\
\usrule{}                   & $\vsecondSig{35.0}\pm0.6$   & $\vsecondSig{41.4}\pm0.4$ & $\vsecondSig{44.9}\pm0.3$ & $\vsecond{46.9}\pm0.5$    & $\vsecondSig{48.2}\pm0.4$ & $51.3\pm0.4$              & $52.7\pm0.6$              \\
\complowrank{}              & $30.6\pm0.7$                & $37.2\pm0.4$              & $40.7\pm0.5$              & $43.4\pm0.6$              & $45.8\pm0.6$              & $\vsecond{51.8}\pm0.7$    & $\vfirst{54.6}\pm4.0$     \\[2pt]
    & \multicolumn{7}{c}{\sla{}} \\ \cmidrule(lr){2-8}
\uslpropGsec{}              & $\vfirst{39.4}\pm0.1$       & $\vfirstSig{44.9}\pm0.2$  & $\vfirstSig{47.8}\pm0.1$  & $\vfirstSig{49.7}\pm0.1$  & $\vfirstSig{51.0}\pm0.1$  & $\vsecondSig{54.4}\pm0.1$ & $\vsecondSig{56.4}\pm0.2$ \\
\usrule{}                   & $35.5\pm0.3$                & $40.7\pm0.1$              & $43.8\pm0.2$              & $46.0\pm0.2$              & $47.3\pm0.2$              & $51.5\pm0.1$              & $54.0\pm0.4$              \\
\complowrank{}              & $\vsecond{39.0}\pm0.3$      & $\vsecondSig{41.8}\pm0.3$ & $\vsecondSig{44.8}\pm0.5$ & $\vsecondSig{48.0}\pm0.6$ & $\vsecondSig{50.5}\pm0.4$ & $\vfirstSig{57.2}\pm0.3$  & $\vfirstSig{59.1}\pm0.6$  \\[2pt]
    & \multicolumn{7}{c}{\epi{}} \\ \cmidrule(lr){2-8}
\uslpropGsec{}              & $\vfirstSig{45.1}\pm0.4$    & $\vsecond{50.0}\pm0.4$    & $\vsecondSig{52.6}\pm0.3$ & $\vsecondSig{54.4}\pm0.2$ & $\vsecondSig{55.8}\pm0.3$ & $\vsecondSig{59.3}\pm0.1$ & $\vsecondSig{62.0}\pm0.2$ \\
\usrule{}                   & $40.7\pm0.6$                & $47.2\pm0.5$              & $50.2\pm0.6$              & $53.1\pm0.2$              & $54.6\pm0.2$              & $58.6\pm0.1$              & $60.9\pm0.2$              \\
\complowrank{}              & $\vsecondSig{44.1}\pm0.1$   & $\vfirst{50.8}\pm0.7$     & $\vfirstSig{55.9}\pm0.5$  & $\vfirstSig{59.3}\pm0.4$  & $\vfirstSig{61.7}\pm0.2$  & $\vfirstSig{66.8}\pm0.2$  & $\vfirstSig{68.4}\pm0.8$  \\[2pt]
    & \multicolumn{7}{c}{\kiw{}} \\ \cmidrule(lr){2-8}
\uslpropGsec{}              & $\vfirstSig{34.4}\pm0.3$    & $\vfirstSig{37.1}\pm0.4$  & $\vfirstSig{37.9}\pm0.3$  & $\vfirstSig{38.4}\pm0.6$  & $\vsecond{38.5}\pm0.4$    & $\vsecondSig{38.9}\pm0.3$ & $\vsecondSig{39.4}\pm0.3$ \\
\usrule{}                   & $\vsecondSig{28.5}\pm0.3$   & $\vsecondSig{32.9}\pm0.2$ & $\vsecondSig{34.8}\pm0.1$ & $35.5\pm0.2$              & $36.1\pm0.1$              & $37.2\pm0.1$              & $37.6\pm0.3$              \\
\complowrank{}              & $26.7\pm0.4$                & $30.9\pm0.4$              & $34.2\pm0.6$              & $\vsecondSig{36.6}\pm0.8$ & $\vfirst{38.7}\pm0.7$     & $\vfirstSig{44.0}\pm0.4$  & $\vfirstSig{46.3}\pm1.8$  \\[2pt]
\bottomrule
\end{tabular}}
\end{adjustwidth}
\end{table}



\subsection{Social balance as a learning bias}
\label{sub:bias_balance}

From a narrow practical point of view, balance theories specify which sign assignments of a triangle
are possible. By forbidding triangles with a single negative edge, weak balance gives rise to a
$K$-\emph{consistent} clusters structure on complete graphs. After making these notions more precise
in the rest of this section, we conclude by precisely exposing our learning bias.

\begin{aside}
We focus here on balance theory, for it is both compelling and well-established.
Indeed, as illustrated by the account of \textcite{CSBalanceSurvey15}, it has been used extensively
in the last half century to study signed networks and their dynamics (an example of this far
reaching influence is the analysis of the relation between characters in fairy
tales~\autocite{fairyTales80}).
However, it has shortcomings, the main one being its inability to deal with directed graphs.
Therefore, alternative theories governing signs formation have been proposed, as surveyed
in~\autocite{Yap2015}.
\end{aside}

In his seminal work on interpersonal relations,
\textcite{Heider46} posits through
psychological and sociological arguments that in order to reduce their cognitive dissonance, three
people always interact in a way that preserve social balance. These social balance requirements can be
succinctly summarized by four statements~\autocite{HeiderBook58}:

\begin{enumerate}
	\item \textcolor{DodgerBlue}{my friend's} \textcolor{Orange}{friend} is
		\textcolor{Olive}{my friend}\hspace{1em}%
		\tikz[baseline=5]{
			\node[draw,circle,inner sep=2pt] (me) at (0,0) {};
			\node[draw,circle,inner sep=2pt] (mine) at (.5,.5) {};
			\node[draw,circle,inner sep=2pt] (alter) at (1,0) {};
			\draw[] (me) -- node[left,DodgerBlue,pos=.9] {$+$} (mine) ;
			\draw[] (mine) -- node[right,Orange,pos=.1] {$+$} (alter) ;
			\draw[white] (me) -- node[text=Olive] {$+$} (alter) ;
		}
	\item \textcolor{DodgerBlue}{my friend's} \textcolor{Orange}{enemy} is
		\textcolor{Olive}{my enemy}\hspace{1em}%
		\tikz[baseline=5]{
			\node[draw,circle,inner sep=2pt] (me) at (0,0) {};
			\node[draw,circle,inner sep=2pt] (mine) at (.5,.5) {};
			\node[draw,circle,inner sep=2pt] (alter) at (1,0) {};
			\draw[] (me) -- node[left,DodgerBlue,pos=.9] {$+$} (mine) ;
			\draw[] (mine) -- node[right,Orange,pos=.1] {$-$} (alter) ;
			\draw[white] (me) -- node[text=Olive] {$-$} (alter) ;
		}
	\item \textcolor{DodgerBlue}{my enemy's} \textcolor{Orange}{friend} is
		\textcolor{Olive}{my enemy}\hspace{1em}%
		\tikz[baseline=5]{
			\node[draw,circle,inner sep=2pt] (me) at (0,0) {};
			\node[draw,circle,inner sep=2pt] (mine) at (.5,.5) {};
			\node[draw,circle,inner sep=2pt] (alter) at (1,0) {};
			\draw[] (me) -- node[left,DodgerBlue,pos=.9] {$-$} (mine) ;
			\draw[] (mine) -- node[right,Orange,pos=.1] {$+$} (alter) ;
			\draw[white] (me) -- node[text=Olive] {$-$} (alter) ;
		}
	\item \textcolor{DodgerBlue}{my enemy's} \textcolor{Orange}{enemy} is
		\textcolor{Olive}{my friend}\hspace{1em}%
		\tikz[baseline=5]{
			\node[draw,circle,inner sep=2pt] (me) at (0,0) {};
			\node[draw,circle,inner sep=2pt] (mine) at (.5,.5) {};
			\node[draw,circle,inner sep=2pt] (alter) at (1,0) {};
			\draw[] (me) -- node[left,DodgerBlue,pos=.9] {$-$} (mine) ;
			\draw[] (mine) -- node[right,Orange,pos=.1] {$-$} (alter) ;
			\draw[white] (me) -- node[text=Olive] {$+$} (alter) ;
		}
\end{enumerate}

This can be readily translated into graph properties. Given a path of length two over three nodes,
note that the first part of each statement (in blue and orange) defines the four possible sign
assignments of such a path. The last part of the statement (in green) then prescribes which sign
should close the triangle to respect the social balance. Such triangles are called strongly balanced.
\begin{definition}[strongly balanced triangle]
	A triangle is strongly balanced if it has zero or two negative signs. Otherwise it is unbalanced.
\end{definition}

We use triangles as the building block of our definition of strongly balanced graphs.
\begin{definition}[strongly balanced complete graphs]
	A complete signed graph is strongly balanced if all its triangles are strongly balanced.
\end{definition}

It is interesting to note that this local property of triangles gives raise to a global structure.
Namely, a strongly balanced complete graph can be split in two (possibly empty) clusters $L$ and $R$
such that all edges within $L$ and $R$ are positive, while all edges between $L$ and $R$ are
negative. Indeed, let us pick an arbitrary node $u$ and let $L=\neip(u)\setminus \{u\}$ and $R=V\setminus L$. Now
let $v,w\in L\setminus \{u\}$ and $x,y\in R$. From \autoref{fig:cc_twostrongcc} and the fact that
every triangle is balanced, we can see that $(v,w)$ and $(x,y)$ are positive edges, while $(w,y)$ is
a negative edge.
\begin{marginfigure}
  \centering
  \includegraphics[width=28mm]{assets/raw/2cc.pdf}
  \caption{A two-clustering of a complete strongly balanced graph}
  \label{fig:cc_twostrongcc}
\end{marginfigure}
We say that $L$ and $R$ are $2$-consistent clusters and more generally, we define
\begin{definition}[$K$-consistent clustering]
  Given a graph $G=(V, E)$, a clustering $\cluster=\{C_1, \ldots, C_K\}$ of $V$ is \emph{consistent}
  with the signs of $E$ if for every edge $(u,v)$ in $E$:
  \begin{enumerate}[(i),nosep]
    \item $(u,v)$ is positive if $\cluster(u)=\cluster(v)$, and
    \item $(u,v)$ is negative if $\cluster(u)\neq\cluster(v)$.
  \end{enumerate}
\end{definition}

We just saw that if a complete graphs is strongly balanced, then it admits a $2$-consistent
clustering (\autoref{thm:structural} proves the converse is true). It is natural to ask under which
conditions a complete signed graph admits a $K>2$ consistent clustering. This is where the notion of
weak balance comes handy. It was noted by
sociologists that among triangles with odd number of negative edges, one is more stable and common
than that the other. Therefore, \textcite{davis1967clustering} relax the strong balance into the
weak balance by considering triangles with three negative edges to be balanced, as illustrated on
\autoref{fig:cc_balance}. Formally, we can modify our two previous definitions of balance to arrive
at weak balance:
\begin{definition}[Weak balance]\label{def:weak_balance}
	A triangle is weakly balanced if it has zero, two or three negative signs. Otherwise it is
	unbalanced. \\
	A complete signed graph is weakly balanced if all its triangles are weakly balanced.
\end{definition}

\begin{figure}[htpb]
	\centering
	\includegraphics[width=0.8\linewidth]{assets/raw/triangle_smaller.pdf}
	\caption{The four possible undirected triads, as classified by the two structural balance theories
	introduced in the main text} \label{fig:cc_balance}
\end{figure}

In that case as well, the local property of triangles has structural implication for the whole graph.
Namely, we can partition a complete weakly balanced graph into $K$ consistent clusters. 
Consider the same construction as in
\autoref{fig:cc_twostrongcc}, putting a node $u$ and its positive neighborhood in one cluster $L$ and
the rest of the nodes in $R$. For $v,w\in L\setminus \{u\}$ and $x,y\in R$, we can still conclude
that $(v,w)$ is positive and $(v,y)$ is negative. On the other hand, this time weak balance allows
$(x,y)$ to be either positive or negative. We can split $R$ further into the positive neighborhood of
$x$ and the rest, and keep doing so until we form the $k$ clusters. In the following we will make
this argument more formal.

We gave the definitions of balance in the context of complete graphs in order to build intuition
about the partition consequences and because it easier to reason on triangles. Furthermore, if we are
given a set of objects and a similarity function, we can indeed build a complete graph. However,
they are situations were it is not realistic to make the completeness assumption, especially for
social networks that are typically very sparse. We therefore extend the concept of balance to
general graphs in the following way:
\begin{definition}[balanced general graphs]
  A general graph is strongly (respectively weakly) balanced if there is a sign assignment of all its
  missing edges such that the resulting complete signed graph is strongly (respectively weakly)
  balanced.
\end{definition}
To recover the characterization of signed graphs by consistent clustering, we need to
consider longer cycles than triangles.
\begin{definition}[balanced cycles]
  A cycle is strongly (respectively weakly) unbalanced if it has an odd number of negative edges
  (respectively exactly one negative edge). Otherwise it is strongly (respectively weakly) balanced.
\end{definition}

Being strongly balanced is equivalent to having a $2$-consistent clustering, as proved in the landmark
theorem of \textcite[Theorem 3]{harary1953}:
\begin{theorem}[Structural Theorem]\label{thm:structural}
  A graph $G=(V,E)$ is strongly balanced if and only if $V$ admits a $2$-consistent clustering.
\end{theorem}

\begin{aside}
According to \textcite{Huffner2010}, a similar theorem was proved earlier by \textcite[Theorem
X.11]{Konig36}, although \textcite{Zaslavsky2012} notes that it was stated \enquote{without the
terminology of signs, while \autocite{harary1953} has the first recognition of the crucial fact that
labelling edges by elements of a group---specifically, the sign group---can lead to a general
theory.}
\end{aside}

A useful characterization, whose proof can be found in~\autocite[page 122]{BookKleinberg2010}, is the
following:
\begin{theorem}
  A graph $G=(V,E)$ is strongly balanced if and only if all its cycles are strongly balanced.
\end{theorem}

Similar results also hold for weak balance.
\begin{theorem}\label{thm:weakbal-consistent}
  A graph $G=(V,E)$ is weakly balanced if and only if $V$ admits a $K$-consistent clustering.
\end{theorem}
\begin{proof}
  Let first assume that $G$ is weakly balanced. According to the definition, we can choose sign for
  the missing edges such that it becomes a complete graph with all its triangles weakly balanced. As
  mentioned earlier, we can then pick a node $u_1$ and let $C_1 = \neip(u) \cup \{u\}$. All
  nodes in $C_1$ are connected positively with each other and negatively with nodes in $V\setminus
  C_1$. Note also that $|V\setminus C_1| < |V|$ and therefore we can repeat this
  procedure until $V$ is exhausted, at which point we have obtained our $K$ consistent clusters.

  Conversely, assume we have $K$ consistent clusters. We can complete the graph $G$ by letting the edge
  $(u,v)$ be positive if $u$ and $v$ are in the same cluster and negative otherwise. Let us the pick
  three arbitrary nodes $u,v,w\in V$. There are three cases: they are either all in the same
  cluster, all in different clusters or two of them are in a first cluster and the third node is in
  a second cluster. In every case, one can check they form a weakly balanced triangle and therefore
  $G$ is weakly balanced.
\end{proof}
\begin{theorem}[Theorem 1 of \autocite{davis1967clustering}]
  A graph $G=(V,E)$ is weakly balanced if and only if all its cycles are weakly balanced.
\end{theorem}

In real data though, we do not expect either strong or weak balance to hold, for they are fairly
demanding model. Indeed, three of the real networks considered in \autoref{chap:troll} have been
repeatedly shown to be unbalanced, although the extent of this unbalance depends on the importance
given to longer cycles compared with triads~\autocites{Facchetti2011isingmodel}{measureUnbalance14}%
{measureUnbalance17}.
% there are also \url{https://arxiv.org/abs/1509.04037} about measuring balance (or frustration
% index) which they claimed has been peer reviewed?
% For the same reason, in virtually any real instances of the \pcc{} problem, the optimal solution
% will incur some disagreements. 
\marginpars{The proof makes essential use of the strong balance (which is ok since next section also
deal with 2-CC) and the completeness of the graph (which is a far less realistic assumption) yet
it would be interesting to see if it can be extended.}
\Textcite[Section 5.5]{BookKleinberg2010}
nonetheless prove that even when only a fraction of the triangle are strongly balanced in a complete
graph, the two clusters structure is still present, although it does not cover the whole graph
any more.

\begin{theorem}
  Let $\epsilon < \nicefrac{1}{8}$ and $\delta = \sqrt[3]\epsilon$. If at least $1-\epsilon$ of all
  triangles in a signed complete graph are strongly balanced, then either
  \begin{enumerate}[(a),nosep]
    \item there is a set consisting of at least $1-\delta$ of the nodes in which at least $1-\delta$
      of all edges are positive, or else
    \item the nodes can be divided into two groups $L$ and $R$ such that
      \begin{enumerate}[(i),nosep]
	\item at least $1-\delta$ of the edges in $L$ are positive,
	\item at least $1-\delta$ of the edges in $R$ are positive, and
	\item at least $1-\delta$ of the edges between $L$ and $R$ are negative.
      \end{enumerate}
  \end{enumerate}
\end{theorem}

\label{text:cc_new_bias}
Using these results on weak balance, we can now make our new bias explicit. Drawing parallel with
the previous generative model, we assume that each node $u$ is endowed with an integer $\cluster(u)
\in \{1, \ldots, K\}$ that specifies its cluster and we let the undirected edge $u,v$ be positive if
$\cluster(u) = \cluster(v)$ and negative otherwise. In other words, this corresponds to picking an
arbitrary partition of $V$ and setting the signs of $E$ in such a way that the partition becomes a
consistent clustering. This can still be seen as a generative process because of the initial integer
drawn \uar{},\Todo{Although I'm not sure how I would write this likelihood to be fair} but one that
avoids the two pitfalls discussed before. Indeed, there is no decision to be made by nodes, since
the previous probability has been replaced by a binary function (either both nodes are in the same
cluster or not). Furthermore, since the clustering function is symmetric, the model is inherently
suited to undirected graphs.

\iffalse
We can conclude that balance theory helps characterize zero disagreement instances of \pcc{}.
However, how is \pcc{} useful in predicting edge sign? At first sight, the two problems seem to be quite
different, for the former is an unsupervised/agnostic clustering problem, while the latter is a
supervised classification problem. Yet, getting an optimal solution of \pcc{} on the training set
would provide a principled heuristic for \esp{}. Unfortunately, as we shall see in
\autoref{sub:state_of_the_art}, exact or fixed parameter solutions are not scalable and one has to
rely on approximations. Under the Unique Games Conjecture, the approximation factor is lower bounded by
$\Theta(\log n)$ in the worst case. However, this is still a viable option, as our bias is that we
will operate on instances that are to some extent balanced. Therefore, we are not in the worst case
and we can find some comfort in results about the noisy (\vpageref{sub:variants_and_extensions}) and
stable (\autoref{ssub:cc_under_stability_assumption}) settings.

Another connection we will leverage in \autoref{sec:low_stretch_trees_and_spanners} is that the
value of the \mind{} objective is a complexity measure of the \esp{} problem~\autocite[Section
4.1]{Cesa-Bianchi2012b}. Before that, we will also present some variants and extensions of \pcc{},
in \autoref{sub:variants_and_extensions}, some of them bearing similarities with the Chapter 3's problem.

strong balance random graph model \url{https://www.lri.fr/~yannis/randsigned.pdf}, line index is basically
the objective value of CC and if I'm reading their bound correctly,  basically you need to flip half
the edges of a random graph to make it balanced (also most of their results hold when the
probability of an existing edge is 1/2, but it was strengthened in a follow up dealing with weak
balance \url{http://people.maths.ox.ac.uk/harutyunyan/weak-balance.pdf})

For the 2 clusters case ($k=2$), characterization proven already in 1936 by \textcite{Konig36}, as noted in
\cite{Zaslavsky2012} (commenting on \cite{harary1953}: \enquote{Although Theorem 3 was anticipated by
\textcite[Theorem X.11]{Konig36}  without the terminology of signs, here is the
first recognition of the crucial fact that labelling edges by elements of a
group---specifically, the sign group---can lead to a general theory.})
and \cite{Huffner2010} (\enquote{\textcite{Konig36} proved the following characterization of
	balanced graphs. For a graph $G = (V, E)$, the following are equivalent:\marginpars{There is a
	proof in \autocite[p. 111]{BookKleinberg2010}, maybe I can rewrite it as well}
	\begin{enumerate}
		\item $V$ can be partitioned into two sets $V_1$ and $V_2$ called sides such that there is no
			negative edge $\{v, w\} \in E$ with both $v, w \in V_1$ or both $v, w \in V_2$ and no positive
			edge $\{v, w\}$ with $v \in V_1$ and $w \in V_2$ .
		\item $V$ can be colored with two colors such that for all $\{v, w\} \in E^-$, the vertices $v$
			and $w$ have different colors, and for all $\{v, w\} \in E^+$, the vertices $v$ and $w$ have
			the same color. The color classes correspond to the sides.
		\item $G$ does not contain cycles with an odd number of negative edges.
	\end{enumerate}
	Using the characterization by a coloring, it is easy to see that balance of a signed
graph can be checked in linear time by depth-first search.})
\fi

