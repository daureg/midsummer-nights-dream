Let us recall first the sign model we used for \dssn{}. Each node $u$ is endowed with two parameters
$p$ and $q$ drawn from an arbitrary joint distribution $\mu$ over $[0,1]\times [0,1]$. $p$ can be
interpreted as the tendency of $u$ to send positive edges to other nodes (\ie{} the
\enquote{niceness} of $u$), while $q$ can be interpreted as the tendency of $u$ to receive positive
edges from other nodes (\ie{} the \enquote{popularity} of $u$). This suggests that nodes in the
graph have a form of agency, and this imposes that edges are directed. We show experimentally that
failing to meet these two assumptions harms the performance of our previous method. This is not
surprising, for our bias is no more justified in that case. Therefore, we suggest a different bias,
drawing heavily on social balance theories, although we shall see later this holds for many
nonsocial graphs as well.

\subsection{Reliance on the sign generative model}
\label{sub:reliance_on_the_sign_generative_model}

The method presented in the previous chapter hinges crucially upon our sign generative model. Yet
one can imagine contexts where this model is not applicable, especially when nodes do not represent
human beings. One way to alleviate this issue was introduced with our online algorithm, where this
time, signs are generated by an arbitrary adversary. However, in that case, we are still facing two
limitations\marginpars{limitations is not the right word, maybe \emph{strong constraints}…}
\begin{enumerate}
	\item Our bias remains that the labeling is regular (recall this means informally that the all
		the outgoing signs from a given node tend to be the same, and likewise for the incoming signs),
		since irregularities are the cost payed by the adversary to make our algorithm mispredict. While
		this bias is well suited to social networks, other applications may require other
		bias\marginpars{namely the \pcc{} bias we'll introduce next}.
	\item In the online setting, we evaluate our performance by the regret\marginpars{ref to regret equation},
		whereas in general we are interested in more classical measures of predictive accuracy.
\end{enumerate}

\subsubsection{Experience on directed biological networks}
\label{ssub:experience_on_directed_biological_networks}

Take for instance a biological network.

According to \href{https://web.stanford.edu/class/cs224w/slides/handout-bionets.pdf}%
{this description of various biological networks}, \emph{gene regulatory network} are directed and
contains activation and inhibition links, as \href{https://en.wikipedia.org/wiki/Gene_regulatory_network#Overview}%
{showed on Wikipedia}. There are some online databases such as
\href{http://regulondb.ccg.unam.mx/menu/download/datasets/index.jsp}%
{RegulonDB} or \href{http://www.pathwaycommons.org/pcviz/}{Pathway Commons} (this last one provide
network in the BioPAX format, which can be visualized by
\href{http://www.cytoscape.org/}{CytoScape}).
Another source is the \href{https://www.ncbi.nlm.nih.gov/pmc/articles/PMC2708159/table/T1}{Table 1}
of \cite{BioSigned09}.
%DasGupta also describe (in
%http://www.sciencedirect.com/science/article/pii/S0303264706001419#sec14) how
%they built a directed signed network from this SBML file
%https://www.ncbi.nlm.nih.gov/pmc/articles/PMC1681468/bin/msb4100014-sd1.xml

other domain? coref, images (is it directed?), entity resolution…\marginpars{Actually it's difficult
to find other kind of graph because all constructed ones (like image, coreference,
deduplication) are inherently symmetric}


\subsection{Need for a directed graph}
\label{sub:need_for_a_directed_graph}

Having two parameters per node, one for outgoing edges and another one for incoming edges, clearly
targets directed graphs. Many online interactions are inherently directed, for instance friendship,
trust or communication. On the other hand, predicting edge signs in undirected graphs is an equally
relevant objective. A prime example of such a situation is when we are given $n$ objects, some
pairwise similarities them, and the similarity function itself, which unfortunately takes an
exorbitant time to be evaluated. There is an underlying graph and being able to predict the sign of
its undirected edges would save us expensive evaluations of the symmetric similarity function.

A trivial way to turn an undirected graph $G$ into a directed graph $G'$ is to let $V'=V$ and, for
every edge $(u,v)$ in $E$, to add both \euv{} and \evu{} to $E'$ with the same sign as $(u,v)$. In
terms of our generative model, this corresponds to putting all the probability mass of $\mu$ on the
$p=q$ diagonal. This is clearly not a very satisfactory solution, for it removes one degree of
freedom from the model. To illustrate this point, we conduct the following experiment. We use our
previous \dssn{} datasets and remove the edge direction. As mentioned earlier, for a few pair of
nodes, there are reciprocal edges of different signs, in which case we pick a sign arbitrarily. Given
those undirected graphs, we orient them using the approach described above and compare our method
with the \complowrank{} approach ran directly on the undirected graphs. This is a fair comparison,
for \complowrank{} is designed to works solely on undirected graphs.

As we can see in \autoref{tab:bias_exp_undir}, our methods perform worse than when running on the
original directed graphs. Looking at \uslpropGsec{} on a 15\% training set as an example, we observe
that the MCC decreases by one (\kiw{}) to almost ten (\epi{}) points. On the other hand, the
\complowrank{} seems to perform better. Recall however that because it is an undirected method in
the first place and because we roughly double the number of signed edges, a 15\% training size means
the algorithm gets to observe more signs and therefore perform better. This small experiment thereby
shows that although the performance remains decent, not having direction information hurts our bias.

\begin{table}[t]
\begin{adjustwidth}{-2cm}{}
\centering
% \setlength{\tabcolsep}{3pt}
% \scriptsize
\small
\caption[MCC on the six previous datasets with direction removed]{MCC results on the six datasets from \autoref{chap:troll}, after removing the directions as
    described in the main text.  These results are presented like in \autoref{tab:all_mcc}, except
    we have transposes the rows and the columns, and we show only three relevant methods.
\label{tab:bias_exp_undir}}
% \hspace*{-0.2in}%
{\renewcommand{\arraystretch}{0.9}%
\begin{tabular}{lrrrrrrr}
\toprule
$\frac{|\trainset{}|}{|E|}$ & $5\%$                       & $10\%$                    & $15\%$                    & $20\%$                    & $25\%$                    & $50\%$                    & $90\%$                    \\
\midrule
                            & \multicolumn{7}{c}{\aut{}} \\ \cmidrule(lr){2-8}
\uslpropGsec{}              & $\vfirstSig{22.2}\pm0.7$    & $\vfirstSig{28.6}\pm0.5$  & $\vfirst{32.0}\pm0.5$     & $\vsecondSig{34.4}\pm0.3$ & $\vsecondSig{36.9}\pm0.5$ & $41.7\pm0.4$              & $44.9\pm0.9$              \\
\usrule{}                   & $\vsecondSig{18.1}\pm0.7$   & $25.3\pm0.8$              & $29.7\pm0.4$              & $32.6\pm0.4$              & $35.8\pm0.4$              & $\vsecondSig{42.2}\pm0.4$ & $\vsecondSig{45.0}\pm1.0$ \\
\complowrank{}              & $17.3\pm0.5$                & $\vsecondSig{25.3}\pm0.3$ & $\vsecond{31.4}\pm0.5$    & $\vfirstSig{36.3}\pm0.8$  & $\vfirstSig{41.2}\pm1.2$  & $\vfirstSig{55.4}\pm0.8$  & $\vfirstSig{60.6}\pm3.9$  \\[2pt]
    & \multicolumn{7}{c}{\adv{}} \\ \cmidrule(lr){2-8}
\uslpropGsec{}              & $\vfirstSig{36.5}\pm0.4$    & $\vfirstSig{41.4}\pm0.5$  & $\vfirstSig{43.9}\pm0.4$  & $\vfirstSig{45.7}\pm0.5$  & $\vfirst{47.0}\pm0.5$     & $\vfirst{50.4}\pm0.4$     & $\vfirst{52.7}\pm0.7$     \\
\usrule{}                   & $\vsecondSig{32.8}\pm0.5$   & $\vsecondSig{38.9}\pm0.7$ & $\vsecondSig{41.8}\pm0.8$ & $\vsecondSig{44.3}\pm0.8$ & $\vsecond{46.3}\pm0.7$    & $\vsecond{49.8}\pm0.5$    & $\vsecond{52.6}\pm0.8$    \\
\complowrank{}              & $28.8\pm0.6$                & $32.2\pm0.4$              & $33.9\pm0.6$              & $35.0\pm0.5$              & $36.3\pm0.7$              & $41.1\pm1.0$              & $46.2\pm3.4$              \\[2pt]
    & \multicolumn{7}{c}{\wik{}} \\ \cmidrule(lr){2-8}
\uslpropGsec{}              & $\vfirstSig{36.3}\pm0.3$    & $\vfirstSig{42.6}\pm0.4$  & $\vfirstSig{45.8}\pm0.2$  & $\vfirst{47.8}\pm0.4$     & $\vfirstSig{49.2}\pm0.2$  & $\vfirst{52.5}\pm0.3$     & $\vsecond{54.1}\pm0.5$    \\
\usrule{}                   & $\vsecondSig{35.0}\pm0.6$   & $\vsecondSig{41.4}\pm0.4$ & $\vsecondSig{44.9}\pm0.3$ & $\vsecond{46.9}\pm0.5$    & $\vsecondSig{48.2}\pm0.4$ & $51.3\pm0.4$              & $52.7\pm0.6$              \\
\complowrank{}              & $30.6\pm0.7$                & $37.2\pm0.4$              & $40.7\pm0.5$              & $43.4\pm0.6$              & $45.8\pm0.6$              & $\vsecond{51.8}\pm0.7$    & $\vfirst{54.6}\pm4.0$     \\[2pt]
    & \multicolumn{7}{c}{\sla{}} \\ \cmidrule(lr){2-8}
\uslpropGsec{}              & $\vfirst{39.4}\pm0.1$       & $\vfirstSig{44.9}\pm0.2$  & $\vfirstSig{47.8}\pm0.1$  & $\vfirstSig{49.7}\pm0.1$  & $\vfirstSig{51.0}\pm0.1$  & $\vsecondSig{54.4}\pm0.1$ & $\vsecondSig{56.4}\pm0.2$ \\
\usrule{}                   & $35.5\pm0.3$                & $40.7\pm0.1$              & $43.8\pm0.2$              & $46.0\pm0.2$              & $47.3\pm0.2$              & $51.5\pm0.1$              & $54.0\pm0.4$              \\
\complowrank{}              & $\vsecond{39.0}\pm0.3$      & $\vsecondSig{41.8}\pm0.3$ & $\vsecondSig{44.8}\pm0.5$ & $\vsecondSig{48.0}\pm0.6$ & $\vsecondSig{50.5}\pm0.4$ & $\vfirstSig{57.2}\pm0.3$  & $\vfirstSig{59.1}\pm0.6$  \\[2pt]
    & \multicolumn{7}{c}{\epi{}} \\ \cmidrule(lr){2-8}
\uslpropGsec{}              & $\vfirstSig{45.1}\pm0.4$    & $\vsecond{50.0}\pm0.4$    & $\vsecondSig{52.6}\pm0.3$ & $\vsecondSig{54.4}\pm0.2$ & $\vsecondSig{55.8}\pm0.3$ & $\vsecondSig{59.3}\pm0.1$ & $\vsecondSig{62.0}\pm0.2$ \\
\usrule{}                   & $40.7\pm0.6$                & $47.2\pm0.5$              & $50.2\pm0.6$              & $53.1\pm0.2$              & $54.6\pm0.2$              & $58.6\pm0.1$              & $60.9\pm0.2$              \\
\complowrank{}              & $\vsecondSig{44.1}\pm0.1$   & $\vfirst{50.8}\pm0.7$     & $\vfirstSig{55.9}\pm0.5$  & $\vfirstSig{59.3}\pm0.4$  & $\vfirstSig{61.7}\pm0.2$  & $\vfirstSig{66.8}\pm0.2$  & $\vfirstSig{68.4}\pm0.8$  \\[2pt]
    & \multicolumn{7}{c}{\kiw{}} \\ \cmidrule(lr){2-8}
\uslpropGsec{}              & $\vfirstSig{34.4}\pm0.3$    & $\vfirstSig{37.1}\pm0.4$  & $\vfirstSig{37.9}\pm0.3$  & $\vfirstSig{38.4}\pm0.6$  & $\vsecond{38.5}\pm0.4$    & $\vsecondSig{38.9}\pm0.3$ & $\vsecondSig{39.4}\pm0.3$ \\
\usrule{}                   & $\vsecondSig{28.5}\pm0.3$   & $\vsecondSig{32.9}\pm0.2$ & $\vsecondSig{34.8}\pm0.1$ & $35.5\pm0.2$              & $36.1\pm0.1$              & $37.2\pm0.1$              & $37.6\pm0.3$              \\
\complowrank{}              & $26.7\pm0.4$                & $30.9\pm0.4$              & $34.2\pm0.6$              & $\vsecondSig{36.6}\pm0.8$ & $\vfirst{38.7}\pm0.7$     & $\vfirstSig{44.0}\pm0.4$  & $\vfirstSig{46.3}\pm1.8$  \\[2pt]
\bottomrule
\end{tabular}}
\end{adjustwidth}
\end{table}


