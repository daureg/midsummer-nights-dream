Given a graph $G$, and following the social balance theory roughly summarized as \enquote{the enemy
of my enemy is my friend}, assume that the labelling of $E$ is consistent with a two-clustering of
$V$. Namely, assume that $V$ can be partitioned in two clusters such that edges within each cluster
are positive and edges across clusters are negative. In that case, the following
\emph{multiplicative rule} holds: for any nodes $u$, $v$ in $V$, and any path $p$ between $u$ and
$v$ in $G$, the sign $\yuv$ is equal to the product of the signs along $p$ (we call this product the
parity of $p$ and denote it $\pi(p)$). While it is a simple and convenient model, this is too strong
of a requirement to be satisfied in practice. Therefore, we relax it by assuming that we start with
a consistent labeling $Y$ but can only observe a randomly perturbed version $Y'$ of $Y$.
Specifically, given a constant $q\in [0, \nicefrac{1}{2})$, each sign of $Y$ is flipped with a
probability smaller than $q$.

In this section, we are interested in active learning algorithms, that first query a subset
$\etrain$ of the edges, observe the signs in $\etrain$ and use them to predict the remaining signs.
More precisely, we focus on an algorithm that queries spanning tree $T$ of $G$ and predicts the sign
of an edge $(u,v) \in \etest = E \setminus E_T$ as the parity of $\pathtuv$. While this
can\Todo{state it formally then ;)} be stated more formally~\autocite[Section
4.1]{Cesa-Bianchi2012b}, it makes sense intuitively that since each sign has been potentially
flipped, the longer the path in $T$, the more likely its parity will be not be equal to the true
sign $\yuv{}$. Therefore we would like each such path to be as short as possible. In the following,
we describe a way to build spanning trees tailored for this situation.

\subsection{\gtx{}: a spanning tree designed for sign prediction}
\label{sub:gtx_a_simple_low_stretch_tree_construction}

The \gtx{} algorithm takes as input a graph topology $G_0=(V_0, E_0)$ and produces a sequence of
graphs $\{G_t\}_{t=1}^T$ of decreasing size until each connected component of $G_0$ is reduced to a
single node. As we will prove later, after reaching this point, the algorithm has selected $|V_0| -
1$ edges that form a spanning tree of $G_0$. $G_{t+1}$ is obtained from $G_t$ by composing two
primitives so that we can informally write $G_{t+1} = \left(\collapseStar{} \circ
\extractStar{}\right)(G_t)$.

\extractStar{} partitions the graph $G_t$ into a set of stars and \collapseStar{} build the graph
made of those stars using the edges in $E_t$. We provide more details on those two operations in the
following, as well as their complexity analysis. Then we state formally the \gtx{} algorithm and prove
its termination and correctness.  Finally, we study its properties, such as the number of iterations
needed to finish and the stretch of the resulting tree. For simplicity and without loss of
generality, we assume that $G_0$ consist of a single connected component.

\medskip

\extractStar{} takes as input a graph $G_t=(V_t, E_t)$, and optionally a \emph{threshold function}
$t_f$ or a \emph{degree function} $d_f$. While the nodeset $V_t$ is not exhausted, it repeatedly samples a
star center $c_i$, creates a star $S_i^t$ with the neighbors of $c_i$, removes all the nodes of $S_i^t$ from
$V_t$ and all the edges incident to $S_i^t$ from $E_t$, and finally decrements accordingly the
degree of the 2-hop neighbors of $c_i$ (see \autoref{fig:gtx_star_simple} for a visual
representation of this notation).
\begin{marginfigure}
  \centering
  \includegraphics[height=0.15\textheight]{assets/tikz/gtx_star_tikz.pdf}
  \caption[A sample star]{A sample star created during the \tth{} extraction level. The black node
    % \tikz{\node[vertex,rare] {$c_i$};}
    is the center $c_i$ of the star $S_i^t$, which is made of the four light gray peripheral nodes
  % \tikz{\node[vertex,medium] {$p_1$};} to \tikz{\node[vertex,medium] {$p_4$};}
  as well as the solid edges. The 2-hops neighbors of $c_i$ are the white nodes
  % \tikz{\node[vertex] {$h_1$};} to \tikz{\node[vertex] {$h_3$};}
  $h_1$ to $h_3$, whose degree will decrease once we $S_i^t$ is removed from $G_t$.}
  \label{fig:gtx_star_simple}
\end{marginfigure}
Upon completion, it returns a list of stars and a mapping of
each node of $V_t$ to the unique star it belongs to. We consider three heuristics to choose centers:

\begin{itemize}%[nosep]
  \item choose the node with the current highest degree, with ties broken arbitrarily
  \item if $n_i$ is the number of node remaining in $V_t$ before choosing the \ith{} center, choose
    a node \uar{} among those with a degree larger than $t_f(n_i)$. Setting the threshold function
    to be the identity therefore recovers the previous strategy, but the idea here is to choose
    among a small set of high degree nodes, for instance by letting $t_f(n) = \sqrt{n}$
  \item if $\degr(u)$ is the degree of node $u$, choose node proportionally to $d_f(\degr(u))$.
    Again, the degree function is designed so that it favors the selection of high degree nodes. For
    instance, one could use $d_f(\degr(u)) = \degr(u)^2$.
\end{itemize}

We now give the pseudo code of \extractStar{} for the highest degree variant.\footnote{Note that for
clarity, we removed some bookkeeping code in all listings, mainly the part related to maintaining
mapping between nodes at different level of contraction. However, the full python implementation
is available at \url{https://github.com/daureg/magnet/blob/master/veverica/new_galaxy.py\#L27}.}
We assume that $G$ is the adjacency list of the graph, so that $G[u]$ is the set of neighbors of
$u$, \ie{} $G[u] \equiv \mathcal{N}(u)$. The other piece of notation is $\textsc{Star}$, which
simply create a star given a center and a list of peripheral nodes.  \vspace{-\baselineskip}

\begin{center}
  \rule{\textwidth}{.3pt}
  \begin{algorithmic}[1]
    \Function{\extractStar{}}{$G_t=(V_t,E_t)$}
      \State Let $Q$ be a max-priority queue. The key of element $x$ is $Q[x]$
      \Let{$stars$}{[]}
      \Let{$remaining$}{$\emptyset$}
      \ForAll{node $u$ in $V_t$}
        % \State $\textsc{Insert}\\left(Q,\,u\right)$ \Comment with the key $\degr(u)$
        \State \Call{Insert}{$Q,\,u$} \Comment with the key $\degr(u)$
        \Let{$remaining$}{$remaining \bigcup \left\{u\right\}$}
      \EndFor
      \While{$Q$ is not empty}
        \Let{$c_i$}{\Call{Extract-Max}{$Q$}}
        \If{$c_i$ not in $remaining$}
          \State \textbf{continue} \Comment{$c_i$ is part of an existing star so there is
          nothing to do}
        \EndIf
        \Let{$periphery$}{$G[c_i] \bigcap remaining$}
        \Let{$stars$}{$stars \bigcap $\Call{Star}{$c_i,\, periphery$}}
        \Let{$remaining$}{$remaining \setminus\left\{c_i\right\} \setminus periphery$}
        \For{$p$ in $periphery$}
          \For{$h$ in $G[p] \bigcap remaining$}
            \State \Call{Decrease-Key}{$Q,\, h,\, Q[h]-1$}
          \EndFor
        \EndFor
      \EndWhile
      \State \textbf{return} $stars$
    \EndFunction
  \end{algorithmic}
  \rule{\textwidth}{.3pt}
\end{center}

\extractStar{} terminates because at each iteration of the while loop line 8, we remove one node
from $Q$ and never add any. Let us analyze the complexity when $|V_t|=n$ and $|E_t|=m$. We first
build a priority queue of all the nodes sorted by their degree (line 5--7), which takes $O(n)$ time.
Then, at each iteration of the inner loop, we find the center of the next star by extracting the
maximum of the queue (line 9), we build the corresponding star (line 12--14) and we decrease the
priority (\ie the degree) of all nodes adjacent to the new star (line 15--17).  Since both
operations require constant time when using a Strict Fibonacci Heap~\autocite{FibonacciHeaps12} and
there are $n$ iterations of that loop, a coarse approximation of the runtime of \extractStar{} is
$O(n^2)$. However, observe that there can be at most $m$ decrease operations (since after that, all
nodes still in the queue have an effective degree of $0$, meaning that $periphery$ will the empty
set and lines 13--17 will run in constant time), reducing the complexity to $O(m+n)$.

The other two variants are more time consuming because they require additional bookkeeping. Their
randomization make them useful in an adversarial context but it also renders their analysis more
challenging, not necessarily for the runtime of \extractStar{} but mostly for the tree construction.
Therefore, we only briefly describe the implementations here.\footnote{Although they are available
online at 
\nolinkurl{https://github.com/daureg/magnet/blob/master/veverica/}%
\{\href{https://github.com/daureg/magnet/blob/master/veverica/ThresholdSampler.py}%
{ThresholdSampler.py}, \href{https://github.com/daureg/magnet/blob/master/veverica/NodeSampler.py}%
{NodeSampler.py}\}.} For the threshold function, we
maintain two queues, $high$ and $low$, containing nodes whose degree is respectively above and below
the current threshold. We select a node \uar{} in $high$, remove the corresponding star from $G_t$,
recompute the new threshold and if necessary, move nodes which fell under the threshold from $high$
to $low$ and those who climb above the threshold from $low$ to $high$. For the degree function, we
can draw any node as center proportionally to its weight (where the weight of node $u$ is defined as
$d_f\left(\degr(u)\right)$), but we cannot use the standard method of computing the cumulative sum
of weights since each iteration change some of them. Therefore, we construct a binary tree whose
leaves are the nodes of $V_t$ and where each tree nodes maintain the sum of weights in its left and
right subtrees. To sample, we draw a random number between $0$ and the total weight of the tree.
When degrees are updated (or graph node removed), we update the weights along a path from the
corresponding leaves to the root of the tree.

\medskip

The second routine, \collapseStar{} takes as input the result of \extractStar{}, along with $E_t$
and an optional $\emph{eccentricity}$ array we will describe soon. It builds a new graph $G_{t+1}$
where each star becomes a node and there is a link between two nodes $s_1$ and $s_2$ if the nodes
making up $s_1$ and $s_2$ are connected in $E_t$. For that, we first shuffle $E_t$ and go through
it. When we find an edge whose endpoints belong to two different stars not yet connected, we use
that edge to connect these two stars. This trivially takes $O(m)$ times.

A variant instead keeps track of all edges connecting each pair of stars to choose one that will
best contribute to our low stretch objective. Namely, when connecting two stars, we would prefer to
join their centers rather than two peripheral points. For that we maintain an eccentricity count for
all of the nodes of the original $G_0$, which is incremented by $1$ each time a node is chosen to be
on the periphery of a star.\Todo{link that to the walk through example.}
For each pair of stars, we thus choose the edge across them with minimal sum of its endpoints'
eccentricity. This requires another pass over the edges, preserving the $O(m)$ runtime.

\begin{figure}[htbp]
  \centering
  \includegraphics[width=0.78\linewidth]{tikz/gtx_eccentricity_tikz.pdf}
  \caption[The hierarchical structure of stars created by \gtx{}]{%
    The execution of the \gtx{} algorithm. The original graph is made of the solid edges
    connecting the nodes labeled by  their index. Edges forming the final spanning tree are in black
    while the others are in gray. The four shades of gray, from white to dark gray
    denote increasing node eccentricity (as computed at the end of the algorithm). The \ith{} star
    created during the \jth{} iteration of the algorithm is denoted $S_i^j$. Refer to the main text
    for a complete walk through.}
  \label{fig:gtx_eccentricity}
\end{figure}

\begin{figure}[bthp]
  \centering
  \begin{subfigure}[b]{0.47\textwidth}
    \centering
    \includegraphics[height=5cm]{tikz/gtx_run_level1_tikz}
    \caption{Resulting graph after the first iteration}\label{fig:gtx_run1}
  \end{subfigure}~
  \begin{subfigure}[b]{0.47\textwidth}
    \centering
    \includegraphics[height=2.2cm]{tikz/gtx_run_level2_tikz}
    \caption{Resulting graph after the second iteration}\label{fig:gtx_run2}
    \vspace{\baselineskip}
    \includegraphics[height=2.2cm]{tikz/gtx_run_level3_tikz}
    \caption{Resulting graph after the third iteration}\label{fig:gtx_run3}
  \end{subfigure}~
  \caption{The other iterations of \gtx{}}\label{fig:gtx_run}
\end{figure}

\medskip

We illustrate the operation of the \gtx{} algorithm on small (and somewhat contrived) example.
Let us start with the initial graph $G_0$ depicted in \autoref{fig:gtx_eccentricity} and initialize
the eccentricity of all nodes to $0$. When running \extractStar{}, we see that the maximum degree is
$4$, achieved at nodes $\{1, 6, 11, 16, 21, 26, 31, 36, 41\}$. For the sake of simplicity, assume
nodes are picked according to their index. First, node $1$ is forms the star
$\textcolor{DodgerBlue}{S_1^1}$ with peripheral nodes $2$, $3$, $4$ and $5$. This increments the
eccentricity of those peripheral nodes by $1$. Then node $6$ forms its star
$\textcolor{DodgerBlue}{S_2^1}$ with $7$, $8$, $9$ and $10$. The process continues until node $41$ is
chosen to be the center of star $\textcolor{DodgerBlue}{S_9^1}$, at which point the max-priority
queue has been exhausted and \extractStar{} finishes.

We then call \collapseStar{}, with the eccentricity reducing variant. This will connect all possible
pairs of star. For instance, the edge between nodes $19$ and $29$ leads to the edge
between $\textcolor{DodgerBlue}{S_4^1}$ and $\textcolor{DodgerBlue}{S_6^1}$. This is actually the
only possible edge between $\textcolor{DodgerBlue}{S_4^1}$ and $\textcolor{DodgerBlue}{S_6^1}$.
Consider on the other hand the case of edges $(2, 6)$ and $(2, 9)$. They both connect
$\textcolor{DodgerBlue}{S_1^1}$ and $\textcolor{DodgerBlue}{S_2^1}$. Yet at this point of the algorithm,
the eccentricity of node $2$ is $1$, the eccentricity of node $6$ is $0$ and the eccentricity of node
$9$ is $1$. The edge $(2, 6)$ has therefore the smallest total eccentricity and is chosen to connect
$\textcolor{DodgerBlue}{S_1^1}$ and $\textcolor{DodgerBlue}{S_2^1}$. The full result of the
\collapseStar{} procedure can be seen on \autoref{fig:gtx_run1}.


\subsection{Related works}
\label{sub:gtx_related_works}

\label{sub:gtx_state_of_the_art}

Looking for a subgraph $H$ of $G$ that best preserve the distance in $G$ while being sparse is an old
problem, driven originally by network design in fields such as transportation~\autocite{RoadNetworks60}
and electrical circuits~\autocite{electricalNetworks60}. The way we define \enquote{preserving the
distance}, and the exact form of $H$ give rise to several problems, which we summarize later in
\autoref{tab:gtx_related_stretch} \vpageref{tab:gtx_related_stretch}. We first give some
definitions, then cover the most relevant problems in details, and finally give some pointers for
the others problems.

Let the distance between $u$ and $v$ in $G$ be
\begin{equation*}
  d_G(u,v) = \sum_{e \in \pathguv} \ell(e)\,,
\end{equation*}
where $\ell(e)$ is the \emph{length} of the edge $e$ and \pathguv{} is the shortest path between $u$
and $v$ in $G$. In the following, we consider only the uniform case, in which the length of an edge
is equal to its weight. The stretch of an edge $(u,v)$ in $H$ is defined as
\begin{equation*}
  \estr(u,v) = \frac{d_H(u,v)}{d_G(u,v)}.
\end{equation*}

We may then want to minimize the stretch of:
\begin{enumerate}[1),nosep]%,leftmargin=*]
  \item some pairs of nodes. That is, given $L$ and $R$ in $V$, minimize $\sum_{u \in L, v\in R}
    \estr(u,v)$
  \item all pairs of nodes corresponding to edges of $G$, \ie{} minimize $\sum_{(u,v) \in E}
    \estr(u,v)$
  \item all pairs of nodes, \ie{}  minimize $\sum_{(u,v) \in V^2} \estr(u,v)$
\end{enumerate}
Note that for an unweighted graph, the second problem reduces to minimizing $\sum_{(u,v) \in E}
|\pathhuv|$. If furthermore $H$ is tree, this is equivalent to minimize the second term of equation
\eqref{eq:stretch_mistakes}. Therefore we focus mainly of that definition of stretch, and consider
the other two only briefly.

The second point affecting the problem is the structure of $H$. The only requirements are that it
must be spanning all the nodes involved in the computation of the chosen stretch, and that $\forall
(u,v) \in E,\, d_H(u,v) \geq d_G(u,v)$. Beside that, $H$ can be a tree of $G$, a general subgraph of
$G$ or even a subset of $V^2$ (\ie{} containing edges not in $E$). We focus mainly on the first two
cases, since they are covered by the \gtx{} algorithm.

% Namely, let $G$ be a graph over vertex set $V$ with $|V|=n$ and edge set $E$. Furthermore, let $T$
% be a spanning tree of $G$ and $\etest{}$ the edges of $G$ not in $T$. Then we define the
% \emph{average test edge stretch} as $\frac{1}{|\etest{}|} \sum_{(u,v) \in \etest{}}
% |\mathrm{path}^T_{u,v}|$, where $|\mathrm{path}^T_{u,v}|$ is the unique path between $u$ and $v$ in
% $T$.


% However, \textcite[Section 3, page 453]{lognMetricBoundConf03} claim a $O(\log n)$ approximation so maybe I'm
% wrong. Turns out, they refer to a distribution over trees and this $O(\log n)$ is the expected
% stretch of a tree sampled from this distribution

% This defines two kind of structures, spanning trees and spanners (which are still sparse subgraphs
% yet containing more than $|V|-1$ edges).

\paragraph{Trees}
\label{par:trees}

One early mention of seeking a low-stretch spanning tree is given by \textcite{Requirements74},
albeit in more general form:
\begin{problem}[Optimal Communication Spanning Tree]
Given a set of nodes $V=\{v_1, \ldots, v_n\}$, a set of distances $d_{ij}$ and a set of requirements
$r_{ij}$ between $v_i$ and $v_j$, find a spanning tree connecting these $n$ nodes such that the
total cost of communication of the spanning tree is a minimum among all spanning trees. The cost of
communication for a pair of nodes is $r_{i,j}$ multiplied by the sum of the distances of arcs which
form the unique path connecting $v_i$ and $v_j$ in the spanning tree. The cost of a spanning tree is
the sum of costs over all pairs of nodes.
\end{problem}
For a weighted graph $G=(V,E,w)$, by letting $d_{ij} = w_{ij}$ and $r_{i,j}= \Ind{(i,j) \in E}$,
finding an Optimal Communication Spanning Tree thus amounts to finding a low-stretch spanning tree.
\autoref{tab:gtx_related} present a list of works where the stretch was improved.

We start with the seminal paper of \textcite{LowerBound95}. It touches on many topics, and frame the
problem in a game theoretic way but here we only focus on two of their results: a lower bound of
$\Omega(\log n)$ for the average stretch of any tree and their construction of a tree with $\exp
O(\sqrt{\log n\log\log n})$ average stretch in time $O(m^2)$. The lower bound follows from an
existing result in extremal graph theory~\autocite[pages 107--109]{ExtremalGraph04}: there is a
positive constant $a$ such that for all $n\in \Nbb$, one can construct a graph $G$ with $n$ vertices
and $2n$ edges such that every cycle $G$ has a length of at least $a\log n$. Now consider any
spanning tree $T$ of $G$.  While all the $n-1$ edges of $T$ have a stretch of $1$, the $n+1$
remaining ones form a cycle in $T$ hence in $G$ as well and thus incur a stretch of at least $a\log
n$. This shows that the average stretch is at least $\frac{1}{2}a\log n$.

They construct a low stretch spanning tree in a bottom up manner like the \gtx{} algorithm. First,
they extend the definition of stretch to multigraph~\autocite[Section 4]{LowerBound95} and then
describe a procedure to transform in linear time any multigraph $G$ with $n$ nodes to a multigraph
$G'$ on the same nodeset with at most $n(n+1)$ edges such the average stretch of $G'$ is at most
twice that of $G$~\autocite[Lemma 5.2]{LowerBound95}. The next ingredient is an algorithm to build a
low diameter decomposition of a multigraph $G$, parametrized by a number $x(n)$ depending of $n$. It
works by repeatedly selecting an arbitrary node and growing a ball around it until the number of
edges leaving the ball is at most a fraction $\nicefrac{1}{x(n)}$ of the number of edges with both
endpoints in the ball. The key property of this decomposition is that it yields a partition of $G$
in clusters such that the radius of each cluster is small (namely at most $O(x(n)\log n)$) and there
is most a fraction $\nicefrac{1}{x(n)}$ of edges between clusters. Finally, the iterative procedure
is a follows: once a partition has been built, we compute a shortest path spanning tree in each
cluster that are then collapsed into super nodes to form the next graph $G'$ and the process repeats.
Another difference from \gtx{}, besides the partition procedure, is that $G'$ is a multigraph,
taking into account the number of edges joining cluster, while \collapseStar{} picks only the most
direct one.

Another interesting idea from this paper is to consider a distribution over trees instead of a
single instance, especially when one is concern about the maximum stretch instead of the average
one. For instance, on a cycle with $n$ nodes, a tree is obtained by removing one edge, and that edge
incurs a stretch of $n-1$. The uniform distribution over such trees has a maximum stretch of
$2\left(1 - \frac{1}{n}\right)$~\autocite{circle2k89}.

\begin{table}[htbp]
  \centering
  \caption{Reproduction of Table 1 from~\autocite{Abraham2012}, showing the evolution of the best
  asymptotic average stretch over time.}\label{tab:gtx_related}
  \begin{tabular}{lll}
    \toprule
    work                      & average stretch                          & time                    \\
    \midrule
    \autocite{LowerBound95}   & $\exp(O(\sqrt{\log n\log\log n}))$       & $O(m^2)$                \\
    \autocite{LowerStretch05} & $O((\log n)^2 \log \log n)$              & $O(m \log^2 n)$         \\
    \autocite{nearlyTight08}  & $O(\log n(\log \log n)^3)$              & $O(m \log^2 n)$         \\
    % \autocite{nearlyTight08}  & $O(\log n \log\log n(\log\log\log n)^3)$ & $O(m^2)$                \\
    \autocite{TighterSDD11}   & $O(\log n(\log \log n)^3 )$              & $O(m \log n\log\log n)$ \\
    \autocite{Abraham2012}    & $O(\log n \log \log n)$                  & $O(m \log n\log\log n)$ \\
    \bottomrule
  \end{tabular}
\end{table}

The idea of recursively partitioning the graph and construction a low-stretch spanning tree in each
part is common to all the papers of \autoref{tab:gtx_related}. \Textcite{LowerStretch05} devise a
$(\delta, \epsilon)$-star decomposition such that all the stars have comparably low radius.
It was modified in~\autocite{nearlyTight08} to improve the stretch. Then \textcite{TighterSDD11}
improve the runtime by rounding the edge weights to the closest power of $2$ and using a modified
implementation of the Dijkstra's algorithm in the case of at most $k$ distinct edge
weights~\autocite{FastPathFewWeights10}. Finally, \textcite{Abraham2012} describe an even more
complex but tighter petal decomposition.
\iffalse
\begin{marginfigure}
  \centering
  \includegraphics[width=.95\textwidth]{assets/raw/star_decomp.pdf}
  \caption{Star decomposition (reproduced from Figure 1 of~\autocite{LowerStretch05})}
  \label{fig:gtx_star_decomp}
\end{marginfigure}

Special case of graph
series parallel
\enquote{In a subsequent paper, \textcite{seriesParallel06} proved that every series-parallel
unweighted graph admits a spanning tree of average stretch $O(log n)$. This bound is tight as it
matches the lower bound established in~\autocite{cutsTrees99}.}

more special cases are in~\autocite{specialCase14}, although it's for the minimum max stretch
$t^\star$.
\enquote{Note also that a number of particular graph classes (like interval
graphs, permutation graphs, asteroidal-triple–free graphs, strongly chordal graphs,
dually chordal graphs, and others) admit tree $t$-spanners for small values of $t$}
\fi


\paragraph{Spanners}
\label{par:spanners}

As we mentioned, by stopping the \gtx{} algorithm before it finishes, we obtain a set of edges
spanning the graphs that is not a tree. Such structure are called \emph{spanner}. More precisely,
the subgraph $H$ is said to be an $t$-spanner of $G$ if, for a parameter $t \geq 1$, and for every
pair $u, v \in V$ of vertices, it holds that $d_H(u, v) \leq t \cdot d_G(u, v)$. The problem was
introduced by \textcites{SpannerFirst89}{SpannerSecond89} and has been extensively studied since
then, for it has many applications in network design. It was also showed to be \NPh{} to
approximate~\autocite{SpannerNPHard07}. The most simple construction is a greedy
algorithm~\autocite{greedySpanner93} that works similarly to the minimum spanning tree construction.
Starting from an empty subgraph $H$, it goes through every edge $(u, v)$ of $G$ sorted by weight and
check if there is a path between $u$ and $v$ in $H$ of length at most $t$. If it is case the edge
$(u,v)$ is dropped, otherwise it is inserted in $H$. This results in a $(2t - 1)$-spanner with
$O(n^{1+1/t})$ edges, which is an optimal trade-off between those two quantifies.  Furthermore on
weighted graphs, the greedy spanner total weight is essentially optimal~\autocite{GreedyOpt16}.
However, the best implementation of it, using a dynamic data structure~\autocite{fastGreedy04} is
not scalable for it runs in $O(t n^{2+\nicefrac{1}{t}})$ and cannot easily be parallelized.
Parallelization therefore requires other kind of approaches~\autocites{parSpanner08}{parSpanner15}.
Recently, \textcite{Spanner17} showed how to obtain, for any $\epsilon > 0$, a $(2t - 1)$-spanner
with $O(n^{1+1/k}/\epsilon)$ edges in $t$ rounds, with probability at least $1 - \epsilon$.

\iffalse
\url{http://www.siam.org/meetings/da17/schedule.html} SODA 13B \url{http://dl.acm.org/citation.cfm?id=3039686}
for instance the Elkin paper~\autocite{Spanner17} \enquote{Our centralized randomized algorithm computes (with
probability close to 1), a $(2k - 1)$-spanner with $n \cdot (1 + O(\frac{\log k}{n}))$ edges in
$O(|E|)$ time, whenever $k = \Omega(\log n)$. Note that when $k = \omega(\log n)$, the number of
edges is $n(1+o(1))$, i.e., in this range the algorithm computes an ultra-sparse spanner in $O(|E|)$
time.} For instance, if $k=5\log n$, we get a $10\log n$-spanner with $n\left(1+O\left(\frac{\log\log
n}{n}\right)\right)$ edges in $O(|E|)$ time.


They have applications in computing approximately shortest
paths [9, 22, 28, 37], routing [48], distance oracles and
labeling schemes [49, 56, 36] and synchronization [7].

From 6:They also appear in biology in the process of reconstructing phylogenetic trees from
matrices, whose entries represent genetic distances among contemporary living species (H. J.
Bandelt, A. W. M. Dress, Reconstructing the Shape of a Tree from Observed Dissimilarity Data, Adv.
in Appl. Math. 7 (1986)). Robotics researchers have studied spanners under the constraints of
Euclidean geometry, where vertices of the graph are points in space, and edges are line segments
joining pairs of points (Chew), (Dobkin), $[DJ], [K], [KG], [LL]$. 

studied in 1; 4; 6; 9; 15; 19; 22; 24; 26; 28; 30; 31; 37; 43; 51; 52; 57; 58



\begin{tabulary}{\textwidth}{LLLLL}
  \toprule
  work  & average stretch & edge size                              & weighted & time                                             \\
  \midrule
  46    & $4k + 1$        & $O(n^{1+\nicefrac{1}{k}})$             & no       & polynomial                                       \\
  6     & $2k +1$         & $O(n\cdot \ceil{n^{\nicefrac{1}{k}}})$ & yes      & $O\left(m(n^{1+\nicefrac{1}{k}}+n\log n)\right)$ \\
  40    & $2k-1$          & $O(n^{1+\nicefrac{1}{k}}+n)$           & no       & $O(m)$                                           \\
  Elkin & $2k-1$          & $n \cdot (1 + O(\frac{\log k}{n}))$    & no       & $O(m)$                                           \\
  \bottomrule
\end{tabulary}

% 22: E. Cohen, "Fast algorithms for constructing t-spanners and paths with stretch t," Proceedings
% of 1993 IEEE 34th Annual Foundations of Computer Science, Palo Alto, CA, 1993, pp. 648-658.  doi:
% 10.1109/SFCS.1993.366822
% We construct t-spanners of size (number of edges) Õ(n 1+(2+\epsilon)/t )
% (for any \epsilon > 0 and t such that t/(2+\epsilon) is integral). These spanners can be constructed
% by a randomized algorithm that runs in Õ(mn (2+\epsilon)/t ) time.


% Halperin and Zwick [40].  Their deterministic algorithm, for an integer parame- ter k ≥ 1,
% computes a (2k − 1)-spanner with n 1+1/k + n edges in O(|E|) time. (Their result improved previous
% pioneering work by [46, 22].)

Describe the greedy algorithm of 6. Using (55 On Dynamic Shortest Paths Problems Liam Roditty, Uri
Zwick, 2004), it runs in $O(\alpha n^{2+\nicefrac{1}{\alpha}})$

peleg 2007 hardness results

46: Peleg, D. and Schäffer, A. A. (1989), Graph spanners. J. Graph Theory, 13: 99–116.
doi:10.1002/jgt.3190130114

study the problem on unweighted graph. Application to routing scheme (48: D. Peleg and E. Upfal, A
tradeoff between space and efficiency for routing tables. 20th ACM Symposium on the Theory of
Computing, Chicago (1988))
Special case for the complete graph weighted by the distance in a 2D plan. Existing $\sqrt{10}$
spanner for the $\ell_1$ metric (L. P. Chew, There is a planar graph almost as good as the complete
graph.  Proceedings of the 2nd ACM Symposium on Computational Geometry, (1986)) (improved to
$\sqrt{4+2\sqrt{2}}$ by N. Bonichon, C. Gavoille, N. Hanusse, L. Perkovic The stretch factor of
$\ell_1$ and $\ell_{\infty}$ Delaunay triangulations European Symposium on Algorithms (ESA) (2012))
and $\phi \pi$ for $\ell_2$ (D. P. Dobkin, S. J. Friedman, and K. J. Supowit, Delaunay graphs are
almost as good as complete graphs. 28th IEEE Symposium on the Foundations of Computer Science,
(1987)) (improved to $1.998$ by Ge Xia. 2011. Improved upper bound on the stretch factor of delaunay
triangulations. In Proceedings of the twenty-seventh annual symposium on Computational geometry
(SoCG '11)). See (Prosenjit Bose, Michiel Smid, On plane geometric spanners: A survey and open
problems, Computational Geometry, Volume 46, Issue 7, 2013) for more on the 2D case.

In undirected graph, finding a spanner with less than $k$ edges is \NPc{} (theorem 2.2)
For $k<n$, one can construct in polynomial time a $(4\log_k n +1)$ spanner with less than $kn$ edges
(theorem 2.4) giving for instance ($k=2$) $O(\log n)$ spanner with $O(n)$ edges and
($k=n^{\nicefrac{1}{r}}$, $r\geq 1$) a $(4r+1)$ spanner with $O(n^{1+\nicefrac{1}{r}})$ edges
(matching lower bound within constant factor). For every $d \geq 0$, the $d$-dimensional cube has a
$3$-spanner with fewer than $7\time 2^d$ edges (Lemma 2.10 from 47: D. Peleg and J. D. Ullman. An
optimal synchronizer for the hypercube. SIAM J. on Comput., 18:740–747, 1989)
For chordal graphs: for every $n$-vertex chordal graph there exists a $2$-spanner with
$O(n\sqrt{n})$ edges (matching lower bound), a $3$-spanner with $O(n \log n)$ edges and a $5$-spanner
with $O(n)$ edges.
Much more difficult for directed graph, according to Theorem 4.2: For every $t \geq 1$ there are
infinitely many $n$-vertex directed graphs for which every $t$-spanner requires
$\Omega(\nicefrac{n^2}{t^2})$ edges.

% check some surveys of the 80's
% http://pubsonline.informs.org/doi/abs/10.1287/trsc.18.1.1
% http://onlinelibrary.wiley.com/doi/10.1002/net.3230190305/full
% early solutions
% http://onlinelibrary.wiley.com/doi/10.1002/net.3230090104/full
% http://onlinelibrary.wiley.com/doi/10.1002/net.3230130309/full
% later solution?
% http://ieeexplore.ieee.org/document/81738


While they have many applications [see first paragraph of \url{https://arxiv.org/pdf/1401.2454.pdf},
which was later merged in a STOC'14 paper] (a major one being solving linear systems), in some
practical situations their advantages are less clear [from
\url{https://link.springer.com/chapter/10.1007/978-3-319-20086-6_16}\enquote{for reasonable inputs
the constant factors make the solver much slower than methods with higher asymptotic complexity.
One other aspect predicted by theory is confirmed by our findings: Spanning trees with lower
stretch indeed reduce the solver's running time. Yet, simple spanning tree algorithms perform
better in practice than those with a guaranteed low stretch.} this is improved by
\url{https://link.springer.com/chapter/10.1007%2F978-3-319-20086-6_17} although they seem to work
	mostly with the Laplacian of the tree ]
\fi

\paragraph{Other problems}

Finding low stretch trees and spanners with respect to the existing edges is the most relevant
problem when addressing the \esp{} problem. For the sake of completeness, we nonetheless give an
overview of some related problems.

For instance, \textcite{Johnson1978} define the following problem, where the stretch is defined over
all possible pairs of nodes\footnote{We adapt their notations
to match ours}:  
\begin{problem}[Network Design Problem]
  \label{prob:gtx_ndp}
  Given an undirected integer-weighted graph $G=(V, E, w)$, a budget $B\in\Nbb$ and a criterion
  threshold $C\in \Nbb$, does there exist a spanning subgraph $G'=(V, E')$ of $G$ with weight
  $w(E') \leq B$ and criterion value $F(G') \leq C$, where the criterion function $F(G')$ denotes
  the sum of the weights of the shortest paths in $G'$ between all vertex pairs?
\end{problem}
They prove that finding such a subgraph is \NPc{}, by exhibiting a reduction from the
\textsc{Knapsack} problem. They also prove that the less general problem of finding a spanning tree
on an unweighted graph, that is
\vspace{-.5\baselineskip}
\begin{problem}[Simple Network Design Problem]
  \autoref{prob:gtx_ndp} with $w$ being the equal to $1$ for all edges in $E$ and $B=|V|-1$.
\end{problem}%
\vspace{-.5\baselineskip}
\noindent is also \NPc{} by reduction from \textsc{Exact 3-Cover}.
However, it has recently been show that this Simple Network Design problem can be approximated to a
constant factor $6$~\autocite{AllPairStrech10}. Moreover, even when the graph is weighted,
\textcite{constantDistortion07} achieve a universal constant bound for any weighted graph.

Another problem appear when the low-stretch structure $H$ can include edges not in $G$ (as long as
the distances in $H$ remain larger than the distances in $G$). This is captured by the following
problem~\autocite{OptimalNetwork69}:
\begin{problem}[Optimal Network Problem]
  \label{prob:gtx_scott}
  Given a set $V$ of $n$ vertices, find a set of spanning edges $E\subset V^2$ that minimizes
  the sum of the length of the shortest paths  between all vertex pairs while the
  total length of the resulting network does not exceed some upper bound $B\in\Nbb$.
\end{problem}
This can be seen as a special case of \autoref{prob:gtx_ndp} with $G$ being the unweighted
$n$-complete graph. \Textcite{OptimalNetwork69} proposes a backtracking solution and two local search approximate
algorithms. Some early branch and bound heuristic solutions to \autoref{prob:gtx_scott} are surveyed
in~\autocite[Section 2.3.2]{networkDesignSurvey89} although they do not come with asymptotic
guarantee on the stretch. Furthermore, \textcite{optimApproxNP80} proves that for any $\epsilon \in
(0,1)$, finding a $|V|^{1-\epsilon}$ approximation is \NPc{}.
However, if we consider the average stretches over a distribution of trees, then this approximation
factor can be reduced to $\Theta(\log n)$~\autocite{lognMetricBoundConf03}.

Finally, the stretch can also be computed for a subset of the edges. This is useful in cases where
we have prior information on the importance of individual nodes or edges.  For instance,
\textcite{RamseyTree17} show that for every $t$, any $n$-nodes graph $G=(V,E)$ has a subset $S$ of
size at least $n^{1 - \nicefrac{1}{k}}$, and a spanning tree that has stretch $O ( k \log \log n)$
between any node in $S$ and any node in $V$. Likewise, \textcite{mLAST17} describe how to maintain a
light subgraph $H$ that minimizes the distance between pairs of source and sink that are given in an
online fashion.

As shown by \autoref{tab:gtx_related_stretch}, those problems defined in the seventies are still
being discussed nowadays in top tier conferences, proving their relevance and impact beyond the
\esp{} problem.

\setlength{\fullpage}{\textwidth+\marginparsep+\marginparwidth}
\begin{table}[htbp]
  \centering
  \caption{A summary of the lowest stretches achievable for various problems.
  \label{tab:gtx_related_stretch}}
  \begin{tabulary}{\fullpage}{LCCL}
    \toprule
    kind of stretch    & \multicolumn{2}{c}{only existing edges}  & extra edges allowed   \\
    \midrule
                       & tree                                     & not tree             &\\
    \cmidrule(r){2-3}
    some pairs         & $O(k\log\log n)$~\autocite{RamseyTree17} & \autocite[Section 4]{mLAST17} & --- \\
    all existing pairs & $O\left(\log n (\log\log n)\right)$~\autocite{Abraham2012}
		       & $(2t - 1)$-spanner, $O(n^{1+1/t})$ edges~\autocite{greedySpanner93}
		       & $\Theta(\log n)$ in expectation~\autocite{lognMetricBoundConf03} \\
    all possible pairs & $6$ for unweighted graphs \autocite{AllPairStrech10} and $O(1)$
                         in general \autocite{constantDistortion07}
		       & ---
		       & no need for extra edges in that case                              \\
    \bottomrule
  \end{tabulary}
\end{table}


\subsection{Empirical evaluation}
\label{sub:gtx_empirical_evaluation}

In this \nameref{sub:gtx_empirical_evaluation}, we provide empirical evidences of the properties of
\gtx{} over several classes of graph, and compare it with a \bfs{} baseline.\marginpars{If time
allows, it would be interesting to implement some methods of \vref{sub:gtx_state_of_the_art} and add
them to the comparison}\todo*{implement more low stretch methods} Namely, we consider three kinds of
graph topology (with both synthetic and real world instances that carry signs on their edges) and
evaluate $(i)$ what average stretch is reached by various trees and $(ii)$ how accurate is the sign
prediction.

\subsubsection{Graph topology}

The three kinds of topology we consider are:
\begin{description}
	\item[\grid{}] which are 2D lattices, where each node has four neighbors except on the boundary.
		The synthetic ones are square, while the \enquote{real world} ones represents the four neighbors
		pixel connectivity of the pictures showed in \autoref{fig:gtx_xp_bwpics}.
	\item[\lpa{}] which are built synthetically according to the model of \textcite{Barabasi1999}.
		While this does not follow the more rigorous specification of \textcite{PAmodel04}, informally,
		we start with a line graph of $m$ nodes and add node one by one until the graph consists of $n$
		nodes. Each time a new node is added, it is connected to $m$ of the existing nodes with a
		probability proportional to their degree. Here we choose $m=3.13$, that is when adding a new
		node, we pick $3$ or $4$ existing neighbors such the initial expected number of neighbors for
		each new nodes is $3.13$. Such graphs are quite sparse and have short diameter, thus providing a
		crude but reasonable approximation of online social networks. Therefore, the real world
		instances of the \lpa{} model are \wik{}, \sla{} and \epi{}\marginpars{as used in the first
		chapter}\todo*{add a ref to first chapter} along with \gplus{}. The last one is constructed from
		ego networks of \gplus{}\footnote{Available at
		\url{http://snap.stanford.edu/data/egonets-Gplus.html}} by keeping the largest connected
		component of users whose gender is known. Basic statistics of those real \lpa{} graphs are
		presented in (\autoref{tab:gtx_xp_dataset}). 
	\item[\triangle{}] which consists of a Delaunay triangulation of random 2D points\footnote{As
		implemented by the \textsf{graph-tool} library (\url{https://graph-tool.skewed.de})}.
\end{description}

\begin{table}[htpb]
	\centering
	\caption{Dataset description }\label{tab:gtx_xp_dataset}
	\begin{tabular}{lrrcc}
		\toprule
             & $|V|$  & $|E|$    & fraction of $+$ edges & $\frac{2|E|}{|V|\cdot(|V|-1)}$ \\
		\midrule
		\wik{}   & \np{7065}   & \np{99936}    & 78.5\%                & $4.00\cdot 10^{3}$             \\
		\gplus{} & \np{74917}  & \np{10130461} & 67.6\%                & $3.61\cdot 10^{3}$             \\
		\sla{}   & \np{82052}  & \np{498527}   & 76.4\%                & $1.48\cdot 10^{4}$             \\
		\epi{}   & \np{119070} & \np{701569}   & 83.2\%                & $9.90\cdot 10^{5}$             \\
		\bottomrule
	\end{tabular}
\end{table}

\begin{figure}[t]
	\centering
	\begin{subfigure}[b]{0.32\textwidth}
		\includegraphics[width=\textwidth]{gtx_exp/zmonastery}
	\end{subfigure}~
	\begin{subfigure}[b]{0.32\textwidth}
		\includegraphics[width=\textwidth]{gtx_exp/zworld}
	\end{subfigure}~
	\begin{subfigure}[b]{0.32\textwidth}
		\includegraphics[width=\textwidth]{gtx_exp/nips_poster}
	\end{subfigure}

	\begin{subfigure}[b]{0.32\textwidth}
		\includegraphics[width=\textwidth]{gtx_exp/zmonastery_bin}
		\caption{monastery}
	\end{subfigure}~
	\begin{subfigure}[b]{0.32\textwidth}
		\includegraphics[width=\textwidth]{gtx_exp/zworld_bin}
		\caption{world}
	\end{subfigure}~
	\begin{subfigure}[b]{0.32\textwidth}
		\includegraphics[width=\textwidth]{gtx_exp/nips_poster_bin}
		\caption{poster}
	\end{subfigure}

	\begin{subfigure}[b]{0.32\textwidth}
		\includegraphics[width=\textwidth]{gtx_exp/nips_logo}
	\end{subfigure}~
	\begin{subfigure}[b]{0.32\textwidth}
		\includegraphics[width=\textwidth]{gtx_exp/space}
	\end{subfigure}~
	\begin{subfigure}[b]{0.32\textwidth}
		\includegraphics[width=\textwidth]{gtx_exp/waterfall}
	\end{subfigure}

	\begin{subfigure}[b]{0.32\textwidth}
		\includegraphics[width=\textwidth]{gtx_exp/nips_logo_bin}
		\caption{logo}
	\end{subfigure}~
	\begin{subfigure}[b]{0.32\textwidth}
		\includegraphics[width=\textwidth]{gtx_exp/space_bin}
		\caption{space}
	\end{subfigure}~
	\begin{subfigure}[b]{0.32\textwidth}
		\includegraphics[width=\textwidth]{gtx_exp/waterfall_bin}
		\caption{waterfall}
	\end{subfigure}
	\caption{Real world pictures and their binarized version}\label{fig:gtx_xp_bwpics}
\end{figure}

\subsubsection{Stretch}

The first property of Galaxy trees we wish to evaluate is their stretch, which depends only of graph
topology. Namely, let $G$ be a graph over vertex set $V$ with $|V|=n$ and edge set
$E$.\Todo[MOVE]{Stretch definition is likely to happen somewhere earlier} Furthermore, let $T$ be a
spanning tree of $G$ and $\etest{}$ the edges of $G$ not in $T$. Then we define the \emph{average
test edge stretch} as $\frac{1}{|\etest{}|} \sum_{(u,v) \in \etest{}} |\mathrm{path}^T_{u,v}|$,
where $|\mathrm{path}^T_{u,v}|$ is the unique path between $u$ and $v$ in $T$.

As we consider unweighted graphs, we compare \gtx{} with a natural baseline, namely a spanning tree
rooted at the highest degree node and obtained through a breadth first visit of the graph. This
involves randomness in order in which nodes are visited. Likewise in \gtx{}, the choice of the edge
linking two stars is not always unique, meaning that we have to break ties at random.  Therefore,
for each graph, we repeat the tree construction 12 times and present the average result, noting that
the variance (showed as error bar in \autoref{fig:gtx_xp_st}) is small.

On \lpa{} and \triangle{}, we see that both trees exhibits logarithmic stretch, although with a
larger constant for \gtx{}. Note that this is also the case for others low stretch tree methods
\autocite[\S 5.3.1]{papplow}. On \grid{} however, \gtx{} preserves this logarithmic stretch growth
while this is visually no longer the case for \bfs{}.
In that case, we cannot expect a better stretch than $\frac{\log n}{2048}$ according to
\autocite[Theorem 6.6]{LowerBound95}.

\begin{figure}[tbh]
	\centering
	\begin{subfigure}[b]{0.9\textwidth}
		\includegraphics[width=\textwidth]{gtx_exp/gridst}
		\caption{\grid{} }\label{fig:gtx_xp_gridst}
	\end{subfigure}

	\begin{subfigure}[b]{0.9\textwidth}
		\includegraphics[width=\textwidth]{gtx_exp/past}
		\caption{\lpa{} }\label{fig:gtx_xp_past}
	\end{subfigure}

	\begin{subfigure}[b]{0.9\textwidth}
		\includegraphics[width=\textwidth]{gtx_exp/trst}
		\caption{\triangle{} }\label{fig:gtx_xp_trst}
	\end{subfigure}
	\caption{Stretch over graphs of increasing size}\label{fig:gtx_xp_st}
\end{figure}

\subsubsection{Sign prediction}

The second design goal of Galaxy trees is to accurately predict the sign of edges in $\etest{}$.
Except for the three real datasets that already include signs\footnote{We nonetheless perform some
preprocessing in order to make them undirected to remove the small proportion of conflicting edges
(e.g. positive from $u$ to $v$ but negative from $v$ to $u$).}, all the other are constructed,
meaning we have to set sign on their edges in the first place. This is done by partitioning the
nodes into two clusters. For \gplus{} we use node gender, for pictures we use node color (black or
white), and for all others, we propagate labels $0$ and $1$ from randomly selected high degree nodes.
Once each node belongs to one of the two clusters, we set the sign of an edge between two nodes to
be $+$ if they are in the same cluster and $-$ otherwise.  Predicting using path parity will thus
gives perfect result. To test performance in real or adversarial situation, we then add noise, that
is we select a fraction of edges uniformly at random and flip their sign. 

We evaluate the performance of our prediction using the Matthews Correlation\Todo{merge this
definition of MCC with the one in first chapter} Coefficient (MCC)~\autocite{MCC00} \[
	\mathrm{MCC} = \frac{ TP \times TN - FP \times FN } {\sqrt{ (TP + FP) ( TP + FN ) (
			TN + FP ) ( TN + FN ) } } = \pm \sqrt{\frac{\chi^2}{n}}
\]
Since we do not have confidence score, we cannot use AUC. Yet we have to account for the large sign
unbalance and thus cannot rely on accuracy or $F_1$ measure.  Therefore we choose MCC, which
combines all the four numbers of the confusion matrix in a single metric. It ranges from $+1$
(perfect prediction) to $-1$ (inverse prediction) through $0$ (random prediction). As a
demonstration of MCC usefulness, predicting all edges but one to be positive on Slashdot gives
$.764$ accuracy, $.886$ $F_1$ score\marginpars{Actually the $F_1$ score is $.866$ for positive
edges, $0$ for negative ones and $.661$ if we can take an average weighted by class size.} but
$-0.0007$ MCC.

As showed in \autoref{fig:gtx_xp_mcc}, when the noise level is low, \gtx{} performs better than
\bfs{}. As the noise level gets higher, they have similar performance. Note also than in
\autoref{fig:gtx_xp_pasynthmcc}, \gtx{} is less sensible to the size of the graph.

\begin{figure}[tbh]
	\centering
	\begin{subfigure}[b]{0.47\textwidth}
		\includegraphics[width=\textwidth]{gtx_exp/grsynthmcc}
		\caption{Synthetic \grid{} }\label{fig:gtx_xp_grsynthmcc}
	\end{subfigure}~
	\begin{subfigure}[b]{0.47\textwidth}
		\includegraphics[width=\textwidth]{gtx_exp/grrwmcc}
		\caption{Pictures \grid{} }\label{fig:gtx_xp_grrwmcc}
	\end{subfigure}
	\begin{subfigure}[b]{0.47\textwidth}
		\includegraphics[width=\textwidth]{gtx_exp/pasynthmcc}
		\caption{Synthetic \lpa{} }\label{fig:gtx_xp_pasynthmcc}
	\end{subfigure}~
	\begin{subfigure}[b]{0.47\textwidth}
		\includegraphics[width=\textwidth]{gtx_exp/trmcc}
		\caption{\triangle{} }\label{fig:gtx_xp_trmcc}
	\end{subfigure}
	\begin{subfigure}[b]{0.47\textwidth}
		\includegraphics[width=\textwidth]{gtx_exp/parwmcc}
		\caption{Real world network }\label{fig:gtx_xp_parwmcc}
	\end{subfigure}
	\caption{MCC over various graphs}\label{fig:gtx_xp_mcc}
\end{figure}

To further assess the quality of our trees, we plug them in them into a successful heuristic method
to predict edge sign: \asym{}~\autocite{Kunegis2009}. \Todo{It might also be interesting to see if
that would be a good training set for our troll method, although it has to be checked it makes sense
from a running time point of view.} It computes the exponential of the adjacency matrix after it has
been reduce to $z$ dimension. This allows to count the sign of all paths between two pairs of nodes
with decreasing weight depending of their length. To simulate an active learning setting, we reveal
only a subset of edge in $A$. This subset can be: $i)$ the edges forming a \bfs{}, $ii)$ the edges
forming a \gtx{} $iii)$ $|V|-1$ edges chosen uniformly at random.

We set the parameter $z$ equal to $15$ because $i)$ it is one of the best in \cite[Fig.
11]{Kunegis2009}, $ii)$ it performs well on real dataset in \cite[Fig.3]{Cesa-Bianchi2012a}, and
$iii)$ it was good in our initial testing (\texttt{20150401\_wed\_spectral.ipynb}).

As the \asym{} has a $O(n^3)$ complexity and uses quite some memory at prediction time, the larger
graphs used previously are not all included. The conclusion of \autoref{fig:gtx_xp_asym} is that
except on social network, it is better to use spanning trees than random edges. Specifically, \gtx{}
on \grid{} and \bfs{} elsewhere.

\begin{figure}[tbh]
	\centering
	\begin{subfigure}[b]{0.47\textwidth}
		\includegraphics[width=\textwidth]{gtx_exp/grsynthasym}
		\caption{Synthetic \grid{} \label{fig:gtx_xp_grsynthasym}}
	\end{subfigure}~
	\begin{subfigure}[b]{0.47\textwidth}
		\includegraphics[width=\textwidth]{gtx_exp/grrwasym}
		\caption{\enquote{Real} \grid{} }\label{fig:gtx_xp_grrwasym}
	\end{subfigure}
	\begin{subfigure}[b]{0.47\textwidth}
		\includegraphics[width=\textwidth]{gtx_exp/pasynthasym}
		\caption{Synthetic \lpa{} }\label{fig:gtx_xp_pasynthasym}
	\end{subfigure}~
	\begin{subfigure}[b]{0.47\textwidth}
		\includegraphics[width=\textwidth]{gtx_exp/trasym}
		\caption{\triangle{} }\label{fig:gtx_xp_trasym}
	\end{subfigure}
	\begin{subfigure}[b]{0.47\textwidth}
		\includegraphics[width=\textwidth]{gtx_exp/parwasym}
		\caption{Real world network }\label{fig:gtx_xp_parwasym}
	\end{subfigure}
	\caption{\asym{} over various graphs}\label{fig:gtx_xp_asym}
\end{figure}

Finally\marginpars{Actually I never did it because \shz{} wasn't implemented at the time, so now is
a good occasion}\todo*{Run shazoo on galaxy tree} we also compare \gtx{} with \bfs{} and \rst{} on
the task of nodes prediction using \shz{} algorithm~\autocite{Vitale2012}.


