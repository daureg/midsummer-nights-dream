We can interpret the description of our sign model as if, when establishing a new link, nodes were
deciding the sign of this link based on their individual preferences. While this makes sense in
social networks where nodes represent human beings, one can imagine contexts where this model is not
applicable. Another model, \eg{} for proteins, could be that proteins belongs to functional groups
and that two proteins interact positively if they belong to the same group, negatively otherwise.
We presented a way to circumvent this notion of node behavior with our online algorithm, where this
time, signs are generated by an arbitrary adversary. However, even in that case, our bias remains
that the labeling is regular. Recall this means informally that all the outgoing signs from a given
node tend to be the same, and likewise for the incoming signs. Indeed, irregularities are the cost
payed by the adversary to make our algorithm mispredict. In other words, regularity hints at a
consistent sign behavior from nodes, that we cannot assume carry out outside of social networks.
Pursuing the proteins example, we can imagine that proteins interact half of the time with proteins
of their own group, and half of time with proteins from other groups. This would correspond to
maximum irregularity, yet it is a plausible situation. We now show experimentally, that in fact,
biological networks do not lend themselves to our sign model bias.

% In the online setting, we evaluate our performance by the regret\marginpars{ref to regret equation},
% whereas in general we are interested in more classical measures of predictive accuracy.

Namely, we consider \emph{gene regulatory networks}. The nodes of such directed graphs are various
chemicals (such as genes, proteins or messenger RNAs) and a directed edge \euv{} indicates that one
node $u$ influences the concentration of another node $v$ through a chemical reaction. This
influence can be positive (that is, an increase of $u$ concentration results in an increase of $v$
concentration) or negative (that is, an increase of $u$ concentration results in a decrease of $v$
concentration). In this context, we could now interpret the \enquote{niceness} of $u$ as it ability
to accelerate chemical reactions, and its \enquote{popularity} as its propensity to take part in
reactions accelerated by other nodes. Intuitively, this is quite far-fetched. To demonstrate this,
we borrow the following three gene regulatory networks from \autocite[Table 1]{BioSigned09}, and
display their statistics in \autoref{tab:bias_bio_dataset}:
\begin{enumerate}[1.]
  \item The signaling pathways of the \hip{} CA1 neuron~\autocite{Hippocamp05}.
  \item The interactions between genes and their products that are involved in \can{} development in
    humans and mouses~\autocite{Cancer07}.
  \item The \reg{} database\autocite{RegulonDB16}, summarizing the connection between transcription
    factors and their targets in the Escherichia coli K-12 bacteria. We download four
    files\footnote{\url{http://regulondb.ccg.unam.mx/menu/download/datasets/index.jsp}} containing
    experimental evidence and keep only the positive and negative edges with \textsf{Strong}
    support.
\end{enumerate}
%\url{http://causalbionet.com/}

\begin{table}[bt]
  \centering
  \small
  \caption{Biological dataset properties. The columns have the same meaning as in
    \autoref{tab:dataset} \vpageref{tab:dataset}\label{tab:bias_bio_dataset}}
  \begin{tabular}{lrrrrrrrr}
    \toprule
    Dataset & \thead{$|V|$}       & \thead{$|E|$}       & \thead{$\frac{|E|}{|V|}$} &
    \thead{$\frac{|E^+|}{|E|}$} & \thead{$\frac{\Psi^2_{G''}(Y)}{|E|}$} &
    \thead{$\frac{\Psi_G(Y)}{|E|}$} & \thead{reciprocal\\ edges} &
    \thead{reciprocal\\ disagreement} \\
    \midrule                                                                                                   
    \hip{} &  \np{501} & \np{1046} &               2.1 &               69.5\% &                       .056 &                    .124 &       0.2\% &              0.0\% \\
    \can{} & \np{1240} & \np{3065} &               2.5 &               78.4\% &                       .047 &                    .108 &       5.5\% &             27.1\% \\
    \reg{} & \np{1700} & \np{2570} &               1.5 &               50.3\% &                       .060 &                    .148 &       0.0\% &              0.0\% \\
    \bottomrule
  \end{tabular}
\end{table}

\begin{table}[t]
\centering
\setlength{\tabcolsep}{3pt}
\scriptsize
\caption{This table is the same as in \autoref{tab:all_mcc} \vpageref{tab:all_mcc}, but this time on
three smaller, directed biological networks.
\label{tab:cc_bio_exp}}
% \hspace*{-0.2in}%
\begin{tabular}{lrcccc|ccccc}
\toprule
                                                  & $\frac{|\trainset{}|}{|E|}$ &             \uslpropGsec{} &                     \usrule{} &     \uslogregp{} &       \usoptim{} &             \complowrank{} &              \compmaxnorm{} &    \comptriads{} &             \compranknodes{} &               \compbayesian{} \\
\midrule
\multirow{9}{*}{\rotatebox[origin=c]{90}{\hip{}}} & $5\%$  &   $\vfirst{14.1}\pm5.7$  &     $\vsecond{11.5}\pm6.3$  &   $9.6\pm6.7$  &   $8.5\pm4.7$  &            $10.9\pm6.3$  &              $2.3\pm3.4$  &   $3.7\pm4.0$  &              $11.4\pm7.4$  &                $5.4\pm5.2$  \\
                                                  & $9\%$  &  $\vsecond{18.1}\pm6.1$  &               $16.9\pm6.0$  &  $15.4\pm4.2$  &  $12.2\pm4.2$  &            $15.3\pm4.9$  &              $1.8\pm4.6$  &   $7.0\pm3.3$  &     $\vfirst{21.2}\pm4.8$  &               $15.0\pm5.7$  \\
                                                  & $15\%$ &  $\vsecond{20.3}\pm5.5$  &               $18.6\pm4.9$  &  $17.6\pm4.4$  &  $14.3\pm4.2$  &            $18.0\pm4.5$  &              $2.7\pm2.6$  &  $12.1\pm4.9$  &     $\vfirst{24.4}\pm3.7$  &               $19.5\pm4.5$  \\
                                                  & $20\%$ &            $22.0\pm4.3$  &               $21.1\pm4.0$  &  $19.0\pm3.1$  &  $17.7\pm4.0$  &            $17.2\pm2.4$  &              $3.0\pm3.4$  &  $15.9\pm3.6$  &  $\vfirstSig{26.7}\pm3.5$  &  $\vsecondSig{23.2}\pm3.5$  \\
                                                  & $25\%$ &            $24.4\pm3.8$  &               $23.2\pm3.2$  &  $21.5\pm3.6$  &  $19.2\pm2.6$  &            $21.5\pm3.5$  &              $3.2\pm3.4$  &  $18.5\pm3.6$  &  $\vfirstSig{28.6}\pm4.7$  &  $\vsecondSig{24.7}\pm3.7$  \\
                                                  & $50\%$ &            $30.5\pm4.5$  &  $\vsecondSig{32.1}\pm4.4$  &  $32.0\pm4.8$  &  $28.1\pm4.2$  &            $31.9\pm4.0$  &             $16.0\pm3.9$  &  $31.5\pm4.1$  &  $\vfirstSig{36.4}\pm3.9$  &               $31.2\pm4.1$  \\
                                                  & $90\%$ &            $34.4\pm5.8$  &               $35.7\pm7.1$  &  $35.5\pm7.4$  &  $31.3\pm7.8$  &  $\vfirst{40.8}\pm12.7$  &  $\vsecond{39.9}\pm11.4$  &  $36.3\pm9.0$  &              $35.9\pm6.6$  &               $37.2\pm9.3$  \\
\midrule
\multirow{9}{*}{\rotatebox[origin=c]{90}{\can{}}} & $5\%$  &   $\vfirst{15.6}\pm4.0$  &  $12.9\pm4.1$  &            $10.6\pm3.9$  &   $9.1\pm2.9$  &  $13.0\pm2.4$  &   $0.7\pm1.9$  &   $5.7\pm4.6$  &           $14.2\pm5.1$  &  $\vsecond{14.7}\pm3.3$  \\
                                                  & $10\%$ &  $\vsecond{21.7}\pm2.6$  &  $18.0\pm3.2$  &            $15.9\pm3.1$  &  $12.1\pm2.6$  &  $16.9\pm2.7$  &   $1.0\pm2.2$  &  $10.3\pm4.3$  &  $\vfirst{21.7}\pm2.9$  &            $15.6\pm3.4$  \\
                                                  & $15\%$ &  $\vsecond{24.6}\pm2.3$  &  $21.3\pm3.5$  &            $20.1\pm2.2$  &  $15.5\pm2.0$  &  $19.6\pm2.7$  &   $0.5\pm2.0$  &  $14.8\pm2.6$  &  $\vfirst{24.8}\pm3.2$  &            $20.3\pm3.0$  \\
                                                  & $20\%$ &  $\vsecond{26.5}\pm2.0$  &  $24.0\pm2.3$  &            $24.2\pm2.2$  &  $19.7\pm2.4$  &  $22.7\pm3.4$  &   $3.5\pm2.5$  &  $18.8\pm3.2$  &  $\vfirst{27.4}\pm2.7$  &            $22.4\pm2.5$  \\
                                                  & $25\%$ &  $\vsecond{29.3}\pm2.1$  &  $25.9\pm1.6$  &            $26.3\pm2.1$  &  $21.4\pm1.2$  &  $24.8\pm2.9$  &   $7.2\pm2.6$  &  $21.2\pm3.0$  &  $\vfirst{29.6}\pm2.0$  &            $24.8\pm3.0$  \\
                                                  & $50\%$ &            $34.6\pm1.7$  &  $34.7\pm1.9$  &  $\vsecond{34.8}\pm1.7$  &  $32.0\pm2.9$  &  $31.3\pm3.2$  &  $22.4\pm2.6$  &  $31.7\pm1.8$  &  $\vfirst{35.3}\pm3.3$  &            $33.7\pm2.0$  \\
                                                  & $90\%$ &            $39.8\pm4.4$  &  $42.4\pm4.5$  &            $42.8\pm4.7$  &  $39.3\pm5.3$  &  $34.4\pm5.3$  &  $34.3\pm4.3$  &  $40.6\pm6.5$  &  $\vfirst{43.9}\pm3.9$  &  $\vsecond{42.8}\pm4.9$  \\
\midrule
\multirow{9}{*}{\rotatebox[origin=c]{90}{\reg{}}} & $5\%$  &               $26.5\pm8.9$  &               $31.5\pm4.7$  &  $20.4\pm13.6$  &  $29.0\pm2.9$  &      $\vfirst{32.6}\pm4.1$  &  $-0.5\pm2.2$  &   $7.7\pm7.6$  &    $\vsecond{31.7}\pm7.3$  &    $5.0\pm7.7$  \\
                                                  & $10\%$ &               $39.3\pm4.9$  &      $\vfirst{43.6}\pm2.9$  &   $38.5\pm5.2$  &  $37.2\pm2.3$  &               $40.1\pm2.3$  &  $-2.3\pm2.2$  &  $19.6\pm7.2$  &    $\vsecond{43.3}\pm3.7$  &    $8.2\pm5.6$  \\
                                                  & $15\%$ &               $44.9\pm2.1$  &  $\vsecondSig{45.6}\pm2.4$  &   $40.9\pm6.0$  &  $40.2\pm2.1$  &               $44.9\pm2.9$  &   $0.5\pm2.3$  &  $29.9\pm4.9$  &  $\vfirstSig{49.2}\pm3.0$  &   $17.4\pm9.6$  \\
                                                  & $20\%$ &  $\vsecondSig{46.8}\pm2.5$  &               $46.1\pm2.3$  &   $42.0\pm3.3$  &  $42.0\pm2.2$  &               $46.1\pm2.9$  &   $3.0\pm2.7$  &  $35.3\pm5.1$  &  $\vfirstSig{51.4}\pm3.5$  &  $20.3\pm11.5$  \\
                                                  & $25\%$ &  $\vsecondSig{49.1}\pm2.4$  &               $48.4\pm3.1$  &   $44.4\pm2.6$  &  $43.8\pm2.7$  &               $48.2\pm2.0$  &   $5.6\pm2.1$  &  $38.6\pm5.6$  &  $\vfirstSig{54.5}\pm2.1$  &   $32.7\pm8.5$  \\
                                                  & $50\%$ &               $52.4\pm2.8$  &               $48.7\pm2.3$  &   $44.6\pm3.2$  &  $47.4\pm2.8$  &  $\vsecondSig{55.8}\pm1.8$  &  $26.0\pm2.6$  &  $50.3\pm4.6$  &  $\vfirstSig{59.7}\pm1.7$  &   $40.0\pm6.7$  \\
                                                  & $90\%$ &               $56.4\pm6.2$  &               $50.4\pm5.6$  &   $47.9\pm6.1$  &  $47.2\pm5.6$  &      $\vfirst{65.8}\pm5.5$  &  $47.2\pm6.4$  &  $59.7\pm6.1$  &    $\vsecond{64.0}\pm4.7$  &   $49.5\pm6.5$  \\
\bottomrule
\end{tabular}
\end{table}


Compared with the \dssn{} of the previous chapter, those biological are smaller, have lower average
degree and little to no reciprocal edges. Trying to solve the \esp{} problem using the same
procedure as in \autoref{s:exp}, we read in \autoref{tab:cc_bio_exp} that our methods have lower
absolute MCC performance (for instance, \uslpropGsec{} with a 15\% training size achieves $20$, $24$
and $45$ compared with values between $36$ and $61$ on \dssn{}).  To further illustrate the mismatch
between the troll bias and biological networks, notice that the standard deviation figures are
several times larger than those reported in \autoref{tab:all_mcc}.

