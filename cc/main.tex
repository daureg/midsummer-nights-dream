\label{chap:cc}

In the previous chapter, we presented a learning bias for the \esp{} problem, namely our sign
generative model. By leveraging its rigorous theoretical guarantees, we have demonstrated
empirically its good performance on real \dssn{}. However, in this chapter, we consider other kinds
of signed networks and show they require another bias and different, more combinatorial algorithms.
More precisely, we start by pointing in \autoref{sec:new_bias} that our previous bias may not apply
to directed graphs from domains other than social science, or to undirected graphs. Motivated by
balance theories, we then introduce a new bias, assuming that ideally, nodes are grouped in $K$
clusters such that all edges within clusters are positive and all edges between cluster are
negative. Recovering such clusters from a given signed graph has been studied extensively in the
last decades under the \pcc{} name. In \autoref{sec:correlation_clustering}, we show how this
difficult combinatorial clustering problem is connected to our learning bias. We survey a wide range
of methods to solve it exactly, approximately or heuristically, paying special attention to settings
where such recovery is easier, such as noisy or stable instances.  Finally, focusing on the
important and convenient special case $K=2$, we develop in \autoref{sec:gtx} an algorithm that,
given an undirected general graph, produce a spanning tree designed to support fast and accurate
edge sign prediction.

\section{A bias for general graphs}
\label{sec:new_bias}

Let us recall first the sign model we used for \dssn{}. Each node $u$ is endowed with two parameters
$p_u$ and $q_u$ drawn from an arbitrary joint distribution $\mu$ over $[0,1]\times [0,1]$. $p_u$ can be
interpreted as the tendency of $u$ to send positive edges to other nodes (\ie{} the
\enquote{niceness} of $u$), while $q_u$ can be interpreted as the tendency of $u$ to receive positive
edges from other nodes (\ie{} the \enquote{popularity} of $u$). The sign of an edge \euv{} is then
chosen to be positive with probability $\frac{p_u+q_v}{2}$. This suggests that nodes in the
graph have a form of agency, and this imposes that edges are directed. We show experimentally that
failing to meet these two assumptions harms the performance of our previous method. This is not
surprising, for our bias is no more justified in that case. Therefore, we suggest a different bias,
drawing heavily on social balance theories, although we shall see later this holds for many
nonsocial graphs as well.

\subsection{Sign generative model and behavior}
\label{sub:bias_sign_behavior}

We can interpret the description of our sign model as if, when establishing a new link, nodes were
deciding the sign of this link based on their individual preferences. While this makes sense in
social networks where nodes represent human beings, one can imagine contexts where this model is not
applicable. Another model, \eg{} for proteins, could be that proteins belongs to functional groups
and that two proteins interact positively if they belong to the same group, negatively otherwise.
We presented a way to circumvent this notion of node behavior with our online algorithm, where this
time, signs are generated by an arbitrary adversary. However, even in that case, our bias remains
that the labeling is regular. Recall this means informally that all the outgoing signs from a given
node tend to be the same, and likewise for the incoming signs. Indeed, irregularities are the cost
payed by the adversary to make our algorithm mispredict. In other words, regularity hints at a
consistent sign behavior from nodes, that we cannot assume carry out outside of social networks.
Pursuing the proteins example, we can imagine that proteins interact half of the time with proteins
of their own group, and half of time with proteins from other groups. This would correspond to
maximum irregularity, yet it is a plausible situation. We now show experimentally, that in fact,
biological networks do not lend themselves to our sign model bias.

% In the online setting, we evaluate our performance by the regret\marginpars{ref to regret equation},
% whereas in general we are interested in more classical measures of predictive accuracy.

Namely, we consider \emph{gene regulatory networks}. The nodes of such directed graphs are various
chemicals (such as genes, proteins or messenger RNAs) and a directed edge \euv{} indicates that one
node $u$ influences the concentration of another node $v$ through a chemical reaction. This
influence can be positive (that is, an increase of $u$ concentration results in an increase of $v$
concentration) or negative (that is, an increase of $u$ concentration results in a decrease of $v$
concentration). In this context, we could now interpret the \enquote{niceness} of $u$ as it ability
to accelerate chemical reactions, and its \enquote{popularity} as its propensity to take part in
reactions accelerated by other nodes. Intuitively, this is quite far-fetched. To demonstrate this,
we borrow the following three gene regulatory networks from \autocite[Table 1]{BioSigned09}, and
display their statistics in \autoref{tab:bias_bio_dataset}:
\begin{enumerate}[1.]
  \item The signaling pathways of the \hip{} CA1 neuron~\autocite{Hippocamp05}.
  \item The interactions between genes and their products that are involved in \can{} development in
    humans and mouses~\autocite{Cancer07}.
  \item The \reg{} database~\autocite{RegulonDB16}, summarizing the connection between transcription
    factors and their targets in the Escherichia coli K-12 bacteria. We download four
    files\footnote{\url{http://regulondb.ccg.unam.mx/menu/download/datasets/index.jsp}} containing
    experimental evidence and keep only the positive and negative edges with \textsf{Strong}
    support.
\end{enumerate}
%\url{http://causalbionet.com/}

\begin{table}[bt]
  \centering
  \small
  \caption{Biological dataset properties. The columns have the same meaning as in
    \autoref{tab:dataset} \vpageref{tab:dataset}\label{tab:bias_bio_dataset}}
  \begin{tabular}{lrrrrrrrr}
    \toprule
    Dataset & \thead{$|V|$}       & \thead{$|E|$}       & \thead{$\frac{|E|}{|V|}$} &
    \thead{$\frac{|E^+|}{|E|}$} & \thead{$\frac{\Psi^2_{G''}(Y)}{|E|}$} &
    \thead{$\frac{\Psi_G(Y)}{|E|}$} & \thead{reciprocal\\ edges} &
    \thead{reciprocal\\ disagreement} \\
    \midrule                                                                                                   
    \hip{} &  \np{501} & \np{1046} &               2.1 &               69.5\% &                       .056 &                    .124 &       0.2\% &              0.0\% \\
    \can{} & \np{1240} & \np{3065} &               2.5 &               78.4\% &                       .047 &                    .108 &       5.5\% &             27.1\% \\
    \reg{} & \np{1700} & \np{2570} &               1.5 &               50.3\% &                       .060 &                    .148 &       0.0\% &              0.0\% \\
    \bottomrule
  \end{tabular}
\end{table}

\begin{table}[t]
\centering
\setlength{\tabcolsep}{3pt}
\scriptsize
\caption{This table is the same as in \autoref{tab:all_mcc} \vpageref{tab:all_mcc}, but this time on
three smaller, directed biological networks.
\label{tab:cc_bio_exp}}
% \hspace*{-0.2in}%
\begin{tabular}{lrcccc|ccccc}
\toprule
                                                  & $\frac{|\trainset{}|}{|E|}$ &             \uslpropGsec{} &                     \usrule{} &     \uslogregp{} &       \usoptim{} &             \complowrank{} &              \compmaxnorm{} &    \comptriads{} &             \compranknodes{} &               \compbayesian{} \\
\midrule
\multirow{9}{*}{\rotatebox[origin=c]{90}{\hip{}}} & $5\%$  &   $\vfirst{14.1}\pm5.7$  &     $\vsecond{11.5}\pm6.3$  &   $9.6\pm6.7$  &   $8.5\pm4.7$  &            $10.9\pm6.3$  &              $2.3\pm3.4$  &   $3.7\pm4.0$  &              $11.4\pm7.4$  &                $5.4\pm5.2$  \\
                                                  & $9\%$  &  $\vsecond{18.1}\pm6.1$  &               $16.9\pm6.0$  &  $15.4\pm4.2$  &  $12.2\pm4.2$  &            $15.3\pm4.9$  &              $1.8\pm4.6$  &   $7.0\pm3.3$  &     $\vfirst{21.2}\pm4.8$  &               $15.0\pm5.7$  \\
                                                  & $15\%$ &  $\vsecond{20.3}\pm5.5$  &               $18.6\pm4.9$  &  $17.6\pm4.4$  &  $14.3\pm4.2$  &            $18.0\pm4.5$  &              $2.7\pm2.6$  &  $12.1\pm4.9$  &     $\vfirst{24.4}\pm3.7$  &               $19.5\pm4.5$  \\
                                                  & $20\%$ &            $22.0\pm4.3$  &               $21.1\pm4.0$  &  $19.0\pm3.1$  &  $17.7\pm4.0$  &            $17.2\pm2.4$  &              $3.0\pm3.4$  &  $15.9\pm3.6$  &  $\vfirstSig{26.7}\pm3.5$  &  $\vsecondSig{23.2}\pm3.5$  \\
                                                  & $25\%$ &            $24.4\pm3.8$  &               $23.2\pm3.2$  &  $21.5\pm3.6$  &  $19.2\pm2.6$  &            $21.5\pm3.5$  &              $3.2\pm3.4$  &  $18.5\pm3.6$  &  $\vfirstSig{28.6}\pm4.7$  &  $\vsecondSig{24.7}\pm3.7$  \\
                                                  & $50\%$ &            $30.5\pm4.5$  &  $\vsecondSig{32.1}\pm4.4$  &  $32.0\pm4.8$  &  $28.1\pm4.2$  &            $31.9\pm4.0$  &             $16.0\pm3.9$  &  $31.5\pm4.1$  &  $\vfirstSig{36.4}\pm3.9$  &               $31.2\pm4.1$  \\
                                                  & $90\%$ &            $34.4\pm5.8$  &               $35.7\pm7.1$  &  $35.5\pm7.4$  &  $31.3\pm7.8$  &  $\vfirst{40.8}\pm12.7$  &  $\vsecond{39.9}\pm11.4$  &  $36.3\pm9.0$  &              $35.9\pm6.6$  &               $37.2\pm9.3$  \\
\midrule
\multirow{9}{*}{\rotatebox[origin=c]{90}{\can{}}} & $5\%$  &   $\vfirst{15.6}\pm4.0$  &  $12.9\pm4.1$  &            $10.6\pm3.9$  &   $9.1\pm2.9$  &  $13.0\pm2.4$  &   $0.7\pm1.9$  &   $5.7\pm4.6$  &           $14.2\pm5.1$  &  $\vsecond{14.7}\pm3.3$  \\
                                                  & $10\%$ &  $\vsecond{21.7}\pm2.6$  &  $18.0\pm3.2$  &            $15.9\pm3.1$  &  $12.1\pm2.6$  &  $16.9\pm2.7$  &   $1.0\pm2.2$  &  $10.3\pm4.3$  &  $\vfirst{21.7}\pm2.9$  &            $15.6\pm3.4$  \\
                                                  & $15\%$ &  $\vsecond{24.6}\pm2.3$  &  $21.3\pm3.5$  &            $20.1\pm2.2$  &  $15.5\pm2.0$  &  $19.6\pm2.7$  &   $0.5\pm2.0$  &  $14.8\pm2.6$  &  $\vfirst{24.8}\pm3.2$  &            $20.3\pm3.0$  \\
                                                  & $20\%$ &  $\vsecond{26.5}\pm2.0$  &  $24.0\pm2.3$  &            $24.2\pm2.2$  &  $19.7\pm2.4$  &  $22.7\pm3.4$  &   $3.5\pm2.5$  &  $18.8\pm3.2$  &  $\vfirst{27.4}\pm2.7$  &            $22.4\pm2.5$  \\
                                                  & $25\%$ &  $\vsecond{29.3}\pm2.1$  &  $25.9\pm1.6$  &            $26.3\pm2.1$  &  $21.4\pm1.2$  &  $24.8\pm2.9$  &   $7.2\pm2.6$  &  $21.2\pm3.0$  &  $\vfirst{29.6}\pm2.0$  &            $24.8\pm3.0$  \\
                                                  & $50\%$ &            $34.6\pm1.7$  &  $34.7\pm1.9$  &  $\vsecond{34.8}\pm1.7$  &  $32.0\pm2.9$  &  $31.3\pm3.2$  &  $22.4\pm2.6$  &  $31.7\pm1.8$  &  $\vfirst{35.3}\pm3.3$  &            $33.7\pm2.0$  \\
                                                  & $90\%$ &            $39.8\pm4.4$  &  $42.4\pm4.5$  &            $42.8\pm4.7$  &  $39.3\pm5.3$  &  $34.4\pm5.3$  &  $34.3\pm4.3$  &  $40.6\pm6.5$  &  $\vfirst{43.9}\pm3.9$  &  $\vsecond{42.8}\pm4.9$  \\
\midrule
\multirow{9}{*}{\rotatebox[origin=c]{90}{\reg{}}} & $5\%$  &               $26.5\pm8.9$  &               $31.5\pm4.7$  &  $20.4\pm13.6$  &  $29.0\pm2.9$  &      $\vfirst{32.6}\pm4.1$  &  $-0.5\pm2.2$  &   $7.7\pm7.6$  &    $\vsecond{31.7}\pm7.3$  &    $5.0\pm7.7$  \\
                                                  & $10\%$ &               $39.3\pm4.9$  &      $\vfirst{43.6}\pm2.9$  &   $38.5\pm5.2$  &  $37.2\pm2.3$  &               $40.1\pm2.3$  &  $-2.3\pm2.2$  &  $19.6\pm7.2$  &    $\vsecond{43.3}\pm3.7$  &    $8.2\pm5.6$  \\
                                                  & $15\%$ &               $44.9\pm2.1$  &  $\vsecondSig{45.6}\pm2.4$  &   $40.9\pm6.0$  &  $40.2\pm2.1$  &               $44.9\pm2.9$  &   $0.5\pm2.3$  &  $29.9\pm4.9$  &  $\vfirstSig{49.2}\pm3.0$  &   $17.4\pm9.6$  \\
                                                  & $20\%$ &  $\vsecondSig{46.8}\pm2.5$  &               $46.1\pm2.3$  &   $42.0\pm3.3$  &  $42.0\pm2.2$  &               $46.1\pm2.9$  &   $3.0\pm2.7$  &  $35.3\pm5.1$  &  $\vfirstSig{51.4}\pm3.5$  &  $20.3\pm11.5$  \\
                                                  & $25\%$ &  $\vsecondSig{49.1}\pm2.4$  &               $48.4\pm3.1$  &   $44.4\pm2.6$  &  $43.8\pm2.7$  &               $48.2\pm2.0$  &   $5.6\pm2.1$  &  $38.6\pm5.6$  &  $\vfirstSig{54.5}\pm2.1$  &   $32.7\pm8.5$  \\
                                                  & $50\%$ &               $52.4\pm2.8$  &               $48.7\pm2.3$  &   $44.6\pm3.2$  &  $47.4\pm2.8$  &  $\vsecondSig{55.8}\pm1.8$  &  $26.0\pm2.6$  &  $50.3\pm4.6$  &  $\vfirstSig{59.7}\pm1.7$  &   $40.0\pm6.7$  \\
                                                  & $90\%$ &               $56.4\pm6.2$  &               $50.4\pm5.6$  &   $47.9\pm6.1$  &  $47.2\pm5.6$  &      $\vfirst{65.8}\pm5.5$  &  $47.2\pm6.4$  &  $59.7\pm6.1$  &    $\vsecond{64.0}\pm4.7$  &   $49.5\pm6.5$  \\
\bottomrule
\end{tabular}
\end{table}


Compared with the \dssn{} of the previous chapter, those biological networks are smaller, have lower average
degree and little to no reciprocal edges. Trying to solve the \esp{} problem using the same
procedure as in \autoref{s:exp}, we read in \autoref{tab:cc_bio_exp} that our methods have lower
absolute MCC performance (for instance, \uslpropGsec{} with a 15\% training size achieves $20$, $24$
and $45$ compared with values between $36$ and $61$ on \dssn{}).  To further illustrate the mismatch
between the troll bias and biological networks, notice that the standard deviation figures are
several times larger than those reported in \autoref{tab:all_mcc}.



\subsection{Directed edges requirement}
\label{sub:need_for_a_directed_graph}

A more serious limitation is that our sign model only applies to directed graphs. Consider
another model, where each node $i$ is endowed with an integer $c_i \in \{1, \ldots, k\}$ that
specifies its cluster and let the sign of the undirected edge $i,j$ be $+$ if $c_i = c_j$ and $-$
otherwise.  This corresponds naturally to the situation modeled by the \pcc{} problem.



\section{\pcc{}}
\label{sec:correlation_clustering}

% *TODO*
% say more clearly there are 3 methods for approximation so far: LP & region growing; LP & distance
% pivot; combinatorial pivot
% say that in practice, general graphs are assumed to be complete with missing edges being negative
% - “The problem is agnostic in the sense that the clustering of the data is not taken into account
%   or even assumed to exist.” another way to look at it is that we're trying to get a binary
%   classifier of edges (+ or -) using an hypothesis class (nodes clustering) with limited
%   expression power (as it has to respect transitivity constraints, whereas signs can be inconsistent)
% - when the graph is weighted, there is no much point separating complete from
%   general since weights can be 0

In this section, we introduce the \pcc{} problem (\autoref{sub:problem_setting}) and motivate it
both by general applications (\autoref{ssub:cc_applications}) and by its close connection with the
\esp{} problem (\autoref{sub:relation_with_edge_sign_prediction}). After discussing its theoretical
complexity and various approximation methods (\autoref{ssub:cc_harness_approx}) along other kinds of
methods (\autoref{ssub:cc_methods}), we will review related problems stemming from the original
formulation (\autoref{sub:variants_and_extensions}). In doing so, we present a thorough and updated
counterpart to existing surveys the same
topic~\autocites{bonchi2014correlation}{surveyCC16}{CCWirth2017}.

% surveyCC16 is about mind on complete graph (ie cluster editing). They give a Russian article from
% 1971 that says it cluster editing on triangle-free graph can be solved in polynomial time, which
% include bipartite graph (so for CC, if G^+ is (complete? I think so) bipartite)
% Fridman, G.Š.: A graph approximation problem. Upravlyaemye Sistemy 8, 73–75 (1971). (in Russian)
% In [14] it was observed that the problem is polynomially solvable for instances of at most 2
% clusters??! Actually it means consensus clustering can be solved in polynomial time when there are
% only 2 clustering as input

\subsection{Problem setting and applications}
\label{sub:problem_setting}
\input{cc_problem_applications}

\subsection{Relation with edge sign prediction}
\label{sub:relation_with_edge_sign_prediction}
Now that we are familiar with the \pcc{} problem and its numerous applications, let us return to the
original problem of this chapter, predicting the edge sign of a signed graph. In order to do so, we
will first focus on the case where a signed graph has no disagreements (that is the minimum of
\eqref{eq:mind} is $0$), and the connection with the balance theory.

In his seminal work on interpersonal relations,
\citeauthor*{Heider46}~\autocites{Heider46}{HeiderBook58} argues by
psychological and sociological arguments that in order to reduce their cognitive dissonance, three
people always interact in a way that preserve social balance. This social balance requirements can be
succinctly summarized by four statements~\autocite{HeiderBook58}:

\begin{enumerate}
	\item \textcolor{DodgerBlue}{my friend's} \textcolor{Orange}{friend} is
		\textcolor{Olive}{my friend}\hspace{1em}%
		\tikz[baseline=5]{
			\node[draw,circle,inner sep=2pt] (me) at (0,0) {};
			\node[draw,circle,inner sep=2pt] (mine) at (.5,.5) {};
			\node[draw,circle,inner sep=2pt] (alter) at (1,0) {};
			\draw[] (me) -- node[left,DodgerBlue,pos=.9] {$+$} (mine) ;
			\draw[] (mine) -- node[right,Orange,pos=.1] {$+$} (alter) ;
			\draw[white] (me) -- node[text=Olive] {$+$} (alter) ;
		}
	\item \textcolor{DodgerBlue}{my friend's} \textcolor{Orange}{enemy} is
		\textcolor{Olive}{my enemy}\hspace{1em}%
		\tikz[baseline=5]{
			\node[draw,circle,inner sep=2pt] (me) at (0,0) {};
			\node[draw,circle,inner sep=2pt] (mine) at (.5,.5) {};
			\node[draw,circle,inner sep=2pt] (alter) at (1,0) {};
			\draw[] (me) -- node[left,DodgerBlue,pos=.9] {$+$} (mine) ;
			\draw[] (mine) -- node[right,Orange,pos=.1] {$-$} (alter) ;
			\draw[white] (me) -- node[text=Olive] {$-$} (alter) ;
		}
	\item \textcolor{DodgerBlue}{my enemy's} \textcolor{Orange}{friend} is
		\textcolor{Olive}{my enemy}\hspace{1em}%
		\tikz[baseline=5]{
			\node[draw,circle,inner sep=2pt] (me) at (0,0) {};
			\node[draw,circle,inner sep=2pt] (mine) at (.5,.5) {};
			\node[draw,circle,inner sep=2pt] (alter) at (1,0) {};
			\draw[] (me) -- node[left,DodgerBlue,pos=.9] {$-$} (mine) ;
			\draw[] (mine) -- node[right,Orange,pos=.1] {$+$} (alter) ;
			\draw[white] (me) -- node[text=Olive] {$-$} (alter) ;
		}
	\item \textcolor{DodgerBlue}{my enemy's} \textcolor{Orange}{enemy} is
		\textcolor{Olive}{my friend}\hspace{1em}%
		\tikz[baseline=5]{
			\node[draw,circle,inner sep=2pt] (me) at (0,0) {};
			\node[draw,circle,inner sep=2pt] (mine) at (.5,.5) {};
			\node[draw,circle,inner sep=2pt] (alter) at (1,0) {};
			\draw[] (me) -- node[left,DodgerBlue,pos=.9] {$-$} (mine) ;
			\draw[] (mine) -- node[right,Orange,pos=.1] {$-$} (alter) ;
			\draw[white] (me) -- node[text=Olive] {$+$} (alter) ;
		}
\end{enumerate}

This can be readily translated into graph properties. Given a path of length two over three nodes,
note that the first part of each statement (in blue and orange) defines the four possible sign
assignment of such a path. The last part of the statement (in green) then prescribes which sign
should close the triangle to respect the social balance. As we can see, this means that triangles can
only have zero or two negative signs, or equivalently, that no triangle can have an odd number
number of negative signs.

From this local observation, we define a graph as strongly balanced if none of its cycle has an odd
number of negative signs
in that case we have \textcite[Theorem 3]{harary1953} stating:
statement
proof
According to \textcite{Huffner2010}, a similar theorem was proved earlier by \textcite[Theorem
X.11]{Konig36}, although \textcite{Zaslavsky2012} notes that it was stated \enquote{without the
	terminology of signs, while \autocite{harary1953}
has the first recognition of the crucial fact that labelling edges by elements of a
group---specifically, the sign group---can lead to a general theory.}

when graph is strongly balanced, the 2 clusters of previous claim gives a natural partition with no
disagreements in the sense of \pcc{}. However, the existence of 2 such clusters is only a sufficient
but not necessary condition for having no disagreement. As hinted in
\autoref{sub:need_for_a_directed_graph}, when the nodes are partitioned into $k$ clusters, letting
edges within clusters to be positive and edges across clusters to be negative will by definition
results in zero disagreement. \Textcite{davis1967clustering} show that this $k>2$ clusters partition
is equivalent to the absence of cycle with a single negative edge, namely
statement
proof
In terms of balance, this corresponds to allowing triangle to have three negative edges, and is called
weak balance, as illustrated on \autoref{fig:cc_balance}.

\begin{figure}[htpb]
	\centering
	\includegraphics[width=0.8\linewidth]{assets/raw/triangle_smaller.pdf}
	\caption{The four possible undirected triads, as classified by the two structural balance theories
	introduced in the main text} \label{fig:cc_balance}
\end{figure}

In real data though, we do not expect neither strong nor weak balance to hold, for they are fairly
demanding model. Indeed, three of the real networks considered in \autoref{chap:troll} have been
repeatedly shown to be unbalanced, although the extent of this unbalance depends of the importance
given to longer cycle compared with triads~\autocites{Facchetti2011isingmodel}{measureUnbalance14}%
{measureUnbalance17} For the same reason, in virtually any real instance of the \pcc{} problem, the
optimal solution will incur some disagreements. \Textcite[Section 5.5]{BookKleinberg2010}
nonetheless prove that even when only a fraction of the triangle are strongly balanced in a complete
graph, the two clusters structure is still present, although it does not cover the whole graph
anymore. Formally statement
The proof makes essential use of the strong balance (which is ok since next section also deal with
2-CC) and the completeness of the graph (which is a far less realistic assumption) yet it would be
interesting to see if it can be extended.

Said that, how is \pcc{} useful in predicting edge sign? At first sight, the two problems seem quite
different for the former is an unsupervised/agnostic clustering problem while the latter is a
supervised classification problem. Yet, getting an optimal solution of \pcc{} on the training set
would provide a principled heuristics for \esp{}. Unfortunately, as we shall see in
\autoref{sub:state_of_the_art}, exact or fixed parameter solution are not scalable and one has to
rely approximation. Under the Unique Games Conjecture, the approximation factor is lower bounded by
$\Theta(\log n)$ in the worst case. However, this is still a viable option, as our bias is that we
will operate on instances that are to some extent balanced. Therefore, we are not in the worst case
and we can find some comfort in results about the noisy (\vpageref{sub:variants_and_extensions}) and
stable (\autoref{ssub:cc_under_stability_assumption}) settings.

Another connection we will leverage in \autoref{sec:low_stretch_trees_and_spanners} is that the
value of the \mind{} objective is a complexity measure of the \esp{} problem~\autocite[Section
4.1]{Cesa-Bianchi2012b}. Before that, we will also present some variants and extentsion of \pcc{},
some of them bearing similarities with the Chapter 3's problem.

there are also \url{https://arxiv.org/abs/1509.04037} about measuring balance (or frustration index)
which they claimed has been peer reviewed? They also mention their method to compute frustration
exactly \url{https://arxiv.org/abs/1611.09030} along with existing work (for $n=50$)
\url{https://doi.org/10.1016/j.socnet.2013.09.002} (leading to an 1996 heuristics for \pcc{} in
section 3 of \url{https://doi.org/10.1016/0378-8733(95)00259-6})

\iffalse
strong balance random graph model \url{https://www.lri.fr/~yannis/randsigned.pdf}, line index is basically
the objective value of CC and if I'm reading their bound correctly,  basically you need to flip half
the edges of a random graph to make it balanced (also most of their results hold when the
probability of an existing edge is 1/2, but it was strengthened in a follow up dealing with weak
balance \url{http://people.maths.ox.ac.uk/harutyunyan/weak-balance.pdf})

For the 2 clusters case ($k=2$), characterization proven already in 1936 by \textcite{Konig36}, as noted in
\cite{Zaslavsky2012} (commenting on \cite{harary1953}: \enquote{Although Theorem 3 was anticipated by
\textcite[Theorem X.11]{Konig36}  without the terminology of signs, here is the
first recognition of the crucial fact that labelling edges by elements of a
group---specifically, the sign group---can lead to a general theory.})
and \cite{Huffner2010} (\enquote{\textcite{Konig36} proved the following characterization of
	balanced graphs. For a graph $G = (V, E)$, the following are equivalent:\marginpars{There is a
	proof in \autocite[p. 111]{BookKleinberg2010}, maybe I can rewrite it as well}
	\begin{enumerate}
		\item $V$ can be partitioned into two sets $V_1$ and $V_2$ called sides such that there is no
			negative edge $\{v, w\} \in E$ with both $v, w \in V_1$ or both $v, w \in V_2$ and no positive
			edge $\{v, w\}$ with $v \in V_1$ and $w \in V_2$ .
		\item $V$ can be colored with two colors such that for all $\{v, w\} \in E^-$, the vertices $v$
			and $w$ have different colors, and for all $\{v, w\} \in E^+$, the vertices $v$ and $w$ have
			the same color. The color classes correspond to the sides.
		\item $G$ does not contain cycles with an odd number of negative edges.
	\end{enumerate}
	Using the characterization by a coloring, it is easy to see that balance of a signed
graph can be checked in linear time by depth-first search.})
\fi


\subsection{State of the art}
\label{sub:state_of_the_art}

\subsubsection{Hardness and approximation}
\label{ssub:cc_harness_approx}
Although the same problem was studied earlier~\autocites{Early96}{Ben-Dor99}, \textcite{Bansal2002}
coined the term \pcc{} and were the first to study this problem complexity. Namely, for complete,
unweighted signed graphs, they show that both \mind{} and \maxa{} are \NPc{}. Along the way, they
give a $17429$-approximation of \mind{} and a PTAS%
\bottomfootnote{An algorithm $\mathcal{A}$ is a \emph{polynomial-time approximation scheme (PTAS)} for a
minimization (respectively maximization) problem $\mathcal{P}$  in NP if given any $\epsilon>0$ and
any instance $x$ of $\mathcal{P}$ of size $n$, $\mathcal{A}$ produces, in time polynomial in $n$, a
solution that is within a factor $1+\epsilon$ (respectively $1-\epsilon$) of being optimal with
respect to $x$. Note that the time is not necessarily polynomial in $\epsilon$, so that a running
time of $O(n^{\frac{1}{\epsilon}})$ would qualify~\autocite[Definition 3.10]{CpxBook99}.}
that, for any $\epsilon \in [0,1]$, runs in
$O(n^2e^{O(\frac{1}{\epsilon^{10}}\log\frac{1}{\epsilon})})$ and returns with probability
$1-\frac{\epsilon}{3}$ a solution with at most $\epsilon n^2$ fewer agreements than the optimal
solution of \maxa{}.

\begin{table}[bt]
   \centering
   \small
   \caption{Hardness results of \pcc{}} \label{tab:cc_cpx}
   \begin{tabulary}{187mm}{lCCCC}
      \toprule
               & \multicolumn{2}{c}{\mind{}}   & \multicolumn{2}{c}{\maxa{}}                                   \\
      \cmidrule(r){2-3}
      \cmidrule(r){4-5}
      graph    & weighted                      & unweighted                                                     & weighted                                                     & unweighted                   \\
      \midrule
      Complete & \APXh{}~\autocite{Charikar2003} & \NPc{}~\autocite{Bansal2002}, \APXh{}~\autocite{Charikar2003}  &                                                              & \NPc{}~\autocite{Bansal2002} \\
      General  & \APXh{}~\autocites{Charikar2003}{Demaine2003} &  \APXh{}~\autocites{Charikar2003}{Emanuel2003}   & \multicolumn{2}{c}{\APXh{}~\autocite[Thm. 9]{Charikar2003}} \\
      \bottomrule
   \end{tabulary}
\end{table}

The next year, several authors independently strengthened these results and extended them to
weighted and general graphs, as summarized in \autoref{tab:cc_cpx}. In the most complete paper,
\textcite{Charikar2003} show that on complete graphs, minimizing disagreements is \APXh{},
\enquote{that is, is NP-hard to approximate within some constant factor greater than one} (it would
be nice to provide some intuition why it's more difficult to minimize disagreements but according to
the authors themselves, their reduction is \enquote{somewhat intricate}). They prove the same result
on general graphs, for \mind{} by using a reduction from the multicut problem \autocite[Theorem
8]{Charikar2003}\footnote{This reduction was actually turned into an equivalence, as we describe
in more details in \autoref{ssub:cc_under_stability_assumption}.}
which asks for the minimum weight set of edges whose removal in $G$ disconnect the
$k$ pairs $(s_i, t_i)$ and for \maxa{} by a reduction from MAX 3SAT~\autocite[Theorem
9]{Charikar2003}. The multicut reduction yields a $O(\log n)$ approximation bound, which is achieved
by rounding a Linear Program, that we now describe. Assign a binary variable $x_{uv}$ to each pair
of nodes (so that $x_{uv}=x_{vu}$). For a given clustering \cluster{}, let $x_{uv} = 0$ if $u$ and
$v$ are in the same cluster and $x_{uv}=1$ is $u$ and $v$ are in different clusters. Noting that
$1-x_{uv}$ is $1$ if the edge $(u,v)$ is within a cluster and $0$ otherwise, the weighted
number of disagreements is then $w(\cluster{}) = \sum_{(u,v)\in E^-} w_{uv}(1-x_{uv}) +
\sum_{(u,v)\in E^+} w_{uv}x_{uv}$. By construction, if edges $(u,v)$ and $(v,w)$ are within the same
cluster, then $(v, w)$ is also within that cluster. In terms of $x$ variable, we have that $x_{uv}=0
\wedge x_{vw}=0 \implies x_{uw} = 0$. For $x$ to be a valid cluster assignment, we thus require that all
variables are either $0$ or $1$ and respect the triangle inequality. We can thus relax the problem into
\begin{align}
   \label{eq:mindLP}
   \text{minimize } & \sum_{(u,v)\in E^-} w_{uv}(1-x_{uv}) + \sum_{(u,v)\in E^+} w_{uv}x_{uv} \\
   \text{subject to}& \quad x_{uw} \leq x_{uv} + x_{vw} \nonumber\\
   \phantom{subject to}& \quad 0 \leq x_{uv} \leq 1  \nonumber \\
   \phantom{subject to}& \quad x_{uv} = x_{vu}  \nonumber
\end{align}
Once the LP \eqref{eq:mindLP} is solved, we interpret $x_{uv}$ as a distance: the larger it is and
the more we want $u$ and $v$ to be in different clusters. We can then use the \regionGrow{}
method~\autocite{RegionGrowing93}. Namely, we pick a ball center $u$ \uar{} with radius \shalf{}: if
the average distance of the nodes in the ball to $u$ is less than $\nicefrac{1}{4}$, the ball forms a
cluster, otherwise $\{u\}$ forms a singleton cluster. We then remove the corresponding nodes from
the graph and repeat until all nodes are clustered. As we shall see, this method has been re used
with variations on the LP or on the two thresholds \shalf{} and $\nicefrac{1}{4}$.

However, \textcite[Theorem 2]{Charikar2003} note that the LP formulation has a poor integrality gap
when it comes to \maxa{}, thus they turn to a Semi Definite Program. Say that each cluster is
associated with a basis vector, then for each node $u$ in a cluster, we set $a_u$ to be the
corresponding basis vector. If $u$ and $v$ are in the same cluster, we then have $a_u\cdot a_v = 1$
while if they belong to different clusters, $a_u\cdot a_v = 0$. The weighted number of agreements
can then be represented by
\begin{align}
   \label{eq:maxaSDP}
   \text{maximize } & \sum_{(u,v)\in E^+} w_{uv}(a_u\cdot a_v) + \sum_{(u,v)\in E^-} w_{uv}(1-a_u\cdot a_v) \\
   \text{subject to}& \quad a_u\cdot a_u=1 \nonumber\\
   \phantom{subject to}& \quad a_u\cdot a_v\geq 0  \nonumber
\end{align}
After solving the SDP, a clustering can be obtained by a general rounding technique $H_t$: pick $t$
random hyperplanes and divides the nodes in $2^t$ clusters. \Textcite[Theorem 3]{Charikar2003} prove
that taking the best results of $H_2$ and $H_3$ gives in a $0.7664$ approximation on general graph.
This was slightly improved to $0.7666$ by \textcite{Swamy2004} with a different rounding: pick $k$
random unit vectors (called \emph{spokes}) and assign each $a_u$ to the closest spoke.

Combining \mind{} and \maxa{}, \textcite[Section 4]{Charikar2004} give a $\Omega(\frac{1}{\log n})$
approximation of the \textsc{MaxCorr} problem, which is maximizing \eqref{eq:maxa} - \eqref{eq:mind}
and can be formulated as a quadratic programming problem solved in polynomial time.

In complete graphs, \textcite[Section 3]{Charikar2003} also give an improved $4$-approximation to
\mind{}, by rounding the same LP and using its solution in randomized algorithm. We will not
describe it in detail since a similar idea was used by \textcite{CCPivotConf05} with a better
approximation. To explain it, we first describe their randomized combinatorial algorithm \ccpivot{},
which gives a $3$-approximation of \mind{} on complete unweighted graphs (it has later been
derandomized while preserving its approximation guarantee~\autocite{derandomCCPivot08}). At each
iteration, we pick a node $u$ \uar{} (called the pivot) and we create a cluster containing $u$ and all its neighbors
linked by a positive edges. On weighted complete graphs, they tweak this algorithm by using the
solution of the  LP~\eqref{eq:mindLP} to obtain different approximation factor depending on the
constraints imposed on the weight.\Todo{Comment on those constraints, especially the triangular ones
in the context of consensus clustering.} Recall that in the general formulation of the problem, each
edge carries two positive numbers: $w_{u,v}^+$ and $w_{u,v}^-$. If the weights respect the
probability constraints stating that for all edge $(u,v)$ in $E$, $w_{u,v}^+ + w_{u,v}^- = 1$, this
tweaking provide a $2.5$-approximation. Note that unweighted graphs naturally fit into that case, as
each edge is either labeled $+$ or $-$. If the weights
additionally respect the triangular inequality constraints stating that $w_{u,v}^- \leq w_{u,w}^- +
w_{w,v}^-$, this become a $2$-approximation. After solving the LP~\eqref{eq:mindLP} with additional
probability constraints, when picking a node $u$, each of its neighbors $v$ is added to the cluster
of $u$ with probability $x_{uv}$. \Textcite{Chawla2014} improve these two factors to respectively
$2.06$ and $1.5$ by exploiting the same idea but setting the probability to include each neighbor
$v$ of $u$ in the cluster of $u$ to be $1-f^+(x_{uv})$ if $(u,v)\in E^+$ and $1-f^-(x_{uv})$ if
$(u,v)\in E^-$, with a careful choice of $f^+$ and $f^-$. They also give a derandomized version of
their algorithm in the full version of the paper~\autocite[Theorem 23]{ChawlaArxiv14}.

\begin{table}
   \begin{tabulary}{187mm}{llcLL}
      \toprule
                                &            & $k$   & \mind{}                                                                & \maxa{}                                                     \\
      \midrule
      \multirow{2}{*}{Complete} & unweighted &       & $2.06$ \autocite{Chawla2014}                                           & PTAS from \textcite{Bansal2002}  \\
      \cmidrule(r){4-4}
                                & weighted   &       & $1.5$ (with triangular inequality) \autocite{Chawla2014}               &  and by setting $k=\Omega(1/\epsilon)$ \autocite{Giotis2006}  \\
      \midrule
   \multirow{3}{*}{General}     & unweighted &       & \multicolumn{2}{c}{No specific results for the unweighted case}                                                                       \\
      \cmidrule(r){4-5}
                                & weighted   & $k=2$ & $O(\sqrt{\log n})$ \autocite{Giotis2006}                               & $0.884$ \autocite{Mitra2009}                                  \\
      \cmidrule(r){4-5}
                                & weighted   &       & $O(\log n)$ \autocite{Charikar2003}, optimal under the UCG conjecture  & $0.7666$ \autocite{Swamy2004}                                 \\
      \bottomrule
   \end{tabulary}
   \caption{Best results on various problem.\label{tab:cc_approx}}
\end{table}

This concludes the presentation of the known approximation results on \pcc{}, that we summarize in
\autoref{tab:cc_approx}. As mentioned earlier, not having to set the number of clusters is an
attractive feature of the \pcc{} problem, but in some cases we may want to use prior knowledge.  The
problem was studied by \textcite{Giotis2006} on general graphs and we compile their results in
\autoref{tab:cc_fixed}. On complete unweighted graphs and with $k$ being the number of clusters, they
provide PTASs for \maxa{}$[k]$ running in $nk^{O(\epsilon^{-3}\log(\frac{k}{\epsilon}))}$ time and for
\mind{}$[k]$ running in $n^{O\left(\epsilon^{-2} 9^k\right)}\log(n)$ time. The latter was improved by
\textcite{LinearMinPTAS09}, with a PTAS running in $n^2 2^{O\left(\epsilon^{-3}k^6\log d\right)}$
and that can handle weighted graphs. For $k=2$, \mind$[2]$ admits a faster local search method with a
factor $2$ approximation~\autocite{Coleman2008}.
For complete weighted graph, \textcite{WeightedMaxAPTAS08} provide a PTAS for
\maxa{}$[k]$ under the condition that the ratio between the largest and smallest weights is bounded by a
constant.

\begin{table}[htpb]
   \centering
   \caption{Approximation results for \pcc{} on general graph with $k$ clusters} \label{tab:cc_fixed}
   \begin{tabulary}{187mm}{lLL}
      \toprule
      $k$	 & 2 & $\geq 3$ \\
      \midrule
      \maxa{} & 0.878 (improved to 0.884 by \autocite{Mitra2009}) & 0.7666 \autocite{Swamy2004} \\
      \mind{} & $O(\sqrt{\log n})$ as it reduces to Min 2CNF Deletion for which \textcite{min2CNF05}
      give such an approximation &
      this can be reduced from $k$-coloring, which for any $\epsilon > 0$ is \NPc{} to approximate
      within $n^{1-\epsilon}$ \autocite{InnaproxChroma07} \\
      \bottomrule
   \end{tabulary}
\end{table}

\subsubsection{Proposed approaches}
\label{ssub:cc_methods}
\input{cc_proposed}

\subsection{Variants and extensions}
\label{sub:variants_and_extensions}
% leftover:
% - something about fuzzy edges, ie we're not sure they are in the graph or not (not super
%   interesting in my opinion) Clustering with partial information, Theoretical Computer Science,
%   Volume 411, Issue 7, 2010, Pages 1202-1211, ISSN 0304-3975,
%   \url{http://dx.doi.org/10.1016/j.tcs.2009.12.016}
% - \paragraph{Hypergraphs} \Textcite{Kim2011} show a LP relaxation on hypergraph and
%   \textcite{Ricatte13} describe a class of hypergraphs that can be reduced to signed graph.  Jörg
%   Hendrik Kappes, Markus Speth, Gerhard Reinelt, Christoph Schnörr, Higher-order segmentation via
%   multicuts, Computer Vision and Image Understanding, Volume 143, 2016, Pages 104-119, ISSN
%   1077-3142, \url{http://dx.doi.org/10.1016/j.cviu.2015.11.005}
% - Interactive \pcc{}?~\autocite{interactiveCC16}, could it go into active?  Geerts, F. & Ndindi,
%   R. Bounded correlation clustering. Int. J. Data Sci. Anal. 1, 17–35 (2016)
%   \url{https://doi.org/10.1007/s41060-016-0005-2}

So far we focused on \pcc{} in its rawest form, that is solving the \mind{} and \maxa{} objectives
in the case where the general binary-labeled graph is known in advance. We will now see first some
special cases, namely when the weights obey the triangle inequality (to solve \msc{}) or when the
graph is bipartite and then move to variants. We will consider more general objectives, when the
edges are labeled categorically instead of binary, when nodes can belong simultaneously to several
clusters or when we optimize local objectives per nodes instead of global ones. Finally we will also look at
clustering in signed graphs in general, using spectral methods or heuristics from the community
detection literature.

% \paragraph{\msc{}}                                | special weights
% \paragraph{Bipartite \pcc{}}                      | special graphs
% \paragraph{Categorical edge labelling}            | more general objective
% \paragraph{Overlapping \pcc{}}                    | more general objective
% \paragraph{Local \pcc{}}                          | more general objective or at least different
% \paragraph{Spectral clustering}                   | related objective
% \paragraph{Communities detection}                 | related objective

% \paragraph{Online \& active setting}              | different setting
% \paragraph{recovery under noise}                  | non worst case analysis
% \subsubsection{\pcc{} under stability assumption} | non worst case analysis

\paragraph{\msc{}} 

In \msc{}, the goal is to output a clustering which best summarizes (or agrees with) the several
given input clusterings of the same set of objects.  Motivations include robustness ---by using an
ensemble of clusterings from diverse methods--- and privacy ---if the clusterings were computed by
different parties each considering only a subset of the objects attributes. We can build the
complete graph of these objects, with weights set to the fraction of clusterings that place two
objects in different clusters, thus representing a kind of distance in the space of clusterings.  As
first show by \textcite{EarlyConsensusClustering03} , finding the optimal clustering is therefore
an instance of \pcc{} where the weights obey the triangular inequality. \textcite{Gionis2007} give a
deterministic $3$-approximation using the \regionGrow{} method. Later \textcite{Bonizzoni2008} show that
the minimization version is \APXh{}, even when the input is made of three clusterings and give a
combinatorial $\nicefrac{4}{5}$-approximation for the maximization problem. Experimental evaluations are
conducted by \textcite{Bertolacci07} and \textcite{Filkov08}. The former describe a scalable
approach that first samples a small portion of the data, runs a (potentially computationally
expensive) approximation algorithm and finally augment the resulting partition by adding to it the
unsampled nodes one by one.  Experiments confirm that the running time is greatly improved compared
with the linear program methods while the resulting objective value is essentially the same. Note,
however, that LP methods can be applied in practice thanks to some tricks~\autocite{ConsensusLP10}.
If we have $k$ input clusterings $\cluster^1,\ldots,\cluster^k$ and we parameterized the problem by
$t$, which is the sum over the input clusterings of the number of pairs of objects that are clustered
differently by a solution $\cluster^\star$ and $\cluster^i$, then there is a polynomial algorithm
running in $O(4.24^{\nicefrac{t}{k}}\cdot \nicefrac{t}{k}^3 +
kn^2)$~\autocite{parameterizedConsensus14}.

\paragraph{Bipartite \pcc{}}

Bipartite graphs are an interesting special case for \pcc{}, as they often appear in the context of
recommendation systems, where users rate products positively or negatively, although in this setting
we cannot expect to have complete bipartite graphs in practice. The first results was given by
\textcite{Amit04}, who obtains an $11$-approximation for \mind{} by adapting the \regionGrow{} method.
The \ccpivot{} is adapted to the bipartite case by
\textcite{Bipartite12}, who prove it results in a randomized $4$-approximation (and provide a matching
deterministic approximation by rounding a LP). By using their idea of rounding the results of the LP
differently for positive, negative and in that case same-side absent edges, \textcite{Chawla2014}
bring down this approximation factor to $3$, even for $K\geq 2$-partite graphs. Through formulating the
\maxa{}$[k]$ problem as a bilinear maximization problem and computing a low-rank approximation of
the graph biadjacency matrix, \textcite{Asteris2016} obtain a efficient PTAS, that is a $(1-\delta)$
approximation running in time exponential in $k$ and $\delta^{-1}$ but linear in $n$. By an
appropriate choice of $k$, it is possible to use this PTAS to solve the general \maxa{} problem.

\paragraph{Categorical edge labelling}

In the so called \textsc{Chromatic}-\pcc{} setting, \enquote{positive} edges are now associated with
one of $L$ possible colors and the goal is to form clusters mostly made up of edges with the one same
color. Namely, a disagreement is now a negative edge between clusters or a within-cluster edge
whose color differs from the majority color of that cluster. This is motivated by edge-labeled graph
in social networks, biology and citation networks and will discuss such applications in
\autoref{chap:vector}. As a generalisation of \pcc{}, it is \NPc{} but \textcite{Bonchi2012a}
present a modification of the \ccpivot{} algorithm that pick edges instead of nodes as pivots, and
grow clusters by adding monochromatic triangles. This gives an approximation factor of six times the
maximum degree of the graph. They also present a method when the number $k$ of clusters is fixed
beforehand, starting with an initial partition and then alternating between finding the majority
color of clusters and finding better clusters. An improved heuristic algorithm is given in
\url{http://ieeexplore.ieee.org/document/7732322/}. Unfortunately, the maximum degree of a graph can be
as large as $n$. However, \textcite{Anava2015} present constant factor approximations. Namely, they
show the problem can be reduced to classical \pcc{} by setting all edges incident to a node $u$ to
negative if they are not of the majority color of $u$. They then apply the regular \ccpivot{} and
show this gives an $11$-approximation to the original problem. Furthermore, they also write a linear
program and round it using the \regionGrow{} method to obtain an
approximation factor of $4$. \Textcite{multiChromatic15} extend this line of work to the case were a
single edge can carry a \emph{set} of labels and adapt their randomized algorithm so that the
approximation factor is multiplied by the size of the input label set.

\paragraph{Overlapping \pcc{}}

While in \pcc{}, each node is assigned to a single cluster, in other settings we may want to relax
this constraint. Given a complete weighted graph, \textcite{Bonchi2012} want to output a clustering
\cluster{} that minimizes the following cost: \[ \sum_{(u,v)\in E} \left| H(\cluster(u),
\cluster(v)) - w_{uv}\right|\] where $H$ is a similarity function between two sets of labels, chosen
here to be the Jaccard similarity or a $0/1$ indicator of nonempty intersection. These problems are
showed \NPc{} and approximated by a local search algorithm, iteratively optimizing the assignment of
one node while all others are fixed. As one of the demonstration on their theoretical work,
\textcite{WeightedTheta15} show a faster solution based on a weighted extension of the Lovász's
theta function, the corresponding geometric embedding of graphs and a solver derived from one-class
SVM, while \textcite{GeneticOCC14} propose a genetic algorithm to solve this problem. Finally,
\Textcite{StochasticCC13} also deals with overlapping clustering by relaxing the problem to a
stochastic setting and using \enquote{the Baum-Eagon inequality, which provides an effective
iterative scheme for maximizing polynomial functions in probability domains}.

\paragraph{Local \pcc{}}

In classical \pcc{}, all nodes have an identical role, in the sense that they contribute equally to
the final objective in terms of (dis)agreements. Here we instead look at approaches where we either
add a local penalty to each in order to better control their behavior or where we altogether modify
the objective to focus on (dis)agreements at specific nodes.

% that one is not really local, except for the node penalty
\Textcite{Puleo2014} adapt the linear program of~\autocite{Charikar2003} and its \regionGrow{}
method to the case
where all clusters have to contain less than $K$ nodes, by assigning to each node $u$ a penalty
$\mu_u$. If $u$ is placed in a cluster $C_i$, the original \mind{} objective is penalized by an
extra $\mu_u\left(|C_i| - (K+1)\right)$. By varying $\mu_v$ between $0$ and $1$ and because the
positive weights are assumed to smaller than $1$, this cluster size constraint can be made hard or
soft. They also handle more general weights, since they allow $w^-_{uv}$ to be as large as $\tau$
for $\tau\in [1,\infty)$ while still guaranteeing a $5-\nicefrac{1}{\tau}$-approximation on complete
graphs, and adapt \ccpivot{} to unweighted graphs with the hard cluster size constraint, obtaining a
randomized $7$-approximation.
These soft constraints are for instance used in a biological application where nodes are genes and
where singleton and giant clusters are uninformative~\autocite{Hou2016}

\Textcite{pmlr-v48-puleo16} also modify the \mind{} objective to make it more general. Based on the
classic \pcc{} linear program, they define a \enquote{\emph{fractional clustering} of $G$ as a
vector $x$ indexed by $V$ such that $x_{uv} \in [0, 1]$ for all $uv \in \binom{V}{2}$ and such that
$x_{vz} \leq x_{vw} + x_{wz}$ for all distinct $v, w, z \in V$}. They also define \enquote{The
error vector $err(x)$ of $x$, as a real vector indexed by $V$ whose coordinates are}
\begin{equation*}
  \mathrm{err}(x)_u = \sum_{v\in\nei^+(u)} x_{uv} + \sum_{v\in\nei^-(u)} (1-x_{uv})
\end{equation*}

Given a function $f: \Rbb^n_{\geq 0} \rightarrow \Rbb$ verifying two elementary conditions, the
problem is then to find a fractional clustering $x$ minimizing $f(err(x))$. The classical \pcc{}
corresponds to setting $f(x) = \lhalf{}\ell^1(x)$ whereas the authors here are interested in Minimax
\pcc{} that arises by setting $f(x) = \ell^\infty(x)$. Minimizing the maximum number of
disagreements incurred by a single node is motivated by the example of recommendation systems: if
errors correspond to unsatisfying recommendations, we do not want a single user to suffer many
of them. Minimax \pcc{} is \NPc{} on both complete graphs and complete bipartite graphs but
by modifying the \regionGrow{} method, the authors respectively a $48$ and
$10$ approximation, the latter for the one-sided error (that only counts disagreements for the nodes
in one of the two clusters).
The idea is to chose pivots not randomly but by maximizing a given criteria and
to grow balls with a radius $\alpha$ computed numerically to optimize the approximation factor.
Interestingly, and in contrast with the classic \pcc{} situation, minimax \maxa{} is not easier than
minimax \mind{} and seems not to have a constant factor approximation, even on complete graphs.
Furthermore, these algorithms are deterministic, as opposed to many \pcc{} approximations, since
bounds on expected disagreements of an edge does not translate easily on their maximum.
\Textcite{Charikar2017} improve these two factors to $7$, using a simpler version of the algorithm
of \textcite{pmlr-v48-puleo16}. Namely, find the ball of radius $\nicefrac{1}{7}$ with the largest
number node and create a cluster from its center with a radius of $\nicefrac{3}{7}$. They also show
that on general weighted graphs, the LP has a large integrality gap of $\nicefrac{n}{2}$ yet they
combine it with a combinatorial approach to reach a $O(\sqrt{n})$ approximation. Finally, they
consider the complementary problem of maximizing the minimum number of agreements reach at a single
node, and provide a $\frac{1}{2+\epsilon}$ approximation.

\paragraph{Spectral Clustering}
\label{par:cc_spectral}

A classic method for clustering graphs is to leverage their spectral properties. Namely, if $A$ is
the adjacency matrix of $G$ and $D$ its degree diagonal matrix (that is $D_{u,u} = \degr(u)$), the
\emph{Laplacian} of $G$, defined by $L_G = D - A$, is a symmetric positive semidefinite matrix. As
such, it has $n$ real non-negative eigenvalues, and its spectrum provides additional information on
the connectivity of $G$. For instance, $0$ is always the smallest eigenvalue and its multiplicity is
equal to number of connected components of $G$, while ---if $G$ is connected--- the second
eigenvalue is the algebraic connectivity of $G$, whose magnitude is an indication of how well
connected is the graph. This matrix is typically used for clustering by computing its first $k$
eigenvectors, which embed the $n$ nodes of $G$ in $\Rbb^k$, where there are then clustered with the
$k$-means algorithm. This can be seen as a relaxation of the discrete \rcut{} objective, which asks
for the partition $\{C_1, \ldots, C_k\}$ minimizing $\frac{1}{2}\sum_{i=1}^k \frac{\mathrm{cut}(C_i,
\bar{C_i})}{|C_i|}$, where $\bar{C_i}$ is the complement of $C_i$ in $V$ and $\mathrm{cut}(B, C) =
\sum_{u\in B, v\in C} w_{uv}$ is the total weight of the edges between $B$ and
$C$~\autocite{tutoSpectralClustering07}. By considering the symmetric normalized Laplacian
$L_{sym}  = D^{\nicefrac{1}{2}}LD^{\nicefrac{1}{2}}$, it is also possible to approximate the
normalized cut objective (\ncut{}), where $|C_i|$ is replaced by $vol(C_i) = \sum_{u\in C_i}
\degr(u)$. We will now see how these kinds of approaches can be extended to signed graphs, noting
first that they require to fix the number of clusters beforehand and are looking for clusters
balanced in size, which makes the problem related but not equivalent to \pcc{}.

The first line of research consider only \mind{}$[2]$. For instance, \textcite{NcutAnd2CC08} show
that both normalized cut and \mind{}$[2]$ objectives can be written as a SDP (or equivalently as
eigenvalue problems) and thus combined, the intuition being that we look for \ncut{} solutions whose
number of disagreements is not too much more than the approximate optimal.

Letting \ncut{}$(C, \bar{C}) = \frac{\mathrm{cut}(C, \bar{C})}{\mathrm{bal}(C)}$ with
$\mathrm{bal}(C) = 2\frac{vol(C)vol(\bar{C})}{vol(V)}$, \textcite{mOneCC12} define a new objective:
\begin{equation*}
  \hat{F}_\gamma(C) = \frac{\mathrm{cut}(C, \bar{C}) + \gamma\left(\hat{M}(C)+\hat{N}(C)\right)}{\mathrm{bal}(C)}
\end{equation*}
where $\gamma \in \Rbb{}_+$ is a parameter, while $\hat{M}(C)$ and $\hat{N}(C)$ are respectively the
number of positive and negative disagreements of the $(C, \bar{C})$ clustering.
They show how to optimize a tight continuous relaxation of $\hat{F}_\gamma$ as the non-negative
ratio of a difference of convex function and a convex function.

On the other hand, one can also adapt these two cut objectives to directly include negative edges.
\Textcite{Luca10} define the signed Laplacian as $\bar{L} = \bar{D} - A$, where $\bar{D}$ is the
signed degree matrix such that $\bar{D}_{uu} = \sum_{v\in \nei(u)} |A_{uv}|$, as well as a signed
variant of the symmetric normalized Laplacian $\bar{L}_{sym}  = \bar{D}^{\nicefrac{1}{2}} \bar{L}
\bar{D}^{\nicefrac{1}{2}}$. They show that the signed Laplacian is positive semidefinite, and
even positive-definite as soon as the graph is unbalanced (that is contains a cycle with an odd
number of negative edges). From positive and negative cuts defined as $\mathrm{cut}^+(B, C) =
\sum_{u\in B, v\in C} w^+_{uv}$ and $\mathrm{cut}^-(B, C) = \sum_{u\in B, v\in C} w^-_{uv}$, a
natural signed cut is $\mathrm{scut}(B, C) = 2\mathrm{cut}^+(B, C) + \mathrm{cut}^-(B, B) +
\mathrm{cut}^-(C, C)$ which can then be used to defined signed \rcut{} and \ncut{}. Arguing that those
definitions force negatively linked nodes to be symmetric around the origin, do not take into
account the balance of negative edges in each cluster and are difficult to extend to more than two
clusters, \textcite{SignedEmbedding15} instead propose two new normalized cuts:
\begin{align*}
  SNScut(C_1, \ldots, C_k) &= \sum_{i=1}^k
  \frac{\mathrm{cut}^+(C_i, \bar{C_i})-\mathrm{cut}^-(C_i, \bar{C_i})}{vol(C_i)}  \\
  BNScut(C_1, \ldots, C_k) &= \sum_{i=1}^k
  \frac{\mathrm{cut}^+(C_i, \bar{C_i})-\mathrm{cut}^-(C_i, \bar{C_i})+vol^-(C_i)}{vol(C_i)}
\end{align*}

Noting that if $x_i\in\Rbb^n$ is the vector indicator of cluster $C_i$ (that is the \uth{} entry of
$x_i$ is $1$ is $u$ belongs to $C_i$ and $0$ otherwise), $x_i^T\bar{L}x_i = 2\mathrm{cut}^-(C_i,
C_i) + \mathrm{cut}^-(C_i, \bar{C}_i) + \mathrm{cut}^+(C_i, \bar{C}_i)$,
\textcite{mSemanticWordCC17} introduce the following cut objective:
\begin{equation*}
  sNcut(C_1, \ldots, C_k) = \sum_{i=1}^k
  \frac{2\mathrm{cut}^-(C_i, C_i) + \mathrm{cut}^-(C_i, \bar{C}_i) + \mathrm{cut}^+(C_i, \bar{C}_i)}{vol(C_i)}
\end{equation*}
Additional cut formulations for $k$-clusters are also presented in~\autocite{moreSignedCut12},
although \textcite{Knyazev2017l} argue that using the non signed Laplacian and considering negative
eigenvalues might be just as effective, citing for instance numerical instability of signed
Laplacian.

Finally, \textcite{mGeometricMean16} show that the Laplacians defined so far can be seen as
arithmetic means of the Laplacian $L^+$ of the positive subgraph $G^+=(V, E^+)$ and the signless
Laplacian $Q^-$ of the negative subgraph $G^-=(V, E^-)$, where $Q^- = D^- + A^-$. They suggest
instead to use a geometric mean defined for two positive matrices $A$ and $B$ as $A\#B =
A^{\shalf{}} \left(A^{-\shalf{}} BA^{-\shalf{}} \right)^{\shalf{}} A^{\shalf{}}$. This suggestion is
based on the fact that if $u$ is a common eigenvector of both $A$ and $B$ with eigenvalue $\lambda$
and $\mu$ respectively, then $u$ is an eigenvector of $A+B$ with eigenvalue $\lambda+\mu$ and an
eigenvector of $A\#B$ with eigenvalue $\sqrt{\lambda\mu}$. Therefore, the $k$ smallest eigenvalues
of the geometric mean Laplacian will be influenced by both smallest eigenvalues of $L^+$
(corresponding to assortative clusters in $G^+$) and of $Q^-$ (corresponding to disassortative
clusters in $G^-$), while this is note the case for the arithmetic mean of Laplacians.

\paragraph{Community detection}

The clustering problem is often named community detection in the context of social networks, and
several methods developed by practitioners have been extended to signed graphs. While they do not
necessarily considered the \pcc{} objective, and especially not its optimum, we still give a brief
overview of them, as they tend to have been more tested experimentally. For instance, to find the
cluster of node $u$, \textcite{Yang2007} use a random walk approach on the positive subgraph $G^+$
to compute the probability of each node to reach $u$ in $T$ steps, sort the nodes accordingly and
then find a threshold  based on the number of disagreements.
The one node moves local heuristic that we described in the Physics-inspired paragraph
\vpageref{par:cc_physics} can also be formulated as genetic algorithms that simultaneously try to
minimize the number of disagreements and maximize a signed variant of the
modularity~\autocites{Li2013}{Amelio2013}. \Textcite{Anchuri2012} also consider these two objectives
by seeing them as eigenvalue problems and devise a iterative splitting procedure.
The overlapping community detection variant is considered by \textcite{Chen14}, who used a signed
probabilistic mixture model. Namely, an edge selects a pair of cluster $r,s$ with probability
$\omega_{rs}$ (where $r=s$ if the edge is positive and $r\neq s$ otherwise) and chooses its endpoints
$u$ and $v$ with probability $\theta_{ru}$ and $\theta_{sv}$. $\theta_{ru}$ is therefore the soft
membership of node $u$ to cluster $r$, and those parameters are estimated using the
expectation-maximization algorithm. The same model is extended to directed graphs by
\textcite{Jiang2015}, who strangely enough names it stochastic blockmodel, although the focus is
still on edge and not nodes.
In a similar spirit to \maxa{}$[k]$, \textcite{SignedGang} focus on finding $k$ subgraphs dense in
positive edges and densely connected by negative edges to each other. They dub such subgraph
\emph{Oppositive Cohesive Groups}, or more vividly \emph{Gangs in War}, and after formulating the
problem as a constrained quadratic optimization, they propose a faster iterative local search
heuristic. When $k=2$, these subgraphs are called antagonistic communities and a specific data
mining approach was proposed by \textcite{quasiGang}.
% relaxed structural balance \autocite{Doreian2009} but used the same approach as \textcite{Early96}


\subsubsection{\pcc{} under stability assumption}
\label{ssub:cc_under_stability_assumption}
\input{cc_multimincut_stable}


\section{Low stretch trees and spanners}
\label{sec:gtx}

We now show how to apply the learning bias of \autoref{sec:new_bias} on undirected graphs.
Furthermore, we assume that the strong balance holds, meaning that there are only two groups
according to \autoref{thm:structural}. In other words, the labelling of $E$ is consistent with a
two-clustering of $V$. Namely, $V$ can be partitioned in two clusters such that edges within each
cluster are positive and edges across clusters are negative.
In that case, the following
\emph{multiplicative rule} holds: for any nodes $u$, $v$ in $V$, and any path $p$ between $u$ and
$v$ in $G$, the sign $\yuv$ is equal to the product of the signs along $p$. Hereafter, we call this
product the parity of $p$, and denote it by $\pi(p)$.
While it is a simple and convenient hypothesis, this is too strong
of a requirement to be satisfied in practice. Therefore, we relax it by assuming that, starting from
a consistent labeling $Y$, we can only observe a randomly perturbed version $Y'$ of $Y$.
Specifically, given a constant $q\in [0, \nicefrac{1}{2})$, every sign of $Y$ is flipped with a
probability smaller than $q$. We denote by $E_{\mathrm{flip}} \subset E$ the set of edges whose sign
has been flipped.

In this section, we are interested in active learning algorithms that first query a subset
$\etrain$ of the edges, observe the signs in $\etrain$ and use them to predict the remaining signs.
More precisely, we focus on an algorithm that queries a spanning tree $T$ of $G$ and predicts the
sign of an edge $(u,v) \in \etest = E \setminus E_T$ as the parity of $\pathtuv$. Intuitively, since
each sign has been potentially flipped, the longer the path in $T$, the more likely its parity will
be not be equal to the true sign $\yuv{}$. Therefore we would like each such path to be as short as
possible.
Formally, the number of mistakes of such an algorithm is upper bounded by~\autocite[Equation
(3)]{Cesa-Bianchi2012b}
\begin{equation*}
  |E_{\mathrm{flip}}| + \sum_{(u,v) \in \etest}
  \sum_{e\in E} \Ind{e \in \pathtuv} \Ind{e \in E_{\mathrm{flip}}}
\end{equation*}
which in expectation is equal to:
\begin{equation}
  \label{eq:stretch_mistakes}
  q\left(|E| + \sum_{(u,v) \in \etest} |\pathtuv| \right)
\end{equation}
In the following, we describe a way to build spanning trees tailored for this situation. More
precisely, we implement and analyze a suggestion made to us by~\autocite{gtxFabio}.

\subsection{\gtx{}: a spanning tree designed for sign prediction}
\label{sub:gtx_algo}

The \gtx{} algorithm takes as input a graph topology $G_0=(V_0, E_0)$ and produces a sequence of
graphs $\{G_t\}_{t=1}^T$ of decreasing size until each connected component of $G_0$ is reduced to a
single node. As we will prove later, after reaching this point, the algorithm has selected $|V_0| -
1$ edges that form a spanning tree of $G_0$. $G_{t+1}$ is obtained from $G_t$ by composing two
primitives so that we can informally write $G_{t+1} = \left(\collapseStar{} \circ
\extractStar{}\right)(G_t)$.

\extractStar{} partitions the graph $G_t$ into a set of stars and \collapseStar{} build the graph
made of those stars using the edges in $E_t$. We provide more details on those two operations in the
following, as well as their complexity analysis. Then we state formally the \gtx{} algorithm and prove
its termination and correctness.  Finally, we study its properties, such as the number of iterations
needed to finish and the stretch of the resulting tree. For simplicity and without loss of
generality, we assume that $G_0$ consist of a single connected component.

\medskip

\extractStar{} takes as input a graph $G_t=(V_t, E_t)$, and optionally a \emph{threshold function}
$t_f$ or a \emph{degree function} $d_f$. While the nodeset $V_t$ is not exhausted, it repeatedly samples a
star center $c_i$, creates a star $S_i^t$ with the neighbors of $c_i$, removes all the nodes of $S_i^t$ from
$V_t$ and all the edges incident to $S_i^t$ from $E_t$, and finally decrements accordingly the
degree of the 2-hop neighbors of $c_i$ (see \autoref{fig:gtx_star_simple} for a visual
representation of this notation).
\begin{marginfigure}
  \centering
  \includegraphics[height=0.15\textheight]{assets/tikz/gtx_star_tikz.pdf}
  \caption[A sample star]{A sample star created during the \tth{} extraction level. The black node
    % \tikz{\node[vertex,rare] {$c_i$};}
    is the center $c_i$ of the star $S_i^t$, which is made of the four light gray peripheral nodes
  % \tikz{\node[vertex,medium] {$p_1$};} to \tikz{\node[vertex,medium] {$p_4$};}
  as well as the solid edges. The 2-hops neighbors of $c_i$ are the white nodes
  % \tikz{\node[vertex] {$h_1$};} to \tikz{\node[vertex] {$h_3$};}
  $h_1$ to $h_3$, whose degree will decrease once we $S_i^t$ is removed from $G_t$.}
  \label{fig:gtx_star_simple}
\end{marginfigure}
Upon completion, it returns a list of stars and a mapping of
each node of $V_t$ to the unique star it belongs to. We consider three heuristics to choose centers:

\begin{itemize}%[nosep]
  \item choose the node with the current highest degree, with ties broken arbitrarily
  \item if $n_i$ is the number of node remaining in $V_t$ before choosing the \ith{} center, choose
    a node \uar{} among those with a degree larger than $t_f(n_i)$. Setting the threshold function
    to be the identity therefore recovers the previous strategy, but the idea here is to choose
    among a small set of high degree nodes, for instance by letting $t_f(n) = \sqrt{n}$
  \item if $\degr(u)$ is the degree of node $u$, choose node proportionally to $d_f(\degr(u))$.
    Again, the degree function is designed so that it favors the selection of high degree nodes. For
    instance, one could use $d_f(\degr(u)) = \degr(u)^2$.
\end{itemize}

We now give the pseudo code of \extractStar{} for the highest degree variant.\footnote{Note that for
clarity, we removed some bookkeeping code in all listings, mainly the part related to maintaining
mapping between nodes at different level of contraction. However, the full python implementation
is available at \url{https://github.com/daureg/magnet/blob/master/veverica/new_galaxy.py\#L27}.}
We assume that $G$ is the adjacency list of the graph, so that $G[u]$ is the set of neighbors of
$u$, \ie{} $G[u] \equiv \mathcal{N}(u)$. The other piece of notation is $\textsc{Star}$, which
simply create a star given a center and a list of peripheral nodes.  \vspace{-\baselineskip}

\begin{center}
  \rule{\textwidth}{.3pt}
  \begin{algorithmic}[1]
    \Function{\extractStar{}}{$G_t=(V_t,E_t)$}
      \State Let $Q$ be a max-priority queue. The key of element $x$ is $Q[x]$
      \Let{$stars$}{[]}
      \Let{$remaining$}{$\emptyset$}
      \ForAll{node $u$ in $V_t$}
        % \State $\textsc{Insert}\\left(Q,\,u\right)$ \Comment with the key $\degr(u)$
        \State \Call{Insert}{$Q,\,u$} \Comment with the key $\degr(u)$
        \Let{$remaining$}{$remaining \bigcup \left\{u\right\}$}
      \EndFor
      \While{$Q$ is not empty}
        \Let{$c_i$}{\Call{Extract-Max}{$Q$}}
        \If{$c_i$ not in $remaining$}
          \State \textbf{continue} \Comment{$c_i$ is part of an existing star so there is
          nothing to do}
        \EndIf
        \Let{$periphery$}{$G[c_i] \bigcap remaining$}
        \Let{$stars$}{$stars \bigcap $\Call{Star}{$c_i,\, periphery$}}
        \Let{$remaining$}{$remaining \setminus\left\{c_i\right\} \setminus periphery$}
        \For{$p$ in $periphery$}
          \For{$h$ in $G[p] \bigcap remaining$}
            \State \Call{Decrease-Key}{$Q,\, h,\, Q[h]-1$}
          \EndFor
        \EndFor
      \EndWhile
      \State \textbf{return} $stars$
    \EndFunction
  \end{algorithmic}
  \rule{\textwidth}{.3pt}
\end{center}

\extractStar{} terminates because at each iteration of the while loop line 8, we remove one node
from $Q$ and never add any. Let us analyze the complexity when $|V_t|=n$ and $|E_t|=m$. We first
build a priority queue of all the nodes sorted by their degree (line 5--7), which takes $O(n)$ time.
Then, at each iteration of the inner loop, we find the center of the next star by extracting the
maximum of the queue (line 9), we build the corresponding star (line 12--14) and we decrease the
priority (\ie the degree) of all nodes adjacent to the new star (line 15--17).  Since both
operations require constant time when using a Strict Fibonacci Heap~\autocite{FibonacciHeaps12} and
there are $n$ iterations of that loop, a coarse approximation of the runtime of \extractStar{} is
$O(n^2)$. However, observe that there can be at most $m$ decrease operations (since after that, all
nodes still in the queue have an effective degree of $0$, meaning that $periphery$ will the empty
set and lines 13--17 will run in constant time), reducing the complexity to $O(m+n)$.

The other two variants are more time consuming because they require additional bookkeeping. Their
randomization make them useful in an adversarial context but it also renders their analysis more
challenging, not necessarily for the runtime of \extractStar{} but mostly for the tree construction.
Therefore, we only briefly describe the implementations here.\footnote{Although they are available
online at 
\nolinkurl{https://github.com/daureg/magnet/blob/master/veverica/}%
\{\href{https://github.com/daureg/magnet/blob/master/veverica/ThresholdSampler.py}%
{ThresholdSampler.py}, \href{https://github.com/daureg/magnet/blob/master/veverica/NodeSampler.py}%
{NodeSampler.py}\}.} For the threshold function, we
maintain two queues, $high$ and $low$, containing nodes whose degree is respectively above and below
the current threshold. We select a node \uar{} in $high$, remove the corresponding star from $G_t$,
recompute the new threshold and if necessary, move nodes which fell under the threshold from $high$
to $low$ and those who climb above the threshold from $low$ to $high$. For the degree function, we
can draw any node as center proportionally to its weight (where the weight of node $u$ is defined as
$d_f\left(\degr(u)\right)$), but we cannot use the standard method of computing the cumulative sum
of weights since each iteration change some of them. Therefore, we construct a binary tree whose
leaves are the nodes of $V_t$ and where each tree nodes maintain the sum of weights in its left and
right subtrees. To sample, we draw a random number between $0$ and the total weight of the tree.
When degrees are updated (or graph node removed), we update the weights along a path from the
corresponding leaves to the root of the tree.

\medskip

The second routine, \collapseStar{} takes as input the result of \extractStar{}, along with $E_t$
and an optional $\emph{eccentricity}$ array we will describe soon. It builds a new graph $G_{t+1}$
where each star becomes a node and there is a link between two nodes $s_1$ and $s_2$ if the nodes
making up $s_1$ and $s_2$ are connected in $E_t$. For that, we first shuffle $E_t$ and go through
it. When we find an edge whose endpoints belong to two different stars not yet connected, we use
that edge to connect these two stars. This trivially takes $O(m)$ times.

A variant instead keeps track of all edges connecting each pair of stars to choose one that will
best contribute to our low stretch objective. Namely, when connecting two stars, we would prefer to
join their centers rather than two peripheral points. For that we maintain an eccentricity count for
all of the nodes of the original $G_0$, which is incremented by $1$ each time a node is chosen to be
on the periphery of a star.\Todo{link that to the walk through example.}
For each pair of stars, we thus choose the edge across them with minimal sum of its endpoints'
eccentricity. This requires another pass over the edges, preserving the $O(m)$ runtime.

\begin{figure}[htbp]
  \centering
  \includegraphics[width=0.78\linewidth]{tikz/gtx_eccentricity_tikz.pdf}
  \caption[The hierarchical structure of stars created by \gtx{}]{%
    The execution of the \gtx{} algorithm. The original graph is made of the solid edges
    connecting the nodes labeled by  their index. Edges forming the final spanning tree are in black
    while the others are in gray. The four shades of gray, from white to dark gray
    denote increasing node eccentricity (as computed at the end of the algorithm). The \ith{} star
    created during the \jth{} iteration of the algorithm is denoted $S_i^j$. Refer to the main text
    for a complete walk through.}
  \label{fig:gtx_eccentricity}
\end{figure}

\begin{figure}[bthp]
  \centering
  \begin{subfigure}[b]{0.47\textwidth}
    \centering
    \includegraphics[height=5cm]{tikz/gtx_run_level1_tikz}
    \caption{Resulting graph after the first iteration}\label{fig:gtx_run1}
  \end{subfigure}~
  \begin{subfigure}[b]{0.47\textwidth}
    \centering
    \includegraphics[height=2.2cm]{tikz/gtx_run_level2_tikz}
    \caption{Resulting graph after the second iteration}\label{fig:gtx_run2}
    \vspace{\baselineskip}
    \includegraphics[height=2.2cm]{tikz/gtx_run_level3_tikz}
    \caption{Resulting graph after the third iteration}\label{fig:gtx_run3}
  \end{subfigure}~
  \caption{The other iterations of \gtx{}}\label{fig:gtx_run}
\end{figure}

\medskip

We illustrate the operation of the \gtx{} algorithm on small (and somewhat contrived) example.
Let us start with the initial graph $G_0$ depicted in \autoref{fig:gtx_eccentricity} and initialize
the eccentricity of all nodes to $0$. When running \extractStar{}, we see that the maximum degree is
$4$, achieved at nodes $\{1, 6, 11, 16, 21, 26, 31, 36, 41\}$. For the sake of simplicity, assume
nodes are picked according to their index. First, node $1$ is forms the star
$\textcolor{DodgerBlue}{S_1^1}$ with peripheral nodes $2$, $3$, $4$ and $5$. This increments the
eccentricity of those peripheral nodes by $1$. Then node $6$ forms its star
$\textcolor{DodgerBlue}{S_2^1}$ with $7$, $8$, $9$ and $10$. The process continues until node $41$ is
chosen to be the center of star $\textcolor{DodgerBlue}{S_9^1}$, at which point the max-priority
queue has been exhausted and \extractStar{} finishes.

We then call \collapseStar{}, with the eccentricity reducing variant. This will connect all possible
pairs of star. For instance, the edge between nodes $19$ and $29$ leads to the edge
between $\textcolor{DodgerBlue}{S_4^1}$ and $\textcolor{DodgerBlue}{S_6^1}$. This is actually the
only possible edge between $\textcolor{DodgerBlue}{S_4^1}$ and $\textcolor{DodgerBlue}{S_6^1}$.
Consider on the other hand the case of edges $(2, 6)$ and $(2, 9)$. They both connect
$\textcolor{DodgerBlue}{S_1^1}$ and $\textcolor{DodgerBlue}{S_2^1}$. Yet at this point of the algorithm,
the eccentricity of node $2$ is $1$, the eccentricity of node $6$ is $0$ and the eccentricity of node
$9$ is $1$. The edge $(2, 6)$ has therefore the smallest total eccentricity and is chosen to connect
$\textcolor{DodgerBlue}{S_1^1}$ and $\textcolor{DodgerBlue}{S_2^1}$. The full result of the
\collapseStar{} procedure can be seen on \autoref{fig:gtx_run1}.


\subsection{Related work}
\label{sub:gtx_related_work}

\label{sub:gtx_state_of_the_art}

Looking for a subgraph $H$ of $G$ that best preserve the distance in $G$ while being sparse is an old
problem, driven originally by network design in fields such as transportation~\autocite{RoadNetworks60}
and electrical circuits~\autocite{electricalNetworks60}. The way we define \enquote{preserving the
distance}, and the exact form of $H$ give rise to several problems, which we summarize later in
\autoref{tab:gtx_related_stretch} \vpageref{tab:gtx_related_stretch}. We first give some
definitions, then cover the most relevant problems in details, and finally give some pointers for
the others problems.

Let the distance between $u$ and $v$ in $G$ be
\begin{equation*}
  d_G(u,v) = \sum_{e \in \pathguv} \ell(e)\,,
\end{equation*}
where $\ell(e)$ is the \emph{length} of the edge $e$ and \pathguv{} is the shortest path between $u$
and $v$ in $G$. In the following, we consider only the uniform case, in which the length of an edge
is equal to its weight. The stretch of an edge $(u,v)$ in $H$ is defined as
\begin{equation*}
  \estr(u,v) = \frac{d_H(u,v)}{d_G(u,v)}.
\end{equation*}

We may then want to minimize the stretch of:
\begin{enumerate}[1),nosep]%,leftmargin=*]
  \item some pairs of nodes. That is, given $L$ and $R$ in $V$, minimize $\sum_{u \in L, v\in R}
    \estr(u,v)$
  \item all pairs of nodes corresponding to edges of $G$, \ie{} minimize $\sum_{(u,v) \in E}
    \estr(u,v)$
  \item all pairs of nodes, \ie{}  minimize $\sum_{(u,v) \in V^2} \estr(u,v)$
\end{enumerate}
Note that for an unweighted graph, the second problem reduces to minimizing $\sum_{(u,v) \in E}
|\pathhuv|$. If furthermore $H$ is tree, this is equivalent to minimize the second term of equation
\eqref{eq:stretch_mistakes}. Therefore we focus mainly of that definition of stretch, and consider
the other two only briefly.

The second point affecting the problem is the structure of $H$. The only requirements are that it
must be spanning all the nodes involved in the computation of the chosen stretch, and that $\forall
(u,v) \in E,\, d_H(u,v) \geq d_G(u,v)$. Beside that, $H$ can be a tree of $G$, a general subgraph of
$G$ or even a subset of $V^2$ (\ie{} containing edges not in $E$). We focus mainly on the first two
cases, since they are covered by the \gtx{} algorithm.

% Namely, let $G$ be a graph over vertex set $V$ with $|V|=n$ and edge set $E$. Furthermore, let $T$
% be a spanning tree of $G$ and $\etest{}$ the edges of $G$ not in $T$. Then we define the
% \emph{average test edge stretch} as $\frac{1}{|\etest{}|} \sum_{(u,v) \in \etest{}}
% |\mathrm{path}^T_{u,v}|$, where $|\mathrm{path}^T_{u,v}|$ is the unique path between $u$ and $v$ in
% $T$.


% However, \textcite[Section 3, page 453]{lognMetricBoundConf03} claim a $O(\log n)$ approximation so maybe I'm
% wrong. Turns out, they refer to a distribution over trees and this $O(\log n)$ is the expected
% stretch of a tree sampled from this distribution

% This defines two kind of structures, spanning trees and spanners (which are still sparse subgraphs
% yet containing more than $|V|-1$ edges).

\paragraph{Trees}
\label{par:trees}

One early mention of seeking a low-stretch spanning tree is given by \textcite{Requirements74},
albeit in more general form:
\begin{problem}[Optimal Communication Spanning Tree]
Given a set of nodes $V=\{v_1, \ldots, v_n\}$, a set of distances $d_{ij}$ and a set of requirements
$r_{ij}$ between $v_i$ and $v_j$, find a spanning tree connecting these $n$ nodes such that the
total cost of communication of the spanning tree is a minimum among all spanning trees. The cost of
communication for a pair of nodes is $r_{i,j}$ multiplied by the sum of the distances of arcs which
form the unique path connecting $v_i$ and $v_j$ in the spanning tree. The cost of a spanning tree is
the sum of costs over all pairs of nodes.
\end{problem}
For a weighted graph $G=(V,E,w)$, by letting $d_{ij} = w_{ij}$ and $r_{i,j}= \Ind{(i,j) \in E}$,
finding an Optimal Communication Spanning Tree thus amounts to finding a low-stretch spanning tree.
\autoref{tab:gtx_related} present a list of works where the stretch was improved.

We start with the seminal paper of \textcite{LowerBound95}. It touches on many topics, and frame the
problem in a game theoretic way but here we only focus on two of their results: a lower bound of
$\Omega(\log n)$ for the average stretch of any tree and their construction of a tree with $\exp
O(\sqrt{\log n\log\log n})$ average stretch in time $O(m^2)$. The lower bound follows from an
existing result in extremal graph theory~\autocite[pages 107--109]{ExtremalGraph04}: there is a
positive constant $a$ such that for all $n\in \Nbb$, one can construct a graph $G$ with $n$ vertices
and $2n$ edges such that every cycle $G$ has a length of at least $a\log n$. Now consider any
spanning tree $T$ of $G$.  While all the $n-1$ edges of $T$ have a stretch of $1$, the $n+1$
remaining ones form a cycle in $T$ hence in $G$ as well and thus incur a stretch of at least $a\log
n$. This shows that the average stretch is at least $\frac{1}{2}a\log n$.

They construct a low stretch spanning tree in a bottom up manner like the \gtx{} algorithm. First,
they extend the definition of stretch to multigraph~\autocite[Section 4]{LowerBound95} and then
describe a procedure to transform in linear time any multigraph $G$ with $n$ nodes to a multigraph
$G'$ on the same nodeset with at most $n(n+1)$ edges such the average stretch of $G'$ is at most
twice that of $G$~\autocite[Lemma 5.2]{LowerBound95}. The next ingredient is an algorithm to build a
low diameter decomposition of a multigraph $G$, parametrized by a number $x(n)$ depending of $n$. It
works by repeatedly selecting an arbitrary node and growing a ball around it until the number of
edges leaving the ball is at most a fraction $\nicefrac{1}{x(n)}$ of the number of edges with both
endpoints in the ball. The key property of this decomposition is that it yields a partition of $G$
in clusters such that the radius of each cluster is small (namely at most $O(x(n)\log n)$) and there
is most a fraction $\nicefrac{1}{x(n)}$ of edges between clusters. Finally, the iterative procedure
is a follows: once a partition has been built, we compute a shortest path spanning tree in each
cluster that are then collapsed into super nodes to form the next graph $G'$ and the process repeats.
Another difference from \gtx{}, besides the partition procedure, is that $G'$ is a multigraph,
taking into account the number of edges joining cluster, while \collapseStar{} picks only the most
direct one.

Another interesting idea from this paper is to consider a distribution over trees instead of a
single instance, especially when one is concern about the maximum stretch instead of the average
one. For instance, on a cycle with $n$ nodes, a tree is obtained by removing one edge, and that edge
incurs a stretch of $n-1$. The uniform distribution over such trees has a maximum stretch of
$2\left(1 - \frac{1}{n}\right)$~\autocite{circle2k89}.

\begin{table}[htbp]
  \centering
  \caption{Reproduction of Table 1 from~\autocite{Abraham2012}, showing the evolution of the best
  asymptotic average stretch over time.}\label{tab:gtx_related}
  \begin{tabular}{lll}
    \toprule
    work                      & average stretch                          & time                    \\
    \midrule
    \autocite{LowerBound95}   & $\exp(O(\sqrt{\log n\log\log n}))$       & $O(m^2)$                \\
    \autocite{LowerStretch05} & $O((\log n)^2 \log \log n)$              & $O(m \log^2 n)$         \\
    \autocite{nearlyTight08}  & $O(\log n(\log \log n)^3)$              & $O(m \log^2 n)$         \\
    % \autocite{nearlyTight08}  & $O(\log n \log\log n(\log\log\log n)^3)$ & $O(m^2)$                \\
    \autocite{TighterSDD11}   & $O(\log n(\log \log n)^3 )$              & $O(m \log n\log\log n)$ \\
    \autocite{Abraham2012}    & $O(\log n \log \log n)$                  & $O(m \log n\log\log n)$ \\
    \bottomrule
  \end{tabular}
\end{table}

The idea of recursively partitioning the graph and construction a low-stretch spanning tree in each
part is common to all the papers of \autoref{tab:gtx_related}. \Textcite{LowerStretch05} devise a
$(\delta, \epsilon)$-star decomposition such that all the stars have comparably low radius.
It was modified in~\autocite{nearlyTight08} to improve the stretch. Then \textcite{TighterSDD11}
improve the runtime by rounding the edge weights to the closest power of $2$ and using a modified
implementation of the Dijkstra's algorithm in the case of at most $k$ distinct edge
weights~\autocite{FastPathFewWeights10}. Finally, \textcite{Abraham2012} describe an even more
complex but tighter petal decomposition.
\iffalse
\begin{marginfigure}
  \centering
  \includegraphics[width=.95\textwidth]{assets/raw/star_decomp.pdf}
  \caption{Star decomposition (reproduced from Figure 1 of~\autocite{LowerStretch05})}
  \label{fig:gtx_star_decomp}
\end{marginfigure}

Special case of graph
series parallel
\enquote{In a subsequent paper, \textcite{seriesParallel06} proved that every series-parallel
unweighted graph admits a spanning tree of average stretch $O(log n)$. This bound is tight as it
matches the lower bound established in~\autocite{cutsTrees99}.}

more special cases are in~\autocite{specialCase14}, although it's for the minimum max stretch
$t^\star$.
\enquote{Note also that a number of particular graph classes (like interval
graphs, permutation graphs, asteroidal-triple–free graphs, strongly chordal graphs,
dually chordal graphs, and others) admit tree $t$-spanners for small values of $t$}
\fi


\paragraph{Spanners}
\label{par:spanners}

As we mentioned, by stopping the \gtx{} algorithm before it finishes, we obtain a set of edges
spanning the graphs that is not a tree. Such structure are called \emph{spanner}. More precisely,
the subgraph $H$ is said to be an $t$-spanner of $G$ if, for a parameter $t \geq 1$, and for every
pair $u, v \in V$ of vertices, it holds that $d_H(u, v) \leq t \cdot d_G(u, v)$. The problem was
introduced by \textcites{SpannerFirst89}{SpannerSecond89} and has been extensively studied since
then, for it has many applications in network design. It was also showed to be \NPh{} to
approximate~\autocite{SpannerNPHard07}. The most simple construction is a greedy
algorithm~\autocite{greedySpanner93} that works similarly to the minimum spanning tree construction.
Starting from an empty subgraph $H$, it goes through every edge $(u, v)$ of $G$ sorted by weight and
check if there is a path between $u$ and $v$ in $H$ of length at most $t$. If it is case the edge
$(u,v)$ is dropped, otherwise it is inserted in $H$. This results in a $(2t - 1)$-spanner with
$O(n^{1+1/t})$ edges, which is an optimal trade-off between those two quantifies.  Furthermore on
weighted graphs, the greedy spanner total weight is essentially optimal~\autocite{GreedyOpt16}.
However, the best implementation of it, using a dynamic data structure~\autocite{fastGreedy04} is
not scalable for it runs in $O(t n^{2+\nicefrac{1}{t}})$ and cannot easily be parallelized.
Parallelization therefore requires other kind of approaches~\autocites{parSpanner08}{parSpanner15}.
Recently, \textcite{Spanner17} showed how to obtain, for any $\epsilon > 0$, a $(2t - 1)$-spanner
with $O(n^{1+1/k}/\epsilon)$ edges in $t$ rounds, with probability at least $1 - \epsilon$.

\iffalse
\url{http://www.siam.org/meetings/da17/schedule.html} SODA 13B \url{http://dl.acm.org/citation.cfm?id=3039686}
for instance the Elkin paper~\autocite{Spanner17} \enquote{Our centralized randomized algorithm computes (with
probability close to 1), a $(2k - 1)$-spanner with $n \cdot (1 + O(\frac{\log k}{n}))$ edges in
$O(|E|)$ time, whenever $k = \Omega(\log n)$. Note that when $k = \omega(\log n)$, the number of
edges is $n(1+o(1))$, i.e., in this range the algorithm computes an ultra-sparse spanner in $O(|E|)$
time.} For instance, if $k=5\log n$, we get a $10\log n$-spanner with $n\left(1+O\left(\frac{\log\log
n}{n}\right)\right)$ edges in $O(|E|)$ time.


They have applications in computing approximately shortest
paths [9, 22, 28, 37], routing [48], distance oracles and
labeling schemes [49, 56, 36] and synchronization [7].

From 6:They also appear in biology in the process of reconstructing phylogenetic trees from
matrices, whose entries represent genetic distances among contemporary living species (H. J.
Bandelt, A. W. M. Dress, Reconstructing the Shape of a Tree from Observed Dissimilarity Data, Adv.
in Appl. Math. 7 (1986)). Robotics researchers have studied spanners under the constraints of
Euclidean geometry, where vertices of the graph are points in space, and edges are line segments
joining pairs of points (Chew), (Dobkin), $[DJ], [K], [KG], [LL]$. 

studied in 1; 4; 6; 9; 15; 19; 22; 24; 26; 28; 30; 31; 37; 43; 51; 52; 57; 58



\begin{tabulary}{\textwidth}{LLLLL}
  \toprule
  work  & average stretch & edge size                              & weighted & time                                             \\
  \midrule
  46    & $4k + 1$        & $O(n^{1+\nicefrac{1}{k}})$             & no       & polynomial                                       \\
  6     & $2k +1$         & $O(n\cdot \ceil{n^{\nicefrac{1}{k}}})$ & yes      & $O\left(m(n^{1+\nicefrac{1}{k}}+n\log n)\right)$ \\
  40    & $2k-1$          & $O(n^{1+\nicefrac{1}{k}}+n)$           & no       & $O(m)$                                           \\
  Elkin & $2k-1$          & $n \cdot (1 + O(\frac{\log k}{n}))$    & no       & $O(m)$                                           \\
  \bottomrule
\end{tabulary}

% 22: E. Cohen, "Fast algorithms for constructing t-spanners and paths with stretch t," Proceedings
% of 1993 IEEE 34th Annual Foundations of Computer Science, Palo Alto, CA, 1993, pp. 648-658.  doi:
% 10.1109/SFCS.1993.366822
% We construct t-spanners of size (number of edges) Õ(n 1+(2+\epsilon)/t )
% (for any \epsilon > 0 and t such that t/(2+\epsilon) is integral). These spanners can be constructed
% by a randomized algorithm that runs in Õ(mn (2+\epsilon)/t ) time.


% Halperin and Zwick [40].  Their deterministic algorithm, for an integer parame- ter k ≥ 1,
% computes a (2k − 1)-spanner with n 1+1/k + n edges in O(|E|) time. (Their result improved previous
% pioneering work by [46, 22].)

Describe the greedy algorithm of 6. Using (55 On Dynamic Shortest Paths Problems Liam Roditty, Uri
Zwick, 2004), it runs in $O(\alpha n^{2+\nicefrac{1}{\alpha}})$

peleg 2007 hardness results

46: Peleg, D. and Schäffer, A. A. (1989), Graph spanners. J. Graph Theory, 13: 99–116.
doi:10.1002/jgt.3190130114

study the problem on unweighted graph. Application to routing scheme (48: D. Peleg and E. Upfal, A
tradeoff between space and efficiency for routing tables. 20th ACM Symposium on the Theory of
Computing, Chicago (1988))
Special case for the complete graph weighted by the distance in a 2D plan. Existing $\sqrt{10}$
spanner for the $\ell_1$ metric (L. P. Chew, There is a planar graph almost as good as the complete
graph.  Proceedings of the 2nd ACM Symposium on Computational Geometry, (1986)) (improved to
$\sqrt{4+2\sqrt{2}}$ by N. Bonichon, C. Gavoille, N. Hanusse, L. Perkovic The stretch factor of
$\ell_1$ and $\ell_{\infty}$ Delaunay triangulations European Symposium on Algorithms (ESA) (2012))
and $\phi \pi$ for $\ell_2$ (D. P. Dobkin, S. J. Friedman, and K. J. Supowit, Delaunay graphs are
almost as good as complete graphs. 28th IEEE Symposium on the Foundations of Computer Science,
(1987)) (improved to $1.998$ by Ge Xia. 2011. Improved upper bound on the stretch factor of delaunay
triangulations. In Proceedings of the twenty-seventh annual symposium on Computational geometry
(SoCG '11)). See (Prosenjit Bose, Michiel Smid, On plane geometric spanners: A survey and open
problems, Computational Geometry, Volume 46, Issue 7, 2013) for more on the 2D case.

In undirected graph, finding a spanner with less than $k$ edges is \NPc{} (theorem 2.2)
For $k<n$, one can construct in polynomial time a $(4\log_k n +1)$ spanner with less than $kn$ edges
(theorem 2.4) giving for instance ($k=2$) $O(\log n)$ spanner with $O(n)$ edges and
($k=n^{\nicefrac{1}{r}}$, $r\geq 1$) a $(4r+1)$ spanner with $O(n^{1+\nicefrac{1}{r}})$ edges
(matching lower bound within constant factor). For every $d \geq 0$, the $d$-dimensional cube has a
$3$-spanner with fewer than $7\time 2^d$ edges (Lemma 2.10 from 47: D. Peleg and J. D. Ullman. An
optimal synchronizer for the hypercube. SIAM J. on Comput., 18:740–747, 1989)
For chordal graphs: for every $n$-vertex chordal graph there exists a $2$-spanner with
$O(n\sqrt{n})$ edges (matching lower bound), a $3$-spanner with $O(n \log n)$ edges and a $5$-spanner
with $O(n)$ edges.
Much more difficult for directed graph, according to Theorem 4.2: For every $t \geq 1$ there are
infinitely many $n$-vertex directed graphs for which every $t$-spanner requires
$\Omega(\nicefrac{n^2}{t^2})$ edges.

% check some surveys of the 80's
% http://pubsonline.informs.org/doi/abs/10.1287/trsc.18.1.1
% http://onlinelibrary.wiley.com/doi/10.1002/net.3230190305/full
% early solutions
% http://onlinelibrary.wiley.com/doi/10.1002/net.3230090104/full
% http://onlinelibrary.wiley.com/doi/10.1002/net.3230130309/full
% later solution?
% http://ieeexplore.ieee.org/document/81738


While they have many applications [see first paragraph of \url{https://arxiv.org/pdf/1401.2454.pdf},
which was later merged in a STOC'14 paper] (a major one being solving linear systems), in some
practical situations their advantages are less clear [from
\url{https://link.springer.com/chapter/10.1007/978-3-319-20086-6_16}\enquote{for reasonable inputs
the constant factors make the solver much slower than methods with higher asymptotic complexity.
One other aspect predicted by theory is confirmed by our findings: Spanning trees with lower
stretch indeed reduce the solver's running time. Yet, simple spanning tree algorithms perform
better in practice than those with a guaranteed low stretch.} this is improved by
\url{https://link.springer.com/chapter/10.1007%2F978-3-319-20086-6_17} although they seem to work
	mostly with the Laplacian of the tree ]
\fi

\paragraph{Other problems}

Finding low stretch trees and spanners with respect to the existing edges is the most relevant
problem when addressing the \esp{} problem. For the sake of completeness, we nonetheless give an
overview of some related problems.

For instance, \textcite{Johnson1978} define the following problem, where the stretch is defined over
all possible pairs of nodes\footnote{We adapt their notations
to match ours}:  
\begin{problem}[Network Design Problem]
  \label{prob:gtx_ndp}
  Given an undirected integer-weighted graph $G=(V, E, w)$, a budget $B\in\Nbb$ and a criterion
  threshold $C\in \Nbb$, does there exist a spanning subgraph $G'=(V, E')$ of $G$ with weight
  $w(E') \leq B$ and criterion value $F(G') \leq C$, where the criterion function $F(G')$ denotes
  the sum of the weights of the shortest paths in $G'$ between all vertex pairs?
\end{problem}
They prove that finding such a subgraph is \NPc{}, by exhibiting a reduction from the
\textsc{Knapsack} problem. They also prove that the less general problem of finding a spanning tree
on an unweighted graph, that is
\vspace{-.5\baselineskip}
\begin{problem}[Simple Network Design Problem]
  \autoref{prob:gtx_ndp} with $w$ being the equal to $1$ for all edges in $E$ and $B=|V|-1$.
\end{problem}%
\vspace{-.5\baselineskip}
\noindent is also \NPc{} by reduction from \textsc{Exact 3-Cover}.
However, it has recently been show that this Simple Network Design problem can be approximated to a
constant factor $6$~\autocite{AllPairStrech10}. Moreover, even when the graph is weighted,
\textcite{constantDistortion07} achieve a universal constant bound for any weighted graph.

Another problem appear when the low-stretch structure $H$ can include edges not in $G$ (as long as
the distances in $H$ remain larger than the distances in $G$). This is captured by the following
problem~\autocite{OptimalNetwork69}:
\begin{problem}[Optimal Network Problem]
  \label{prob:gtx_scott}
  Given a set $V$ of $n$ vertices, find a set of spanning edges $E\subset V^2$ that minimizes
  the sum of the length of the shortest paths  between all vertex pairs while the
  total length of the resulting network does not exceed some upper bound $B\in\Nbb$.
\end{problem}
This can be seen as a special case of \autoref{prob:gtx_ndp} with $G$ being the unweighted
$n$-complete graph. \Textcite{OptimalNetwork69} proposes a backtracking solution and two local search approximate
algorithms. Some early branch and bound heuristic solutions to \autoref{prob:gtx_scott} are surveyed
in~\autocite[Section 2.3.2]{networkDesignSurvey89} although they do not come with asymptotic
guarantee on the stretch. Furthermore, \textcite{optimApproxNP80} proves that for any $\epsilon \in
(0,1)$, finding a $|V|^{1-\epsilon}$ approximation is \NPc{}.
However, if we consider the average stretches over a distribution of trees, then this approximation
factor can be reduced to $\Theta(\log n)$~\autocite{lognMetricBoundConf03}.

Finally, the stretch can also be computed for a subset of the edges. This is useful in cases where
we have prior information on the importance of individual nodes or edges.  For instance,
\textcite{RamseyTree17} show that for every $t$, any $n$-nodes graph $G=(V,E)$ has a subset $S$ of
size at least $n^{1 - \nicefrac{1}{k}}$, and a spanning tree that has stretch $O ( k \log \log n)$
between any node in $S$ and any node in $V$. Likewise, \textcite{mLAST17} describe how to maintain a
light subgraph $H$ that minimizes the distance between pairs of source and sink that are given in an
online fashion.

As shown by \autoref{tab:gtx_related_stretch}, those problems defined in the seventies are still
being discussed nowadays in top tier conferences, proving their relevance and impact beyond the
\esp{} problem.

\setlength{\fullpage}{\textwidth+\marginparsep+\marginparwidth}
\begin{table}[htbp]
  \centering
  \caption{A summary of the lowest stretches achievable for various problems.
  \label{tab:gtx_related_stretch}}
  \begin{tabulary}{\fullpage}{LCCL}
    \toprule
    kind of stretch    & \multicolumn{2}{c}{only existing edges}  & extra edges allowed   \\
    \midrule
                       & tree                                     & not tree             &\\
    \cmidrule(r){2-3}
    some pairs         & $O(k\log\log n)$~\autocite{RamseyTree17} & \autocite[Section 4]{mLAST17} & --- \\
    all existing pairs & $O\left(\log n (\log\log n)\right)$~\autocite{Abraham2012}
		       & $(2t - 1)$-spanner, $O(n^{1+1/t})$ edges~\autocite{greedySpanner93}
		       & $\Theta(\log n)$ in expectation~\autocite{lognMetricBoundConf03} \\
    all possible pairs & $6$ for unweighted graphs \autocite{AllPairStrech10} and $O(1)$
                         in general \autocite{constantDistortion07}
		       & ---
		       & no need for extra edges in that case                              \\
    \bottomrule
  \end{tabulary}
\end{table}


\subsection{Empirical evaluation}
\label{sub:gtx_empirical_evaluation}

In this \nameref{sub:gtx_empirical_evaluation}, we provide empirical evidences of the properties of
\gtx{} over several classes of graph, and compare it with a \bfs{} baseline.\marginpars{If time
allows, it would be interesting to implement some methods of \vref{sub:gtx_state_of_the_art} and add
them to the comparison}\todo*{implement more low stretch methods} Namely, we consider three kinds of
graph topology (with both synthetic and real world instances that carry signs on their edges) and
evaluate $(i)$ what average stretch is reached by various trees and $(ii)$ how accurate is the sign
prediction.

\subsubsection{Graph topology}

The three kinds of topology we consider are:
\begin{description}
	\item[\grid{}] which are 2D lattices, where each node has four neighbors except on the boundary.
		The synthetic ones are square, while the \enquote{real world} ones represents the four neighbors
		pixel connectivity of the pictures showed in \autoref{fig:gtx_xp_bwpics}.
	\item[\lpa{}] which are built synthetically according to the model of \textcite{Barabasi1999}.
		While this does not follow the more rigorous specification of \textcite{PAmodel04}, informally,
		we start with a line graph of $m$ nodes and add node one by one until the graph consists of $n$
		nodes. Each time a new node is added, it is connected to $m$ of the existing nodes with a
		probability proportional to their degree. Here we choose $m=3.13$, that is when adding a new
		node, we pick $3$ or $4$ existing neighbors such the initial expected number of neighbors for
		each new nodes is $3.13$. Such graphs are quite sparse and have short diameter, thus providing a
		crude but reasonable approximation of online social networks. Therefore, the real world
		instances of the \lpa{} model are \wik{}, \sla{} and \epi{}\marginpars{as used in the first
		chapter}\todo*{add a ref to first chapter} along with \gplus{}. The last one is constructed from
		ego networks of \gplus{}\footnote{Available at
		\url{http://snap.stanford.edu/data/egonets-Gplus.html}} by keeping the largest connected
		component of users whose gender is known. Basic statistics of those real \lpa{} graphs are
		presented in (\autoref{tab:gtx_xp_dataset}). 
	\item[\triangle{}] which consists of a Delaunay triangulation of random 2D points\footnote{As
		implemented by the \textsf{graph-tool} library (\url{https://graph-tool.skewed.de})}.
\end{description}

\begin{table}[htpb]
	\centering
	\caption{Dataset description }\label{tab:gtx_xp_dataset}
	\begin{tabular}{lrrcc}
		\toprule
             & $|V|$  & $|E|$    & fraction of $+$ edges & $\frac{2|E|}{|V|\cdot(|V|-1)}$ \\
		\midrule
		\wik{}   & \np{7065}   & \np{99936}    & 78.5\%                & $4.00\cdot 10^{3}$             \\
		\gplus{} & \np{74917}  & \np{10130461} & 67.6\%                & $3.61\cdot 10^{3}$             \\
		\sla{}   & \np{82052}  & \np{498527}   & 76.4\%                & $1.48\cdot 10^{4}$             \\
		\epi{}   & \np{119070} & \np{701569}   & 83.2\%                & $9.90\cdot 10^{5}$             \\
		\bottomrule
	\end{tabular}
\end{table}

\begin{figure}[t]
	\centering
	\begin{subfigure}[b]{0.32\textwidth}
		\includegraphics[width=\textwidth]{gtx_exp/zmonastery}
	\end{subfigure}~
	\begin{subfigure}[b]{0.32\textwidth}
		\includegraphics[width=\textwidth]{gtx_exp/zworld}
	\end{subfigure}~
	\begin{subfigure}[b]{0.32\textwidth}
		\includegraphics[width=\textwidth]{gtx_exp/nips_poster}
	\end{subfigure}

	\begin{subfigure}[b]{0.32\textwidth}
		\includegraphics[width=\textwidth]{gtx_exp/zmonastery_bin}
		\caption{monastery}
	\end{subfigure}~
	\begin{subfigure}[b]{0.32\textwidth}
		\includegraphics[width=\textwidth]{gtx_exp/zworld_bin}
		\caption{world}
	\end{subfigure}~
	\begin{subfigure}[b]{0.32\textwidth}
		\includegraphics[width=\textwidth]{gtx_exp/nips_poster_bin}
		\caption{poster}
	\end{subfigure}

	\begin{subfigure}[b]{0.32\textwidth}
		\includegraphics[width=\textwidth]{gtx_exp/nips_logo}
	\end{subfigure}~
	\begin{subfigure}[b]{0.32\textwidth}
		\includegraphics[width=\textwidth]{gtx_exp/space}
	\end{subfigure}~
	\begin{subfigure}[b]{0.32\textwidth}
		\includegraphics[width=\textwidth]{gtx_exp/waterfall}
	\end{subfigure}

	\begin{subfigure}[b]{0.32\textwidth}
		\includegraphics[width=\textwidth]{gtx_exp/nips_logo_bin}
		\caption{logo}
	\end{subfigure}~
	\begin{subfigure}[b]{0.32\textwidth}
		\includegraphics[width=\textwidth]{gtx_exp/space_bin}
		\caption{space}
	\end{subfigure}~
	\begin{subfigure}[b]{0.32\textwidth}
		\includegraphics[width=\textwidth]{gtx_exp/waterfall_bin}
		\caption{waterfall}
	\end{subfigure}
	\caption{Real world pictures and their binarized version}\label{fig:gtx_xp_bwpics}
\end{figure}

\subsubsection{Stretch}

The first property of Galaxy trees we wish to evaluate is their stretch, which depends only of graph
topology. Namely, let $G$ be a graph over vertex set $V$ with $|V|=n$ and edge set
$E$.\Todo[MOVE]{Stretch definition is likely to happen somewhere earlier} Furthermore, let $T$ be a
spanning tree of $G$ and $\etest{}$ the edges of $G$ not in $T$. Then we define the \emph{average
test edge stretch} as $\frac{1}{|\etest{}|} \sum_{(u,v) \in \etest{}} |\mathrm{path}^T_{u,v}|$,
where $|\mathrm{path}^T_{u,v}|$ is the unique path between $u$ and $v$ in $T$.

As we consider unweighted graphs, we compare \gtx{} with a natural baseline, namely a spanning tree
rooted at the highest degree node and obtained through a breadth first visit of the graph. This
involves randomness in order in which nodes are visited. Likewise in \gtx{}, the choice of the edge
linking two stars is not always unique, meaning that we have to break ties at random.  Therefore,
for each graph, we repeat the tree construction 12 times and present the average result, noting that
the variance (showed as error bar in \autoref{fig:gtx_xp_st}) is small.

On \lpa{} and \triangle{}, we see that both trees exhibits logarithmic stretch, although with a
larger constant for \gtx{}. Note that this is also the case for others low stretch tree methods
\autocite[\S 5.3.1]{papplow}. On \grid{} however, \gtx{} preserves this logarithmic stretch growth
while this is visually no longer the case for \bfs{}.
In that case, we cannot expect a better stretch than $\frac{\log n}{2048}$ according to
\autocite[Theorem 6.6]{LowerBound95}.

\begin{figure}[tbh]
	\centering
	\begin{subfigure}[b]{0.9\textwidth}
		\includegraphics[width=\textwidth]{gtx_exp/gridst}
		\caption{\grid{} }\label{fig:gtx_xp_gridst}
	\end{subfigure}

	\begin{subfigure}[b]{0.9\textwidth}
		\includegraphics[width=\textwidth]{gtx_exp/past}
		\caption{\lpa{} }\label{fig:gtx_xp_past}
	\end{subfigure}

	\begin{subfigure}[b]{0.9\textwidth}
		\includegraphics[width=\textwidth]{gtx_exp/trst}
		\caption{\triangle{} }\label{fig:gtx_xp_trst}
	\end{subfigure}
	\caption{Stretch over graphs of increasing size}\label{fig:gtx_xp_st}
\end{figure}

\subsubsection{Sign prediction}

The second design goal of Galaxy trees is to accurately predict the sign of edges in $\etest{}$.
Except for the three real datasets that already include signs\footnote{We nonetheless perform some
preprocessing in order to make them undirected to remove the small proportion of conflicting edges
(e.g. positive from $u$ to $v$ but negative from $v$ to $u$).}, all the other are constructed,
meaning we have to set sign on their edges in the first place. This is done by partitioning the
nodes into two clusters. For \gplus{} we use node gender, for pictures we use node color (black or
white), and for all others, we propagate labels $0$ and $1$ from randomly selected high degree nodes.
Once each node belongs to one of the two clusters, we set the sign of an edge between two nodes to
be $+$ if they are in the same cluster and $-$ otherwise.  Predicting using path parity will thus
gives perfect result. To test performance in real or adversarial situation, we then add noise, that
is we select a fraction of edges uniformly at random and flip their sign. 

We evaluate the performance of our prediction using the Matthews Correlation\Todo{merge this
definition of MCC with the one in first chapter} Coefficient (MCC)~\autocite{MCC00} \[
	\mathrm{MCC} = \frac{ TP \times TN - FP \times FN } {\sqrt{ (TP + FP) ( TP + FN ) (
			TN + FP ) ( TN + FN ) } } = \pm \sqrt{\frac{\chi^2}{n}}
\]
Since we do not have confidence score, we cannot use AUC. Yet we have to account for the large sign
unbalance and thus cannot rely on accuracy or $F_1$ measure.  Therefore we choose MCC, which
combines all the four numbers of the confusion matrix in a single metric. It ranges from $+1$
(perfect prediction) to $-1$ (inverse prediction) through $0$ (random prediction). As a
demonstration of MCC usefulness, predicting all edges but one to be positive on Slashdot gives
$.764$ accuracy, $.886$ $F_1$ score\marginpars{Actually the $F_1$ score is $.866$ for positive
edges, $0$ for negative ones and $.661$ if we can take an average weighted by class size.} but
$-0.0007$ MCC.

As showed in \autoref{fig:gtx_xp_mcc}, when the noise level is low, \gtx{} performs better than
\bfs{}. As the noise level gets higher, they have similar performance. Note also than in
\autoref{fig:gtx_xp_pasynthmcc}, \gtx{} is less sensible to the size of the graph.

\begin{figure}[tbh]
	\centering
	\begin{subfigure}[b]{0.47\textwidth}
		\includegraphics[width=\textwidth]{gtx_exp/grsynthmcc}
		\caption{Synthetic \grid{} }\label{fig:gtx_xp_grsynthmcc}
	\end{subfigure}~
	\begin{subfigure}[b]{0.47\textwidth}
		\includegraphics[width=\textwidth]{gtx_exp/grrwmcc}
		\caption{Pictures \grid{} }\label{fig:gtx_xp_grrwmcc}
	\end{subfigure}
	\begin{subfigure}[b]{0.47\textwidth}
		\includegraphics[width=\textwidth]{gtx_exp/pasynthmcc}
		\caption{Synthetic \lpa{} }\label{fig:gtx_xp_pasynthmcc}
	\end{subfigure}~
	\begin{subfigure}[b]{0.47\textwidth}
		\includegraphics[width=\textwidth]{gtx_exp/trmcc}
		\caption{\triangle{} }\label{fig:gtx_xp_trmcc}
	\end{subfigure}
	\begin{subfigure}[b]{0.47\textwidth}
		\includegraphics[width=\textwidth]{gtx_exp/parwmcc}
		\caption{Real world network }\label{fig:gtx_xp_parwmcc}
	\end{subfigure}
	\caption{MCC over various graphs}\label{fig:gtx_xp_mcc}
\end{figure}

To further assess the quality of our trees, we plug them in them into a successful heuristic method
to predict edge sign: \asym{}~\autocite{Kunegis2009}. \Todo{It might also be interesting to see if
that would be a good training set for our troll method, although it has to be checked it makes sense
from a running time point of view.} It computes the exponential of the adjacency matrix after it has
been reduce to $z$ dimension. This allows to count the sign of all paths between two pairs of nodes
with decreasing weight depending of their length. To simulate an active learning setting, we reveal
only a subset of edge in $A$. This subset can be: $i)$ the edges forming a \bfs{}, $ii)$ the edges
forming a \gtx{} $iii)$ $|V|-1$ edges chosen uniformly at random.

We set the parameter $z$ equal to $15$ because $i)$ it is one of the best in \cite[Fig.
11]{Kunegis2009}, $ii)$ it performs well on real dataset in \cite[Fig.3]{Cesa-Bianchi2012a}, and
$iii)$ it was good in our initial testing (\texttt{20150401\_wed\_spectral.ipynb}).

As the \asym{} has a $O(n^3)$ complexity and uses quite some memory at prediction time, the larger
graphs used previously are not all included. The conclusion of \autoref{fig:gtx_xp_asym} is that
except on social network, it is better to use spanning trees than random edges. Specifically, \gtx{}
on \grid{} and \bfs{} elsewhere.

\begin{figure}[tbh]
	\centering
	\begin{subfigure}[b]{0.47\textwidth}
		\includegraphics[width=\textwidth]{gtx_exp/grsynthasym}
		\caption{Synthetic \grid{} \label{fig:gtx_xp_grsynthasym}}
	\end{subfigure}~
	\begin{subfigure}[b]{0.47\textwidth}
		\includegraphics[width=\textwidth]{gtx_exp/grrwasym}
		\caption{\enquote{Real} \grid{} }\label{fig:gtx_xp_grrwasym}
	\end{subfigure}
	\begin{subfigure}[b]{0.47\textwidth}
		\includegraphics[width=\textwidth]{gtx_exp/pasynthasym}
		\caption{Synthetic \lpa{} }\label{fig:gtx_xp_pasynthasym}
	\end{subfigure}~
	\begin{subfigure}[b]{0.47\textwidth}
		\includegraphics[width=\textwidth]{gtx_exp/trasym}
		\caption{\triangle{} }\label{fig:gtx_xp_trasym}
	\end{subfigure}
	\begin{subfigure}[b]{0.47\textwidth}
		\includegraphics[width=\textwidth]{gtx_exp/parwasym}
		\caption{Real world network }\label{fig:gtx_xp_parwasym}
	\end{subfigure}
	\caption{\asym{} over various graphs}\label{fig:gtx_xp_asym}
\end{figure}

Finally\marginpars{Actually I never did it because \shz{} wasn't implemented at the time, so now is
a good occasion}\todo*{Run shazoo on galaxy tree} we also compare \gtx{} with \bfs{} and \rst{} on
the task of nodes prediction using \shz{} algorithm~\autocite{Vitale2012}.



\section{Conclusions}
\label{sec:cc_conclusions}

\begin{itemize}
	\item improve stable \pcc{}
	\item give guarantees for \gtx{}
	\item handle weights in \gtx{}
\end{itemize}

