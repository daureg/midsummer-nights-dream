Now that we are familiar with the \pcc{} problem and its numerous applications, let us return to the
original problem of this chapter, predicting the edge sign of a signed graph. In order to do so, we
will first focus on the case where a signed graph has no disagreements (that is the minimum of
\eqref{eq:mind} is $0$), and the connection with the balance theory.

In his seminal work on interpersonal relations,
\citeauthor*{Heider46}~\autocites{Heider46}{HeiderBook58} argues by
psychological and sociological arguments that in order to reduce their cognitive dissonance, three
people always interact in a way that preserve social balance. This social balance requirements can be
succinctly summarized by four statements~\autocite{HeiderBook58}:

\begin{enumerate}
	\item \textcolor{DodgerBlue}{my friend's} \textcolor{Orange}{friend} is
		\textcolor{Olive}{my friend}\hspace{1em}%
		\tikz[baseline=5]{
			\node[draw,circle,inner sep=2pt] (me) at (0,0) {};
			\node[draw,circle,inner sep=2pt] (mine) at (.5,.5) {};
			\node[draw,circle,inner sep=2pt] (alter) at (1,0) {};
			\draw[] (me) -- node[left,DodgerBlue,pos=.9] {$+$} (mine) ;
			\draw[] (mine) -- node[right,Orange,pos=.1] {$+$} (alter) ;
			\draw[white] (me) -- node[text=Olive] {$+$} (alter) ;
		}
	\item \textcolor{DodgerBlue}{my friend's} \textcolor{Orange}{enemy} is
		\textcolor{Olive}{my enemy}\hspace{1em}%
		\tikz[baseline=5]{
			\node[draw,circle,inner sep=2pt] (me) at (0,0) {};
			\node[draw,circle,inner sep=2pt] (mine) at (.5,.5) {};
			\node[draw,circle,inner sep=2pt] (alter) at (1,0) {};
			\draw[] (me) -- node[left,DodgerBlue,pos=.9] {$+$} (mine) ;
			\draw[] (mine) -- node[right,Orange,pos=.1] {$-$} (alter) ;
			\draw[white] (me) -- node[text=Olive] {$-$} (alter) ;
		}
	\item \textcolor{DodgerBlue}{my enemy's} \textcolor{Orange}{friend} is
		\textcolor{Olive}{my enemy}\hspace{1em}%
		\tikz[baseline=5]{
			\node[draw,circle,inner sep=2pt] (me) at (0,0) {};
			\node[draw,circle,inner sep=2pt] (mine) at (.5,.5) {};
			\node[draw,circle,inner sep=2pt] (alter) at (1,0) {};
			\draw[] (me) -- node[left,DodgerBlue,pos=.9] {$-$} (mine) ;
			\draw[] (mine) -- node[right,Orange,pos=.1] {$+$} (alter) ;
			\draw[white] (me) -- node[text=Olive] {$-$} (alter) ;
		}
	\item \textcolor{DodgerBlue}{my enemy's} \textcolor{Orange}{enemy} is
		\textcolor{Olive}{my friend}\hspace{1em}%
		\tikz[baseline=5]{
			\node[draw,circle,inner sep=2pt] (me) at (0,0) {};
			\node[draw,circle,inner sep=2pt] (mine) at (.5,.5) {};
			\node[draw,circle,inner sep=2pt] (alter) at (1,0) {};
			\draw[] (me) -- node[left,DodgerBlue,pos=.9] {$-$} (mine) ;
			\draw[] (mine) -- node[right,Orange,pos=.1] {$-$} (alter) ;
			\draw[white] (me) -- node[text=Olive] {$+$} (alter) ;
		}
\end{enumerate}

This can be readily translated into graph properties. Given a path of length two over three nodes,
note that the first part of each statement (in blue and orange) defines the four possible sign
assignment of such a path. The last part of the statement (in green) then prescribes which sign
should close the triangle to respect the social balance. As we can see, this means that triangles can
only have zero or two negative signs, or equivalently, that no triangle can have an odd number
number of negative signs.

From this local observation, we define a graph as strongly balanced if none of its cycle has an odd
number of negative signs
in that case we have \textcite[Theorem 3]{harary1953} stating:
statement
proof
According to \textcite{Huffner2010}, a similar theorem was proved earlier by \textcite[Theorem
X.11]{Konig36}, although \textcite{Zaslavsky2012} notes that it was stated \enquote{without the
	terminology of signs, while \autocite{harary1953}
has the first recognition of the crucial fact that labelling edges by elements of a
group---specifically, the sign group---can lead to a general theory.}

when graph is strongly balanced, the 2 clusters of previous claim gives a natural partition with no
disagreements in the sense of \pcc{}. However, the existence of 2 such clusters is only a sufficient
but not necessary condition for having no disagreement. As hinted in
\autoref{sub:need_for_a_directed_graph}, when the nodes are partitioned into $k$ clusters, letting
edges within clusters to be positive and edges across clusters to be negative will by definition
results in zero disagreement. \Textcite{davis1967clustering} show that this $k>2$ clusters partition
is equivalent to the absence of cycle with a single negative edge, namely
statement
proof
In terms of balance, this corresponds to allowing triangle to have three negative edges, and is called
weak balance, as illustrated on \autoref{fig:cc_balance}.

\begin{figure}[htpb]
	\centering
	\includegraphics[width=0.8\linewidth]{assets/raw/triangle_smaller.pdf}
	\caption{The four possible undirected triads, as classified by the two structural balance theories
	introduced in the main text} \label{fig:cc_balance}
\end{figure}

In real data though, we do not expect neither strong nor weak balance to hold, for they are fairly
demanding model. Indeed, three of the real networks considered in \autoref{chap:troll} have been
repeatedly shown to be unbalanced, although the extent of this unbalance depends of the importance
given to longer cycle compared with triads~\autocites{Facchetti2011isingmodel}{measureUnbalance14}%
{measureUnbalance17} For the same reason, in virtually any real instance of the \pcc{} problem, the%
% there are also \url{https://arxiv.org/abs/1509.04037} about measuring balance (or frustration
% index) which they claimed has been peer reviewed?
optimal solution will incur some disagreements. \Textcite[Section 5.5]{BookKleinberg2010}
nonetheless prove that even when only a fraction of the triangle are strongly balanced in a complete
graph, the two clusters structure is still present, although it does not cover the whole graph
anymore. Formally statement
The proof makes essential use of the strong balance (which is ok since next section also deal with
2-CC) and the completeness of the graph (which is a far less realistic assumption) yet it would be
interesting to see if it can be extended.

Said that, how is \pcc{} useful in predicting edge sign? At first sight, the two problems seem quite
different for the former is an unsupervised/agnostic clustering problem while the latter is a
supervised classification problem. Yet, getting an optimal solution of \pcc{} on the training set
would provide a principled heuristics for \esp{}. Unfortunately, as we shall see in
\autoref{sub:state_of_the_art}, exact or fixed parameter solution are not scalable and one has to
rely approximation. Under the Unique Games Conjecture, the approximation factor is lower bounded by
$\Theta(\log n)$ in the worst case. However, this is still a viable option, as our bias is that we
will operate on instances that are to some extent balanced. Therefore, we are not in the worst case
and we can find some comfort in results about the noisy (\vpageref{sub:variants_and_extensions}) and
stable (\autoref{ssub:cc_under_stability_assumption}) settings.

Another connection we will leverage in \autoref{sec:low_stretch_trees_and_spanners} is that the
value of the \mind{} objective is a complexity measure of the \esp{} problem~\autocite[Section
4.1]{Cesa-Bianchi2012b}. Before that, we will also present some variants and extentsion of \pcc{},
some of them bearing similarities with the Chapter 3's problem.


\iffalse
strong balance random graph model \url{https://www.lri.fr/~yannis/randsigned.pdf}, line index is basically
the objective value of CC and if I'm reading their bound correctly,  basically you need to flip half
the edges of a random graph to make it balanced (also most of their results hold when the
probability of an existing edge is 1/2, but it was strengthened in a follow up dealing with weak
balance \url{http://people.maths.ox.ac.uk/harutyunyan/weak-balance.pdf})

For the 2 clusters case ($k=2$), characterization proven already in 1936 by \textcite{Konig36}, as noted in
\cite{Zaslavsky2012} (commenting on \cite{harary1953}: \enquote{Although Theorem 3 was anticipated by
\textcite[Theorem X.11]{Konig36}  without the terminology of signs, here is the
first recognition of the crucial fact that labelling edges by elements of a
group---specifically, the sign group---can lead to a general theory.})
and \cite{Huffner2010} (\enquote{\textcite{Konig36} proved the following characterization of
	balanced graphs. For a graph $G = (V, E)$, the following are equivalent:\marginpars{There is a
	proof in \autocite[p. 111]{BookKleinberg2010}, maybe I can rewrite it as well}
	\begin{enumerate}
		\item $V$ can be partitioned into two sets $V_1$ and $V_2$ called sides such that there is no
			negative edge $\{v, w\} \in E$ with both $v, w \in V_1$ or both $v, w \in V_2$ and no positive
			edge $\{v, w\}$ with $v \in V_1$ and $w \in V_2$ .
		\item $V$ can be colored with two colors such that for all $\{v, w\} \in E^-$, the vertices $v$
			and $w$ have different colors, and for all $\{v, w\} \in E^+$, the vertices $v$ and $w$ have
			the same color. The color classes correspond to the sides.
		\item $G$ does not contain cycles with an odd number of negative edges.
	\end{enumerate}
	Using the characterization by a coloring, it is easy to see that balance of a signed
graph can be checked in linear time by depth-first search.})
\fi
