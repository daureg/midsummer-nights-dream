\documentclass[a4paper,11pt,oneside,onecolumn,openany,final,svgnames]{memoir}
\usepackage[utf8]{inputenc}
\usepackage[T1]{fontenc}
\usepackage{csquotes}
\usepackage{babel}
\usepackage[heightrounded]{geometry}
\usepackage{setspace}
\usepackage{microtype}
\usepackage[citestyle=alphabetic, bibstyle=ieee-alphabetic, isbn=false, maxnames=1, minnames=1,
sorting=nyvt, backref=true, backend=biber, defernumbers=true]{biblatex}
\AtEveryBibitem{
   \clearfield{arxivId}
   % \clearfield{booktitle}
   % \clearfield{doi}
   \clearfield{eprint}
   \clearfield{eventdate}
   \clearfield{isbn}
   \clearfield{issn}
   % \clearfield{journaltitle}
   \clearfield{month}
   % \clearfield{number}
   % \clearfield{pages}
   \clearfield{series}
   \clearfield{url}
   \clearfield{urldate}
   \clearfield{venue}
   % \clearfield{volume}
   \clearlist{location} % alias to field 'address'
   \clearlist{publisher}
   \clearname{editor}
}
\addbibresource{/home/orphee/data/projects/biblio/library.bib}
\addbibresource{more.bib}
\addbibresource{gtx.bib}
\usepackage{xcolor}
\usepackage{amsmath}
\usepackage{amsthm}
\usepackage{mathtools}
\usepackage{sidenotes}
\renewcommand{\footnote}[1]{\sidenote{\footnotesize #1}}
\usepackage{enumitem}
\usepackage{tabulary}
\usepackage{multirow}
\usepackage[marginpar]{todo}
\usepackage[np,autolanguage]{numprint}
\usepackage{listings}
\lstset{frame=none, language=Python, tabsize=4,}
% \usepackage{inconsolata}
\usepackage{nicefrac}
\usepackage{tikz}
\usetikzlibrary{arrows,automata,calc}
\usetikzlibrary{arrows.meta}
\usetikzlibrary{decorations.pathreplacing}
\usetikzlibrary{backgrounds}
\tikzset{%
  show curve controls/.style={
    postaction={
      decoration={
        show path construction,
        curveto code={
          \draw [blue] 
            (\tikzinputsegmentfirst) -- (\tikzinputsegmentsupporta)
            (\tikzinputsegmentlast) -- (\tikzinputsegmentsupportb);
          \fill [red, opacity=0.5] 
            (\tikzinputsegmentsupporta) circle [radius=.25ex]
            (\tikzinputsegmentsupportb) circle [radius=.25ex];
        }
      },
      decorate
}}}
\tikzstyle{vertex}=[draw,circle,black,inner sep=2pt]
\tikzstyle{edge}=[line width=1.3pt,color=Black]
\tikzstyle{rare}=[fill=black,text=white]
\tikzstyle{medium}=[fill=black!15!white]


\usepackage{algorithm}
\usepackage[noend]{algpseudocode}
\newcommand*\Let[2]{\State #1 $\gets$ #2}
\def\algorithmautorefname{Algorithm}

\usepackage{varioref}
\usepackage{hyperref}
\hypersetup{%
    % draft,    % = no hyperlinking at all (useful in b/w printouts)
    colorlinks=true, linktocpage=true, pdfstartpage=3, pdfstartview=FitV,%
    % uncomment the following line if you want to have black links (e.g., for printing)
    %colorlinks=false, linktocpage=false, pdfborder={0 0 0}, pdfstartpage=3, pdfstartview=FitV,%
    breaklinks=true, pdfpagemode=UseNone, pageanchor=true, pdfpagemode=UseOutlines,%
    plainpages=false, bookmarksnumbered, bookmarksopen=true, bookmarksopenlevel=1,%
    hypertexnames=true, pdfhighlight=/O,%nesting=true,%frenchlinks,%
    urlcolor=Chocolate, linkcolor=RoyalBlue, citecolor=LimeGreen, %pagecolor=RoyalBlue,%
}
\newcommand{\marginpars}[1]{\marginpar{\small#1}}
\usepackage{caption}
\usepackage[margin=0pt,font+=small,labelformat=parens,labelsep=space,
skip=6pt,list=false,hypcap=false]{subcaption}
\captionsetup{compatibility=false}
\usepackage{graphicx}
\usepackage{booktabs}
\graphicspath{{./assets/}}
% \usepackage[capitalize,noabbrev]{cleveref}
\usepackage[]{extdash}

\newcommand{\asym}{\emph{A sym exp}}
\newcommand{\bfs}{\textsc{Breadth First Tree}}
\newcommand{\ccPivot}{\textsc{CC-Pivot}}
\newcommand{\epi}{\textsc{Epinion}}
\newcommand{\etest}{\ensuremath{E_{\mathrm{test}}}}
\newcommand{\etrain}{\ensuremath{E_{\mathrm{train}}}}
\newcommand{\gplus}{\textsc{Google+}}
\newcommand{\grid}{\textsc{Grid}}
\newcommand{\gtx}{\textsc{Galaxy Tree}}
\newcommand{\extractStar}{\textsc{Extract-Stars}}
\newcommand{\collapseStar}{\textsc{Collapse-Stars}}
\newcommand{\lpa}{\textsc{Preferential Attachment}}
\newcommand{\pcc}{\textsc{Correlation Clustering}}
\newcommand{\rst}{\textsc{Random Spanning Tree}}
% \newcommand{\sgt}{\textsc{Galaxy Tree}}
\newcommand{\shz}{\textsc{Shazoo}}
\newcommand{\sla}{\textsc{Slashdot}}
\newcommand{\wik}{\textsc{Wikipedia}}
\renewcommand{\triangle}{\textsc{Triangle}}

\newcommand{\ith}{\ensuremath{i^{\mathrm{th}}}}
\newcommand{\jth}{\ensuremath{j^{\mathrm{th}}}}
\newcommand{\tth}{\ensuremath{t^{\mathrm{th}}}}
\newcommand{\uar}{uniformly at random}
\newcommand{\ie}{i.e.\@}
\DeclareMathOperator{\degr}{deg}
\DeclareMathOperator*{\argmin}{arg\,min}

\newcommand{\starone}[1]{\ensuremath{\textcolor{DodgerBlue}{S_{#1}^1}}}
\newcommand{\startwo}[1]{\ensuremath{\textcolor{Orange}{S_{#1}^2}}}
\newcommand{\starthree}[1]{\ensuremath{\textcolor{Green}{S_{#1}^3}}}

\newcommand{\euv}{\ensuremath{u\rightarrow v}}
\newcommand{\evu}{\ensuremath{v\rightarrow u}}
\newcommand{\yuv}{\ensuremath{y_{u, v}}}
% \newcommand{\pathtuv}{\ensuremath{\text{\textsc{Path}}^T_{u,v}}}
\DeclareMathOperator{\pathm}{path}
\newcommand{\pathtuv}{\ensuremath{\pathm^T(u, v)}}

\title{PhD plans}
\title{Géraud Le Falher}
\begin{document}
\maketitle
\begin{description}
	\item[Oct. 14] State of the art for \pcc{} (see
		\href{run:support/cc.pdf}{A few words on Correlation Clustering})
	\item[Nov. 14 --- Jan. 15] Trying different ways of closing the
		triangles of a general graph without introducing bad cycle in
		order to transfer the \ccPivot{} algorithm (see
		\href{run:support/december.pdf}{Combinatorial
		Correlation Clustering on general unweighted graphs} and the Dating presentation
		\href{run:support/presentationeng.pdf}{Signed graphs: clustering and link prediction})
	\item[Jan. 15 --- Apr. 15]{Predicting link signs by building a low
		stretch spanning tree dubbed \gtx{}, which works by extracting
		stars, connecting them in a graph, and repeating on this reduced graph
		(see \href{run:support/nips2015.pdf}{some experimental results}).}
	\item[Mar. 15] Visit to Claudio. The setting was a graph where nodes have profiles and be
		similar along various ways, and the problems were either answering online
		similarity queries or recovering hidden profiles (see
		\href{run:support/contextual_clustering.pdf}{Contextual Clustering}).
	\item[May. 15] Implementing \shz{}~\autocite{Vitale2012} (actually it took a week, but I wasn't so
		much in the lab in May)
	\item[Jun. 15 --- Aug. 15] 
		I supervised the internship of Paul Dennetiere on the implementation of a parallel
		algorithm for \pcc{} described by \textcite{Pan2014} (see the
		\href{run:support/stage_cc.pdf}{internship description}, a
		\href{run:support/draft_paul.pdf}{draft of the technical part of Paul's report} and
		some ideas on \href{run:support/parallel_postprocessing.pdf}{parallel post
		processing} we didn't have time to explore).

		I worked on building trees that minimized the stretch of the distances between a
		set of revealed nodes and the other ones (see a
		\href{run:support/sujet2_lowstretch.pdf}{a MVA project topic} and
		\href{run:support/constrained_low_stretch.pdf}{some initial experiments}).

		Two others summaries of what I did until then can be found in my
		\href{run:support/postermlss.pdf}{MLSS poster} and my
		\href{run:support/progress.pdf}{first year progress report}.
	\item[Sep. 15]
		When Fabio came back from Finland, he told me about a problem on graph
		reconstruction. I can't fully remember the setting, but I think the general idea
		was to make some queries about the existence of certain edges. Trivially, by
		making $\Theta(|V|^2)$ queries, we know the full topology so the question of
		was can we do better. I didn't write much about except
		\href{run:support/permutation.pdf}{one thing about counting permutation on a
		line graph}, but this is very incomplete.
	\item[Oct. 15 --- Nov. 15] Following Michael's visit during the summer, we
		worked on a binary node classification problem where the bias is that a node
		is labeled $+1$ iff at least two of its neighbors are labeled $+1$. This
		lead to a propagation algorithm based on triangles chain (see
		\href{run:support/NodeClassification_triangle_mail_fabio_nov15.pdf}{some Fabio writeup on the topic}).
	\item[Nov. 15 --- May. 16] Edge classification in the troll trust model (see
		the \href{https://arxiv.org/pdf/1606.00182.pdf}{arxiv version of the AISTATS paper})
	\item[Jun.16 --- Dec. 16] Edge clustering
	\item[Dec. 16] Students of the MVA did a project on \pcc{} under perturbation,
		here is the \href{run:support/sujet1_ccpert.pdf}{topic description} and
		\href{run:support/Dubreuil_Ramzi_report.pdf}{their report}.
	\item[Jan. 17 --- Feb. 17] We worked with Fabio on a more robust version of
		the \shz{} algorithm that decreases its confidence to labeled nodes when
		they are implicated in further prediction mistake but
		\href{run:support/trees.pdf}{experimental results} are not very satisfying.
	\item[Mar. 17 --- May. 17] Edge clustering
\end{description}

During my thesis, I also published my Master's thesis in a
conference~\autocite{Thesis15} and kept working on Urban Informatics, which
resulted in poster~\autocite{WWWDemo16} and a journal
paper~\autocite{GeotopicTBD16}.

Furthermore I (loosely) maintained a list of references on
\href{run:support/dist.pdf}{distributed graph processing frameworks} and
\href{run:support/embedding.pdf}{neural network based graph embedding methods}.

\begingroup
% \setstretch{0.9}
\setlength\bibitemsep{2pt}
\printbibliography
\endgroup
\end{document}
