Looking for a subgraph $H$ of $G$ that best preserve the distance in $G$ while being sparse is an old
problem, especially driven by network design in fields such as transportation~\autocite{RoadNetworks60}
and electrical circuits~\autocite{electricalNetworks60}. For instance, \textcite{Johnson1978} define
the following problem\footnote{We adapt their notations to match ours}:  
\begin{problem}[Network Design Problem]
  \label{prob:gtx_ndp}
  Given an undirected integer-weighted graph $G=(V, E, w)$, a budget $B\in\Nbb$ and a criterion
  threshold $C\in \Nbb$, does there exist a spanning subgraph $G'=(V, E')$ of $G$ with weight
  $w(E') \leq B$ and criterion value $F(G') \leq C$, where the criterion function $F(G')$ denotes
  the sum of the weights of the shortest paths in $G'$ between all vertex pairs?
\end{problem}
Note that their definition of stretch includes all possible edges in $G$ and not only those in $E$.
They prove that this problem is \NPc{} by exhibiting a reduction from the \textsc{Knapsack}
problem. Finally they prove that the simpler problem of finding a spanning tree on an unweighted
graph, that is
\vspace{-.5\baselineskip}
\begin{problem}[Simple Network Design Problem]
  \autoref{prob:gtx_ndp} with $w$ being the equal to $1$ for all edges in $E$ and $B=|V|-1$.
\end{problem}%
\vspace{-.5\baselineskip}
\noindent is also \NPc{} by reduction from \textsc{Exact 3-Cover}.

\begin{problem}[Optimal Communication Spanning Tree~\autocite{Requirements74}]
Given a set of nodes $V=\{v_1, \ldots, v_n\}$, a set of distances $d_{ij}$ and a set of requirements
$r_{ij}$ between $v_i$ and $v_j$, find a spanning tree connecting these $n$ nodes such that the
total cost of communication of the spanning tree is a minimum among all spanning trees. The cost of
communication for a pair of nodes is $r_{i,j}$ multiplied by the sum of the distances of arcs which
form the unique path connecting $v_i$ and $v_j$ in the spanning tree. The cost of a spanning tree is
the sum of costs over all pairs of nodes.
\end{problem}

Given a weighted graph $G=(V,E,w)$, if we let $d_{ij}$ be $w((i,j))$ and $r_{ij}$ be $1$ if
$(i,j)\in E$ and $0$ otherwise, it seems to me this correspond to the weighted low stretch problem.
However, \textcite[Section 3, page 453]{lognMetricBoundConf03} claim a $O(\log n)$ approximation so maybe I'm
wrong.

This defines two kind of structures, spanning trees and spanners (which are still sparse subgraphs
yet containing more than $|V|-1$ edges).

\paragraph{Trees}
\label{par:trees}

Earlier but slightly less related, \textcite{OptimalNetwork69} studies the following problem:
\begin{problem}[Optimal Network Problem]
  \label{prob:gtx_scott}
  Given a set $V$ of $n$ vertices, find a set of spanning edges $E\subset V^2$ that minimizes
  the sum of the length of the shortest paths  between all vertex pairs while the
  total length of the resulting network does not exceed some upper bound $B\in\Nbb$.
\end{problem}
This can be seen as a special case of \autoref{prob:gtx_ndp} with $G$ being the unweighted
\marginpars{according to the experiments, it's not clear whether there are weights or
not…} $n$-complete graph. He proposes a backtracking solution and two local search approximate
algorithms. Some early branch and bound heuristic solutions to \autoref{prob:gtx_scott} are surveyed
in~\autocite[Section 2.3.2]{networkDesignSurvey89} although they do not come with asymptotic
guarantee on the stretch. Furthermore, \textcite{optimApproxNP80} proves that for any $\epsilon \in
(0,1)$, finding a $|V|^{1-\epsilon}$ approximation is \NPc{}.

\paragraph{Spanners}
\label{par:spanners}


% check some surveys of the 80's
% http://pubsonline.informs.org/doi/abs/10.1287/trsc.18.1.1
% http://onlinelibrary.wiley.com/doi/10.1002/net.3230190305/full
% early solutions
% http://onlinelibrary.wiley.com/doi/10.1002/net.3230090104/full
% http://onlinelibrary.wiley.com/doi/10.1002/net.3230130309/full
% later solution?
% http://ieeexplore.ieee.org/document/81738
