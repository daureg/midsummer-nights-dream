The framework we proposed is agnostic to the question of the interplay between connection and
linking attributes, although this is an interesting and active research question.

Homophily and Contagion Are Generically Confounded in Observational Social Network Studies
Cosma Rohilla Shalizi, Andrew C. Thomas
Sociological Methods \& Research
Vol 40, Issue 2, pp. 211-239
DOI: 10.1177/0049124111404820

Digital Footprints: Opportunities and Challenges for Online Social Research
Annual Review of Sociology
Vol. 40:129-152 
First published online as a Review in Advance on June 16, 2014
\url{https://doi.org/10.1146/annurev-soc-071913-043145}

Here is what the “Digital Footprints” writes about the previous paper
An important limitation is the difficulty distinguishing between homophily and contagion. Homophily
refers to a variety of selection mechanisms by which a social tie is more likely between individuals
with similar attributes and environmental exposures (McPherson et al.  2001). Contagion refers to
influence mechanisms (e.g., imitation or peer pressure) by which traits diffuse along network edges.
Homophily and contagion offer competing explanations for network autocorrelation, which refers to
the greater similarity in the attributes of closely connected nodes. Based on simulated networks,
Shalizi \& Thomas (2011, p. 216) conclude that “there is just no way to separate selection from
influence observationally” (see also Manski 1993). This does not mean that observational studies
using online networks are useless, but researchers need to refrain from assuming that the observed
network autocorrelation reflects contagion effects and to acknowledge that the similarity between
adjacent nodes may reflect the mutually reinforcing effects of influence and selection whose
separate contributions may be impossible to tease apart.  For example, although Ugander et al.
(2012) controlled for demographic similarity (sex, age, and nationality), there are countless other
ways in which shared environments, affiliations, interests, and personality traits might cause two
friends to join Facebook independently but not on the same day, making it look like the “early
adopter” influenced the friend they invited who would have joined anyway.

Furthermore, we could also consider the role of time with dynamic networks. For instance, while the
photo similarity between two Flickr users rises before they connect, it later decreases if they have
the same level of popularity, as they try to differentiate themselves
Zeng and Wei: Social Ties and User Content Generation: Evidence from Flickr, Information Systems
Research 24(1), pp. 71–87
