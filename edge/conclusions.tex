\section{Open directions}
\label{sec:edge_conc}

We could extend our approach to more general types of graphs and tasks:

\begin{itemize}[leftmargin=*]

  \item First, as we mentioned earlier, \textbf{networks are dynamic} and evolve over time, with
    nodes and edges constantly appearing and disappearing~\autocite{networkEvolution14}. Slicing
    time into successive periods is one way to create edge types, one per period. For instance,
    \textcite{TimedUSSenate10} study the \np{110} congresses of the US senate between the years 1789
    and 2008, where \np{1884} individual Senators are linked by their voting similarity. They show
    how it improves the detection of relevant communities. Beyond predicting the relative order of
    links, it would also be interesting to include time into node similarity itself.
    %
    For instance, while the photo similarity between two Flickr users rises before they connect, it
    later decreases if they have the same level of popularity, as both try to differentiate
    themselves~\autocite{dynamicFlickr13}.
    %
    More generally, by defining our problem on a static snapshot of a graph, our model is agnostic
    to the question of whether edges were created because of the similarity/dissimilarity of
    the linked nodes' attributes or because attributes started to change after the edge creation.
    This is traditionally framed as dichotomy between homophily and contagion, and is an active topic
    of research in social networks analysis~\autocites{InfluenceVsHomo08}{ConfoundedHomophily11} as
    noted by \textcite{OSNreview14}: \textquote{Homophily refers to a variety of selection
    mechanisms by which a social tie is more likely between individuals with similar attributes and
    environmental exposures~\autocite{Homophily01}. Contagion refers to influence mechanisms (\eg{}
    imitation or peer pressure) by which traits diffuse along network edges. Homophily and contagion
    offer competing explanations for network autocorrelation, which refers to the greater similarity
    in the attributes of closely connected nodes.}

  \item Another rich class of graphs where predicting edge types would be a challenging and rewarding
    task is \textbf{multigraph}, which allow several edges between two
    nodes~\autocite{typedMultigraph11}. It can also be seen as
    the flattening of several graphs sharing the same nodeset, but where the provenance of each edge
    would have been lost and need to be recovered. With our original formulation of goodness as
    ${s_{uv}}^T w_{uv}$, only one direction can achieve the higher score (discarding the rare case
    of tie). One alternative we proposed was to let each $w_{uv}$ be a linear combination of a small
    number of base directions.
    Symmetrically, another idea could be to have more than one way to compute the similarity
    $s_{uv}$ between two nodes.

  \item Generalizing the balance theory from signed graphs, we could imagine that in triangles or
    short cycles, only certain combinations of directions are valid (or at least desirable).
    Interestingly, this would likely require different optimization algorithms, probably non convex.
    However, while this is a general concept, its exact implementation might be
    application-dependant. Furthermore, in our current setting, we do not know in advance the
    semantic of each direction, nor do we have labeled edges. Therefore, it is not clear how those
    constraints would be specified, or whether they can be learned, if applicable. A limited solution
    exists in the context of heterogeneous information networks, where in addition to having several
    types of edges, there are also several types of nodes. Since not all types of node can be
    connected with any types of edge, the possible connections are themselves represented as a meta
    graph called the network schema~\autocite[Definition 3]{HINSurvey17}. This approach would need
    to be extended to deal with higher order patterns, for instance in a manner reminiscent of
    composition of paths in multilayer graphs~\autocites{metapath11}[Section
    4.2]{KnowledgeGraphSurvey17}.

  \item Among tasks that could benefit from labeling the edge of complex networks, we mention two.
    The first is \textbf{graph summarization} where the idea is to transmit a fraction of a large
    graph plus additional information that can be used to reconstruct the original graph. Namely,
    our goal here is to send only a few node profiles along with the edge directions, enabling the
    missing profiles to be reconstructed from the known goodness values. Note that we spare
    information regarding the profiles but still transmit the full topology. Therefore, this differs
    from lossless schemes~\autocites{compressSN09}{compressSN12} that leverage typical edge locality
    patterns to encode edges using as few bits as possible, or from lossy graph sketching
    methods~\autocite{graphSketch12} discussed in \autoref{sub:gtx_state_of_the_art}, which preserve
    spectral and topological properties of the original graph.
    %
    The second is \textbf{node embedding}, where every node $u$ is represented by a vector of size
    $k$ containing the relative proportion of each direction with the edges incident to $u$.
\end{itemize}
