In this chapter, we extend the problem of predicting edge types beyond the binary case.
Furthermore, we assume that we are provided with node attributes in addition to the mere graph
topology we used in the previous chapters. Our motivation is twofold: intrinsic and extrinsic.
First, we want an interpretable model that, given two connected nodes, can explain why they are
linked. This leads us to use a linear model, for it is both easily amenable to
interpretation and computationally efficient. Second, edge-labeled graphs are susceptible to further
studies by existing networks analysis methods, as we will describe shortly. Thus, our method is a
useful way of enriching data to extract new insights from it.

As demonstrated by our prior extensive coverage, predicting edge sign is an interesting and
challenging problem in its own right. Yet it is natural to extend and generalize it.  First, there
are other edge binary classification problems that arise in graphs. For instance, predicting whether
a link in a social network is strong or weak~\autocites{communityWeakTies14}{communityWeakTies17};
predicting whether two authors in a citation network are advisor-advisee or regular
coauthors~\autocite{Advisor10}; predicting whether two persons in a company network have a
relationship of manager-subordinate or not~\autocite{Manager07}; predicting whether two proteins
interact physically or genetically; or whether two neurons have a chemical or ionic link between
them~\autocite{BioMultiplex15}.

Second, and more generally, we are also interested in multiclass problems, within the context of
multilayer graphs~\autocites{Kivela2014}{multiSurvey14}. While \textcite{Kivela2014} define an
elaborate and highly expressive notion of multilayer graph, in this chapter we shall restrict
ourselves to the following\footnote{Which roughly corresponds to the \emph{single aspect},
\emph{node-aligned} multilayer graph of \autocite{Kivela2014}.}:
\begin{definition}[$k$-multilayer graph]
  A $k$-multilayer graph $G$ is a pair $(V, \{E_1, \ldots, E_k \})$, where $V$ is a set of $n$ nodes
  and each $E_i \subset V\times V$ is a set of edges, for $i \in \rangesk$. We call the subgraph $G_i =
  (V, E_i)$ the \ith{} layer of $G$ and it will be convenient to think of $G$ as the superposition
  of its $k$ layers, so that by a slight abuse of notation we can write $ G = \bigcup_{i=1}^k G_i $.
\end{definition}%\vspace{-0.25\baselineskip}\noindent%
We naturally identify each layer with a type of relation, so that our classification problem is to
decide to which layer a given edge belongs. Furthermore, we make the simplifying assumption that two
nodes can be connected in at most one layer (\ie $\forall u,v \in V,\, (u,v)\in E_i \Rightarrow
\forall j \neq i,\, (u,v) \notin E_j$), meaning that the resulting superposition of all layers does
not turn into a multigraph. As mentioned in the introduction, such multigraphs can be used to model
different relation types in networks related to social
interactions~\autocites{Szell2010}{RecoFlickrMulti11}, bibliographical
citation~\autocites{communityDBLPbyConf05}{articlesMultiSim11}, economic
exchange~\autocites{ports3kindofships10}{worldTradeNetwork10}{KantPeace15} and
biology~\autocites{bioLayerExp11}{EcoChile15}{Neuroscience16}.
This additional information on the nature of the relationships allow traditional graph tasks to be
performed more precisely and with higher granularity. Consider for instance node ranking~\autocite{HRRanking13};
node clustering \autocites[Section 2]{surveyAttributedClustering15}[Section 4.5.1]{Kivela2014} and
the study of information spreading \autocite{SpreadingMultiSurvey15}. However, to produce such finer
results, the methods we just referenced require labeled edges that in many cases are not available,
thus justifying the usefulness of our work. Indeed, in \autoref{tab:edge_apps}, we give some examples of
networks whose edges are currently unlabeled and what type of relations they could possibly denotes.
In other words, given a small list of possible reasons in terms of attributes and two nodes, why are
they connected?

\setlength{\fullpage}{179mm}
\begin{table}[h]
\begin{adjustwidth}{-2cm}{}
  \small
  \setlength{\tabcolsep}{4pt}
  \centering
  \caption{Real attributed graphs that we have collected or are easily accessible. Unfortunately,
    none of them comes with known edge types, which implies the last column is merely speculation.
  \label{tab:edge_apps}}
  {\renewcommand{\arraystretch}{0.95}%
  \begin{tabulary}{\fullpage}{lLLLL}
    \toprule
    {Source} & {Node} & {Edge} & {Node profiles} & {Possible edge meanings} \\
    \midrule
    DBLP & author & citation & keyword count of their papers & scientific topic \\
    Github & developer & following & popularity, number of lines written in various programming languages &
    common hobby, involvement in the same open source project, same
    workplace, famous role model \\
    Amazon & video game products & \enquote{frequently bought together} & price, genre, time of release, popularity and evaluation &
    part of a series, cheap bundle, same genre and level of satisfaction \\
    Wikipedia & article & internal link & category, length, edit history, bag of words &
    generalization, specialization, organisation \\
    Flickr & user & following & activity statistics, groups membership, count of photo tags &
    geography, same tags distribution, famous role model,  \\ 
    Foursquare & venue & \enquote{frequently visited together} & location, popularity, category, the time distribution at which they are visited &
    closeness, part of a \enquote{series} (\eg{} movie theater $\rightarrow$ restaurant $\rightarrow$
    bar $\rightarrow$ club) \\
    IMDB & movie & sharing actors & genre, release date and popularity &
    part of a series, common theme, same time period and genre \\
    BlaBlaCar & driver or passenger & shared ride & experience, location, preferences, rating &
    geography, professional, leisure \\
    \bottomrule
  \end{tabulary}}
\end{adjustwidth}
\end{table}

\medskip
Besides predicting potentially more than two classes, the other difference with the setting of the
previous chapters is that we now assume that nodes are associated with attributes, which we call
\emph{profiles}. One could argue this makes our framework less general, for attributes are not
always available, or may come with missing values. Another recent finding is that, despite a natural
assumption, the nodes profiles in attributed graphs are not necessarily correlated with known ground
truth communities~\autocite{FortunatoGroundTruth14}, casting further doubt on the usefulness of that
extra information.
This raises the question of why we cannot simply extend the troll/trust method of counting the type
of edges incident to every node and learn a model from such features.
A similar tension between relying solely on always available albeit sparse topological information or
leveraging possibly richer attributes information also exists in link
prediction~\autocite{linkPredSurvey16}. %\stodo{What they actually write is \enquote{Topology-based
%techniques are more versatile than attribute-based methods since they are not domain specific} which
%sounds a bit unfair: only the attributes are domain specific, not necessarily the methods
%themselves}.
% examples of using node attributes for link prediction, taken from that survey:
% https://doi.org/10.1007/s11432-014-5237-y
% - http://dl.acm.org/citation.cfm?id=2124378
% - using graph kernel https://doi.org/10.1016/j.dss.2012.09.019
% - matrix factorization “these latent features (learned from topology) may be combined with
%   optional explicit features for nodes or edges, which yields better performance than using either
%   type of feature exclusively.” https://doi.org/10.1007/978-3-642-23783-6_28
However, the presence of profiles is motivated by the last fundamental difference with the previous
problems. In this chapter, we do not have access to any direct supervision in the form of labels.
Rather than a strict classification problem, we are thus faced with unsupervised clustering, guided
by the information contained in node profiles. Note that contrary to the relation between \pcc{} and
\esp{}, this time the objects we seek to cluster are edges and not nodes. Furthermore, we stress
that the profiles crucially increase the interpretability of our models.

It would be tempting to assume that nodes are connected because they share the same attributes.
Indeed, a common learning bias in graphs is
that nodes are connected together because they are similar, or as often said proverbially:
\enquote{birds of a feather flock together}. Formally this is called the homophily
principle~\autocite{Homophily01} and has been consistently verified, both in
offline~\autocites{homoAttitude78}{homoEdu85} and online social
networks~\autocites{homoSN09}{homoSN12}. While our every day life experience makes homophily not
surprising in social networks, the same assortative patterns~\autocite{AssortativeMixing02} are
present in other kinds of networks~\autocite{AssortativeMixing09}. Here though, we assume that links
cannot only be explained by a global homophily along all attributes but rather by partial homophily,
partial heterophily or even both. An example of this combination is the balanced news
diet problem~\autocite{balancedNews17}, where we try to connect users of a social network to prevent them
from being locked into topical echo chambers. In that situation, a user $A$ might be more likely to listen to
the opposite point of view of user $B$ about gun control if they both share common demographics and
interests. This suggests, as we make it formal in the next section, that node attributes can include
polarized opinions, \ie{} both positive and negative numerical values.

\medskip

After having motivated our problem of predicting multiple edge types in attributed graphs, we carry
on by stating more formally the \ecp{} problem in \autoref{sec:edge_problem}. Namely, we define a
similarity between nodes profiles and seek a small number of explanations (corresponding to edge
types) achieving good scores as prescribed by a linear model. After explaining how this problem relates
to the \esp{} problem on 2-clusters signed graphs
\vpageref{par:generalization_of_signed_graphs}, we introduce in \autoref{sec:edge_methods} several
methods to solve it in practice. This ranges from tailoring the \kmeans{} algorithm to our goodness
measure (\autoref{sub:edge_baseline}), to a convex relaxation of our objective function
(\autoref{sub:edge_vector}) all the way through a richer and higher capacity matrix formulation allowing
overlapping clustering of edges (\autoref{sub:edge_matrix}). Because finding publicly available real
data with both node attributes and edge types proved difficult, we describe a synthetic model in
\autoref{sec:edge_exp}, on which we perform extensive experiments. We then survey in
\autoref{sec:edge_related} works that are related to the \ecp{}, while noting that very few address
it directly. Finally, we suggest in \autoref{sec:edge_conc} several directions in which our model
could be extended to handle more general classes of graphs.
