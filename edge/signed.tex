\paragraph{Generalization of signed graphs}
\label{par:generalization_of_signed_graphs}

Here we show why \autoref{p:edge_full} can model signed graphs as a special case, namely with $k=2$
and specifically crafted profiles.
Remember the learning
bias we introduced in the previous chapter \vpageref{text:cc_new_bias}: every node belongs to
exactly one of the $K$ clusters available, is connected positively with the other nodes belonging to
the same cluster, and is connected negatively with the nodes belonging to every other clusters. Now
let the dimension of profiles be $d=K$ and the number of directions $k=2$. Moreover, we define the
two direction vectors as $w_1 = \frac{\onev}{\sqrt{d}} = w_+$ and $w_2 = -w_+ = w_-$, where
$\onev$ is the vector of dimension $d$ with only $1$ coordinates. Finally, for a node $u$ in
cluster $i$, we assume its profile $x_u$ is set as follow: all component are equal to $-b$ except
for the \ith{} one, which is equal to $a$, where $a$ and $b$ are real constants we define now.

First, because $x_u$ is a unit vector, we have that $a^2 + (K-1)b^2=1$, from which $a =
\sqrt{1-(K-1)b^2}$ whenever $1-(K-1)b^2 \geq 0$, or equivalently $b \leq \frac{1}{\sqrt{K-1}}$.
Next, we look at $\left(x_u \circ x_v\right)^T \onev = {s_{uv}}^T \onev = \frac{1}{\sqrt{d}}
\sum_{i=1}^d  {s_{uv}}_{;i}$ for any edge $(u,v) \in E$. If $u$ and $v$ are in the same cluster, then
$\sum_{i=1}^d  {s_{uv}}_{;i} = a^2 + (K-1)b^2 = 1$, implying that ${s_{uv}}^T w_+ = \frac{1}{d} >
{s_{uv}}^T w_- = -\frac{1}{d}$. Otherwise, $\sum_{i=1}^d  {s_{uv}}_{;i} = (K-2)b^2 - 2ab = (K-2)b^2 -
2b\sqrt{1-(K-1)b^2} = f_K(b)$. By computing the derivative of $f_K$ with respect to $b$, we find
that it reaches a minimum of $-\frac{1}{K-1}$ at $b = \frac{1}{\sqrt{K(K-1)}} \leq
\frac{1}{\sqrt{K-1}}$. In this case, ${s_{uv}}^T w_+ = -\frac{1}{(K-1) \sqrt{d}} < {s_{uv}}^T w_- =
\frac{1}{(K-1) \sqrt{d}}$. To maximize $\sum_{(u,v)\in E} \max_{\ell \in \{1, 2\}} {s_{uv}}^T
w_\ell$, we therefore assign $w_+$ to edges whose both endpoints belong to the same cluster, and
$w_-$ to edges joining two different clusters.

Moreover, given this assignment, we can show that $w_+$ and $w_-$ are the optimal directions, under
the additional uniformity assumption that there are exactly $p$ edges within every cluster, and $q$
edges between any two clusters. First, we define
\begin{equation*}
  s_{\mathrm{inner}} = \sum_{(u,v) \in E;\, \cluster(u)=\cluster(v)} s_{uv} \qquad \text{and} \qquad
  s_{\mathrm{outer}} = \sum_{(u,v) \in E;\, \cluster(u)\neq \cluster(v)} s_{uv}
\end{equation*}
to be the sum of all edge vectors within and between clusters respectively. By our choice of $a$ and
$b$, we have that $s_{\mathrm{inner}} = p\onev$ and $s_{\mathrm{outer}} = -\frac{q}{2}\onev$. Now
recall that the solution of $\argmax_{w\in \dsphere} w^T c$ is $\frac{c}{||c||}$. Therefore, the
best direction to assign to inner edges is indeed $w_+$, and so is $w_-$ to outer edges.

\begin{aside}
For inner edge vectors, $s_{uv}$ in cluster $i$ has $a^2$ as its \ith{} component, and $b^2$
elsewhere. There are $p$ of them in cluster $i$, so summing over all clusters we get
$s_{\mathrm{inner}} = p\left(a^2 + (K-1)b^2\right)\onev = p\onev$.
For outer edge vectors, we first consider the $q$ edges between $C_1$ and $C_2$,
whose $s_{uv}$ is
$\begin{pmatrix}
  -ab \\
  -ab \\
  b^2 \\
  b^2 \\
  \vdots \\  
\end{pmatrix}$. Likewise, between $C_1$ and $C_3$ we have
$s_{uv}=\begin{pmatrix}
  -ab \\
  b^2 \\
  -ab \\
  b^2 \\
  \vdots \\  
\end{pmatrix}$. We thus we see that the first component is always $-ab$, that there is another
such term at the other cluster index, while the rest is $b^2$. Summing all the vectors incident to
$C_1$, we get
$q\begin{pmatrix}
  -ab(K-1) \\
  b^2(K-2) -ab \\
  b^2(K-2) -ab \\
  \vdots \\  
\end{pmatrix}$. Similarly, the sum of all vectors incident to $C_2$ is
$q\begin{pmatrix}
  b^2(K-2) -ab \\
  -ab(K-1) \\
  b^2(K-2) -ab \\
  \vdots \\  
\end{pmatrix}$, and so on for every cluster $i$. Summing over all clusters and dividing by $2$ to
avoid double counting, we get $s_{\mathrm{outer}} = \frac{q}{2}\left( (K-1)\left(b^2(K-2) -ab\right)
-ab(K-1) \right)\onev = \frac{q(K-1)}{2}\left(b^2(K-2) -2ab\right)\onev =
\frac{q(K-1)}{2}f_K(b)\onev = -\frac{q(K-1)}{2} \frac{1}{K-1} \onev = -\frac{q}{2}\onev$.
\end{aside}

Granted, the situation where all the nodes of a cluster have exactly the same profile and where the
connections are so regular is highly idealized. Moreover, we did not formally prove that there is
not another assignment (with different direction vectors) that would give a better objective
value\footnote{Although given the symmetry of the problem, that would be surprising.}. Our point is
rather that profiles and directions can be seen as a \enquote{latent} explanation for the signs in
a balanced signed graphs, like our generative model of the first chapter.

