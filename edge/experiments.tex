\paragraph{Synthetic data}
\label{par:synthetic_data}

While there exists several methods to generate attributed graph with community
structure~\autocites{Yang2013}{XuBayesian14}{Kataoka2016}, here we present the one we adopted for
our initial experiments, as it is specifically tailored to the model introduced in the previous
section. Given a number of directions $k$, a maximum number $k_\mathrm{local}$ of directions
incident to each node, the dimension of profiles $d$ and a number of node $n$, it return a graph
$G=(V,E)$, a profile matrix $X\in \Rbb^{n\times d}$, a set of $k$ directions $\mathcal{D}_k$ and an
assignment from every edge $(u,v)$ to a direction $w_{uv} \in \mathcal{D}_k$ such that in most
cases, $\forall w_\ell \in \mathcal{D}_k \setminus \{w_{uv}\}\,, {s_{uv}}^T w_{uv} \geq  {s_{uv}}^T
w_\ell$.

We first generate the $k$ unit-norm directions $\mathcal{D}_k=\{w_1, \ldots, w_k\}$, independently
of the graph. Furthermore, we control the number of coordinates $n_o$ that overlap with each other.
That is, when $n_o = 0$, each direction has exactly \nicefrac{k}{d}\footnote{We choose $d$ to be a
multiple of $k$.} non-zero dimension that are disjoint. On the other hand, if for instance $n_o=5$,
then $5$ of the $d$ dimensions will have two directions with non-zero components on it. Those
non-zero components are drawn \uar{} from $[-1, 1]$ and the each directions is then normalized. For
instance, with $d=6$, $k=3$ and $n_o=2$, we could generate the following directions, where the dots
represents non-zero numbers and the red ones are overlapping:
\vspace{-.5\baselineskip}
\begin{center}
	\begin{tabular}{ccc}
		$w_1$ & $w_2$ & $w_3$ \\
		$\begin{pmatrix}
			\bullet \\
			0 \\
			\bullet \\
			0 \\
			0 \\
			\textcolor{Red}{\bullet}
		\end{pmatrix}$ & 
		$\begin{pmatrix}
			0 \\
			\bullet \\
			0 \\
			\textcolor{Red}{\bullet} \\
			\bullet \\
			0
		\end{pmatrix}$ &
		$\begin{pmatrix}
			0 \\
			0 \\
			0 \\
			\bullet \\
			0 \\
			\bullet
		\end{pmatrix}$
	\end{tabular}
\end{center}

The next step is to create the graph topology, which we do by assembling small Erdős-Rényi
subgraphs~\autocites{erdos1959random}{gilbertRG59} that we call \emph{blocks}. Furthermore, driven
by a topological constraint mentioned earlier, we assign to each node a $k_\mathrm{local}$-tuple of
directions such that two adjacent node have \emph{at least} one common direction that can later be
assigned to the edge between them. For convenience, let us first identify directions with colors and
thus call such a $k_\mathrm{local}$-tuple of directions a \emph{palette}. For a given palette $p$,
we call adjacent palettes, denoted $\adj(p)$, all the palettes different from $p$\footnote{We ruled
out $p$ being adjacent to itself to avoid the trivial solution of all nodes being assigned the same
palette.} but sharing one
color with $p$. Finally, we say that an edge $u,v$ is \emph{colorless} if the palette of $u$ and $v$ have
no color in common. Because in the general case, it is not always possible to assign palette to
every nodes such that there is no colorless edge,\footnote{For instance, consider a $4$-clique with $k=4$
and $k_\mathrm{local}=2$. Without loss of generality, say we assign the palette $(1,2)$ to node
$1$. The adjacent palettes are then $(1,3)$, $(1,4)$, $(2,3)$ and $(2,4)$. If we assign those
starting by $1$ (respectively $2$) to the second and third node, then the fourth palette has to
contain a $1$ (respectively $2$), which is not possible (because two connected nodes cannot have the
same palette). On the other hand, if we assign $(1,3)$ to the second, the third node can only have
$(2,3)$, meaning the fourth node has again no palette available. The same situation with $(1,4)$ and
$(2,4)$.} we now describe a simple heuristic. It performs a breath first visit of a subgraph,
starting from a random node. Upon visiting an uncolored node $u$, it builds the intersection $P$ of
all the adjacent palettes of the palette of its colored neighbors (\ie{} $P=\bigcap_{v\in \nei(u)}
\adj(p_v)$). Then it selects \uar{} a palette from $P$, or an arbitrary palette if $P$ is empty (see
\begin{marginfigure}
  \centering
	\includegraphics[width=\textwidth]{tikz/edge_palette_tikz.pdf}
	\caption{A small example of the node palette assignment, with $k=4$ colors (blue, green, red and
		orange) and palette of size $k_\mathrm{local}=3$. We assume nodes $1$ and $2$ have already been
		visited, and got assigned the palette $(\text{blue, green, red})$ and $(\text{blue, green,
		orange})$ respectively. Moreover, we are currently at $3$ while $4$ is yet uncolored. In that
		case, there are two possible palettes for $3$. If we select $(\text{green, red, orange})$, a
		possible color assignment for the edges $(1,2)$, $(1, 3)$ and $(2, 3)$ is respectively blue,
		red, and green.
	\label{fig:edge_palette}}
\end{marginfigure}
\autoref{fig:edge_palette} for a small illustration). Once all nodes have been colored, we count the
number of edges that are colorless. We repeat this procedure, keeping track of the assignment
minimizing the number of colorless edges. Once we colored the nodes of every block, we look at pair
of blocks, build a list of all edges between blocks that are not colorless and sample from it to
connect blocks. The last step is to a assign a color to each edge from the shared color of its
endpoint (or an arbitrary color for colorless edges). Note this is reminiscent of the stochastic
block model, although the probabilities of edges between blocks are not uniform anymore.

The directions $\mathcal{D}_k$, and the graph $G$ along its edge assignment are the two sources of
randomness we will consider in the following.

\bigskip

Now that we have a direction $w_{uv} \in \mathcal{D}_k$ assigned to every edge $(u,v)$, we want to
find a set of user profiles $X=\{x_u\}_{u\in V}$ that maximize the edge score and minimize the node
loss, defined respectively by equations \eqref{eq:edge_full}\stodo{not really} and \eqref{} in the
previous section as: 
\begin{align*}
  \mathcal{L}_{\mathrm{edge}} &=
  \sum_{u,v \in E} \left(x_u \circ x_v\right)^T w_{uv} \quad \text{and} \\
  \mathcal{L}_{\mathrm{node}} &=
  \sum_{u\in V} \left|\left| x_u - b_u - \sum_{v \in \nei(u)} a_{uv} w_{uv} \right|\right|^2\,,
\end{align*}
where for simplicity, we fix $b_u = \zerov$ and $a_{uv} = 1$. Furthermore, to ensure that the
assigned direction of any edge $(u,v)$ achieve a better score than the other directions (that is
$\forall w_\ell \in \mathcal{D}_k \setminus \{w_{uv}\}\,, {s_{uv}}^T w_{uv} \geq  {s_{uv}}^T
w_\ell$) we also minimize cross-entropy loss commonly used in non-binary classification
problem~\autocite[Section 4.3.4]{PRML06}.  Specifically, recall that the goodness of direction
$w_\ell$ for the edge $(u,v)$ is $g(s_{uv}, w_{uv})$. If we denote by $p_{uv,\ell} = \frac{\exp
\left( g(s_{uv}, w_{uv}) \right)}{\sum_{\ell=1}^k \exp \left( g(s_{uv}, w_\ell) \right)}$ the
softmax \enquote{probability} of $s_{uv}$ being explained by direction $\ell$, the cross-entropy
loss is defined by
\begin{equation*}
  \mathcal{L}_{\mathrm{cross-entropy}} = -\sum_{u,v \in E} \sum_{l=1}^k
  \Ind{w_{uv} = w_\ell} p_{uv,\ell}
\end{equation*}

In practice, we first minimize $\lambda \mathcal{L}_{\mathrm{cross-entropy}} -
\mathcal{L}_{\mathrm{edge}}$ with respect to $X$ using the Adam algorithm~\autocite{Adam15} and
automatic differentiation~\autocite{autograd15} as implemented by the Pytorch
package\footnote{\url{http://pytorch.org/}} and projecting the current iterate back to the set of
matrix with unit $L_2$ norm columns $\mathbb{M}^{n\times d}$ at each step.

Finally, to minimize $\mathcal{L}_{\mathrm{node}}$, we iterate over the nodes and take some gradient
steps to minimize $\left|\left| x_u - \sum_{v\in\nei{u}} w_{uv}\right|\right|^2$, but only to the
extent that $w_{uv}$ remains the direction with the largest score on the edge $u,v \in E$.

\subparagraph{Results}
\label{par:edge_synth_results}

In the following we generate one graph topology, along with palette assignment to nodes and
direction assignment to edges. Then we generate a set of unit-norm directions,
parametrized by four numbers (and their default value): $k=7$, $d=35$, $n_o=0$ and
$k_{\mathrm{local}}=3$. To account for the randomness of this generation process, we repeat our
measurement over 200 such $\mathcal{D}_k$. On the other hand, we verify experimentally that
generating various graphs while keeping $\mathcal{D}_k$ fixed gives the same results, hence we will
not report them in the main text.
We also try other configurations to study the parameter sensibility of our methods. Specifically, we
experiment with less directions ($k=5$) or more directions ($k=9, d=36$)\footnote{We add one
dimension to ensure that $d$ remains a multiple of $k$}, with two higher levels of overlap between
the directions ($n_o=6$ and $n_o=12$), with more directions per node ($k_{\mathrm{local}}=4$) and in
larger dimension ($d=77$).

After having assigned a single direction to every edge and generated the user profiles, we can
measure how those two quantities *agree* with the edge scores. Namely, recall we predict the
direction of an edge $(u,v)$ as $\argmax_{\ell \in \rangesk} g(s_{uv}, w_\ell) = {s_{uv}}^T w_\ell$.
However, in the following, we do not evaluate such predictions using accuracy. Indeed, while the
methods we evaluate all assign one among $k$ directions to every edge, there is no guarantee those
directions are recovered in the same order as they were generated. In case of strong disagreement
between the predictions and ground truth labels, it might additionally be difficult to find a
permutation to conciliate them.  Therefore, we see the problem as a clustering one with known ground
truth, and turn to the Adjusted Mutual Information (AMI)~\autocite{AMI09}. It is an information
theoretic measure that enjoys several properties: it is invariant to permutations of the labels,
invariant to the shape of clusters and it is bounded between $0$ and $1$, where an AMI of $1$
indicates perfect correlation with the ground truth labels and $0$ indicates perfect independence
with the ground truth labels.

First, as we can see in the table below, in the \default{} configuration, we do not always
generate user profiles leading to a perfect assignment of edge according to their score/goodness. In
general this is not always possible. Consider a $k$-regular subgraph, where the $k$ edges incident
to every node are each assigned a different directions. By symmetry, it is impossible to find node
profiles that would achieve maximum goodness for all the edges. This is partly what motivated the
$k_{\mathrm{local}}$ local constraint. Even so, and more pragmatically, whereas those AMI scores
could be made higher by increasing the coefficient in front of the cross entropy loss, this would
imply a lower edge goodness. Even with our choice, only a few percent of the edges have a mismatch
between the directions we assigned them and the one with the highest goodness. Note also that
depending of the parameters of the generation, the AMI is not the same. Therefore, in the following,
we divide all the AMI scores by this natural score, in order to obtain a normalized measure.
\newline\vspace{.5\baselineskip}\noindent%
\begin{tabular}{lccccccc}
  \toprule
  {}  & {\smallk{}}    & {\default{}}   & {\largek{}}    & {\smallo{}}    & {\largeo{}}    & {\fdirs{}}     & {\larged{}}    \\
  \midrule
  AMI & $.925 \pm .03$ & $.922 \pm .03$ & $.905 \pm .02$ & $.880 \pm .04$ & $.874 \pm .05$ & $.893 \pm .03$ & $.973 \pm .01$ \\
  \bottomrule
\end{tabular}
\vspace{.5\baselineskip}

Besides this normalized AMI, we also measure how far the recovered directions $\mathcal{\wh{D}}_k$
are from the actual ones, generated initially. More precisely, we associate each vector $w_\ell \in
\mathcal{D}_k$ with the closest one in $\mathcal{\wh{D}}_k$ and report the average $\ell_2$ distance
between the two elements of these $k$ pairs, that is 
\begin{equation*}
  d(\mathcal{D}_k, \mathcal{\wh{D}}_k) =
  \frac{1}{k} \sum_{\ell \in \rangesk} \min_{w'_i \in \mathcal{\wh{D}}_k} \| w_\ell - w'_i \|_2
\end{equation*}

In total we compare five of the methods introduced earlier\stodo{make sure it's clear how we obtain
directions and prediction from each method}:
\begin{enumerate}[1.]
  \item \textsc{$k$-means} clustering of the normalized $s_{uv}$ edge similarity vectors;
  \item \textsc{Lloyd}-like iterative refinement of the $k$-means solution;
  \item \textsc{Combined} optimization of $\mathcal{L}_{\mathrm{edge}}$ and
    $\mathcal{L}_{\mathrm{node}}$;
  \item \textsc{Frank--Wolfe} maximization of $\mathcal{L}_{\mathrm{edge}}^{\mathrm{mat}}(W)$ over
    the $\delta$-ball of the nuclear norm, with a regularization of the $\ell_2$ norm of the columns
    of $W$. In other words, denoting by $\langle A, B \rangle_\mathrm{F} = \tr\left(A^T B\right)$
    the Frobenius inner product of two real matrices, we solve
    \begin{equation*}
      \min_{\|W\|_* \leq \delta} -\frop{S^T}{W} + \mu \frop{W}{W}\,\text{; and}
    \end{equation*}
  \item \textsc{Explicit} low-rank $PQ^T$ factorization, initialized with the solution of
    \textsc{Combined}\stodo{describe formally} and minimizing the same objective in matric form.
\end{enumerate}

As showed in \autoref{tab:edge_Wvaries_nami}, \combined{} is always the best method and, excluding
\pqt{} for now, the \lloyd{} heuristic is second except in one case, while \kmeans{} is third. Note
that although the differences are not large in absolute value, they are generally statistically
significant. Coming back to \pqt{}, not only does it have the same performance as \combined{}, but
we verified that it actually returns the same predictions, up to a few edges that are classified
differently. This can be explained because they optimize the same objective and \pqt{} is
initialized with the solution from \combined{}. However, we also verified that both edge score and
node loss are respectively higher and lower at the end of the optimization. Finally, the results of
\fwa{} are much worse than all the others, despite the optimization leading to edge scores
comparable to the other methods. This suggests that the subsequent clustering is not
adequate\stodo{Maybe it would be better to save the list of atoms or do a SVD at the end and cluster
from there.}. Regarding the generative parameters, and compared with the default configuration, the
problem is easier when the $k$ directions are spread in a larger number of dimension, and more
difficult when each nodes is involved in four directions instead of three. On the other hand,
whereas we expect the performance to decrease with increase in the overall number of directions or
their overlap, there is not consistent evidence of that.


\begin{table}[htb]
  \centering
  % \setlength{\tabcolsep}{3pt}
  \small
  \caption{Normalized AMI of 5 methods, when generating direction with 7 configurations. For each
    configuration, we generate directions 200 times, and report the mean normalized AMI along with
    the standard deviation. Among the first three methods (that do not use matrix formulation), we
    highlight the best one in \textbf{\textcolor{brown}{bold brown}} and the second best one in
    \textit{\textcolor{red}{italic red}}. When the difference between a score and the next best one
    is statistically significant (that is we can confidently reject the hypothesis that the two
    distribution have the same mean), we furthermore underline that score and show in parenthesis the
    $p$-value of a paired Student's $t$-test.
  \label{tab:edge_Wvaries_nami}}
  \begin{tabular}{llll|ll}
    \toprule
    {} &         \thead{\kmeans{}} &                                  \thead{\lloyd{}} &                              \thead{\combined{}} &  \thead{\fwa{}} &  \thead{\pqt{}} \\
    \midrule
    {\smallk{}}  &            $.836 \pm .08$ &                          $\vsecond{.838 \pm .07}$ &  $\vfirstSig{.875 \pm .06}\,\,\spval{2.17}{-58}$ &  $.213 \pm .11$ &  $.875 \pm .06$ \\
    {\default{}} &            $.818 \pm .06$ &  $\vsecondSig{.873 \pm .05}\,\,\spval{1.25}{-63}$ &  $\vfirstSig{.893 \pm .04}\,\,\spval{5.68}{-33}$ &  $.381 \pm .05$ &  $.893 \pm .04$ \\
    {\largek{}}  &            $.803 \pm .04$ &  $\vsecondSig{.881 \pm .04}\,\,\spval{2.66}{-94}$ &  $\vfirstSig{.894 \pm .04}\,\,\spval{8.98}{-17}$ &  $.421 \pm .04$ &  $.894 \pm .04$ \\
    {\smallo{}}  &            $.813 \pm .07$ &   $\vsecondSig{.824 \pm .06}\,\,\spval{7.57}{-6}$ &  $\vfirstSig{.856 \pm .06}\,\,\spval{2.99}{-57}$ &  $.378 \pm .05$ &  $.855 \pm .06$ \\
    {\largeo{}}  &  $\vsecond{.827 \pm .07}$ &                                    $.823 \pm .06$ &  $\vfirstSig{.852 \pm .06}\,\,\spval{1.90}{-25}$ &  $.370 \pm .06$ &  $.851 \pm .06$ \\
    {\fdirs{}}   &            $.772 \pm .07$ &  $\vsecondSig{.814 \pm .07}\,\,\spval{6.02}{-42}$ &  $\vfirstSig{.853 \pm .06}\,\,\spval{2.13}{-47}$ &  $.320 \pm .06$ &  $.853 \pm .06$ \\
    {\larged{}}  &            $.905 \pm .05$ &  $\vsecondSig{.933 \pm .04}\,\,\spval{1.32}{-31}$ &  $\vfirstSig{.941 \pm .03}\,\,\spval{1.77}{-22}$ &  $.222 \pm .10$ &  $.931 \pm .04$ \\
    \bottomrule
  \end{tabular}
\end{table}

Similar conclusions carry out when we evaluate methods according to their ability of recovering the
original directions. Keeping in mind that in the case of perfect recovery, $d(\mathcal{D}_k,
\mathcal{\wh{D}}_k)$ would be equal to zero, while the distance between two $d$-dimensional unit
vectors is $\sqrt{2}$ in expectation\footnote{One can derive this from the fact after normalization,
vectors whose coordinates are drawn from a standard normal distribution are uniformly distributed on
the unit sphere~\autocite{gaussianSphere59}.}, we see in \autoref{tab:edge_Wvaries_dst} that no
method gets very close to the original directions. \combined{} is again the closer overall, but now
\kmeans{} and \lloyd{} are alternating at the second place. The fact $\mathcal{D}_k$ cannot be
completely recover is not surprising, for in absence of prior information the problem is
under-constrained. Indeed, while \combined{} and \pqt{} delivers almost the exact same prediction,
their directions are clearly different.

\begin{table}[hbt]
  \centering
  % \setlength{\tabcolsep}{3pt}
  \small
  \caption{Same as \autoref{tab:edge_Wvaries_nami}, but reporting $d(\mathcal{D}_k,
  \mathcal{\wh{D}}_k)$, which should be as close as possible to $0$.
  \label{tab:edge_Wvaries_dst}}
    \begin{tabular}{llll|ll}
      \toprule
      {} &                                 \thead{\kmeans{}} &                                  \thead{\lloyd{}} &                              \thead{\combined{}} &   \thead{\fwa{}} &  \thead{\pqt{}} \\
      \midrule
      {\smallk{}}  &  $\vsecondSig{.581 \pm .08}\,\,\spval{4.78}{-30}$ &                                    $.606 \pm .09$ &  $\vfirstSig{.560 \pm .05}\,\,\spval{8.07}{-24}$ &  $1.007 \pm .05$ &  $.633 \pm .06$ \\
      {\default{}} &                                    $.533 \pm .08$ &   $\vsecondSig{.528 \pm .09}\,\,\spval{9.61}{-4}$ &   $\vfirstSig{.516 \pm .06}\,\,\spval{4.95}{-5}$ &   $.877 \pm .05$ &  $.564 \pm .06$ \\
      {\largek{}}  &                                    $.541 \pm .07$ &  $\vsecondSig{.521 \pm .07}\,\,\spval{5.08}{-23}$ &                          $\vfirst{.520 \pm .05}$ &   $.811 \pm .06$ &  $.551 \pm .06$ \\
      {\smallo{}}  &  $\vsecondSig{.554 \pm .08}\,\,\spval{2.74}{-52}$ &                                    $.589 \pm .08$ &  $\vfirstSig{.538 \pm .05}\,\,\spval{2.38}{-12}$ &   $.908 \pm .05$ &  $.596 \pm .07$ \\
      {\largeo{}}  &  $\vsecondSig{.554 \pm .09}\,\,\spval{1.50}{-58}$ &                                    $.595 \pm .08$ &   $\vfirstSig{.545 \pm .06}\,\,\spval{4.08}{-5}$ &   $.935 \pm .05$ &  $.604 \pm .08$ \\
      {\fdirs{}}   &                                    $.565 \pm .08$ &                          $\vsecond{.564 \pm .09}$ &   $\vfirstSig{.546 \pm .05}\,\,\spval{6.30}{-8}$ &   $.928 \pm .06$ &  $.588 \pm .06$ \\
      {\larged{}}  &  $\vsecondSig{.571 \pm .07}\,\,\spval{2.02}{-89}$ &                                    $.602 \pm .07$ &                          $\vfirst{.567 \pm .05}$ &  $1.033 \pm .06$ &  $.641 \pm .05$ \\
      \bottomrule
    \end{tabular}
\end{table}

\iffalse
\begin{table}[hbt]
  \centering
  % \setlength{\tabcolsep}{3pt}
  \small
  \caption{Same as \autoref{tab:edge_Wvaries_nami}, but when $\mathcal{D}_k$ is fixed and generating
    200 graphs.  \label{tab:edge_Gvaries_nami}}
    \begin{tabular}{llll|ll}
      \toprule
      {} &                                 \thead{\kmeans{}} &                                  \thead{\lloyd{}} &                              \thead{\combined{}} &  \thead{\fwa{}} &  \thead{\pqt{}} \\
      \midrule
      {\smallk{}}  &  $\vsecondSig{.678 \pm .07}\,\,\spval{4.16}{-22}$ &                                    $.657 \pm .07$ &  $\vfirstSig{.729 \pm .08}\,\,\spval{1.13}{-62}$ &  $.330 \pm .08$ &  $.729 \pm .08$ \\
      {\default{}} &                                    $.745 \pm .04$ &  $\vsecondSig{.799 \pm .04}\,\,\spval{2.88}{-65}$ &  $\vfirstSig{.839 \pm .03}\,\,\spval{1.75}{-68}$ &  $.379 \pm .04$ &  $.839 \pm .03$ \\
      {\largek{}}  &                                    $.769 \pm .03$ &  $\vsecondSig{.843 \pm .04}\,\,\spval{5.06}{-84}$ &  $\vfirstSig{.870 \pm .04}\,\,\spval{8.54}{-40}$ &  $.431 \pm .04$ &  $.870 \pm .04$ \\
      {\smallo{}}  &                                    $.813 \pm .06$ &  $\vsecondSig{.845 \pm .04}\,\,\spval{1.44}{-27}$ &  $\vfirstSig{.869 \pm .04}\,\,\spval{3.98}{-55}$ &  $.371 \pm .05$ &  $.868 \pm .04$ \\
      {\largeo{}}  &                                    $.788 \pm .06$ &   $\vsecondSig{.799 \pm .05}\,\,\spval{1.54}{-6}$ &  $\vfirstSig{.830 \pm .05}\,\,\spval{2.36}{-57}$ &  $.408 \pm .05$ &  $.830 \pm .05$ \\
      {\fdirs{}}   &                                    $.678 \pm .05$ &  $\vsecondSig{.700 \pm .06}\,\,\spval{1.65}{-24}$ &  $\vfirstSig{.761 \pm .06}\,\,\spval{2.73}{-80}$ &  $.336 \pm .05$ &  $.761 \pm .06$ \\
      {\larged{}}  &                                    $.863 \pm .06$ &  $\vsecondSig{.893 \pm .05}\,\,\spval{8.68}{-44}$ &  $\vfirstSig{.907 \pm .05}\,\,\spval{5.58}{-31}$ &  $.276 \pm .08$ &  $.895 \pm .05$ \\
      \bottomrule
    \end{tabular}
\end{table}

\begin{table}[hbt]
  \centering
  % \setlength{\tabcolsep}{3pt}
  \small
  \caption{Same as \autoref{tab:edge_Wvaries_dst}, but when $\mathcal{D}_k$ is fixed and generating
    200 graphs.  \label{tab:edge_Gvaries_dst}}
    \begin{tabular}{llll|ll}
      \toprule
      {} &                                 \thead{\kmeans{}} &                                  \thead{\lloyd{}} &                              \thead{\combined{}} &   \thead{\fwa{}} &  \thead{\pqt{}} \\
      \midrule
      {\smallk{}}  &  $\vsecondSig{.702 \pm .08}\,\,\spval{2.27}{-69}$ &                                    $.754 \pm .06$ &  $\vfirstSig{.647 \pm .06}\,\,\spval{1.59}{-81}$ &   $.975 \pm .06$ &  $.734 \pm .07$ \\
      {\default{}} &                          $\vsecond{.607 \pm .06}$ &                                    $.608 \pm .07$ &  $\vfirstSig{.567 \pm .04}\,\,\spval{2.55}{-69}$ &   $.897 \pm .04$ &  $.622 \pm .05$ \\
      {\largek{}}  &                                    $.584 \pm .06$ &  $\vsecondSig{.563 \pm .07}\,\,\spval{2.79}{-24}$ &                          $\vfirst{.561 \pm .05}$ &   $.804 \pm .05$ &  $.582 \pm .06$ \\
      {\smallo{}}  &  $\vsecondSig{.530 \pm .08}\,\,\spval{1.20}{-67}$ &                                    $.571 \pm .07$ &   $\vfirstSig{.518 \pm .05}\,\,\spval{5.10}{-8}$ &   $.894 \pm .05$ &  $.572 \pm .06$ \\
      {\largeo{}}  &  $\vsecondSig{.607 \pm .08}\,\,\spval{1.24}{-70}$ &                                    $.647 \pm .07$ &  $\vfirstSig{.578 \pm .05}\,\,\spval{6.06}{-35}$ &   $.896 \pm .06$ &  $.636 \pm .07$ \\
      {\fdirs{}}   &  $\vsecondSig{.655 \pm .08}\,\,\spval{2.30}{-32}$ &                                    $.677 \pm .08$ &  $\vfirstSig{.608 \pm .06}\,\,\spval{6.88}{-71}$ &   $.938 \pm .04$ &  $.671 \pm .07$ \\
      {\larged{}}  &  $\vsecondSig{.603 \pm .07}\,\,\spval{9.25}{-95}$ &                                    $.636 \pm .07$ &  $\vfirstSig{.589 \pm .05}\,\,\spval{6.97}{-12}$ &  $1.003 \pm .07$ &  $.662 \pm .06$ \\
      \bottomrule
    \end{tabular}
\end{table}
\fi
