\section{Synthetic experiments}
\label{sec:edge_exp}

In the previous section, we introduced five methods to address \autoref{p:edge_full}, which we now
compare in our experiments. Specifically, this includes:
\begin{enumerate}[1.,nosep]
\item the \kmeans{} baseline,
\item our \lloyd{}-like iterative refinement,
\item the \combined{} optimization of the two terms in \eqref{eq:edge_soft},
\item the \fwa{} convex optimization of \eqref{eq:edge_FW} and
\item the \pqt{} explicit low-rank factorization, optimizing \eqref{eq:edge_PQ}.
\end{enumerate}
They all take as input a node-attributed graph, that is a pair $G=(V, E)$ and $X$.
For evaluation purposes, we extract from their output a set of $k$ directions and an
assignment of every edge to one of these directions.
We try those five methods on data generated synthetically to fit our model described previously in
\autoref{sub:edge_setting}, and study how they perform under various conditions.

\subsection{Data generation}
\label{sub:edge_generation}

While there exists several methods to generate attributed graph with community
structure~\autocites{Yang2013}{XuBayesian14}{Kataoka2016}, here we present the one we devised for
our initial experiments, as it is specifically tailored to the model introduced in \autoref{sub:edge_setting}.
It takes the following parameters as input:
\begin{itemize}[nosep]
  \item the number of nodes $n$,
  \item the dimension of their profiles $d$,
  \item the total number of directions $k$,
  \item the maximum number $k_\mathrm{local}$ of directions incident to each node, and
  \item an integer $n_o$ controlling to which extent two distinct directions explain common dimensions.
\end{itemize}
It then returns:
\begin{itemize}[nosep]
  \item a graph topology $G=(V,E)$,
  \item a profile matrix $X\in \Rbb^{n\times d}$,
  \item a set of $k$ directions $\mathcal{D}_k \subset \dball$, and
  \item an assignment from every edge $(u,v)$ to a direction index $y_{uv} \in \rangesk$ such that in
    most cases, $\mathcal{E}(u,v) = y_{uv}$. In other words, the direction assigned to $(u,v)$ is
    indeed the one achieving maximum goodness. We will explain later why this is not always the
    case.
\end{itemize}
Our methods essentially proceeds as follow:
\begin{enumerate}[1),nosep]
  \item We create an appropriate number of small Erdős-Rényi subgraphs and assign to each node a set of
    $k_\mathrm{local}$ directions such that most pairs of adjacent nodes have at least one
    directions in common (as we motivated after having introduced equation \eqref{eq:edge_local_loss}
    \vpageref{eq:edge_local_loss}).
  \item We connect the blocks by several pairs of nodes having one direction in common, and pick for
    each edge $(u,v)$ a direction index $y_{uv} \in \rangesk$ among the ones shared by its endpoint.
  \item We draw the $k$ directions at random, ensuring they respect the parameter $n_o$.
  \item Finally, we optimize the profiles so as to simultaneously maximize the edges goodness,
    minimize the term $\mathcal{L}_{\mathrm{node}}$ of equation \eqref{eq:edge_node_loss} and
    enforces as much as possible that for every edge $(u,v) \in E$, $\mathcal{E}(u,v) = y_{uv}$.
\end{enumerate}
Steps 1 and 2 generate a graph topology $(V, E)$, while steps 3 and 4 generates directions and
profiles $(\mathcal{D_k}, X)$. Note however that step 3 is independent of the steps 1 and 2, while
step 4 is dependent on all the previous steps.
We now describe each of these steps in more details.

\begin{enumerate}[1),leftmargin=*]
\item
We first create small Erdős-Rényi
subgraphs~\autocites{erdos1959random}{gilbertRG59} that we call \emph{blocks}. Then we
assign to each node a $k_\mathrm{local}$-tuple of
directions such that two adjacent nodes have \emph{at least} one direction in common.
For convenience, let us first identify directions with colors and
thus call such a $k_\mathrm{local}$-tuple of directions a \emph{palette}. For a given palette $p$,
we call adjacent palettes, denoted $\adj(p)$, all the palettes different from $p$\footnote{We ruled
out $p$ being adjacent to itself to avoid the trivial solution of all nodes being assigned the same
palette.} but sharing one
color with $p$. Finally, we say that an edge $u,v$ is \emph{colorless} if the palette of $u$ and $v$ have
no color in common. Because in the general case, it is not always possible to assign a palette to
every node such that there is no colorless edge,\footnote{For instance, consider a $4$-clique with $k=4$
and $k_\mathrm{local}=2$. Without loss of generality, say we assign the palette $(1,2)$ to node
$1$. The adjacent palettes are then $(1,3)$, $(1,4)$, $(2,3)$ and $(2,4)$. If we assign those
starting by $1$ (respectively $2$) to the second and third node, then the fourth palette has to
contain a $1$ (respectively $2$), which is not possible (because two connected nodes cannot have the
same palette). On the other hand, if we assign $(1,3)$ to the second, the third node can only have
$(2,3)$, meaning the fourth node has again no palette available. The same situation arises with $(1,4)$ and
$(2,4)$ respectively.} we now describe a simple heuristic. It performs a breath first visit of a subgraph,
starting from a random node. Upon visiting an uncolored node $u$, it builds a set $P$ of palette
respecting the colouring constraint. Namely, for every colored neighbor $v$ of $u$, it retrieves the
adjacen palettes of $v$'s palette and builds their intersection:
\begin{equation*}
  P=\bigcap_{v\in \nei(u)} \adj(p_v)\,.
\end{equation*}
Then it selects \uar{} a palette from $P$, or an arbitrary palette if $P$ is empty: see
\autoref{fig:edge_palette} for a small illustration. Once all nodes have been colored, we count the
number of edges that are colorless. We repeat this procedure, keeping track of the palettes assignment
minimizing the number of colorless edges.
\begin{figure}[ht]
  \centering
  \includegraphics[width=0.6\textwidth]{tikz/edge_palette_tikz.pdf}
  \caption[Finding the $k_\mathrm{local}$ directions of node when generating synthetic graph]{A
    small example of the node palette assignment, with $k=4$ colors (blue, green, red and
    orange) and palettes of size $k_\mathrm{local}=3$. We assume nodes $1$ and $2$ have already been
    visited, and got assigned the palette $(\text{blue, green, red})$ and $(\text{blue, green,
    orange})$ respectively. Moreover, we are currently visiting node $3$ while node $4$ is yet uncolored. In that
    case, there are two possible palettes for node $3$. If we select $(\text{green, red, orange})$, a
    possible color assignment for the edges $(1,2)$, $(1, 3)$ and $(2, 3)$ is respectively blue,
    red, and green.
  \label{fig:edge_palette}}
\end{figure}

\item
Once we colored the nodes of every block, we look at pair
of blocks, build a list of all edges between blocks that are not colorless and sample from it to
connect blocks. The last step is to a assign a color $y_{uv}$ to each edge $(u,v)$ from the shared color of its
endpoint (or an arbitrary color for colorless edges). We note that this is reminiscent of the stochastic
block model, although the probabilities of edges between blocks are not uniform, for they depend of
the palette assigned to every node.

  \item 
We then generate $k$ sparse, unit-norm directions $\mathcal{D}_k=\{w_1, \ldots, w_k\}$,
independently of the graph. An underlying assumption of our model is that each direction should
provide an explanation of why two nodes are connected that is markedly different from the
explanation of another direction. A way to achieve that is to create directions with only a few
non-zeros components, and ensure that the non-zero components of one direction do not appear at the
same position as the non-zero components of another direction. Therefore, we control through $n_o$
the number of dimensions where more than one non-zero component exists across the $k$ directions.
That is, when $n_o = 0$, each direction has exactly \nicefrac{k}{d} non-zero components that are
disjoint.\footnote{We choose $d$ to be a multiple of $k$.} On the other hand, if for instance $n_o=5$,
then $5$ of the $d$ dimensions will have two directions with non-zero components on it. Those
non-zero components are drawn \uar{} from $[-1, 1]$ and each direction is then normalized. For
instance, with $d=6$, $k=3$ and $n_o=2$, we could generate the following directions, where the blue
non-zero components are \enquote{overlapping} with the black ones:
% \vspace{-.5\baselineskip}

\begin{center}
  \begin{tabular}{ccc}
    $w_1$ & $w_2$ & $w_3$ \\
    $\begin{pmatrix}
      \phantom{-}0.729 \\
      0    \\
      0    \\
      0    \\
      \phantom{-}\textcolor{DodgerBlue}{0.450} \\
      -0.516 \\
    \end{pmatrix}$ &
    $\begin{pmatrix}
      0    \\
      0    \\
      \textcolor{DodgerBlue}{-0.483} \\
      -0.533 \\
      \phantom{-}0.694 \\
      0    \\
    \end{pmatrix}$ &
    $\begin{pmatrix}
      0    \\
      -0.639 \\
      -0.769 \\
      0    \\
      0    \\
      0    \\
    \end{pmatrix}$
  \end{tabular}
\end{center}

\item
Returning to directions instead of colors, we have a direction $w_{y_{uv}} \in \mathcal{D}_k$
assigned to every edge $(u,v)$. For easy of notation, in that paragraph, we will let $w_{uv} = w_{y_{uv}}$.
The final step in generating our node-attributed graph is to
find a set of user profiles $X=\{x_u\}_{u\in V}$ that maximizes the edge goodness and minimizes the node
loss term, defined respectively as:
\begin{align*}
  \mathcal{L}_{\mathrm{edge}} &=
  \sum_{u,v \in E} {s_{uv}}^T w_{uv} \quad \text{and} \\
  \mathcal{L}_{\mathrm{node}} &=
  \sum_{u\in V} \left|\left| x_u - b_u - \sum_{v \in \nei(u)} a_{uv} w_{uv} \right|\right|^2\,,
\end{align*}
where for simplicity, we fix as before $b_u = \zerov$ and $a_{uv} = \frac{1}{\deg(u)}$. Furthermore, to ensure that the
direction assigned to any edge $(u,v)$ achieves a higher goodness than the other directions (that is
$\forall w_\ell \in \mathcal{D}_k \setminus \{w_{uv}\}\,, g(s_{uv}, w_{uv}) \geq  g(s_{uv},
w_\ell)$), we also minimize a cross-entropy loss commonly used in non-binary classification
problem~\autocite[Section 4.3.4]{PRML06}.  Specifically, denoting by $p_{uv,\ell} = \frac{\exp
\left( g(s_{uv}, w_{uv}) \right)}{\sum_{\ell=1}^k \exp \left( g(s_{uv}, w_\ell) \right)}$ the
softmax \enquote{probability} of $s_{uv}$ being explained by direction $\ell$, the cross-entropy
loss is defined by:
\begin{equation}
  \label{eq:edge_cross}
  \mathcal{L}_{\mathrm{cross-entropy}} = -\sum_{u,v \in E} \sum_{l=1}^k
  \Ind{w_{uv} = w_\ell} p_{uv,\ell}\,.
\end{equation}

In practice, we first minimize
\begin{equation}
  \label{eq:edge_cross_edge}
  \lambda \mathcal{L}_{\mathrm{cross-entropy}} - \mathcal{L}_{\mathrm{edge}}
\end{equation}
with respect to $X$ using the Adam algorithm~\autocite{Adam15} and
automatic differentiation~\autocite{autograd15}\footnote{Both being implemented by the Pytorch
package: \url{http://pytorch.org/}.}, projecting the current iterate back to the set of
matrix with unit $L_2$ norm columns $\mathbb{M}^{n\times d}$ at each step.

Finally, to minimize $\mathcal{L}_{\mathrm{node}}$, we iterate over the nodes and take some gradient
steps to minimize $\left|\left| x_u - \frac{1}{\deg(u)} \sum_{v\in\nei{u}} w_{uv}\right|\right|^2$, but only to the
extent that $w_{uv}$ remains the direction with the highest goodness on the edge $(u,v) \in E$.
\end{enumerate}

\subsection{Results}
\label{sub:edge_synth_results}

In this section, we start by describing how we generate test instances of \autoref{p:edge_full} for
the five methods we want to compare.  Then we present two evaluation measures: how well those
methods classify edges, and how well they recover known directions. We use the first measure to
comment on the quality of the generated instances. Finally we present the results of the five
methods in \autoref{tab:edge_Wvaries_nami} and \autoref{tab:edge_Wvaries_dst}.
% - describe how we generate instances
% - present our two evaluation measures how well we classify edges and how well we recover known direction
% - comment on their quality (but this requires AMI…)
% - comparison of the 5 methods

Our test instances are generated as follow. We first choose $n=500$, and generate once a graph
topology with directions assigned to all edges according to steps 1 and 2. This results in a graph
$G$ with roughly $\np{1350}$ edges. We then generate the directions $\mathcal{D}_k$ according to
step 3 and the following choice of parameters: $d=35$, $k=7$, $k_{\mathrm{local}}=3$ and $n_o=0$.
Finally, we create the profiles $X$ as described in step 4.
To account for the randomness of this generation process, we repeat our
measurement over \np{200} such generations of $\mathcal{D}_k$ and $X$. On the other hand, we verify experimentally that
generating various graphs while keeping $\mathcal{D}_k$ fixed gives the same results, hence we will
not report them in the main text.

In the rest of this section, we will refer to the set of parameters $d=35$, $k=7$,
$k_{\mathrm{local}}=3$ and $n_o=0$ as the \emph{default} configuration.
We also try other configurations to study the parameter sensibility of our methods. Specifically, we
experiment with less directions ($k=5$) or more directions ($k=9, d=36$)\footnote{We add one
dimension to ensure that $d$ remains a multiple of $k$.}, with two higher levels of overlap between
the directions ($n_o=6$ and $n_o=12$), with more directions per node ($k_{\mathrm{local}}=4$) and in
larger dimension ($d=77$).

\medskip
% After having assigned a single direction to every edge and generated the user profiles, we can
% measure how the assignment and the profiles match, in the sense that the assigned direction of every
% edge is the one with the largest goodness. Indeed, recall we predict the
% direction of an edge $(u,v)$ as $\mathcal{E}(u,v) = \argmax_{\ell \in \rangesk} g(s_{uv}, w_\ell)$.
% To do so, we however do not rely on the accuracy measure.

Let us now describe how we evaluate the results of our methods on such instances. As said at the
beginning of this section, we extract from the output of each method a set $\mathcal{\wh{D}}_k =
\{\wh{w}_1, \ldots, \wh{w}_k\}$ of $k$ directions, and assign a 
direction to an edge $(u,v)$ as $\mathcal{E}(u,v) = \argmax_{\ell \in \rangesk} g(s_{uv}, \wh{w}_\ell)$.
Recall that during the second of the step of the instance generation, we assigned a direction index
$y_{uv}$ to every edge.
However, we cannot simply evaluate our predictions by simply comparing $\mathcal{E}(u,v)$ and
$y_{uv}$. Indeed, here is no guarantee that the $\mathcal{\wh{D}}_k$
directions are recovered in the same order as they were generated. In case of strong disagreement
between the predictions and ground truth labels, it might additionally be difficult to find a
permutation to conciliate them.  Therefore, we see the problem as a clustering one with known ground
truth, and turn to the Adjusted Mutual Information (AMI)~\autocite{AMI09}. It is an information
theoretic measure that enjoys several properties: it is invariant to permutations of the labels,
invariant to the shape of clusters and it is bounded between $0$ and $1$, where an AMI of $1$
indicates perfect correlation with the ground truth labels and $0$ indicates perfect independence
with the ground truth labels.

Besides the AMI score, we also measure how far the recovered directions $\mathcal{\wh{D}}_k$
are from the actual ones, generated initially. More precisely, we associate each vector $w_\ell \in
\mathcal{D}_k$ with the closest one in $\mathcal{\wh{D}}_k$ and report the average $\ell_2$ distance
between the two elements of these $k$ pairs, that is
\begin{equation*}
  d(\mathcal{D}_k, \mathcal{\wh{D}}_k) =
  \frac{1}{k} \sum_{\ell \in \rangesk} \min_{w'_i \in \mathcal{\wh{D}}_k} \| w_\ell - w'_i \|_2
\end{equation*}

\medskip

Before moving to the results, let us first note that,
in the \default{} configuration, we do not always
generate user profiles leading to a perfect assignment of edge according to their goodness. This can
be seen in \autoref{tab:edge_badami}. In
general, this is not always possible. Consider a $k$-regular subgraph, where the $k$ edges incident
to every node are each assigned a different directions. By symmetry, it is impossible to find node
profiles that would achieve maximum goodness for all the edges. This is partly what motivated the
$k_{\mathrm{local}}$ local constraint. Even so, and more pragmatically, whereas those AMI scores
could be made higher by increasing the coefficient $\lambda$ in~\eqref{eq:edge_cross_edge}, this would
imply a lower edge goodness. With our choice of $\lambda$, only a few percent of the edges have a mismatch
between the directions we assigned them and the one with the largest goodness. Note also that
depending of the parameters of the generation, the AMI is not the same. Therefore, in the following,
we divide all the AMI scores by this natural score, in order to obtain a \enquote{standardized} measure.
% \newline\vspace{1\baselineskip}\noindent%

\begin{table}[htb]
\begin{adjustwidth}{-1.5cm}{}
  \centering
  \small
  \caption[Qualite of our synthetic instances according to generation parameters]{The degree to
    which the edge assignment of step 2 agree with the directions and profiles
    of steps 3 and 4. This is measured by the AMI between the assignment $\{y_{uv}\}_{uv \in E}$ and
    what maximal goodness would assign, \ie{} $\{\mathcal{E}(u,v)\}_{uv \in E}$. We report the
    average AMI and standard deviation over 200 generations of $\mathcal{D}_k$, for different
    configurations.  \label{tab:edge_badami}}
\begin{tabular}{lccccccc}
  \toprule
  {$\mathcal{D}_k$ parameters}  & {\default{}}   & {\smallk{}}    & {\largek{}}    & {\smallo{}}    & {\largeo{}}    & {\fdirs{}}     & {\larged{}}    \\
  \midrule
  AMI & $.922 \pm .03$ & $.925 \pm .03$ & $.905 \pm .02$ & $.880 \pm .04$ & $.874 \pm .05$ & $.893 \pm .03$ & $.973 \pm .01$ \\
  \bottomrule
\end{tabular}
\end{adjustwidth}
\end{table}

As showed in \autoref{tab:edge_Wvaries_nami}, \combined{} is always the best method and, excluding
\pqt{} for now, the \lloyd{} heuristic is second except in one case, while \kmeans{} is third. Note
that although the differences are not large in absolute value, they are generally statistically
significant. Coming back to \pqt{}, not only does it have the same performance as \combined{}, but
we verified that it actually returns the same predictions, up to a few edges that are classified
differently. This can be explained because they optimize the same objective and \pqt{} is
initialized with the solution from \combined{}. However, we also verified that both edge score and
node loss are respectively higher and lower at the end of the optimization. Finally, the results of
\fwa{} are much worse than all the others, despite the optimization leading to edge scores
comparable to the other methods. This suggests that the subsequent clustering is not
adequate. Regarding the generative parameters, and compared with the default configuration, the
problem is easier when the $k$ directions are spread in a larger number of dimensions, and more
difficult when each nodes is involved in four directions instead of three. On the other hand,
whereas we expect the performance to decrease with increase in the overall number of directions or
their overlap, there is not consistent evidence of that.

\begin{table*}[htb]
\begin{adjustwidth}{-2cm}{}
  \centering
  % \setlength{\tabcolsep}{3pt}
  \small
  \caption[Performance of our proposed methods on synthetic instances]{Standardized AMI of 5
    methods, when generating directions with 7 configurations. For each
    configuration, we generate directions 200 times, and report the mean standardized AMI along with
    the standard deviation. Among the first three methods (that do not use matrix formulation), we
    highlight the best one in \textbf{\textcolor{brown}{bold brown}} and the second best one in
    \textit{\textcolor{red}{italic red}}. When the difference between a score and the next best one
    is statistically significant (\ie{} when we can confidently reject the hypothesis that the two
    distributions have the same mean), we furthermore show in parenthesis the supporting
    $p$-value of a paired Student's $t$-test.
  \label{tab:edge_Wvaries_nami}}
  \begin{tabulary}{179mm}{llll|ll}
    \toprule
    \thead{$\mathcal{D}_k$ parameters} &         \thead{\kmeans{}} &                                  \thead{\lloyd{}} &                              \thead{\combined{}} &  \thead{\fwa{}} &  \thead{\pqt{}} \\
    \midrule
    {\default{}} &            $.818 \pm .06$ &  $\esecondSig{.873 \pm .05}\,\,\spval{1.25}{-63}$ &  $\efirstSig{.893 \pm .04}\,\,\spval{5.68}{-33}$ &  $.381 \pm .05$ &  $.893 \pm .04$ \\
    {\smallk{}}  &            $.836 \pm .08$ &                          $\esecond{.838 \pm .07}$ &  $\efirstSig{.875 \pm .06}\,\,\spval{2.17}{-58}$ &  $.213 \pm .11$ &  $.875 \pm .06$ \\
    {\largek{}}  &            $.803 \pm .04$ &  $\esecondSig{.881 \pm .04}\,\,\spval{2.66}{-94}$ &  $\efirstSig{.894 \pm .04}\,\,\spval{8.98}{-17}$ &  $.421 \pm .04$ &  $.894 \pm .04$ \\
    {\smallo{}}  &            $.813 \pm .07$ &   $\esecondSig{.824 \pm .06}\,\,\spval{7.57}{-6}$ &  $\efirstSig{.856 \pm .06}\,\,\spval{2.99}{-57}$ &  $.378 \pm .05$ &  $.855 \pm .06$ \\
    {\largeo{}}  &  $\esecond{.827 \pm .07}$ &                                    $.823 \pm .06$ &  $\efirstSig{.852 \pm .06}\,\,\spval{1.90}{-25}$ &  $.370 \pm .06$ &  $.851 \pm .06$ \\
    {\fdirs{}}   &            $.772 \pm .07$ &  $\esecondSig{.814 \pm .07}\,\,\spval{6.02}{-42}$ &  $\efirstSig{.853 \pm .06}\,\,\spval{2.13}{-47}$ &  $.320 \pm .06$ &  $.853 \pm .06$ \\
    {\larged{}}  &            $.905 \pm .05$ &  $\esecondSig{.933 \pm .04}\,\,\spval{1.32}{-31}$ &  $\efirstSig{.941 \pm .03}\,\,\spval{1.77}{-22}$ &  $.222 \pm .10$ &  $.931 \pm .04$ \\
    \bottomrule
  \end{tabulary}
\end{adjustwidth}
\end{table*}

Similar conclusions carry out when we evaluate methods according to their ability of recovering the
original directions. Keeping in mind that in the case of perfect recovery, $d(\mathcal{D}_k,
\mathcal{\wh{D}}_k)$ would be equal to zero, while the distance between two $d$-dimensional unit
vectors is $\sqrt{2}$ in expectation\footnote{One can derive this from the fact after normalization,
vectors whose coordinates are drawn from a standard normal distribution are uniformly distributed on
the unit sphere~\autocite{gaussianSphere59}.}, we see in \autoref{tab:edge_Wvaries_dst} that no
method gets very close to the original directions. \combined{} is again the closer overall, but now
\kmeans{} and \lloyd{} are alternating at the second place. The fact $\mathcal{D}_k$ cannot be
completely recover is not surprising, for in absence of prior information the problem is
under-constrained. Indeed, while \combined{} and \pqt{} deliver almost the exact same prediction,
their directions are clearly different.

\begin{table*}[hbt]
\begin{adjustwidth}{-2cm}{}
  \centering
  % \setlength{\tabcolsep}{3pt}
  \small
  \caption{Same as \autoref{tab:edge_Wvaries_nami}, but reporting $d(\mathcal{D}_k,
  \mathcal{\wh{D}}_k)$, which should be as close as possible to $0$.
  \label{tab:edge_Wvaries_dst}}
  \begin{tabulary}{179mm}{llll|ll}
      \toprule
      \thead{$\mathcal{D}_k$ parameters} &                                 \thead{\kmeans{}} &                                  \thead{\lloyd{}} &                              \thead{\combined{}} &   \thead{\fwa{}} &  \thead{\pqt{}} \\
      \midrule
      {\default{}} &                                    $.533 \pm .08$ &   $\esecondSig{.528 \pm .09}\,\,\spval{9.61}{-4}$ &   $\efirstSig{.516 \pm .06}\,\,\spval{4.95}{-5}$ &   $.877 \pm .05$ &  $.564 \pm .06$ \\
      {\smallk{}}  &  $\esecondSig{.581 \pm .08}\,\,\spval{4.78}{-30}$ &                                    $.606 \pm .09$ &  $\efirstSig{.560 \pm .05}\,\,\spval{8.07}{-24}$ &  $1.007 \pm .05$ &  $.633 \pm .06$ \\
      {\largek{}}  &                                    $.541 \pm .07$ &  $\esecondSig{.521 \pm .07}\,\,\spval{5.08}{-23}$ &                          $\efirst{.520 \pm .05}$ &   $.811 \pm .06$ &  $.551 \pm .06$ \\
      {\smallo{}}  &  $\esecondSig{.554 \pm .08}\,\,\spval{2.74}{-52}$ &                                    $.589 \pm .08$ &  $\efirstSig{.538 \pm .05}\,\,\spval{2.38}{-12}$ &   $.908 \pm .05$ &  $.596 \pm .07$ \\
      {\largeo{}}  &  $\esecondSig{.554 \pm .09}\,\,\spval{1.50}{-58}$ &                                    $.595 \pm .08$ &   $\efirstSig{.545 \pm .06}\,\,\spval{4.08}{-5}$ &   $.935 \pm .05$ &  $.604 \pm .08$ \\
      {\fdirs{}}   &                                    $.565 \pm .08$ &                          $\esecond{.564 \pm .09}$ &   $\efirstSig{.546 \pm .05}\,\,\spval{6.30}{-8}$ &   $.928 \pm .06$ &  $.588 \pm .06$ \\
      {\larged{}}  &  $\esecondSig{.571 \pm .07}\,\,\spval{2.02}{-89}$ &                                    $.602 \pm .07$ &                          $\efirst{.567 \pm .05}$ &  $1.033 \pm .06$ &  $.641 \pm .05$ \\
      \bottomrule
    \end{tabulary}
\end{adjustwidth}
\end{table*}

\iffalse
\begin{aside}

% \begin{table}[hbt]
%   \centering
  % \setlength{\tabcolsep}{3pt}
  % \caption{%
\begin{center}
  \small
   \captionsetup{font=small}
   \captionof{table}{%
    Same as \autoref{tab:edge_Wvaries_nami}, but when $\mathcal{D}_k$ is fixed and generating
    200 graphs.  \label{tab:edge_Gvaries_nami}}
    \begin{tabular}{llll|ll}
      \toprule
      {} &                                 \thead{\kmeans{}} &                                  \thead{\lloyd{}} &                              \thead{\combined{}} &  \thead{\fwa{}} &  \thead{\pqt{}} \\
      \midrule
      {\smallk{}}  &  $\esecondSig{.678 \pm .07}\,\,\spval{4.16}{-22}$ &                                    $.657 \pm .07$ &  $\efirstSig{.729 \pm .08}\,\,\spval{1.13}{-62}$ &  $.330 \pm .08$ &  $.729 \pm .08$ \\
      {\default{}} &                                    $.745 \pm .04$ &  $\esecondSig{.799 \pm .04}\,\,\spval{2.88}{-65}$ &  $\efirstSig{.839 \pm .03}\,\,\spval{1.75}{-68}$ &  $.379 \pm .04$ &  $.839 \pm .03$ \\
      {\largek{}}  &                                    $.769 \pm .03$ &  $\esecondSig{.843 \pm .04}\,\,\spval{5.06}{-84}$ &  $\efirstSig{.870 \pm .04}\,\,\spval{8.54}{-40}$ &  $.431 \pm .04$ &  $.870 \pm .04$ \\
      {\smallo{}}  &                                    $.813 \pm .06$ &  $\esecondSig{.845 \pm .04}\,\,\spval{1.44}{-27}$ &  $\efirstSig{.869 \pm .04}\,\,\spval{3.98}{-55}$ &  $.371 \pm .05$ &  $.868 \pm .04$ \\
      {\largeo{}}  &                                    $.788 \pm .06$ &   $\esecondSig{.799 \pm .05}\,\,\spval{1.54}{-6}$ &  $\efirstSig{.830 \pm .05}\,\,\spval{2.36}{-57}$ &  $.408 \pm .05$ &  $.830 \pm .05$ \\
      {\fdirs{}}   &                                    $.678 \pm .05$ &  $\esecondSig{.700 \pm .06}\,\,\spval{1.65}{-24}$ &  $\efirstSig{.761 \pm .06}\,\,\spval{2.73}{-80}$ &  $.336 \pm .05$ &  $.761 \pm .06$ \\
      {\larged{}}  &                                    $.863 \pm .06$ &  $\esecondSig{.893 \pm .05}\,\,\spval{8.68}{-44}$ &  $\efirstSig{.907 \pm .05}\,\,\spval{5.58}{-31}$ &  $.276 \pm .08$ &  $.895 \pm .05$ \\
      \bottomrule
    \end{tabular}
% \end{table}
\end{center}

% \begin{table}[hbt]
%   \centering
  % \setlength{\tabcolsep}{3pt}
\begin{center}
  \small
   \captionsetup{font=small}
   \captionof{table}{%
  % \caption{%
    Same as \autoref{tab:edge_Wvaries_dst}, but when $\mathcal{D}_k$ is fixed and generating
    200 graphs.  \label{tab:edge_Gvaries_dst}}
    \begin{tabular}{llll|ll}
      \toprule
      {} &                                 \thead{\kmeans{}} &                                  \thead{\lloyd{}} &                              \thead{\combined{}} &   \thead{\fwa{}} &  \thead{\pqt{}} \\
      \midrule
      {\smallk{}}  &  $\esecondSig{.702 \pm .08}\,\,\spval{2.27}{-69}$ &                                    $.754 \pm .06$ &  $\efirstSig{.647 \pm .06}\,\,\spval{1.59}{-81}$ &   $.975 \pm .06$ &  $.734 \pm .07$ \\
      {\default{}} &                          $\esecond{.607 \pm .06}$ &                                    $.608 \pm .07$ &  $\efirstSig{.567 \pm .04}\,\,\spval{2.55}{-69}$ &   $.897 \pm .04$ &  $.622 \pm .05$ \\
      {\largek{}}  &                                    $.584 \pm .06$ &  $\esecondSig{.563 \pm .07}\,\,\spval{2.79}{-24}$ &                          $\efirst{.561 \pm .05}$ &   $.804 \pm .05$ &  $.582 \pm .06$ \\
      {\smallo{}}  &  $\esecondSig{.530 \pm .08}\,\,\spval{1.20}{-67}$ &                                    $.571 \pm .07$ &   $\efirstSig{.518 \pm .05}\,\,\spval{5.10}{-8}$ &   $.894 \pm .05$ &  $.572 \pm .06$ \\
      {\largeo{}}  &  $\esecondSig{.607 \pm .08}\,\,\spval{1.24}{-70}$ &                                    $.647 \pm .07$ &  $\efirstSig{.578 \pm .05}\,\,\spval{6.06}{-35}$ &   $.896 \pm .06$ &  $.636 \pm .07$ \\
      {\fdirs{}}   &  $\esecondSig{.655 \pm .08}\,\,\spval{2.30}{-32}$ &                                    $.677 \pm .08$ &  $\efirstSig{.608 \pm .06}\,\,\spval{6.88}{-71}$ &   $.938 \pm .04$ &  $.671 \pm .07$ \\
      {\larged{}}  &  $\esecondSig{.603 \pm .07}\,\,\spval{9.25}{-95}$ &                                    $.636 \pm .07$ &  $\efirstSig{.589 \pm .05}\,\,\spval{6.97}{-12}$ &  $1.003 \pm .07$ &  $.662 \pm .06$ \\
      \bottomrule
    \end{tabular}
% \end{table}
\end{center}

\end{aside}
\fi
