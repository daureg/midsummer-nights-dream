Stattner, Erick, and Martine Collard. "Link Clustering for Extracting Collaborative Patterns in a Scientific Co-Authored Network.", FAB workshop, ANOSAM 2017
they use their method of conceptual links

Groups of nodes are first created when they have common attributes. The set of links between two
groups is called a conceptual link in the sense that it corresponds to a relationship between two
sets of attributes, namely two concepts in the field of formal concept analysis [7]. The number of
links between two groups is then evaluated and when the frequency is greater than a given support
threshold $\beta$, we talk about frequent conceptual link. Note that we only keep maximal frequent conceptual links (mfcl), i.e. those who are not included in others.
Finally, the set of maximal frequent conceptual links is used to create a new network structure called a conceptual view which summarizes all the knowledge extracted from the initial network. In a conceptual view, each node corresponds to a set of attributes (in this context we call them Meta-nodes) and a link corresponds to maximal frequent conceptual links.
A formal presentation of frequent conceptual links can be found in [18].

\Textcite[section 7.2 edge labelling]{nodeClassif11} mention a cool paper from Darja Krushevskaja
but I can't find it online.

\fullcite{Zhuang2012} and 
section 2.3 of Semantic Mining of Social Networks (same thing I believe.
Actually they also talk about a WSDM paper on infering ties using transfer
learning~\autocite{transferTiesPred16})

\Textcite[Section 4.2.2]{miningSocialTheoriesSurvey14} is about \emph{Social Tie
Prediction}, but they only mention \autocite{transferTiesPred16} and
\autocite{Yang2012}.

H. Liang, K. Wang and F. Zhu, "Mining social ties beyond homophily," 2016 IEEE 32nd International Conference on Data Engineering (ICDE), Helsinki, 2016, pp. 421-432.
\url{http://doi.org/10.1109/ICDE.2016.7498259} very datamining approach, they
look for link pattern with support and confidence above some threshold, that
also diverge from homophily by excludingi attributes taking the same value (so
it's not exactly like us once again)

The kind of graphs with talked in this paper is a special case of what is
called \emph{heterogenous information network} in \autocite{HINSurvey17}.
